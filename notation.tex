%!TEX root = main.tex

\chapter*{Notation}

\begin{itemize}
    \item We use the normalization of the Fourier transform given by the formula
    %
    \[ \widehat{f}(\xi) = \int f(x) e^{- 2 \pi i \xi \cdot x}\; dx. \]

    \item We use the translation operators $\text{Trans}_y f(x) = f(x - y)$, and $L^\infty$-normalized dilations $\text{Dil}_t f(x) = f(x/t)$. Overloading notation, we also consider dyadic dilations of the form $\text{Dil}_j f(x) = f(x/2^j)$. The dyadic dilation operator will only be used along with symbols that stand for integers, like $n$, $m$, $j$, or $k$, whereas the other dilation operator will be used in all other cases, so the operator used should be clear from context.

    \item On $\RR^d$, we use $\partial_j$ to denote the usual partial derivative operators, and $D_j$ to denote the self-adjoint normalization $D_j f = (2 \pi i)^{-1} \partial_j$. This notation has the convenience that for a polynomial $P$,
    %
    \[ P(D_j) \{ f \} = \int P(\xi) \widehat{f}(\xi) e^{2 \pi i \xi \cdot x}\; d\xi, \]
    %
    and simplifies many formulas associated with Fourier integral operators.

    \item We will often use the Japanese bracket $\langle x \rangle \coloneqq (1 + |x|^2)^{1/2}$ for $x \in \RR^d$.

    \item A function $f: \RR^n \times \RR^p \to \RR$ is a \emph{symbol of order $s$} if it satisfies bounds of the form
    %
    \[ | \partial_x^\alpha \partial_\theta^\beta f (x,\theta) | \lesssim_{\alpha,\beta} \langle \theta \rangle^{s - |\beta|} \]
    %
    for all multi-indices $\alpha$ and $\beta$.

    \item For a measure space $X$, $L^\infty(X)$ is the Banach space of essentially-bounded functions, defined almost everywhere. For a set $X$, $l^\infty(X)$ is the Banach space of bounded functions on $X$, defined everywhere.

    \item All operators we consider are Schwartz operators between manifolds equipped with a canonical measure. Abusing notation, we thus identify an operator with the distribution that gives it's Schwartz kernel. Thus, for a Schwartz operator $A$ from a manifold $Y$ to a manifold $X$, we might write
    %
    \[ A f(x) = \int_Y A(x,y) f(y)\; dy. \]

    \item For $1 \leq p \leq \infty$ and $s \in \RR$, we consider the Sobolev spaces
    %
    \[ \| f \|_{W^{s,p}(\RR^d)} = \| (-\Delta)^{s/2} f \|_{L^p(\RR^d)}, \]
    %
    and for $1 \leq r \leq \infty$, we consider the Besov spaces
    %
    \[ \| f \|_{B^{s,p}_r(\RR^d)} = \| P_k f \|_{l^r_k L^p(\RR^d)}, \]
    %
    where $\{ P_k \}$ are Littlewood-Paley projections.
\end{itemize}