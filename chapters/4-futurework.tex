%!TEX root = ../main.tex

\section{Exploiting Tangency Bounds in $\RR^3$ and $\RR^4$}

The results of Heo, Nazarov, and Seeger only apply when $d \geq 4$. Cladek found a method to get an initial radial multiplier conjecture result in $\RR^3$, and an improvmeent of the bounds obtained by Heo, Nazarov, and Seeger when $d = 3$. The idea is to exploit the fact that one need only prove a version of \ref{lemma2} for a set $\mathcal{E} = \mathcal{E}_X \times \mathcal{E}_R$, where $\mathcal{E}_X$ is a one-separated family of points, and $\mathcal{E}_R$ are a family of radii. One can then exploit this Cartesian product structure when analyzing functions of the form
%
\[ F = \sum_{(x,r) \in \mathcal{E}} \chi_{x,r}, \]
%
in particular, improving upon the result of \cite{HeoandNazarovandSeeger}.

\subsection{Result in 3 Dimensions}

As in \cite{HeoandNazarovandSeeger}, Cladek first performs a density decomposition, i.e. writing
%
\[ F = \sum F_k^m \]
%
where
%
\[ F_k^m = \sum_{(x,r) \in \mathcal{E}_k(2^m)} \chi_{x,r}. \]
%
Cladek then interpolates between an $L^0$ bound and an $L^2$ bound on the resulting functions. The $L^0$ bound is exactly the same bound used in \cite{HeoandNazarovandSeeger}.

\begin{theorem}
    For the function $F$, we have
    %
    \[ |\text{supp}(F_k^m)| \lesssim 2^{-m} 4^k \# \mathcal{E}_k \]
    %
    and thus
    %
    \[ |\text{supp}(F^m)| \lesssim \sum_k 2^{-m} 4^k \# \mathcal{E}_k. \]
\end{theorem}

The $L^2$ bound is improved upon, which is what allows us to obtain a new result in three dimensions.

\begin{lemma} \label{cladeksl2}
    Suppose $\mathcal{E} = \bigcup_k \mathcal{E}_k$ is a one-separated set, where $\mathcal{E}_k \subset \RR^d \times [2^k,2^{k+1})$ is a set of density type $(2^m, 2^k)$. Then
    %
    \[ \left\| \sum_{(x,r) \in \mathcal{E}} \chi_{x,r} \right\|_{L^2(\RR^d)} \lesssim_\varepsilon 2^{[(11/13) + \varepsilon] m} \sum_k 4^k \# \mathcal{E}_k. \]
\end{lemma}

Interpolation thus yields that for a set of density type $2^m$ as in this Lemma,
%
\[ \| \sum_{(x,r) \in \mathcal{E}} \chi_{x,r} \|_{L^p(\RR^d)} \lesssim_\varepsilon 2^{-m(1/p - 12/13 - \varepsilon)} ( \sum_k 4^k \# \mathcal{E}_k )^{1/p}. \]
%
If $1 < p < 13/12$, this sum is favorable in $m$, and may be summed without harm to prove the radial multiplier conjecture for unit scale radial multipliers in this range.

\begin{proof} [Proof of Lemma \ref{cladeksl2}]
    Write
    %
    \[ F_k = \sum_{(x,r) \in \mathcal{E}_k} \chi_{x,r}. \]
    %
    As before, we can throw away terms for $k \leq 10 m$, i.e. obtaining that
    %
    \[ \| \sum F_k \|_{L^2(\RR^d)} \lesssim m^{1/2} \left( \sum_k \| F_k \|_{L^2(\RR^d)}^2 + \sum_{10m < k < k'} |\langle F_k, F_{k'} \rangle| \right)^{1/2}. \]
    %
    Our proof thus splits into two cases: where the radii are incomparable, and where the radii are comparable.

    TODO:
\end{proof}

\subsection{Results in 4 Dimensions}

TODO

\section{Manifolds With Periodic Geodesic Flow}

\section{Abstract Reductions In Manifolds With Constant Sectional Curvature}