%!TEX root = ../main.tex

\chapter{Multipliers of an Elliptic Operator}

Let us now more precisely describe the problem of this thesis. Let $X$ be a compact manifold equipped with a volume density $d\omega$, and let $P$ be an elliptic operator on $X$ of order $a > 0$, formally positive-definite in the sense that
%
\[ \langle Pf, g \rangle = \langle f, Pg \rangle \quad\text{and}\quad \langle Pf, f \rangle \geq 0 \quad\quad\text{for all $f,g \in C^\infty(M)$}. \]
%
By general properties of elliptic operators, $1 + P$ is an isomorphism between the Sobolev space $H^a(X)$ and $L^2(X)$,
%, and because $P$ is positive-definite, we have an estimate of the form
%
%\[ \big\| (1 + P) f \big\|_{L^2(X)} \geq \| f \|_{L^2(X)}, \]
%
%which implies $1 + P$ is a bijection between $H^a(X)$ and $L^2(X)$.
and so the inverse $(1 + P)^{-1}$, viewed as a map from $L^2(X)$ to itself, is compact by a form of the Rellich-Kondrachov embedding theorem. By the spectral theorem for compact operators, there exists a discrete set $\Lambda \subset [0,\infty)$, and an orthogonal decomposition
%
\[ L^2(X) = \bigoplus\nolimits_{\lambda \in \Lambda} \mathcal{V}_\lambda, \]
%
where $\mathcal{V}_\lambda$ is a finite dimensional subspace of $C^\infty(X)$, such that $Pf = \lambda f$ for all $f \in \mathcal{V}_\lambda$. The Weyl Law for such operators tells us that $\# ( \Lambda \cap [0,R] ) = \text{Vol}(M) R^d + O(R^{d-1})$, where
%
\[ \text{Vol}(M) = \int_{T^* M} \II[ p(x,\xi) \leq 1 ]\; d\xi\; dx, \]
%
the function $p: T^* M \to [0,\infty)$ being the principal symbol of $P$. See TODO for details.

We use this decomposition to define a functional calculus for the operator $P$; given a bounded function $a: \Lambda \to \CC$, we can define an operator $m(P)$ on $L^2(X)$ so that $a(P) f = a(\lambda) f$ for all $f \in \mathcal{V}_\lambda$; such an operator is bounded on $L^2(X)$ by orthogonality. To study further regularity of $m(P)$, we introduce the spaces $M^{p,q}(X)$, consisting of all functions $a: [0,\infty) \to \CC$ for which the operator $a(P)$ is  bounded from $L^p(X)$ to $L^q(X)$. The space $M^{p,q}(X)$ is then a seminorm space with respect to the seminorm
%
\[ \| a \|_{M^{p,q}(X)} \coloneqq \sup \left\{ \frac{\| a(P) f \|_{L^q(X)}}{\| f \|_{L^p(X)}} : f \in C^\infty(X) \right\}. \]
%
For notational convenience, we write $M^p(X) = M^{p,p}(X)$. As mentioned above, orthogonality immediate implies that $\| a \|_{M^2(X)} = \| a \|_{l^\infty(\Lambda)}$.

In the sequel, we will restrict our study to formally positive elliptic operators \emph{of order one}; this does not restrict the scope of our analysis; given a formally positive elliptic operator $P$ of order $s$, the operator $P^{1/s}$ defined by the functional calculus is a formally positive elliptic operator of order one\footnote{See Theorem 3.3.1 of \cite{Sogge} for details, based on a technique of \cite{Seeley} initially developed for elliptic differential operators.}, and the spectral theory of $P$ is identical with the spectral theory of $P^{1/s}$. Fixing the order of our operator reflects the fact that the theory of Fourier integral operators we will eventually employ is most elegant when the resulting oscillatory integrals have homogeneous phases of order one.

The primary example of such operators are obtained from a Laplace-Beltrami operator on a compact Riemannian manifold $M$; the operator $-\Delta$ is self-adjoint and positive-definite with respect to the volume form $dV$ on $M$, because of the `integration by parts' identity $\langle \Delta f, g \rangle = - \langle \nabla f, \nabla g \rangle$. Since we restrict to elliptic operators of order one, the canonical object of study is the operator $P = \sqrt{-\Delta}$.

The discrete spectrum of operators on a compact manifold makes it difficult to apply methods of oscillatory integrals to the problem. Fortunately, certain semiclassical heuristics tell us that these problems disappear for 'high frequency inputs'. In order to take advantage of this fact, rather than studying what conditions ensure the operators $m(P)$ are bounded, we study conditions that ensure the operators $a_R(P) = a(P/R)$ are uniformly bounded as $R \to \infty$, since these operators depend more and more on 'high frequency behaviour' in the limit. To prevent pathological examples from arising under dilation, we restrict ourselves to \emph{regulated functions} $a$, i.e. functions $a: [0,\infty) \to \CC$ such that
%
\[ a(\lambda_0) = \lim_{\delta \to 0} \fint_{|\lambda - \lambda_0| \leq \delta} a(\lambda)\; d\lambda \quad \text{for all $\lambda_0 \in [0,\infty)$}. \]
%
%For the manifolds, elliptic operators, and exponents that we focus on in this thesis, for any regulated function $a$,
%
%\[ \sup\nolimits_R \| a_R \|_{M^{p,q}(X)} \sim \limsup\nolimits_{R \to \infty} \| a_R \|_{M^{p,q}(X)}. \]  
%
%Thus uniform boundedness as $R \to \infty$ gives uniform boundedness for all $R > 0$,
Define the closed subspace $M^{p,q}_{\text{Dil}}(X) \subset M^{p,q}(X)$ of regulated functions $a: [0,\infty) \to \CC$ such that the norm $\| a \|_{M^{p,q}_{\text{Dil}}(X)} = \sup_R \| a_R \|_{M^{p,q}(X)}$ is finite.

With notation introduced, we can now more precisely state the main problem that will be studied in this thesis:
\begin{changemargin}{2cm}{2cm}
\begin{center}
  \emph{What conditions are necessary and sufficient for\\ a function to be contained in $M^{p,q}_{\text{Dil}}(X)$ for $(p,q) \neq (2,2)$.}
\end{center}
\end{changemargin}
%
Before the results of this thesis, no such results were known for $X \neq \TT^d$, aside from the simple bound $\| a \|_{M^2_{\text{Dil}}(X)} = \| a \|_{L^\infty[0,\infty)}$ which holds by orthogonality. We will obtain results under two main assumptions. The first assumption is a curvature condition: for each $x_0 \in M$, the cosphere
%
\[ S_{x_0} = \{ \xi \in T^*_{x_0} M : p(x_0,\xi) = 1 \} \]
%
must be a hypersurface in $T^*_x M$ with non-vanishing Gauss curvature. The second assumption we make is that the eigenvalues of the operator $P$ are contained in an arithmetic progression.

For most manifolds $M$ and $P$, elements of $\mathcal{V}_\lambda$ are difficult to describe explicitly; even for the relatively simple case of the sphere $S^d$ many problems about such functions (e.g. such as problems involving nodal domains) remain open. Many arguments in harmonic analysis involve an interplay between spatial and frequential control, and without explicit descriptions of eigenfunctions, spatial control becomes difficult. Nonetheless, we will find we can obtain some spatial control by utilizing the wave equation on $M$, which carries geometric information via the behaviour of wave propogation. Under the curvature assumptions we make, wave propogation has sufficient smoothing properties to match certain necessary conditions that multipliers need in order to be bounded. And the fact that the eigenvalues of the operator $P$ are contained in an arithmetic progression implies that the wave equation is periodic, which simplifies the large time analysis of the wave equation.

%Up to lower order pertubations, the Laplace-Beltrami operator is the unique elliptic operator of order two whose principal symbol is equal to the square of the norm induced on $T^* M$ from the Riemannian metric on $M$. In Chapter BLAH, we will see that this property is true of \emph{all} elliptic operators that satisfy the first assumption of our problem, provided that we are willing to generalize our study to \emph{Finsler manifolds} rather than just Riemannian manifolds. The second assumption is then closely related to the behaviour of geodesics on the manifold, in particular, that all geodesics and closed and have common length.

We begin this thesis by describing what is currently known for the boundedness problem on $\TT^d$, where eigenfunctions are explicit, and one can study the behaviour of spectral multipliers via the Fourier transform and radial convolution operators on $\RR^d$, in order to gain intuition and motivate potential hypotheses in the general setting.

%Under this assumption, the function $p$ defines a \emph{Finsler metric} on $M$, and thus we can define a theory of geodesics on $M$. The second assumption we make is that with respect to this metric, all geodesics on $M$ are closed, and have common length.

%
%\begin{itemize}
%    \item Consider what seems a relatively simple case, the sphere $S^d$ equipped with the Laplacian. Then $\Lambda = \{ k(k+d-1) : k \geq 0 \}$, and for $\lambda = k(k+d-1)$, the space $\mathcal{V}_\lambda$ consists of the spherical harmonics of degree $k$, restrictions of harmonic, homogeneous, degree $k$ polynomials on $\RR^{d+1}$ to $S^d$. There is a basis for such a space defined in terms of the associated Legendre polynomials, but this is still relatively non-explicit.

%    \item On the other hand, on $\TT^d$ with it's Laplacian $-\Delta = - \partial_1^2 - \cdots - \partial_d^2$, for each $\lambda \in \Lambda$, $\mathcal{V}_\lambda$ has an orthogonal basis consisting of linear combinations of the exponentials $x \mapsto e^{2 \pi i k \cdot x}$, and thus multipliers of the Laplacian can be understood using the Fourier series.
%\end{itemize}
%

\chapter{Radial Multipliers on Euclidean Space}

Consider the elliptic operator $P = \sqrt{-\Delta}$ on $\TT^d$, where $\Delta = \partial_1^2 + \cdots + \partial_d^2$ is the usual Laplacian. In such a setting, we have an explicit basis for the eigenfunctions of $\Delta$: for a given eigenvalue $\lambda > 0$, the space $\mathcal{V}_\lambda$ has an orthonormal basis consisting of the exponentials $e^{2 \pi i n \cdot x}$, where $n \in \ZZ^d$ and $|n| = \lambda$. Since the Fourier series of a function gives the expansion of the function in this basis, it follows that we can expand the spectral multiplier operator $T = a(P)$ using a Fourier series, i.e. writing
%
\[ Tf(x) = \sum\nolimits_{n \in \ZZ^d} a( |n| ) \widehat{f}(n) e^{2 \pi i n \cdot x}. \]
%
Thus a multiplier of $\Delta$ on $\TT^d$ is nothing more than a Fourier multiplier operator on $\TT^d$ whose symbol is radial, i.e. depending only on the magnitude of the frequency. Methods of transference\footnote{See Section 3.6.2 of Grafakos \cite{Grafakos}, based on methods of de Leeuw \cite{deLeeuw}} show that the operators $m_R(P)$ are uniformly bounded from $L^p(\RR^d)$ to $L^q(\RR^d)$ if and only if the Fourier multiplier operator on $\RR^d$ given by
%
\[ Tf(x) = \int_{\RR^d} a \big(|\xi| \big) \widehat{f}(\xi) e^{2 \pi i \xi \cdot x}\; d\xi \]
%
is bounded from $L^p(\RR^d)$ to $L^q(\RR^d)$, and so in this section we will focus on operators of this type, which we call \emph{radial Fourier multiplier operators}.

The study of the regularity of Fourier multiplier operators has proved central to the development of modern harmonic analysis and the theory of linear partial differential operators. This is because essentially any translation invariant operator $T$ is a Fourier multiplier operator, i.e. we can find a tempered distribution $m$ on $\RR^d$, the \emph{symbol} of $T$, such that for any Schwartz function $f$,
%
\[ Tf(x) = \int_{\RR^d} m(\xi) \widehat{f}(\xi) e^{2 \pi i \xi \cdot x}\; d\xi. \]
%
%Applying the notation of spectral calculus, one might also write this operator as $m(D)$, where $D = (2 \pi i)^{-1} \nabla$ is a self-adjoint normalization of the gradient operator. Thus the study of the boundedness of translation invariant operators is closely connected to the study of the interactions of the operators
%
%\[ E_\xi f(x) \coloneqq \widehat{f}(\xi) e^{2 \pi i \xi \cdot x}, \]
%
%which act as projections onto the common eigenspaces of the components of $D$, since we can write $m(D)$ as a vector-valued integral of the form
%
%\[ m(D) = \int_{\RR^d} m(\xi) E_\xi\; d\xi. \]
%
%Thus $m(D)$ is represented as a weighted average of the operators $\{ E_\xi \}$.
The study of translation invariant operators emerges from classical questions in analysis, such as the convergence of Fourier series, and problems in mathematical physics related to the study of the heat, wave, and Schr\"{o}dinger equations. These physical equations also often have \emph{rotational} symmetry, so it is natural to restrict our attention to translation-invariant operators which are also rotation-invariant. These operators are precisely the family of radial multiplier operators.
%Such operators are precisely those operators associated with \emph{radial} symbols $m: \RR^d \to \CC$, i.e. symbols for which there exists a function $h: [0,\infty) \to \RR$ such that
%
%\[ m(\xi) = h( |\xi|) \]
%
%for some function $h: [0,\infty) \to \CC$. This is the class of \emph{radial Fourier multipliers}. The study of radial multipliers is closely connected to interactions between the operators
%
%\[ E_\lambda f(x) \coloneqq \int_{|\xi| = \lambda} \widehat{f}(\xi) e^{2 \pi i \xi \cdot x}\; d\xi, \]
%
%for $0 < \lambda < \infty$, which are the projection operators onto the eigenspaces of $\Delta$.  Similar to the study of $m(D)$, we then have
%
%\[ h \left( P \right) = \int_0^\infty h(\lambda) E_\lambda\; d\lambda. \]
%
%Thus studying the regularity of radial Fourier multipliers allows us to understand the interactions between the operators $\{ E_\lambda \}$.

\section{Convolution Kernels of Fourier Multipliers}

It is often useful to study spatial representations of these operators, since one can often exploit certain geometry to obtain useful results. Given any translation invariant operator $T = a(P)$ on $\RR^d$, we can associate a tempered distribution $k$, the \emph{convolution kernel} of $T$, such that
%
\[ Tf(y) = \int_{\RR^d} k(x) f(y-x)\; dx \quad\quad\text{for any Schwartz $f \in \mathcal{S}(\RR^d)$}. \]
%
If $T$ is radial, then so is $k$, and so we can write $k(x) = b(|x|)$ for some distribution $b$ on $[0,\infty)$, and then we have a representation
%
\[ T = \int_0^\infty b(r) S_r\; dr,\quad\text{where}\quad S_rf(x) \coloneqq \int_{|y| = r} f(x + y)\; dy \]
%
are the \emph{spherical averaging operators}. Thus problems about radial Fourier multiplier operators are also connected to spherical averaging problems.

With the notation as above, the function $k$ is the Fourier transform of $m$, i.e. $k = \mathcal{F}_d m$, and the function $b$ is a Bessel transform of $a$, i.e. $b = \mathcal{B}_d a$, where
%
\[ \mathcal{B}_d a(r) \coloneqq (2 \pi) r^{1-d/2} \int_0^\infty a(\lambda) J_{d/2 - 1}(2 \pi \lambda r) \lambda^{d/2}\; d\lambda. \]
%
Here $J_{d/2-1}$ is the Bessel function of order $d/2 - 1$, given by the formula
%
\[ J_\alpha(\lambda) \coloneqq \frac{(\lambda / 2)^\alpha}{\Gamma(\alpha + 1/2)} \int_{-1}^1 e^{i \lambda s} (1 - s^2)^{\alpha - 1/2}\; ds. \]
%
Using the theory of stationary phase, for each $d$ we can write
%
\[ J_d(\lambda) = e^{2 \pi i \lambda} s_1(\lambda) + e^{-2 \pi i \lambda} s_2(\lambda). \]
%
for symbols $s_1$ and $s_2$ of order $-1/2$. The presence of $e^{2 \pi i \lambda}$ and $e^{-2 \pi i \lambda}$ allows one to relate the Bessel transform of $h$ to it's Fourier transform, to a certain extent. In particular, we record the following result of Garrigos and Seeger \cite{GarrigosandSeeger}.

\begin{theorem}
    Suppose $d > 1$ and $1 < p < 2d/(d+1)$, and fix $\phi \in C_c^\infty[0,\infty)$. Then for any function $a: [0,\infty) \to \CC$,
    %
    \[ \left( \int_0^\infty |\mathcal{B}_d \{ \phi a \} (t)|^p t^{d-1} \right)^{1/p} \sim_{p,d,\phi} \left( \int_0^\infty |\mathcal{C}_d \{ \phi a \}(t)|^p \langle t \rangle^{(d-1)(1 - p/2)}\; dt \right)^{1/p}, \]
    %
    where $\mathcal{C}_d f(t) = \int_0^\infty f(\lambda) \cos(2 \pi \lambda t)\; dt$ is the cosine transform of $f$.
\end{theorem}

\section{Multipliers on Euclidean Space}

The general study of the boundedness properties of Fourier multiplier operators in multiple variables was initiated in the 1950s, as connections of the theory to partial differential equations became more fully realized\footnote{See \cite{Hormander1} for a discussion of what was discovered at this time, and for a more detailed exposition of the content of this section.}. It was quickly realized that the most fundamental estimates were $L^p \to L^q$ estimates for such operators.
%
%for $1 \leq p \leq 2$, and $q \geq p$, which by duality are equivalent to bounds
%
%\[ \| Tf \|_{L^{p^*}(\RR^d)} \lesssim \| f \|_{L^{q^*}(\RR^d)}, \]
%
It is therefore natural to introduce the space $M^{p,q}(\RR^d)$, consisting of all symbols $m$ which induce a Fourier multiplier operator $T$ bounded from $L^p(\RR^d)$ to $L^q(\RR^d)$, with the operator norm of $T$ giving a norm on the space. The transference principles mentioned in the previous section imply that for any regulated function $m$,
%
\[ \| a \|_{M^{p,q}_{\text{Dil}}(\TT^d)} = \| a(|\cdot|) \|_{M^{p,q}(\RR^d)}. \]
%
Duality implies that $M^{p,q}(\RR^d)$ is isometric to $M^{q^* \negmedspace ,p^*}(\RR^d)$, where $p^*$ and $q^*$ are the conjugates to $p$ and $q$. And the fact these operators are translation invariant, together with Littlewood's Principle, implies that $M^{p,q}(\RR^d) = \{ 0 \}$, unless $q \geq p$. Combining these reductions, we see that it suffices to study the spaces $M^{p,q}(\RR^d)$ where $1 \leq p \leq 2$ and where $q \geq p$, or alternatively, the spaces $M^{p,q}(\RR^d)$, where $2 \leq p \leq \infty$ and $q \geq p$. We only know simple characterizations of $M^{p,q}(\RR^d)$ for very particular $p$ and $q$:
%
\begin{itemize}
    \item The spaces $M^{1,q}(\RR^d) = M^{q^*,\infty}(\RR^d)$ are characterized by virtue of the fact that the study of the boundedness of operators with domain $L^1(\RR^d)$ or range $L^\infty(\RR^d)$ is often simple; we have
    %
    \[ M^{1,q}(\RR^d) = \widehat{L^q}(\RR^d) \quad\text{for $q > 1$,}\quad \text{and}\quad M^{1,1}(\RR^d) = \widehat{M}(\RR^d), \]
    %
    Here $M(\RR^d)$ is the space of all finite signed Borel measures, equipped with the total variation norm. The proof follows from Schur's Lemma, which often gives tight estimates when bounding for operators in terms of the $L^1$ norm of their input.

    \item The unitary nature of the Fourier transform implies $M^2(\RR^d) = L^\infty(\RR^d)$. The proof follows from Parseval's identity.
    %
    % | m f^ |_{L^2} <= | m |_q | f |_p
    % If f^ is an extremizer for m
    % where $q = 2p/(2-p)$ for $1 \leq p < 2$, and $q = \infty$ for $p = 2$.
\end{itemize}
%
It is perhaps surprising that these are the \emph{only} known characterizations of the spaces $M^{p,q}(\RR^d)$. No necessary and simple conditions for boundedness are known for any other values of $p$ and $q$, and perhaps no simple characterization exists.

\section{The Radial Multiplier Conjecture}

Despite the continuing lack of a complete characterization of the classes $M^{p,q}(\RR^d)$, it is surprising that we \emph{can} conjecture a characterization of the subspace of $M^{p,q}(\RR^d)$ consisting of \emph{radial symbols}, for an appropriate range of exponents. The conjectured range of estimates was first suggested by the result of \cite{GarrigosandSeeger}, which concerned radial multipliers $m$ whose associated operator $T$ is bounded from the $L^p$ norm to the $L^q$ norm \emph{restricted to radial functions}, i.e. such that the norm
%
\[ \| m \|_{M^{p,q}_{\text{Rad}}(\RR^d)} = \sup \left\{ \frac{\| m(P)f \|_{L^q(\RR^d)}}{\| f \|_{L^p(\RR^d)}} : \text{$f$ is radial} \right\} \]
%
is finite. For $d > 1$, in the range $1/p - 1/2 > 1/2d$, for $p \leq q < 2$, and for a radial multiplier $m$, Garrigos and Seeger proved that
%
\begin{equation} \label{RadialMultiplierRadialBound}
    \| m \|_{M^{p,q}_{\text{Rad}}(\RR^d)} \sim \sup\nolimits_{j \in \ZZ} 2^{jd(1/p - 1/q)} \| k_j \|_{L^q(\RR^d)},
\end{equation}
%
where $m_j(t) = \chi(t) m(t / 2^j)$ for a smooth compactly supported $\chi$ with $1 = \sum \chi(t/2^j)$, and $k_j$ is the convolution kernel of $m_j(P)$.

The upper bound of \eqref{RadialMultiplierRadialBound} follows from rescaling the convolution identity $k_j * \psi = k_j$, where $\psi$ is a radial Schwartz function whose Fourier transform is equal to one on the Fourier support of $\chi$. The hard part of the result of \cite{GarrigosandSeeger} is that such a condition is sufficient. The range of $p$ and $q$ in this result is sharp. The result cannot hold for $p \geq 2d/(d+1)$, because, adapting the analysis of Fefferman \cite{Fefferman} to a family of multiplier operators whose symbols are smooth and adapted to a $\delta$ neighborhood of an annulus as $\delta \to 0$, such a result would imply all Kakeya sets have positive measure. TODO: WHAT HAPPENS WHEN $q > 2$.

%hold. On the other hand, by Plancherel, we can find a radial function $f$ with compact Fourier support such that $\| \kappa_t * f \|_{L^2(\RR^d)} \gtrsim 1$ uniformly as $t \to \infty$ (consider such $f$ with Fourier support concentrated near the maximum value of the Fourier transform of $\kappa_0$). By Bernstein's inequality,
%
%\[ \| m_t \|_{M^{p,q}(\RR^d)} \| \kappa_t * f \|_{L^p(\RR^d)} \gtrsim 1 \]

%for each $t$ we can find a radial function $f_t$ with Fourier support near $1$, with $\| f_t \|_{L^2(\RR^d)} = 1$ and with $\| \kappa_t * f_t \|_{L^2(\RR^d)} \gtrsim 1$. By Sobolev embedding we must also have $\| \kappa_t * f_t \|_{L^p(\RR^d)} \gtrsim 1$ and $\| f_t \|_{L^p(\RR^d)} \lesssim 1$.

%And one can see that the condition is not sufficient for $q > \min \left(2, \tfrac{d-1}{d+1} p' \right)$ by testing the same multiplier operators above on `Knapp examples', i.e. smooth functions whose Fourier transforms are adapted to a $\delta$ cap on a $\delta$-neighborhood of the annulus upon which $m$ is supported.

It is natural to conjecture that the same constraint continues to hold when we remove the constraint that our inputs $f$ are radial, i.e so that for a radial function $m$, for $d > 1$, $1/2d < 1/p - 1/2 < 1/2$, and for $p \leq q < 2$,
%
\[ \| m \|_{M^{p,q}(\RR^d)} \sim_{p,q,d} \sup\nolimits_{j \in \ZZ} 2^{jd(1/p - 1/q)} \| k_j \|_{L^q(\RR^d)}. \]
%
%and for general locally integrable symbols $m$,
%
%\[ \| m \|_{M^{p,q}} \sim_{p,q,d} \| k \|_{\dot{B}_{-d/p^*}^{q,\infty}} \]
%
However, for $1/2d < 1/p - 1/2 < 1/(d+1)$ an additional counterexample is obtained by considering the multipliers supported on $\delta$ annuli above, and testing them against smooth functions whose Fourier transforms are adapted to $\delta$ caps on the annuli. These counterexamples show that the condition $q < (d-1)/(d+1) \cdot p'$ is also necessary. In the sequel, we call this characterization of the radial functions in $M^{p,q}(\RR^d)$ the \emph{radial multiplier conjecture} on $\RR^d$.

We now know, by results of Heo, Nazarov, and Seeger \cite{HeoandNazarovandSeeger} that the radial multiplier conjecture is true when $d \geq 4$ and when $1/p - 1/2 > 1/(d+1)$. A summary of the proof strategies of this argument is provided in Section \ref{HeoNazarovSeeger}, and a major part of our bounds for spectral multipliers follow by adapting this argument to the non-Euclidean setting. Partial improvements were obtained by Cladek \cite{Cladek} for symbols $m$ compactly supported away from the origin, obtaining results for $d = 4$ and $1 < p < 36/29$, and establishing \emph{restricted weak type} bounds when $d = 3$ and $1 < p < 13/12$. Cladek's argument is described in Section \ref{Cladek}. But the radial multiplier conjecture has not yet been completely resolved in any dimension $d$, we do not have any strong type $L^p$ bounds when $d = 3$, and no bounds whatsoever are known when $d = 2$.

\section{Radial-Multiplier Bounds By Density Decompositions}

In this section, we give an overview of the proof of the radial multiplier bounds obtained by Heo, Nazarov, and Seeger in \cite{HeoandNazarovandSeeger}.

\begin{theorem} \label{HeoNazarovSeegerTheorem}
    Suppose $1 < p < 2(d-1)/(d+1)$, and $m: \RR^d \to \CC$ is a radial function. Then for $p \leq q \leq 2$,
    %
    \begin{equation} \label{HeoNazarovSeegerMainInequality}
        \| m \|_{M^{p,q}(\RR^d)} \lesssim \sup\nolimits_{j \in \ZZ} 2^{jd(1/p - 1/q)} \| k_j \|_{L^q(\RR^d)},
    \end{equation}
    %
    where $m_j(t) = \chi(t) m(t / 2^j)$ for a smooth compactly supported $\chi$ with $1 = \sum \chi(t/2^j)$, and $k_j$ is the convolution kernel of $m_j(P)$.
\end{theorem}

We begin by establishing a single scale version of this result, i.e. that if the Fourier transform of $k$ is supported on the annulus $|\xi| \sim 1$, then
%
\begin{equation} \label{HeoNazarovSeegerSingleScaleInequality}
    \| k * f \|_{L^p(\RR^d)} \lesssim \| k \|_{L^p(\RR^d)} \| f \|_{L^p(\RR^d)}.
\end{equation}
%
We reduce this to an inequality for sums of functions oscillating on spheres. Let $\sigma_r$ be the surface measure for the sphere of radius $r$ centered at the origin in $\RR^d$. Also fix a nonzero, radial, compactly supported function $\psi \in \mathcal{S}(\RR^d)$ whose Fourier transform is non-negative, and vanishes to high order at the origin. Given $x \in \RR^d$ and $r \geq 1$, define $\chi_{x, r} = \text{Trans}_x (\sigma_r * \psi)$. Then $\chi_{x,r}$ is a smooth function adapted to a thickness $O(1)$ annulus of radius $r$ centered at $x$, which is \emph{slightly oscillating}. We will verify the following lemma.

\begin{lemma} \label{lemma1}
    For any $a : \RR^d \times [1,\infty) \to \CC$, and $1 \leq p < p_d$,
    %
    \[ \left\| \int_{\RR^d} \int_1^\infty a(x,r) \chi_{x, r}\; dx\; dr \right\|_{L^p(\RR^d)} \lesssim \left( \int_{\RR^d} \int_1^\infty |a(x,r)|^p r^{d-1} dr dx \right)^{1/p}. \]
    %
    The implicit constant here depends on $p$, $d$, and $\psi$.
\end{lemma}

Lemma \ref{lemma1} implies \eqref{HeoNazarovSeegerSingleScaleInequality}. Suppose $k(\cdot) = b(|\cdot|)$ for some function $b: [0,\infty) \to \CC$. If we set $a(x,r) = f(x) b(r)$ for any function $f: \RR^d \to \CC$, then
%
\[ k * \psi * f = \int_{\RR^d} \int_1^\infty a(x,r) \chi_{x, r}\; dx\; dr, \]
%
Lemma \ref{lemma1} thus says that
%
\[ \| k * \psi * f \|_{L^p(\RR^d)} \lesssim \| k \|_{L^p(\RR^d)} \| f \|_{L^p(\RR^d)}. \]
%
If we choose $\psi$ so that $\widehat{\psi}$ is non-vanishing on the support of $k$, then the function $1/\widehat{\psi}(\cdot)$ is smooth on the support of $m$; if $T$ is a Fourier multiplier operator with a smooth, compactly supported symbol agreeing with $1/\widehat{\psi}(\cdot)$ on the support of $m$, then $T$ is bounded on $L^p(\RR^d)$, and $T(k * \psi * f) = k * f$, and so we conclude that
%
\[ \| k * \psi \|_{L^p(\RR^d)} = \| T(k * \psi * f) \|_{L^p(\RR^d)} \lesssim \| k * \psi * f \|_{L^p(\RR^d)} \lesssim \| k \|_{L^p(\RR^d)} \| f \|_{L^p(\RR^d)}. \]
%
To prove Lemma \ref{lemma1}, it suffices to prove the following discretized estimate where we replace integrals with sums.

\begin{theorem} \label{lemma2}
    Fix a 1-separated set $\mathcal{E} \subset \RR^d \times [1,\infty)$. Then for any $a: \mathcal{E} \to \CC$ and $1 \leq p < 2(d - 1)/(d+1)$, 
    %
    \[ \left\| \sum\nolimits_{(x,r) \in \mathcal{E}} a(x,r) \chi_{x, r} \right\|_{L^p(\RR^d)} \lesssim \left( \sum\nolimits_{(x,r) \in \mathcal{E}} |a(x,r)|^p r^{d-1} \right)^{1/p}, \]
    %
    where the implicit constant is independent of $\mathcal{E}$.
\end{theorem}

\begin{proof}[Proof of Lemma \ref{lemma1} from Lemma \ref{lemma2}]
    For any $a: \RR^d \times [1,\infty) \to \CC$, if we consider the vector-valued function $\mathbf{a}(x,r) = a(x,r) \chi_{x,r}$, then
    %
    \[ \int_{\RR^d} \int_1^\infty \mathbf{a}(x,r)\; dr\; dx = \int_{[0,1)^d} \int_0^1 \sum\nolimits_{n \in \ZZ^d} \sum\nolimits_{m > 0} \text{Trans}_{n,m} \mathbf{a}(x,r)\; dr\; dx \]
    %
    The triangle inequality and the increasing property of norms on $[0,1)^d \times [0,1]$ imply that
    %
    \begin{align*}
    &\left\| \int_{\RR^d} \int_1^\infty \mathbf{a}(x,r)\; dr\; dx \right\|_{L^p(\RR^d)}\\
    &\quad \leq \int_{[0,1)^d} \int_0^1 \left\| \sum\nolimits_{n \in \ZZ^d} \sum\nolimits_{m > 0} \text{Trans}_{n,m} \mathbf{a}(x,r) \right\|_{L^p(\RR^d)}\; dr\; dx\\
    &\quad \lesssim \int_{[0,1)^d} \int_0^1 \left( \sum\nolimits_{n \in \ZZ^d} \sum\nolimits_{m > 0} |a(x - n, r + m)|^p r^{d-1} \right)^{1/p}\; dr\; dx\\
    &\quad \leq \left( \int_{[0,1)^d} \int_0^1 \sum\nolimits_{n \in \ZZ^d} \sum\nolimits_{m > 0} |a(x - n, r + m)|^p r^{d-1}\; dr\; dx \right)^{1/p}\\
    &\quad = \left( \int_{\RR^d} \int_1^\infty |a(x,r)|^p r^{d-1} dr dx \right)^{1/p}. \qedhere
    \end{align*}
\end{proof}

Lemma \ref{lemma2} is further reduced by considering it as a bound on the operator
%
\[ a \mapsto \sum\nolimits_{(x,r) \in \mathcal{E}} a(x,r) \chi_{x,r}. \]
%
In particular, applying real interpolation, since we are proving a result for an open range of $p$, it suffices for us to prove a restricted strong type bound. Given any discretized set $\mathcal{E}$, let $\mathcal{E}_k$ be the set of $(x,r) \in \mathcal{E}$ with $2^k \leq r < 2^{k+1}$. Then Lemma \ref{lemma2} is implied by the following Lemma.

\begin{lemma} \label{lemma3}
    For any $1 \leq p < 2(d - 1)/(d+1)$ and $k \geq 1$,
    %
    \[ \left\| \sum\nolimits_{(x,r) \in \mathcal{E}} \chi_{x,r} \right\|_{L^p(\RR^d)} \lesssim \left( \sum\nolimits_{k \geq 1} 2^{k(d-1)} \#(\mathcal{E}_k) \right)^{1/p}. \]
\end{lemma}

\begin{remark}
    Note that if $r \sim 2^k$, then $\| \chi_{x,r} \|_{L^p(\RR^d)} \sim 2^{k(d-1)/p}$, and so Lemma \ref{lemma3} says
    %
    \[ \left\| \sum\nolimits_{(x,r) \in \mathcal{E}} \chi_{x,r} \right\|_{L^p(\RR^d)} \lesssim_p \left( \sum\nolimits_{(x,r) \in \mathcal{E}} \| \chi_{x,r} \|_{L^p(\RR^d)}^p \right)^{1/p}. \]
    %
    Thus we are proving a $p$th root cancellation bound for the norm of sums of functions supported on different annuli with slight oscillation.
\end{remark}

\begin{comment}
\begin{proof}[Proof of Lemma \ref{lemma2} from Lemma \ref{lemma3}]
    Let
    %
    \[ F = \sum_{(x,r) \in \mathcal{E}} \chi_{x,r} \]
    %
    and then for $k \geq 1$, let
    %
    \[ F_k = \sum_{(x,r) \in \mathcal{E}_k} \chi_{x,r}. \]
    %
    Then $F = \sum_k F_k$, and. Applying a dyadic interpolation result (Lemma 2.2 of that paper), the bound
    %
    \[ \| F_k \|_{L^r(\RR^d)} \lesssim 2^k (2^{k(d-r-1)} \#(\mathcal{E}_k)^{1/r}) \]
    %
    which holds for $r$ to the left and right of $p$, can be interpolated to yield that
    %
    \[ \| F \|_{L^p(\RR^d)} \lesssim \left( \sum_k 2^{kp} ( 2^{k(d-r-1)} ) \right)^{1/p} \]


    Applying a dyadic interpolation result (Lemma 2.2 of the paper), Lemma \ref{lemma3} implies that
    %
    \[ \left\| \sum_{(x,r) \in \mathcal{E}} \chi_{x,r} \right\| \]

    %
    \[ \left\| \sum_{(x,r) \in \mathcal{E}} \chi_{x,r} \right\|_{L^p(\RR^d)} \lesssim \left( \sum 2^{kp} 2^{k(d-p-1)} \#(\mathcal{E}_k) \right)^{1/p} = \left( \sum 2^{k(d-1)} \#(\mathcal{E}_k) \right)^{1/p} \]
    %
    This is a restricted strong type bound for Lemma \ref{lemma2}, which we can then interpolate.
\end{proof}
\end{comment}

To control these sums, we apply a `density decomposition', somewhat analogous to a Calderon Zygmund decomposition, which will enable us to obtain $L^2$ bounds. We say a 1-separated set $\mathcal{E}$ in a metric space $X$ is of \emph{density type} $(u,r)$ if
%
\[ \#(B \cap \mathcal{E}) \leq u \cdot \diam(B) \]
%
for each ball $B$ in $X$ with diameter at most $r$.

%A covering argument then shows that for any ball $B$,
%
%\[ \#(B \cap \mathcal{E}) \lesssim_d u \cdot \left( 1 + \frac{\diam(B)}{R} \right)^d \cdot \diam(B). \]
%
%(NOTE: WE MIGHT BE ABLE TO DO BETTER USING THE FACT THAT $\mathcal{E} \subset \RR^d \times [R,2R)$, USING THE VALUE $R$).

\begin{theorem} \label{DecompositionTheorem}
    For any 1-separated set $\mathcal{E}_k \subset \RR^d \times [2^k,2^{k+1})$, there exists a decomposition $\mathcal{E}_k = \bigcup_{m = 1}^\infty \mathcal{E}_k(2^m)$ with the following properties:
    %
    \begin{itemize}
        \item For each $m$, $\mathcal{E}_k(2^m)$ has density type $(2^m,2^k)$.

        \item If $B$ is a ball of radius $r \leq 2^k$ containing at least $2^m \cdot r$ points of $\mathcal{E}_k$, then
        %
        \[ B \cap \mathcal{E}_k \subset \bigcup\nolimits_{m' \geq m} \mathcal{E}_k(2^{m'}). \]

        \item For each $m$, there are disjoint balls $\{ B_i \}$, with radii $\{ r_i \}$, each at most $2^k$, such that
        %
        \[ \sum\nolimits_i r_i \leq 2^{-m} \# \mathcal{E}_k, \]
        %
        and such that $\bigcup B_i^*$ covers $\bigcup_{m' \geq m} \mathcal{E}_k(2^{m'})$, where $B_i^*$ denotes the ball with the same center as $B_i$ but 5 times the radius.
    \end{itemize}
\end{theorem}
\begin{proof}
    Define a function $M: \mathcal{E}_k \to [0,\infty)$ by setting
    %
    \[ M(x,r) = \sup \left\{ \frac{\#(\mathcal{E}_k \cap B)}{\text{rad}(B)} : (x,r) \in B\ \text{and}\ \text{rad}(B) \leq 2^k \right\}. \]
    %
    We can establish a kind of weak $L^1$ estimate for $M$ using a Vitali type argument. Let
    %
    \[ \widehat{\mathcal{E}}_k(2^m) = \{ (x,r) \in \mathcal{E}_k : M(x,r) \geq 2^m \}. \]
    %
    We can therefore cover $\widehat{\mathcal{E}}_k(2^m)$ by a family of balls $\{ B \}$ such that $\#(\mathcal{E}_k \cap B) \geq 2^m \text{rad}(B)$. The Vitali covering lemma allows us to find a disjoint subcollection of balls $B_1,\dots,B_N$ such that $B_1^* ,\dots, B_N^*$ covers $\widehat{\mathcal{E}}_k(2^m)$. We find that
    %
    \[ \#(\mathcal{E}_k) \geq \sum_i \#(B_i \cap \mathcal{E}_k) \geq 2^m \sum_i \text{rad}(B_i), \]
    %
    Setting $\mathcal{E}_k = \widehat{\mathcal{E}}_k(2^m) - \bigcup_{k' > k} \widehat{\mathcal{E}}_{k'}(2^m)$ thus gives the required result.
\end{proof}

To prove Lemma \ref{lemma3}, we perform a decomposition of $\mathcal{E}_k$ for each $k$, into the sets $\mathcal{E}_k(2^m)$, and then define $\mathcal{E}^m = \bigcup_{k \geq 1} \mathcal{E}_k^m$. For appropriate exponents, we will prove $L^p$ bounds on the functions
%
\[ F^m = \sum_{(x,r) \in \mathcal{E}^m} \chi_{x,r} \]
%
which are exponentially decaying in $m$, i.e. that
%
\[ \| F^m \|_{L^p(\RR^d)} \lesssim m \cdot 2^{-m(1/p - 1/p_d)} \left( \sum_k 2^{k(d-1)} \#(\mathcal{E}_k) \right)^{1/p}. \]
%
Thus summing in $m$ using the triangle inequality gives a bound on $F = \sum_m F^m$, in the range $1 < p < p_d$, i.e. that
%
\[ \| F \|_{L^p(\RR^d)} \lesssim \left( \sum_k 2^{k(d-1)} \#(\mathcal{E}_k) \right)^{1/p}, \]
%
proving Lemma \ref{lemma3}. To get the bound on $F^m$, we interpolate between an $L^2$ bound for $F^m$, and an $L^0$ bound (i.e. a bound on the measure of the support of $F^m$). First, we calculate the support of $F^m$.

\begin{lemma} \label{lemma5}
    For each $k$,
    %
    \[ |\text{supp}(F^m_k)| \lesssim 2^{-m} 2^{k(d-1)} \# \mathcal{E}_k. \]
    %
    Thus we have
    %
    \[ |\text{supp}(F^m)| \leq \sum_k |\text{supp}(F^m_k)| \lesssim \sum_k 2^{-m} 2^{k(d-1)} \# \mathcal{E}_k. \]
\end{lemma}
\begin{proof}
    We recall that for each $k$ and $m$, we can find disjoint balls $B_1,\dots,B_N$ with radii $r_1,\dots,r_N \leq 2^k$ such that
    %
    \[ \sum_{i = 1}^N r_i \leq 2^{-m} \# \mathcal{E}_k, \]
    %
    where $\mathcal{E}_k(2^m)$ is covered by the expanded balls $B_1^* \cup \dots \cup B_N^*$. If we write
    %
    \[ F^m_{k,i} = \sum_{(x,r) \in \mathcal{E}_k(2^m) \cap B_i^*} \chi_{x,r}, \]
    %
    then $\text{supp}(F^m_k) \subset \bigcup_i \text{supp}(F^m_{k,i})$. For each $(x,r) \in B_i^* \cap \mathcal{E}_k(2^m)$, the support of $\chi_{x,r}$, an annulus of thickness $O(1)$ and radius $r$, is contained in an annulus of thickness $O(r_i)$ and radius $O(2^k)$ with the same centre as $B_i$. Thus we conclude that
    %
    \[ |\text{supp}(F^m_{k,i})| \lesssim r_i 2^{k(d-1)}, \]
    %
    and it follows that
    %
    \[ |\text{supp}(F^m_k)| \leq \sum\nolimits_i r_i 2^{k(d-1)} \leq 2^{-m} 2^{k(d-1)} \# \mathcal{E}_k. \qedhere \]
\end{proof}

Interpolating, it suffices to prove the following $L^2$ estimate on the function $F^m$.

\begin{lemma} \label{lemma6}
    Suppose $\mathcal{E} = \bigcup_k \mathcal{E}_k$ is a 1-separated set, where $\mathcal{E}_k \subset \RR^d \times [2^k,2^{k+1})$ is a set of density type $(2^m, 2^k)$. Then
    %
    \[ \left\| \sum_{(x,r) \in \mathcal{E}} \chi_{x,r} \right\|_{L^2(\RR^d)} \lesssim \sqrt{m} \cdot 2^{ \frac{m}{d-1} } \left( \sum_k 2^{k(d-1)} \#(\mathcal{E}_k) \right)^{1/2}. \]
\end{lemma}

% The L2 norms of the chi_{x,r} are equal to 2^{k(d-1)/2}, so the
% triangle ienquality implies that the LHS is bounded by sum_k 2^{k(d-1)/2} \#(E_k)

The $L^2$ bound in Lemma \ref{lemma6} gets worse and worse as $m$ grows, whereas the $L^0$ bound in Lemma \ref{lemma5} gets better and better, since annuli are concentrating in a small set, which is bad from the perspective of constructive interference, but absolutely fine from the perspective of a support bound. Interpolation gives a bound exponentially decaying in $m$ for $1 < p < p_d$.

\begin{comment}
\begin{proof}[Proof of Lemma \ref{lemma3} from Lemma \ref{lemma6}]
    Write $F = \sum_{(x,r) \in \mathcal{E}_k} \chi_{x,r}$, and then perform a decomposition $\mathcal{E}_k = \bigcup_{m \geq 0} \mathcal{E}_k(2^m)$, and thus define $F = \sum_{m \geq 0} F_m$, where
    %
    \[ F_m = \sum_{(x,r) \in \mathcal{E}(2^m)} \chi_{x,r}. \]
    %
    We have
    %
    \[ \| F_m \|_{L^2(\RR^d)} \lesssim 2^{\frac{m}{d-1} + \frac{k(d-1)}{2}} \log(2 + 2^m)^{1/2} \cdot \#(\mathcal{E}_k)^{1/2}. \]
    %
    If we interpolate this bound with the support bound for $F_m$, a kind of $L^0$ norm estimate, we conclude that for $0 < p \leq 2$,
    % ( int |F_m|^p )^{1/p} <= |S|^{1/pq^*} int |F_m|^{pq} )^{1/pq}
    % pq = 2
    % Then q = 2/p so 1/q^* = 1 - 1/q = 1 - p/2 = (2 - p)/2
    % so q^* = 2/(2-p)
    % 1/pq^* = (2-p)/2p = (1/p - 1/2)
    \begin{align*}
        \| F_m \|_{L^p(\RR^d)} &\leq |\text{Supp}(F_m)|^{1/p - 1/2} \| F_m \|_{L^2(\RR^d)}\\
        &\lesssim ( 2^{k(d-1) - m})^{1/p - 1/2} 2^{\frac{m}{d-1} + \frac{k(d-1)}{2}} \log(2 + 2^m)^{1/2} \cdot \#(\mathcal{E}_k)^{1/p} \\
        &\lesssim 2^{m(1/p_d - 1/p)} \log(2 + 2^m)^{1/2} 2^{\frac{k(d-1)}{p}} \#(\mathcal{E}_k)^{1/p}.
    \end{align*}
    % int |F_m|^p <= |S|^{1/p-1/2} ( int |F_m|^2 )^{1/2}
    %
    where $p_d = 2(d-1)/(d+1)$. This bound is summable in $m$ for $p < p_d$, which enables us to conclude that
    %
    \[ \| F \|_{L^p(\RR^d)} \lesssim 2^{\frac{k(d-1)}{p}} \#(\mathcal{E}_k)^{1/p}. \]
    %
    Thus for $1 \leq p < p_d$, we obtain the bound stated in Lemma \ref{lemma3}.
\end{proof}
\end{comment}

To prove the $L^2$ bound, we require an analysis of the interference patterns of pairs of the functions $\chi_{x,r}$, as provided by the following lemma.

%If $\psi$ is compactly supported, and $r$ is sufficiently large depending on the size of this support, then $\chi_{x,r}$ is supported on an annulus with centre $x$, radius $r$, and thickness $O(1)$. Thus $\| \chi_{x,r} \|_{L^p(\RR^d)} \sim r^{(d-1)/p}$, which implies that
%
%\[ \left\| \sum_{(x,r) \in \mathcal{E}_k} \chi_{x,r} \right\|_{L^p(\RR^d)} \gtrsim 2^{k(d-1)/p} \#(\mathcal{E}_k)^{1/p}. \]
%
%Thus this bound can only be true if $p \geq 1$, and becomes tight when $p = 1$, where we actually have
%
%\[ \left\| \sum_{(x,r) \in \mathcal{E}_k} \chi_{x,r} \right\|_{L^1(\RR^d)} \sim 2^{k(d-1)} \#(\mathcal{E}_k) \]
%

\begin{lemma} \label{lemma4}
    For any $N > 0$, $x_1,x_2 \in \RR^d$ and $r_1,r_2 \geq 1$,
    %with $|x_1 - x_2| \geq 1$ or $x_1 = x_2$, and $r_1,r_2 > 1$,
    %
    \begin{align*}
        |\langle \chi_{x_1,r_1}, \chi_{x_2,r_2} \rangle| &\lesssim_N \left( \frac{r_1r_2}{\langle (x_1,r_1) - (x_2,r_2) \rangle} \right)^{\frac{d-1}{2}} \sum\nolimits_{\pm,\pm} \langle |x_1 - x_2| \pm r_1 \pm r_2 \rangle^{-N}.
    \end{align*}
\end{lemma}

\begin{remark}
    Suppose $r_1 \leq r_2$. Then Lemma \ref{lemma4} implies that $\chi_{x_1,r_1}$ and $\chi_{x_2,r_2}$ are roughly uncorrelated, except when they are supported on annuli that roughly have the same radii and centers, and in addition, one of the following two properties hold:
    %
    \begin{itemize}
        \item $r_1 + r_2 \approx |x_1 - x_2|$, which holds when the two annuli are `approximately' externally tangent to one another.

        \item $r_2 - r_1 \approx |x_1 - x_2|$, which holds when the two annuli are `approximately' internally tangent to one another.
    \end{itemize}
    %
    Heo, Nazarov, and Seeger do not exploit the tangency information, though utilizing the tangencies seems important to improve the results they obtain. Cladek exploits this tangency information further, to obtain improved results.
\end{remark}

%    Note that if we take too comparable radii $r_1,r_2 \sim 2^k$, then the functions $\{ 1_{x,r} \}$, which are the non-oscillating analogues of the functions $\{ \chi_{x,r} \}$ we are studying in this problem, if they correspond to tangent annuli, satisfy bounds  of the form
    %
%    \[ \langle 1_{x_1,r_1}, 1_{x_2,r_2} \rangle \lesssim 2^{- \left( \frac{d-1}{2} \right) k} \]
    %
%    whereas for internal tangencies we have
    %
%    \[ \langle \chi_{x_1,r_1}, \chi_{x_2,r_2} \rangle \lesssim 2^{(d-1)k} |x_1 - x_2|^{- \frac{d-1}{2}} \]
    %
%    In the Clamshell example, one has
    %
%    \[ \left\| \sum_{k = 1}^N k^{d-1} 1_{k,k} \right\|_{L^2(\RR^d)} \]

    % delta = 2^{-k}
    % 3[d-1]/2
%    \[ 2^{(d-1)k} 2^{dk} 2^{-\frac{d+1}{2}} \]
    %
%    \[ \langle r_1^{d-1} 1_{x_1,r_1}, r_2^{d-1} 1_{x_2,r_2} \rangle \lesssim r_1^{d-1} r_2^{d-1}. \]
%    \end{comment}

\begin{proof}
%    We may assume $|x_1 - x_2| \geq 1$, for otherwise the inequality holds trivially since unless $|r_1 - r_2| \lesssim 1$, $f_{x_1r_1}$ and $f_{x_2r_2}$ have disjoint support, and if $|r_1 - r_2| \lesssim 1$ then Cauchy Schwartz implies that
    %
%    \begin{align*}
%        |\langle f_{x_1r_1}, f_{x_2r_2} \rangle| &\lesssim (r_1 r_2)^{(d-1)/2}\\
%        &\lesssim_{N,d} (r_1r_2)^{(d-1)/2} (1 + |r_1 - r_2| + |x_1 - x_2|)^{-(d-1)/2} \sum_{\pm,\pm} (1 + ||x_1 - x_2| \pm r_1 \pm r_2|)^{-N}
%    \end{align*}
%
    We write
    %
    \begin{align*}
        \langle \chi_{x_1 r_1}, \chi_{x_2 r_2} \rangle &= \left\langle \widehat{\chi}_{x_1 r_1}, \widehat{\chi}_{x_2 r_2} \right\rangle\\
        &= \int_{\RR^d} \widehat{\sigma_{r_1} * \psi}(\xi) \cdot \overline{\widehat{\sigma_{r_2} * \psi}(\xi)} e^{2 \pi i (x_2 - x_1) \cdot \xi}\; d\xi\\
        &= (r_1 r_2)^{d-1} \int_{\RR^d} \widehat{\sigma}(r_1 \xi) \overline{\widehat{\sigma}(r_2 \xi)} |\widehat{\psi}(\xi)|^2 e^{2 \pi i (x_2 - x_1) \cdot \xi}\; d\xi.
    \end{align*}
    %
    Define functions $A$ and $B$ such that $B(|\xi|) = \widehat{\sigma}(\xi)$, and $A(|\xi|) = |\widehat{\psi}(\xi)|^2$. Then
    %
    \[ \langle \chi_{x_1, r_1}, \chi_{x_2, r_2} \rangle = C_d (r_1r_2)^{d-1} \int_0^\infty s^{d-1} A(s) B(r_1 s) B(r_2 s) B(|x_2 - x_1| s)\; ds. \]
    %
    Using well known asymptotics for the Fourier transform for the spherical measure, we have
    %
    \[ B(s) = s^{-(d-1)/2} \sum_{n = 0}^{N-1} (c_{n,+} e^{2 \pi i s} + c_{n,-} e^{-2 \pi i s}) s^{-n} + O_N(s^{-N}). \]
    %
    But now, assuming $A(s)$ vanishes to order $\gtrsim N$ at the origin, we conclude that
    %
    \begin{align*}
        \langle \chi_{x_1 r_1}, \chi_{x_2 r_2} \rangle &= C_d \left( \frac{r_1r_2}{|x_1 - x_2|} \right)^{(d-1)/2} \sum_{n,\tau} c_{n,\tau}  r_1^{-n_1} r_2^{-n_2} |x_2 - x_1|^{-n_3}\\
        &\quad\quad\quad \Bigg\{ \int_0^\infty A(s) s^{-(d-1)/2}  s^{-n_1-n_2-n_3} e^{2 \pi i (\tau_1 r_1 + \tau_2 r_2 + \tau_3 |x_2 - x_1|) s}\; ds \Bigg\}\\
        &\lesssim_N \left( \frac{r_1r_2}{|x_1 - x_2|} \right)^{\frac{d-1}{2}} \left(1 + \frac{1}{|x_1 - x_2|^N} \right) \sum_{\tau} \left( 1 + |\tau_1 r_1 + \tau_2 r_2 + \tau_3 |x_2 - x_1|| \right)^{-5N}\\
        &\lesssim_N \left( \frac{r_1r_2}{|x_1 - x_2|} \right)^{\frac{d-1}{2}} \left(1 + \frac{1}{|x_1 - x_2|^N} \right) \sum_\tau \left( 1 + |\tau_1 r_1 + \tau_2 r_2 + |x_2 - x_1|| \right)^{-5N}.
    \end{align*}
    %
    This gives the result provided that $1 + |x_1 - x_2| \geq |r_1 - r_2| / 10$ and $|x_1 - x_2| \geq 1$. If $1 + |x_1 - x_2| \leq |r_1 - r_2| / 10$, then the supports of $\chi_{x_1,r_1}$ and $\chi_{x_2,r_2}$ are disjoint, so the inequality is trivial. On the other hand, if $|x_1 - x_2| \leq 1$, then the bound is trivial by the last sentence unless $|r_1 - r_2| \leq 10$, and in this case the inequality reduces to the simple inequality
    %
    \[ \langle \chi_{x_1,r_1}, \chi_{x_2,r_2} \rangle \lesssim_N (r_1 r_2)^{(d-1)/2}. \] 
    %
    But this follows immediately from the Cauchy-Schwartz inequality.
\end{proof}

The exponent $(d-1)/2$ in Lemma \ref{lemma4} is too weak to apply almost orthogonality directly to obtain $L^2$ bounds on $\sum_{(x,r) \in \mathcal{E}_k} \chi_{xr}$ on it's own, but together with the density decomposition assumption we will be able to obtain Lemma \ref{lemma6}.

\begin{proof}[Proof of Lemma \ref{lemma6}]
    Without loss of generality, we may assume that the set of $k$ such that $\mathcal{E}_k \neq \emptyset$ is $10$-separated. Write
    %
    \[ F = \sum_{(x,r) \in \mathcal{E}} \chi_{x,r} \]
    %
    and $F_k = \sum_{(x,r) \in \mathcal{E}_k} \chi_{x,r}$. First, we deal with $F_{\lesssim m} = \sum_{k \leq 10 m} F_k$ trivially, i.e. writing
    %
    \begin{align*}
        \| F \|_{L^2(\RR^d)} &\lesssim m^{1/2} \left( \sum_{k \leq 10m} \| F_k \|_{L^2(\RR^d)}^2 + \| \sum_{k > 10m} F_k \|_{L^2(\RR^d)} \right)^{1/2}.
    \end{align*}
    %
    We then decompose
    %
    \[ \| \sum_{k > 10 m} F_k \|_{L^2(\RR^d)}^2 \leq \sum_{k > 10 m} \| F_k \|_{L^2(\RR^d)}^2 + 2 \sum_{k' > k > 10m} |\langle F_k, F_{k'} \rangle|. \]
    %
    Let us analyze $\langle F_k, F_{k'} \rangle$. The term will become a sum of the form $\langle \chi_{x,r}, \chi_{y,s} \rangle$, where $r \sim 2^k$ and $s \sim 2^{k'}$. Because of our assumption of being 10-separated, we have $r \leq s / 2^{10}$. If $\langle \chi_{x,r}, \chi_{y,s} \rangle \neq 0$, then since the support of $\chi_{y,s}$ is an annulus of radius $s$ centered at $y$, with thickness $O(1)$, and $\chi_{x,r}$ has support on an annulus of radius $r$ centered at $x$, with thickness $O(1)$, the fact that $r$ is comparatively smaller than $s$ implies that $(x,r)$ must be contained in the annulus of radius $s$ centered at $y$, with thickness $O(2^k)$. Such an annulus is covered by $O( 2^{(k'-k)(d-1)} )$ balls of radius $2^k$. Each ball can only contain $2^{k + m}$ points $(x,r)$, and so there can be at most
    %
    \[ O(2^{k'(d-1)} 2^{-k(d-1)} 2^{k+m} ) = O( 2^{k'(d-1) - k(d-2) + m} ). \]
    %
    pairs $(x,r) \in \mathcal{E}_k$ for which $\langle \chi_{x,r}, \chi_{y,s} \rangle \neq 0$. For such pairs we have
    %
    \[ |\langle \chi_{x,r}, \chi_{y,s} \rangle| \lesssim \left( \frac{2^k 2^{k'}}{2^{k'}} \right)^{\frac{d-1}{2}} = 2^{\frac{k(d-1)}{2}}. \]
    %
    Thus we conclude that
    %
    \[ |\langle F_k, \chi_{y,s} \rangle| \lesssim 2^{-k ( \frac{d-3}{2} ) + k'(d-1) + m }. \]
    %
    Summing over $10m < k < k'$, we conclude that since $d \geq 4$,
    %
    \[ \sum_{10m < k < k'} |\langle F_k, \chi_{y,s} \rangle| \lesssim 2^{k'(d-1) + m} \sum_{10m < k < k'} 2^{-k \frac{d-3}{2}} \lesssim 2^{k'(d-1) + m} 2^{-5m} \lesssim 2^{k'(d-1)}. \]
    %
    But this means that
    %
    \[ \sum_{10m < k < k'} |\langle F_k, F_{k'} \rangle| \lesssim 2^{k'(d-1)} \cdot \# ( \mathcal{E}_{k'} ). \]
    %
    This means that
    %
    \[ \| \sum_{k > 10m} F_k \|_{L^2(\RR^d)}^2 \lesssim \sum_{k > 10m} \| F_k \|_{L^2(\RR^d)}^2 + \sum_{k'} 2^{k'(d-1)} \# (\mathcal{E}_{k'}), \]
    %
    and it now suffices to deal with estimates the $\| F_k \|_{L^2(\RR^d)}$, i.e. the interactions of functions supported on radii of comparable magnitude. To deal with these, we further decompose the radii, writing $[2^k,2^{k+1})$ as the disjoint union of intervals $I_{k,\mu} = [2^k + (\mu - 1) 2^{am}, 2^k + \mu 2^{am}]$, for some $a$ to be chosen later. These interval induces a decomposition $\mathcal{E}_k = \bigcup_\mu \mathcal{E}_{k,\mu}$. Again, incurring a constant loss at most, we may assume that the $\mu$ such that $\mathcal{E}_{k,\mu} \neq \emptyset$ are $10$ separated. We write $F_k = \sum F_{k,\mu}$, and we have
    %
    \[ \| F_k \|_{L^2(\RR^d)}^2 = \sum_\mu \| F_{k,\mu} \|_{L^2(\RR^d)}^2 + \sum_{\mu < \mu'} |\langle F_{k,\mu}, F_{k,\mu'} \rangle|. \]
    %
    We now consider $\chi_{x,r}$ and $\chi_{y,s}$ with $r \in I_{k,\mu}$ and $s \in I_{k,\mu'}$. Then we must have $|x - y| \lesssim 2^k$ and $2^{am} \leq |r - s| \lesssim 2^k$, and so we have
    %
    \begin{align*}
        |\sum_{\mu < \mu'} \langle F_{k,\mu}, \chi_{y,s} \rangle| &\lesssim 2^{k(d-1)} \sum_{\substack{(x,r) \in \mathcal{E}_k\\ 2^{am} \leq |(x,r) - (y,s)| \lesssim 2^k}} |(x,r) - (y,s)|^{- \frac{d-1}{2}}\\
        &\lesssim 2^{k(d-1)} \sum_{am \leq l \leq k} 2^{-l(d-1)/2} \# \{ (x,r) \in \mathcal{E}_k: |(x,r) - (y,s)| \sim 2^l \}.
    \end{align*}
    %
    Using the density assumption,
    %
    \[ \# \{ (x,r) \in \mathcal{E}_k: |(x,r) - (y,s)| \sim 2^l \} \lesssim 2^{l + m} \]
    %
    and so we obtain that, again using the assumption that $d \geq 4$,
    %
    \[ |\sum_{\mu < \mu'} \langle F_{k,\mu}, \chi_{y,s} \rangle| \lesssim 2^{k(d-1)} 2^{m(1-a(d-3)/2)}. \]
    %
    Now summing over all $(y,s)$, we obtain that
    %
    \[ |\sum_{\mu < \mu'} \langle F_{k,\mu}, F_{k,\mu'} \rangle| \lesssim 2^{k(d-1)} 2^{m(1 - a(d-3)/2)} \#(\mathcal{E}_{k,\mu'}). \]
    %
    and now summing over $\mu'$ gives that
    %
    \[ \| F_k \|_{L^2(\RR^d)}^2 \lesssim \sum_\mu \| F_{k,\mu} \|_{L^2(\RR^d)}^2 + 2^{k(d-1)} 2^{m(1 - a(d-3)/2)} \# \mathcal{E}_k, \]
    %
    which is a good enough bound if we pick $a$ to be large enough. Now we are left to analyze $\| F_{k,\mu} \|_{L^2(\RR^d)}$, i.e. analyzing interactions between annuli which have radii differing from one another by at most $O(2^{am})$. Since the family of all possible radii are discrete, the set $\mathcal{R}_{k,\mu}$ of all possible radii has cardinality $O(2^{am})$. We do not really have any orthogonality to play with here, so we just apply Cauchy-Schwartz, writing $F_{k,\mu} = \sum_{r \in \mathcal{R}_{k,\mu}} F_{k,\mu,r}$, to write
    %
    \[ \| F_{k,\mu} \|_{L^2(\RR^d)}^2 \lesssim 2^{am} \sum_r \| F_{k,\mu,r} \|_{L^2(\RR^d)}^2. \]
    %
    Recall that $\chi_{x,r} = \text{Trans}_x(\sigma_r * \psi)$, where $\psi$ is a compactly supported function whose Fourier transform is non-negative and vanishes to high order at the origin. In particular, we now make the additional assumption that $\psi = \psi_{\circ} * \psi_{\circ}$ for some other compactly function $\psi_{\circ}$ whose Fourier transform is non-negative and vanishes to high order at the origin. Then we find that $F_{k,\mu,r}$ is equal to the convolution of the function
    %
    \[ A_r = \sum_{(x,r) \in \mathcal{E}} \text{Trans}_x \psi_{\circ} \]
    %
    with the function $\sigma_r * \psi_{\circ}$. Using the standard asymptotics for the Fourier transform of $\sigma_r$, i.e. that for $|\xi| \geq 1$,
    %
    \[ |\widehat{\sigma_r}(\xi)| \lesssim r^{d-1} (1 + r |\xi|)^{- \frac{d-1}{2}}, \]
    %
    and since $|\widehat{\psi_\circ}(\xi)| \lesssim_N |\xi|^N$, we get that if $r \geq 1$, then for $|\xi| \leq 1/r$,
    %
    \[ |\widehat{\sigma_r}(\xi) \widehat{\psi_\circ}(\xi)| \lesssim_N r^{d-1-N} \]
    %
    and for $|\xi| \geq 1/r$,
    %
    \[ |\widehat{\sigma_r}(\xi) \widehat{\psi_\circ}(\xi)| \lesssim_N r^{\frac{d-1}{2}} |\xi|^{-N}. \]
    %
    Thus in particular,the $L^\infty$ norm of the Fourier transform of $\sigma_r * \psi_\circ$ is $O(r^{(d-1)/2})$. Now the functions $\psi_{\circ}$ are compactly supported, so since the set of $x$ such that $(x,r) \in \mathcal{E}$ is one-separated, we find that
    %
    \[ \| A_r \|_{L^2(\RR^d)} \lesssim \# \{ x : (x,r) \in \mathcal{E} \}^{1/2}. \]
    %
    But this means that
    %
    \[ \| F_{k,\mu,r} \|_{L^2(\RR^d)} = \| A_r * (\sigma_r * \psi_{\circ}) \|_{L^2(\RR^d)} \lesssim r^{\frac{d-1}{2}} \# \{ x : (x,r) \in \mathcal{E} \}^{1/2}. \]
    %
    Thus we have that
    %
    \[ \| F_{k,\mu} \|_{L^2(\RR^d)}^2 = 2^{am} \cdot \# \mathcal{E}_{k,\mu} \cdot 2^{k(d-1)}. \]
    %
    Summing over $\mu$ gives that
    %
    \[ \| F_k \|_{L^2(\RR^d)}^2 = 2^{k(d-1)} \# \mathcal{E}_k (2^{am}  + 2^{m(1 - a(d-3)/2)}). \]
    %
    Picking $a = 2/(d-1)$ optimizes this bound, giving
    %
    \[ \| F_k \|_{L^2(\RR^d)} \lesssim 2^{m/(d-1)} 2^{k(d-1)/2} (\# \mathcal{E}_k)^{1/2}. \]
    %
    Plugging this into the estimates we got for $F$ gives the required bound.
\end{proof}

This completes a proof of the single scale estimates of the paper. The paper then uses an atomic decomposition method to combine these scales and thus complete the proof of Theorem \ref{HeoNazarovSeegerTheorem}. Rather than discuss these methods, we instead discuss an iteration of this method developed by the same authors in a follow up paper \cite{HeoandNazarovandSeeger2}.

\section{Combining Scales With Atomic Decompositions}

Consider a radial Fourier multiplier operator $T$ as in Theorem \ref{HeoNazarovSeegerTheorem}. The last section proves that if we write $T = \sum T_j$, where $T_j$ is the radial Fourier multiplier operator with convolution kernel $k_j(\cdot/2^j)$, then applying Bernstein's inequality and the results of the last section (appropriately rescaled), we conclude that
\begin{equation} \label{individualscaleoperatorbound}
    \| T_j \|_{L^p(\RR^d) \to L^q(\RR^d)} \lesssim 2^{jd(1/p - 1/q)} \| T_j \|_{L^q(\RR^d) \to L^q(\RR^d)} \lesssim 2^{jd(1/p - 1/q)} \| k_j \|_{L^q(\RR^d)}.
\end{equation}
%
The goal of this section is to prove that we can bound the sum of the operators $T_j$ by the suprema of the quantities on the right hand side in \eqref{individualscaleoperatorbound} which, morally speaking, is a bound of the form
%
\[ \left\| \sum\nolimits_j T_j \right\| \lesssim \sup\nolimits_j \| T_j \|, \]
%
and thus implies that the operators $\{ T_j \}$ do not constructively interfere with one another to a significant extent.

The classical example of combining scales of this form occurs in variants of H\"{o}rmander-Mikhlin type multipliers. Here one can use the Littlewood-Paley inequality
%
\[ \left\| \left( \sum\nolimits_j |f_j|^2 \right)^{1/2} \right\|_{L^p(\RR^d)} \lesssim \left\| \sum\nolimits_j f_j \right\|_{L^p(\RR^d)} \]
%
which implies that for the average $x$, the square root cancellation bound
%
\[ \left| \sum\nolimits_j f_j(x) \right| \sim \left( \sum\nolimits_j |f_j(x)|^2 \right)^{1/2} \]
%
holds pointwise. Multipliers of H\"{o}rmander-Mikhlin type are \emph{pseudolocal}, i.e. the 

Thus $\sup_j \| T_j \|_{L^p(\RR^d) \to L^q(\RR^d)} \lesssim \sup_j 2^{jd(1/p - 1/q)} \| k_j \|_{L^q(\RR^d)}$, and our goal is to find a way to prove that the operators $\{ T_j \}$ do not interact significantly enough, so that

The last section proves that if $T$ is a Fourier multiplier operator, and we consider a dyadic decomposition $T = \sum T_j$, where $T_j$ is a Fourier multiplier operator 

\section{Exploiting Tangency Bounds in $\RR^3$ and $\RR^4$}

The results of Heo, Nazarov, and Seeger only apply when $d \geq 4$. Cladek found a method to get an initial radial multiplier conjecture result in $\RR^3$, and an improvmeent of the bounds obtained by Heo, Nazarov, and Seeger when $d = 3$. The idea is to exploit the fact that one need only prove a version of \ref{lemma2} for a set $\mathcal{E} = \mathcal{E}_X \times \mathcal{E}_R$, where $\mathcal{E}_X$ is a one-separated family of points, and $\mathcal{E}_R$ are a family of radii. One can then exploit this Cartesian product structure when analyzing functions of the form
%
\[ F = \sum_{(x,r) \in \mathcal{E}} \chi_{x,r}, \]
%
in particular, improving upon the result of \cite{HeoandNazarovandSeeger}.

\subsection{Result in 3 Dimensions}

As in \cite{HeoandNazarovandSeeger}, Cladek first performs a density decomposition, i.e. writing
%
\[ F = \sum F_k^m \]
%
where
%
\[ F_k^m = \sum_{(x,r) \in \mathcal{E}_k(2^m)} \chi_{x,r}. \]
%
Cladek then interpolates between an $L^0$ bound and an $L^2$ bound on the resulting functions. The $L^0$ bound is exactly the same bound used in \cite{HeoandNazarovandSeeger}.

\begin{theorem}
    For the function $F$, we have
    %
    \[ |\text{supp}(F_k^m)| \lesssim 2^{-m} 4^k \# \mathcal{E}_k \]
    %
    and thus
    %
    \[ |\text{supp}(F^m)| \lesssim \sum_k 2^{-m} 4^k \# \mathcal{E}_k. \]
\end{theorem}

The $L^2$ bound is improved upon, which is what allows us to obtain a new result in three dimensions.

\begin{lemma} \label{cladeksl2}
    Suppose $\mathcal{E} = \bigcup_k \mathcal{E}_k$ is a one-separated set, where $\mathcal{E}_k \subset \RR^d \times [2^k,2^{k+1})$ is a set of density type $(2^m, 2^k)$. Then
    %
    \[ \left\| \sum_{(x,r) \in \mathcal{E}} \chi_{x,r} \right\|_{L^2(\RR^d)} \lesssim_\varepsilon 2^{[(11/13) + \varepsilon] m} \sum_k 4^k \# \mathcal{E}_k. \]
\end{lemma}

Interpolation thus yields that for a set of density type $2^m$ as in this Lemma,
%
\[ \| \sum_{(x,r) \in \mathcal{E}} \chi_{x,r} \|_{L^p(\RR^d)} \lesssim_\varepsilon 2^{-m(1/p - 12/13 - \varepsilon)} ( \sum_k 4^k \# \mathcal{E}_k )^{1/p}. \]
%
If $1 < p < 13/12$, this sum is favorable in $m$, and may be summed without harm to prove the radial multiplier conjecture for unit scale radial multipliers in this range.

\begin{proof} [Proof of Lemma \ref{cladeksl2}]
    Write
    %
    \[ F_k = \sum_{(x,r) \in \mathcal{E}_k} \chi_{x,r}. \]
    %
    As before, we can throw away terms for $k \leq 10 m$, i.e. obtaining that
    %
    \[ \| \sum F_k \|_{L^2(\RR^d)} \lesssim m^{1/2} \left( \sum_k \| F_k \|_{L^2(\RR^d)}^2 + \sum_{10m < k < k'} |\langle F_k, F_{k'} \rangle| \right)^{1/2}. \]
    %
    Our proof thus splits into two cases: where the radii are incomparable, and where the radii are comparable.

    TODO:
\end{proof}

\subsection{Results in 4 Dimensions}

TODO







\chapter{The Radial Multiplier Conjecture on Compact Manifolds}

We now return to the study of spectral multipliers on compact manifolds. In a certain sense, one can think of the Euclidean situation studied in the last Chapter as a limiting case of multipliers in the `high frequency limit'. In particular, a transference principle of Mitjagin \cite{Mitjagin} (See \cite{KenigStantonTomas} for a similar result, written in English and available online) shows that if $X$ is a compact Riemannian manifold, and $h: (0,\infty) \to \CC$ is regulated, then
%
\[ \| h(|\cdot|) \|_{M^{p,q}(\RR^d)} \lesssim_{X,p,q} \| h \|_{M^{p,q}_{\text{Dil}}(X)}. \]
%
Thus, in order for $h$ to be an element of $M^{p,q}_{\text{Dil}}(X)$, it is necessary for the radial multiplier $h(|\cdot|)$ to be bounded from $L^p(\RR^d)$ to $L^q(\RR^d)$, and thus it is necessary for the quantities 

Whether an analogous result remains true for more general Riemannian manifolds remains unclear, since the family of eigenfunctions to the Laplacian can take on various different forms on these manifolds, that can look quite different to the Euclidean case (TODO: Does the existence of low dimension Kakeya sets on certain manifolds show that the radial multiplier conjecture cannot be true in general).

Directly translating the assumptions of the radial multiplier conjecture to this setting yields the following statement: If $h: [0,\infty) \to \RR$ is a function supported at the unit scale, and we define
%
\[ C_q(h) = \left( \int |\widehat{h}(s)|^q \langle s \rangle^{(d-1)(1 - q/2)}\; ds \right)^{1/q}, \]
%\[ C_{p,q}(h) = \sup_{t > 0} t^{d(1/p - 1/q)} \left( \int |\widehat{h}_t(s)|^q (1 + |s|)^{(d-1)(1 - q/2)}\; ds \right)^{1/q}, \]
%
then for what values of $p$ and $q$ is is true that the inequality
%
\[ \| h \|_{M^{p,q}_{\text{Dil}}(X)} \lesssim C_q(h) \]
%
holds. Mitjagin's result implies that we require $1 < p < 2d/(d+1)$ and $p \leq q < 2$, and we conjecture that, perhaps under appropriate assumptions on $X$, we can achieve similar ranges of exponents as have been obtained for the Euclidean radial multiplier conjecture.

On general compact manifolds, there are difficulties arising from a generalization of the radial multiplier conjecture, connected to the fact that analogues of the Kakeya / Nikodym conjecture are false in this general setting \cite{Minicozzi}. But these problems do not arise for constant curvature manifolds, like the sphere. The sphere also has over special properties which make it especially amenable to analysis, such as the fact that solutions to the wave equation on spheres are periodic. Best of all, there are already results which achieve the analogue of \cite{GarrigosandSeeger} on the sphere. Thus it seems reasonable that current research techniques can obtain interesting results for radial multipliers on the sphere, at least in the ranges established in \cite{HeoandNazarovandSeeger} or even those results in \cite{Cladek}.

\section{Connections to Local Smoothing Bounds}

\section{Fourier Integral Operator Techniques}

\section{Mockenhaupt Seeger Sogge: Exploiting Periodicity}

The main goal of the paper \emph{Local Smoothing of Fourier Integral Operators and {C}arleson-{S}j\"{o}lin Estimates} is to prove local regularity theorems for a class of Fourier integral operators in $I^\mu(Z,Y;\mathcal{C})$, where $Y$ is a manifold of dimension $n \geq 2$, and $Z$ is a manifold of dimension $n+1$, which naturally arise from the study of wave equations. A consequence of this result will be a local smoothing result for solutions to the wave equation, i.e. that if $2 < p < \infty$, then there is $\delta$ depending on $p$ and $n$, such that if $T: Y \to Y \times \RR$ is the solution operator to the wave equation, and $Y$ is a compact manifold whose geodesics are periodic, then $T$ is continuous from from $L^p_c(Y)$ to $L^p_{\alpha,\text{loc}}(Y \times \RR)$ for $\alpha \leq -(n-1)|1/2 - 1/p| + \delta$. Such a result is called local smoothing, since if we define $Tf(t,x) = T_tf(x)$, then the operator $T_t$ is, for each $t$, a Fourier integral operator of order zero, with canonical relation
%
\[ \mathcal{C}_t = \{ (x,y;\xi,\xi) : x = y + t \widehat{\xi} \}, \]
%
where $\widehat{\xi} = \xi / |\xi|$ is the normalization of $\xi$. Standard results about the regularity of hyperbolic partial differential equations show that each of the operators $T_t$ is continuous from $L^p_c(Y)$ To $L^p_{\alpha,\text{loc}}(Y \times \RR)$ for $\alpha \leq -(n-1)|1/2 - 1/p|$, and that this bound is sharp. Thus $T$ is \emph{smoothing} in the $t$ variable, so that for any $f \in L^p$, the functions $T_t f$ `on average' gain a regularity of $\delta$ over the worst case regularity at each time. The local smoothing conjecture states that this result is true for any $\delta < 1/p$.

The class of Fourier integral operators studied are those satisfying the following condition: as is standard, the canonical relation $\mathcal{C}$ is a conic Lagrangian manifold of dimension $2n + 1$. The fact that $\mathcal{C}$ is Lagrangian implies $\mathcal{C}$ is locally parameterized by $(\nabla_\zeta H(\zeta, \eta), \nabla_\eta H(\zeta, \eta),\zeta,\eta)$, where $H$ is a smooth, real homogeneous function of order one. If we assume $\mathcal{C} \to T^* Y$ is a submersion, then $D_\xi [\nabla_\eta H(\zeta,\eta)]$ has full rank, which implies $D_\eta [\nabla \xi H(\zeta, \eta)] = (D_\xi [\nabla_\eta H(\zeta, \eta)])^t$ has full rank, and thus the projection $\mathcal{C} \to T^* Z$ is an immersion. We make the further assumption that the projection $\mathcal{C} \to Z$ is a submersion, from which it follows that for each $z$ in the image of this projection, the projection of points in $\mathcal{C}$ onto $T^*_z Z$ is a conic hypersurface $\Gamma_z$ of dimension $n$. The final assumption we make is that all principal curvatures of $\Gamma_z$ are non-vanishing.

\begin{remark}
    The projection properties of $\mathcal{C}$ imply that, in $T^* (Z \times Y)$, there exists a smooth phase $\phi$ defined on an open subset of $Z \times T^* Y$, homogeneous in $T^* Y$, such that locally we can write $\mathcal{C}$ as $(z, \nabla_z \phi(z,\eta), \nabla_\eta \phi(z,\eta), \eta)$ for $\eta \neq 0$. Then, working locally on conic sets,
    %
    \[ \Gamma_z = \{ (\nabla_z \phi(z,\eta)) \}, \]
    %
    and the curvature condition becomes that the Hessian $H_{\eta \eta} \langle \nabla_z \phi, \nu \rangle$ has constant rank $n-1$, where $\nu$ is the normal vector to $\Gamma_z$. This is a natural homogeneous analogue of the Carleson-Sj\"{o}lin condition for non-homogeneous oscillatory integral operators, i.e. the Carleson-Sj\"{o}lin condition is allowed to assume $H_{\eta \eta} \phi$ has rank $n$, which cannot be possible in our case, since $\phi$ is homogeneous here. An approach using the analytic interpolation method of Stein or the Strichartz / Fractional Integral approach generalizes the Carleson-Sj\"{o}lin theorem to show that for any smooth, non-homogeneous phase function $\Phi: \RR^{n+1} \times \RR^n \to \RR$, and any compactly supported smooth amplitude $a$ on $\RR^{n+1} \times \RR^n$. Consider the operators
    %
    \[ T_\lambda f(z) = \int a(z,y) e^{2 \pi i \lambda \Phi(z,y)} f(y)\; dy. \]
    %
    If the associated canonical relation $\mathcal{C}$, if $\mathcal{C}$ projects submersively onto $T^* \RR^n$, so that for each $z \in \RR^{n+1}$ in the image of the projection map $\mathcal{C}$, the set $S_z \subset \RR^{n+1}$ obtained from the inverse image of the projection of $\mathcal{C} \to Z$ at $z$ is a $n$ dimensional hypersurface with $k$ non-vanishing curvatures. Then for $1 \leq p \leq 2$,
    %
    \[ \| T_\lambda f \|_{L^q(\RR^{n+1})} \lesssim \lambda^{-(n+1)/q} \| f \|_{L^p(\RR^n)}. \]
    %
    where $q = p^*(1 + 2/k)$.
\end{remark}

\begin{remark}
    We can also see these assumptions as analogues in the framework of cinematic curvature, splitting the $z$ coordinates into `time-like' and 'space-like' parts. Working locally, because $\mathcal{C} \to T^* Y$ is a submersion, we can consider coordinates $z = (x,t)$ so that, with the phase $\phi$ introduced above, $D_x (\nabla_\eta \phi)$ has full rank $n$, and that $\partial_t \phi(x,t,\eta) \neq 0$. Then for each $z = (x,t)$, we can locally write $\partial_t \phi(x,t,\eta) = q(x,t,\nabla_x \phi(x,t,\eta))$, homogeneous in $\eta$, and then
    %
    \[ \mathcal{C} = \{ (x,t,y;\xi,\tau,\eta) : (x,\xi) = \chi_t(y,\eta), \tau = q(x,t,\xi) \}, \]
    %
    where $\chi_t$ is a canonical transformation. Our curvature conditions becomes that $H_{\xi \xi} q$ has full rank $n-1$. This is the cinematic curvature condition introduced by Sogge. %TODO: READ SOGGE, PROPOGATION OF SINGULARITIES AND MAXIMAL FUNCTIONS IN THE PLANE, WHICH INTRODUCES CINEMATIC CURVATURE?
\end{remark}

Under these assumptions, the paper proves that any Fourier integral operator $T$ in $I^{\mu - 1/4}(Z,Y;\mathcal{C})$ maps $L^2_c(Y)$ to $L^q_{\text{loc}}(Z)$ if
%
\[ 2 \left( \frac{n+1}{n-1} \right) \leq q < \infty \quad\text{and}\quad \mu \leq - n (1/2 - 1/q) + 1/q. \]
%
and maps $L^p_c(Y)$ to $L^p_{\text{loc}}(Z)$ if
%
\[ p > 2 \quad\text{and}\quad \mu \leq -(n-1)(1/2 - 1/p) + \delta(p,n). \]
%
If we introduce time and space variables locally as in the remark above, any operator in $I^{\mu - 1/4}(Z,Y;\mathcal{C})$ can be written locally as a finite sum of operators of the form
%
\[ Tf(x) = \int_{-\infty}^\infty T_t f(x), \]
%
where
%
\[ T_t f(x) = \int a(t,x,\eta) e^{2 \pi i \phi(x,t,y,\eta)} f(y)\; dy\; d\eta. \]
%
is a Fourier integral operator whose canonical relation is a locally a canonical graph, then the general theory implies that each of the maps $T_t$ maps $L^2_c(Y)$ to $L^q_{\text{loc}}(X)$ if
%
\[ 2 \leq q \leq \infty \quad\text{and}\quad \mu \leq -n(1/2 - 1/q) \]
%
so that here we get local smoothing of order $1/q$, and also maps $L^p_c(Y)$ to $L^p_{\text{loc}}(X)$ if
%
\[ 1 < p < \infty \quad\text{and}\quad \mu \leq -(n-1)|1/p - 1/2| \]
%
so we get $\delta(p,n)$ smoothing. A consequence of the smoothing, via Sobolev embedding, is a maximal theorem result for the operator $T_t$, i.e. that for any finite interval $I$, the operator
%
\[ Mf = \sup_{t \in I} |T_t f| \]
%
maps $L^p_c(Y)$ to $L^p_{\text{loc}}(X)$ if $\mu < -(n-1)(1/2 - 1/p) - (1/p - \delta(p,n))$. If the local smoothing conjecture held, we would conclude that, except at the endpoint $T^*$ has the same $L^p_c(Y)$ to $L^p_{\text{loc}}(X)$ mapping properties as each of the operators $T_t$. We also get square function estimates, such that for any finite interval $I$, if we consider
%
\[ Sf(x) = \left( \int_I |T_t f(x)|^2\; dt \right)^{1/2}, \]
%
then for
%
\[ 2 \frac{n+1}{n-1} \leq q < \infty \quad\text{and}\quad \mu \leq -n(1/2 - 1/q) + 1/2, \]
%
the operator $S$ is bounded from $L^2_c(Y)$ to $L^q_{\text{loc}}(X)$.

Our main reason to focus on this paper is the results of the latter half of the paper applying these techniques to radial multipliers on compact manifolds with periodic geodesics. Thus we consider a compact Riemannian manifold $M$, such that the geodesic flow is periodic with minimal period $2 \pi \cdot \Pi$. We consider $m \in L^\infty(\RR)$, such that $\sup_{s > 0} \| \beta \cdot \text{Dil}_s m \|_{L^2_\alpha(\RR)} = A_\alpha$ is finite for some $\alpha > 1/2$ and some $\beta \in C_c^\infty(\RR)$. We define a `radial multiplier' operator
%
\[ Tf = \sum_\lambda m(\lambda) E_\lambda f \]
%
where $E_\lambda$ is the projection of $f$ onto the space of eigenfunctions for the operator $\sqrt{-\Delta}$ on $M$ with eigenvalue $\lambda$. We can also write this operator as $m(\sqrt{-\Delta})$. Then the wave propogation operator $e^{2 \pi i t \sqrt{-\Delta}}$ is periodic of period $\Pi$. The Weyl formula tells us that the number of eigenvalues of $\sqrt{-\Delta}$ which are smaller than $\lambda$ is equal to $V(M) \cdot \lambda^n + O(\lambda^{n-1})$.

\begin{theorem}
    Let $m \in L^2_\alpha(\RR)$ be supported on $(1,2)$, and assume $\alpha > 1/2$, then for $2 \leq p \leq 4$, $f \in L^p(M)$, and for any integer $k$,
    %
    \[ \left\| \sup_{2^k \leq \tau \leq 2^{k+1}} |\text{Dil}_\tau m(\sqrt{-\Delta}) f| \right\|_{L^p(M)} \lesssim_\alpha \| m \|_{L^2_\alpha(M)} \| f \|_{L^p(M)}. \]
\end{theorem}
\begin{proof}
    To understand the radial multipliers we apply the Fourier transform, writing
    %
    \[ T_\tau f = (\text{Dil}_\tau m)(\sqrt{-\Delta}) f = m(\sqrt{-\Delta} / \tau) f = \int_{-\infty}^\infty \tau \widehat{m}(t \tau) e^{2 \pi i t \sqrt{-\Delta}} f\; dt. \]
    %
    If we define $\beta \in C_c^\infty((1/2,8))$ such that $\beta(s) = 1$ for $1 \leq s \leq 4$, and set $L_k f = \text{Dil}_{2^k} \beta(\sqrt{-\Delta}) f$, then for $2^k \leq \tau \leq 2^{k+1}$
    %
    \[ T_\tau f = (\text{Dil}_\tau m)(\sqrt{-\Delta}) f = (\text{Dil}_\tau m \cdot \text{Dil}_{2^k} \beta)(\sqrt{-\Delta}) = T_\tau L_k f. \]
    %
    so Cauchy-Schwartz implies that
    %
    \begin{align*}
        |T_\tau f(x)| &= \left| \int_{-\infty}^\infty \tau \widehat{m}(\tau) e^{2 \pi i t \sqrt{-\Delta}} L_k f(x)\; dt \right|\\
        &\leq \| m \|_{L^2_\alpha(M)} \left( \int_{-\infty}^\infty \frac{\tau}{(1 + |t \tau|^2)^\alpha} |e^{2 \pi i t \sqrt{-\Delta}} L_k f(x)|^2 \right)^{1/2}\\
        &\leq \| m \|_{L^2_\alpha(M)} \left( \int_{-\infty}^\infty \frac{2^k}{(1 + |2^k t|^2)^\alpha} |e^{2 \pi i t \sqrt{-\Delta}} L_k f(x)|^2 \right)^{1/2}
    \end{align*}
    %
    Because of periodicity, if we set $w_k(t) = 2^k / (1 + |2^k t|^2)^\alpha$, it suffices to prove that for $\alpha > 1/2$,
    %
    \[ \left\| \left( \int_0^\Pi w_k(t) |e^{2 \pi i t \sqrt{-\Delta}} L_k f(x)|^2\; dt \right)^{1/2} \right\|_{L^p(M)} \lesssim_{\alpha,p} \| f \|_{L^p(M)}. \]
    %
    This is a weighted combination of the wave propogators, roughly speaking, assigning weight $2^k$ for $t \lesssim 1/2^k$, and assigning weight $1/t$ to values $t \gtrsim 1/2^k$.

    For a fixed $0 < \delta$, we can split this using a partition of unity into a region where $t \gtrsim \delta$ and a region where $t \lesssim \delta$, where $\delta$ is independent of $k$. For each $t$, the wave propogation $e^{2 \pi i t \sqrt{-\Delta}}$ is a Fourier integral operator of order zero (we have an explicit formula for small $t$, and the composition calculus for Fourier integral operators can then be used to give a representation of the propogation operators for all times $t$, such that the symbols of these operators are locally uniformly bounded in $S^0$). Thus the square function estimate above can be applied in the region where $t \gtrsim \delta$, because the weighted square integral above has weight $O_\delta(1)$ uniformly in $k$.

    Next, we move onto the region $t \lesssim 1/2^k$. The symbol of the operator $e^{2 \pi i t \sqrt{-\Delta}}$

    Finally we move onto the region $1/2^k \lesssim t \lesssim \delta$. On this region we have $w_k(t) \sim 1/t$, which hints we should try using dyadic estimates. In particular, suppose that for $\gamma \leq \delta$, we have a family of dyadic estimates of the form
    %
    \[ \left\| \left( \int_\gamma^{2\gamma} |e^{2 \pi i t \sqrt{-\Delta}} L_k f|^2\; dt \right)^{1/2} \right\|_{L^p(M)} \lesssim \gamma^{1/2} (1 + \gamma 2^k)^\varepsilon \cdot \| f \|_{L^p(M)}. \]
    %
    Summing over the $O(k)$ dyadic numbers between $1/2^k$ and $\delta$ gives
    %
    \[ \left\| \left( \int_{1/2^k \lesssim t \lesssim \delta} |e^{2 \pi i t \sqrt{-\Delta}} L_k f|^2\; \frac{dt}{t} \right)^{1/2} \right\|_{L^p(M)} \lesssim 2^{\varepsilon k} \| f \|_{L^p(M)} \]






    If we were able to obtain this inequality for some $\varepsilon > 0$, then we could bound


     that for all $0 < \gamma < \Pi/2$


    If we localize near $t \lesssim 1/2^k$ by multiplying by $\phi(2^k t)$ for some compactly supported smooth $\phi$ supported on $|t| \lesssim 1$, then for $t$ on the support of $\phi(2^k t)$ we have a weight proportional to $2^k$, and rescaling shows that it suffices to bound the quantities
    %
    \[ \left\| \left( \int \phi(t) |e^{2 \pi i (t/2^k) \sqrt{-\Delta}} L_k f(x)|^2\; dt \right)^{1/2} \right\| \]

     the family of functions
    %
    \[ \left\| \left( \int |\phi(t) e^{2 \pi i (t / 2^k) \sqrt{-\Delta}} L_k f(x)|^2\; Dt \right)^{1/2} \right\|_{L^p_x} \lesssim \sup \| e^{2 \pi i (t / 2^k) \sqrt{-\Delta}} L_k f \|_{L^p_x} \]


    $a_k(t) = 2^{-k/2} \widehat{\phi}(t/2^k) \beta(\tau/2^k)$

    it suffices to uniformly bound quantities of the form
    %
    \[ \left\| \left( \int 2^k \phi(2^k t) |e^{2 \pi i \sqrt{-\Delta}} L_k f(x)|^2\; dt \right)^{1/2} \right\|_{L^p(M)} \lesssim_{\alpha,p} \| f \|_{L^p(M)} \]
    %
    We now apply a dyadic decomposition to deal with the smaller values of $t$. Let us assume for simplicity of notation that $\delta < 1$, and then consider a partition of unity $1 = \sum_{j = 1}^\infty \phi(2^j t)$ for $0 \leq t \leq 1$, and such that $\phi$ is localized near $1/4 \leq t \leq 2$, then our goal is to bound the quantities
    %
    \[ \left\| \left( \int_{-\infty}^\infty \phi(2^j t) \frac{2^k}{(1 + |2^k t|^2)^\alpha} |A_t L_k f(x)|^2\; dt \right)^{1/2} \right\|_{L^p(M)}, \]
    %
    which are each proportional to
    \[ s \]
\end{proof}

\section{Lee Seeger: Decomposition Arguments for FIOs with Cinematic Curvature}

Let's now discuss a paper \cite{LeeSeeger} entitled \emph{Lebesgue Space Estimates For a Class of Fourier Integral Operators Associated With Wave Propogation}. In this paper, Lee and Seeger prove a variable coefficient version of the result of Heo, Nazarov, and Seeger, i.e. generalizing that result as it applies to sharp local smoothing on $\RR^d$ to the local smoothing of Fourier integral operators satisfying the cinematic curvature condition.

We consider two manifolds $Y$ and $Z$, of dimension $d$ and $d+1$ respectively, and consider a Fourier integral operator
%
\[ T : \mathcal{D}(Y) \to \mathcal{D}^*(Z) \]
%
of order $\mu - 1/4$, whose canonical relation $\mathcal{C} \subset T^* Y \times T^* Z$ satisfies the following properties:
%
\begin{itemize}
    \item The projection map $\pi_R: \mathcal{C} \to T^* Y$ is a submersion. It follows that around any point $p = (z_0,y_0;\zeta_0,\eta_0)$ we can choose coordinate systems $y$ and $(x,t)$ on $Y$ and $Z$ respectively, centered at $z_0$ and $y_0$, such that $\zeta_0 = dx_1$, $\eta_0 = dy_1$, and the tangent plane $T_p \mathcal{C} \subset T_p( T^* Y \times T^* Z)$ is the common zero set of the equations
    %
    \[ dx = dy \quad\text{and}\quad d\xi = d\eta \quad\text{and}\quad d\tau = 0, \]
    %
    i.e. the tangent space has a basis given by
    %
    \[ \{ \partial_t \} \cup \{ \partial_{x_j} + \partial_{y_j} : 1 \leq j \leq d \} \cup \{ \partial_{\xi_j} + \partial_{\eta_j} : 1 \leq j \leq d \}. \]
    %
    This means in particular that the projection map $\pi_L: \mathcal{C} \to Z$ is a submersion, and we can locally define a function $\phi(z,\eta)$, homogeneous in $\eta$, such that
    %
    \[ \mathcal{C} = \Big\{ (z, \nabla_z \phi(z,\eta)) \times ( \nabla_\eta \phi(z,\eta), \eta ) \in T^* Z \times T^* Y \Big\}. \]
    %
    By assumption on the tangent space of $\mathcal{C}$,
    %
    \[ (D_\eta \nabla_x) \phi(0, e_1) = I \quad (D_\eta \partial_t) \phi(0,e_1) = 0. \]
    %
    \[ \nabla_\eta \phi(0,e_1) = 0 \quad \nabla_x \phi(0,e_1) = e_1 \quad \partial_t \phi(0,e_1) = 0. \]
    %
    The equivalence of phase theorem implies, after appropriately microlocalizing the inputs and outputs of the operator $T$, we can write
    %
    \[ Tf(x,t) = \int a(x,t,y,\eta) e^{2 \pi i [\phi(x,t,\eta) - y \cdot \eta]} f(y)\; d \eta\; dy, \]
    %
    where $a$ is a symbol of order $\mu$, compactly supported in the $(x,t,y)$ variables.

    \item Because $\pi_L$ is a submersion, for each $z_0 \in Z$, the set
    %
    \[ \Sigma_{z_0} = \pi_L^{-1}(z_0) \]
    %
    is a $d$ dimensional submanifold of $\mathcal{C}$. Because $\mathcal{C}$ is conic, $\Sigma_{z_0}$ is also a conic subset of $Z$. One can see from our choice of coordinates that the projection map of $\Sigma_{z_0}$ onto $T^* Z$ is an immersion, whose image is an immersed hypersurface $\Gamma_{z_0}$ of $T^*_{z_0}$. Indeed, the tangent plane to $\Sigma_{z_0}$ at the point above is given in coordinates by
    %
    \[ dx = dy = dt = d\tau = 0 \quad\text{and}\quad d\xi = d\eta. \]
    %
    And this is projected injectively to the plane defined by $d\tau = 0$ in $T^*_{z_0} Z$.

    We will make one more assumption about $\mathcal{C}$, the \emph{cinematic curvature condition}, that for each $z_0 \in Z$, the hypersurface $\Sigma_{z_0}$ is a cone with $l$ nonvanishing principal curvatures, for some $1 \leq l \leq d-1$. Since
    %
    \[ \Sigma_{z_0} = \Big\{ (z_0; \nabla_z \phi(z_0,\eta_0) \Big\}, \]
    %
    the curvature assumptions amount to the fact that the Hessian matrix
    %
    \[ H_\eta \{ \partial_t \phi \} \]
    %
    has rank at least $l$ on a neighborhood of the microsupport of $a$.
\end{itemize}
%
Given these assumptions, the following result is obtained in \cite{LeeSeeger}.

\begin{theorem}
    If
    %
    \[ l \geq 3 \quad\text{and}\quad 1/l < 1/2 - 1/q < 1/2 \quad\text{and}\quad \mu \leq d(1/q - 1/2) + 1/2 \]
    %
    then $T$ maps $L^q_c(Y)$ into $L^q_{\text{loc}}(Z)$.
\end{theorem}

If we take $l = d-1$, we get the full assumption of `cinematic curvature' and we can use this to get results about local smoothing of the wave equation on compact Riemannian manifolds, which, as a special case, recovers the local smoothing result of Heo, Nazarov, and Seeger obtained in their paper on radial Fourier multipliers.

\begin{theorem}
    If $M^d$ is a compact Riemannian manifold,
    %
    \[ d \geq 4 \quad\text{and}\quad \frac{1}{d-1} < 1/2 - 1/q < 1/2, \]
    %
    and we set $\alpha = (d-1)/2 - d/q$, then for any interval $I$,
    %
    \[ \| e^{it \sqrt{-\Delta}} f \|_{L^q_t(I) L^q_x(M)} \lesssim_I \| f \|_{L^q_\alpha(M)}. \]
\end{theorem}
\begin{proof}
    The solution operator for the half-wave equation is a Fourier integral operator of order $- 1/4$, where $\mu = 0$, associated with the canonical relation
    %
    \[ \mathcal{C} = \{ (x,t,y; \eta, \omega, \eta) : x = \exp_y(t \hat{\xi} )\ \text{and}\ \omega = |\xi|_g \}, \]
    %
    where $\hat{\xi} = \xi / |\xi|$. It is immediate that the projection maps $\mathcal{C} \to T^* Y$ and $\mathcal{C} \to Z$ are submersions. For each $z_0 = (x_0,t_0)$,
    %
    \[ \Gamma_{z_0} = \{ (\xi, \omega) : \omega = |\xi|_g \} \]
    %
    is a spherical cone, and thus has $d-1$ nonvanishing principal curvatures. Consider the operator
    %
    \[ T = L \circ (I - \Delta)^{-\alpha/2}. \]
    %
    Then $T$ is a Fourier integral operator of order $-1/4 + \alpha$, associated with the same canonical relation $\mathcal{C}$. Applying the main theorem of \cite{LeeSeeger}, we conclude that for the $\alpha$ specified, $T$ maps $L^q(M)$ into $L^q_{t,\text{loc}}(\RR) L^q_x(M)$. But this implies that if $g = (I - \Delta)^{\alpha/2} f$, then
    %
    \[ \| e^{it \sqrt{-\Delta}} f \|_{L^q_{t,\text{loc}} L^q_x(M)} = \| Tg \|_{L^q_{t,\text{loc}} L^q_x(M) } \lesssim \| g \|_{L^q(M)} = \| f \|_{L^q_\alpha(M)}. \qedhere \]
%An example situation of where the above results apply is to the Lax parametrix for the half-wave equation, which is a Fourier integral operator with $\mu = 0$. The phase $\phi$ above cannot be specified in elementary functions even in this practical examples, except in very special situations, like the half-wave equation on Euclidean space, in which case $\varphi(x,t,\eta) = x \cdot \eta - t |\eta|^2 / 2$, in which case $H_\eta \{ \partial_t \phi \}$ is the identity matrix, and thus has full rank everywhere.
%
%
% { (x,xi,t,tau,y,eta) : (x,xi) = exp_t(y,eta) and tau = |xi|_g = |eta|_g
%
% varphi(x,t,eta)
% D_t varphi(x,t,eta) = |eta|_g
%
% D_x varphi(x,t,eta) = xi
% D_eta varphi(x,t,eta) = y
% s.t. (x,xi) = exp_t(y,eta)
%
% y lies on the geodesic sphere about x with radius t
% If we work in geodesic normal coordinates about x
% then y = x - t xi
% 
% xi = eta
% x = y + t eta
% D_x varphi(x,t,eta) = eta
% D_eta varphi(x,t,eta) = x - t eta
% partial_t varphi(x,t,eta) = |eta|
%
% varphi(x,t,eta) = x * eta - t |eta|^2 / 2
%
\end{proof}

Now to the proof. We begin by trying to establish a frequency localized theorem, i.e. that
%
\[ T_R f(x,t) = \int a(x,t,y,\eta) \chi(\eta / R) e^{2 \pi i [ \phi(x,t,\eta) - y \cdot \eta ]}\; f(y)\; d\eta\; dy, \]
%
then
%
\[ \| T_R f \|_{L^q(\RR^d)} \lesssim R^{\mu - d (1/q - 1/2) + 1/2} \| f \|_{L^q(\RR^d)}. \]
%
For $q = \infty$, the bound
%
\[ \| T_R f \|_{L^\infty(\RR^d)} \lesssim R^{\mu + \frac{d-1}{2}} \| f \|_{L^\infty(\RR^d)} \]
%
is a result of Seeger, Sogge, and Stein \cite{SeegerSoggeStein}. By interpolation, it suffices to prove a weak type bound for the operator $T_R$ for $1/l < 1/2 - 1/q < 1/2$. We will actually prove a bound for the adjoint $T_R^*$, which we can write as
%
\[ T_R^* f(x) = \int a_R(x,y,\xi) e^{2 \pi i R [ x \cdot \xi - \phi(y,t,\xi) ]} f(y,t)\; d\xi\; dy\; dt, \]
%
where $|\partial_x^\alpha \partial_y^\beta \partial_\xi^\lambda a_R(x,y,\xi)| \lesssim_{\alpha,\beta,\lambda} R^{d + \mu}$ and has support on $|\xi| \sim 1$. Using the fact that $a_R$ is smooth and has compact support in the $x$ and $y$ variables, we can apply a Fourier series argument, which means, without loss of generality, it suffices to study operators of the form
%
\[ T_R^* f(x) = \int a_R(\xi) e^{2 \pi i R [ x \cdot \xi - \phi(y,t,\xi) ]} f(y,t)\; d\xi\; dy\; dt. \]
%
Then
%
\[ \widehat{T_R^* f}(\xi) = R^{-d} a_R(\xi / R) \int e^{- 2 \pi i \phi(y,t,\xi)} f(y,t) \; dy\; dt. \]
%
We will prove a restricted weak type inequality for this operator.

% a_R(x,y,\xi) = 2^{jd} a(x,y,2^j \xi) \chi(\xi)
% xi derivative is either 2^{j(d + mu)}


%
By interpolating this bound, to obtain the result above, it is sufficient to obtain a restricted weak-type bound at the endpoint value of $q$.

\subsection{Frequency Localization and Discretization}

Let us describe the idea of the proof. Let $K(z,y)$ denote the kernel of $T$, i.e.
%
\[ K(z,y) = \int a(z,y,\eta) e^{2 \pi i [\phi(z,\eta) - y \cdot \eta]}\; d\eta. \]
%
Without loss of generality, we may assume that $a$ is supported on $|\eta| \geq 1$, since integrals over small frequencies give a smoothing operator. We localize in frequency, writing, working modulo smoothing operators by ignoring small frequencies,
%
\[ K(x,t,y) = \sum_{j \geq 100} 2^{j \mu} K_{2^j}(z,y), \]
%
where, for $R \geq 1$,
%
\begin{align*}
    K_R(z,y) &= R^{- \mu} \int a(z,y,\eta) \chi(\eta / R) e^{2 \pi i [ \phi(z,\eta) - y \cdot \eta ]}\; d\eta\\
    &= \int a_R(z,y,\eta / R) e^{2 \pi i [ \phi(z,\eta) - y \cdot \eta ]}\; d\eta
\end{align*}
%
where $a_R(z,y,\eta) = R^{-\mu} a(z,y,R \eta) \chi(\eta)$ is chosen to have uniform compact support, and such that
%
\[ |D^\alpha_{z,y,\eta} a_R(z,y,\eta)| \lesssim_\alpha 1 \]
%
holds uniformly in $R$. Let $T_R$ be the operator with kernel $K_R$.

\subsection{Discretizing the Problem}

It is more natural to establish estimates for the adjoint operator $T_R^*$. We will establish restricted estimates, i.e. bounding the behaviour of $T_R^* \{ \chi_E \}$ for a measurable set $E \subset \RR^{d+1}$. We have
%
\[ T_R^* \{ \chi_E \}(y) = \int \chi_E(x,t) \overline{a_R(z,y,\eta / R)} e^{2 \pi i [ y \cdot \eta - \phi(z,\eta) ]}\; d\eta\; dx\; dt. \]
%
The majority of the behaviour of $T_R^*$ depends on the behaviour of $\chi_E$ at frequencies with magnitude $\Theta(R)$, so we perform a discretization at a scale $1/R$. Taking a Fourier series in the $y$ variable, assuming $\text{supp}_y(a_R)$ is contained in $[-1/2,1/2]^d$, we write
%
\[ a_R(z,y,\eta) = \sum_{\nu \in \ZZ^d} a_{R,\nu}(z,\eta / R) \chi(y) e^{2 \pi i \nu \cdot y}, \]
%
where $\chi$ is some smooth, compactly supported function equal to one on $\text{supp}_y(K)$. We may then write
%
\[ T_R^* \{ \chi_E \}(y) = \sum_{\nu \in \ZZ^d} T_{R,\nu}^* \{ \chi_E \}(y) e^{2 \pi i \nu \cdot y}, \]
%
where $T_{R,\nu}$ is the operator with kernel 
%
\[ K_{R,\nu}(z,y) = \int a_{R,\nu}(z,\eta / R) \chi(y) e^{2 \pi i [ \phi(z,\eta) - y \cdot \eta ]}\; d\eta. \]
%
The smoothness and support properties of the symbols $\{ a_R \}$ imply that
%
\[ |D^\alpha_{x,t,\eta} a_{R,\nu}(z,\eta)| \lesssim_{\alpha,N} |\nu|^{-N} \quad\text{for all $N > 0$}, \]
%
and so we can likely ignore the interactions between the operators $T_{R,\nu}^*$ as we vary $\nu$, i.e. by using the triangle inequality.

Let us now study the behaviour of the function $T_{R,\nu}^* \{ \chi_E \}$ on the set $\text{supp}_y(K)$. To do this, we discretize our operator at a scale $1/R$. Let $\mathcal{Z}_R$ be the set of points on the lattice $R^{-1} \ZZ^{d+1}$ which lies in some neighborhood of $\supp_z(K)$. For $\zeta \in \mathcal{Z}_R$, we let $Q_\zeta$ be the sidelength $R^{-1}$ cube centred at $\zeta$. Then $\{ Q_\zeta \}$ is an almost disjoint family of cubes covering $\supp_z(K)$. Define
%
\[ a_{R,\nu,\zeta}(\eta) = \int_{Q_\zeta \cap E} a_{R,\nu}(z,\eta / R) \chi(y) e^{2 \pi i [ \phi(\zeta,\eta) - \phi(z,\eta) ]}\; dz. \]
%
Then if we define
%
\begin{align*}
    S_{R,\nu,\zeta}(y) &= \int a_{R,\nu,\zeta}(\eta) e^{2 \pi i [ y \cdot \eta - \phi(\zeta,\eta) ]}\; d\eta,
\end{align*}
%
then for $y \in \text{supp}_y(K)$,
%
\[ T_{R,\nu}^* \{ \chi_E \}(y) = \sum_\zeta S_{R,\nu,\zeta}(y). \]
%
Now for $z \in Q_\zeta$ and $|\eta| \sim R$,
%
\[ |\partial_\eta^\alpha \{ \phi(\zeta, \eta) - \phi(z,\eta) \}| \lesssim_\alpha R^{-\alpha} \quad\text{for all $\alpha$}, \]
%
which allows us to conclude that
%
\[ |D^\alpha_\eta a_{R,\nu,\zeta}(\eta)| \lesssim_{\alpha,N} |Q_\zeta \cap E| \cdot R^{-\alpha} \langle \nu \rangle^{-N} \quad\text{for all $\alpha$ and $\nu > 0$}. \]
%
We are thus reduced to the analysis of the quantities
%
\[ \sum_{\zeta \in \mathcal{Z}_R} S_{R,\nu,\zeta}. \]
%
If, for $m \geq 0$, we write
%
\[ Z_{R,m} = \{ \zeta: |Q_\zeta \cap E| \sim 2^{-m} (1/R)^{d+1} \}, \]
%
then in the next section, we will obtain estimates of the form
%
\[ \left\| \sum_{\zeta \in Z_{R,m}} S_{R,\nu,\zeta} \right\|_{L^{p,\infty}} \lesssim_N \langle \nu \rangle^{-N} [2^{-m} (1/R)^{d+1}] \cdot R^{ \frac{d+1}{2} - \frac{1}{p} } \#(Z_{R,m})^{1/p}. \]
%
If we set $E_{R,m} = \bigcup_{\zeta \in Z_{R,m}} Q_\zeta \cap E$, then $E = \bigcup_m E_{R,m}$. For each $m$,
%
\[ \#(Z_{R,m}) \sim 2^m R^{d+1} |E_{R,m}|, \]
%
and so the estimate above implies that
%
\[ \left\| \sum_{\zeta \in Z_{R,m}} S_{R,\nu,\zeta} \right\|_{L^{p,\infty}} \lesssim_N \langle \nu \rangle^{-N} 2^{-m / q} R^{d(1/p - 1/2) - 1/2} |E_{R,m}|^{1/p}. \]
%
Summing in $m \geq 0$, and using H\"{o}lder's inequality to prove that for any non-negative numbers $\{ a_m \}$, we have
%
\[ \sum_m 2^{-m/q} a_m^{1/p} \lesssim_q ( \sum a_m )^{1/p}, \]
%
we conclude that
%
\begin{align*}
    \| T_{R,\nu}^* \{ \chi_E \} \|_{L^{p,\infty}} &= \left\| \sum_{\zeta \in \mathcal{Z}_R} S_{R,\nu,\zeta} \right\|_{L^{p,\infty}}\\
    &\lesssim \sum_m \left\| \sum_{\zeta \in Z_{R,m}} S_{R,\nu,\zeta} \right\|_{L^{p,\infty}}\\
    &\lesssim_N \langle \nu \rangle^{-N} R^{d(1/p - 1/2) - 1/2} |E|^{1/p}.
\end{align*}
%
Summing over $\nu$, we conclude that
%
\[ \| T_R^* \{ \chi_E \} \|_{L^{p,\infty}} \lesssim R^{d(1/p - 1/2) - 1/2} |E|^{1/p}. \]
%
Thus we have proved a restricted weak-type bound for the operators $T_R^*$.

\subsection{Interactions of Discretized Operators}

Our proof now rests on the following problem. We fix a large constant $M > 0$, to be specified later. Our goal is to prove the following Lemma.

\begin{lemma} \label{lemmaDiscreteBound}
    For each $Z \subset \mathcal{Z}_R$, and each $C > 0$, suppose there exists a symbol $a_\zeta(\eta)$ supported on $|\eta| \sim R$, and satisfying derivative bounds of the form
    %
    \[ |(\partial_\xi^\alpha a_\zeta)(\eta)| \leq C R^{-\alpha} \quad\text{for $|\alpha| \leq M$}. \]
    %
    If we define
    %
    \[ S_\zeta(y) = \int a_\zeta(\eta) e^{2 \pi i [y \cdot \eta - \phi(\zeta, \eta)]}\; d\eta, \]
    %
    then
    %
    \[ \left\| \sum_{\zeta \in Z} S_\zeta \right\|_{L^{p,\infty}} \lesssim C R^{ \frac{d+1}{2} - \frac{1}{p} } \#(Z)^{1/p}, \]
    %
    where the implicit constant is independent of $C$, $R$, and $Z$.
\end{lemma}

Lemma \ref{lemmaDiscreteBound} is clearly sufficient to prove the estimate mentioned in the last section. So let's now prove it. By linearity, we may assume without loss of generality that $C = 1$.

We will obtain the Lemma by interpolating a combination of more elementary estimates, namely, a pair of $L^1$ and $L^\infty$ bounds for the sum, which do not take advantage of the curvature in the problem, and an $L^2$ bound which does take into account this curvature.

% If $\{ P_R \}$ is a family of Littlewood-Paley projections on $\RR^{d+1}$ induced by a function equal to one on a suitably thick region of the unit annulus, then the principle of nonstationary phase implies that the operator $G = K_R^* \circ (I - P_R)$ is a smoothing operator, with derivative estimates
%
%\[ |(\nabla^M_{y,z} G)(y,z)| \lesssim_{N,M} R^{-N} \]
%
%for all $N,M \geq 0$. Thus we may restrict our analysis to $K_R^* \circ P_R$, i.e. we may study $K_R^*$ from the perspective that all inputs are Fourier localized to frequencies with magnitude $\Theta(R)$. The uncertainty principle thus leads us to imagine that these inputs are locally constant at a scale $1/R$, which motivates us to believe that we can write
%
%\[ K_R^* f \approx \sum_{\zeta \in Z_R} c_\zeta S_\zeta, \]
%
%where $Z_R$ is a $1/R$ discretized subset of some small ball about the origin in $\RR^{d+1}$, and for each $\zeta \in Z_R$,
%
%\[ S_\zeta(y) = \int a_\zeta(\eta) e^{2 \pi i R (y \cdot \eta - \varphi(\zeta,\eta))}\; d\eta, \]
%
%for some smooth, compactly supported functions $\{ a_\zeta : \zeta \in Z_R \}$ with uniform bounds on their derivatives. We are thus motivated to consider the behaviour of the functions $\{ S_\zeta \}$ and their interactions.

To understand the individual behaviour of the functions $\{ S_\zeta \}$, we perform a `double dyadic decomposition', i.e. taking a $R^{-1/2}$ discretized subset $\Theta_R$ of unit vectors in $\RR^d$, consider some smooth partition of unity adapted to the $O(1/R^{1/2})$ neighborhoods of these unit vectors, and thus consider the associated decomposition $a_\zeta = \sum_\theta a_{\zeta,\theta}$. Applyig non-stationary phase, we get that, for the resulting decomposition $S_\zeta = \sum_\theta S_{\zeta,\theta}$, the function $S_{\zeta,\theta}$ has the majority of it's support on a cap about the point $\nabla_\xi \varphi(\zeta,\theta)$, with thickness $R^{-1}$ in the $\theta$-direction, and thickness $R^{-1/2}$ in the directions orthogonal to $\theta$. Moreover, on this cap, $S_{\zeta,\theta}$ has magnitude $O(R^{(d+1)/2})$. Completely ignoring the interactions between these caps, which do not matter anyhow given we are taking the $L^1$ norm of the functions, the triangle inequality implies that
%
\[ \| S_\zeta \|_{L^1} \leq \sum_\theta \| S_{\zeta,\theta} \|_{L^1} \lesssim R^{\frac{d-1}{2}} \cdot R^{\frac{d+1}{2}} \cdot R^{-1} R^{- \frac{d-1}{2}} = R^{\frac{d-1}{2}}.  \]
%
We interpolate this bound with some $L^2$ orthogonality bounds for the family $\{ S_\zeta \}$ to prove the required result.

We will use Fourier analysis to obtain these $L^2$ bounds, which is more simple given that the symbols we now have in our Fourier integral operators are independent of $y$. Namely, we have
%
\[ \widehat{S}_\zeta(\eta) = a_\zeta(\eta) e^{- 2 \pi i \phi(\zeta,\eta)}. \]
%
Fix $\zeta = (x,t)$ and $\zeta' = (x',t')$. By the multiplication formula,
%
\begin{align*}
    \langle S_\zeta, S_{\zeta'} \rangle &= \langle \widehat{S}_\zeta, \widehat{S}_{\zeta'} \rangle\\
    &= \int a_\zeta(\eta) a_{\zeta'}(\eta) e^{2 \pi i [ \phi(\zeta', \eta) - \phi(\zeta,\eta) ]}\; d\eta\\
    &= R^d \int a_\zeta( R \eta) a_{\zeta'}( R \eta ) e^{2 \pi i R [ \phi(\zeta', \eta) - \phi(\zeta, \eta) ] }.
\end{align*}
%
Let us write $\phi_{\zeta,\zeta'}(\eta)$ for the phase in this oscillatory integral. Then
%
\[ \nabla_\eta \phi_{\zeta,\zeta'}(\eta) = \nabla_\eta \phi(\zeta', \eta) - \nabla_\eta \phi(\zeta, \eta). \]
%
Provided we have localized our analysis to a suitably small neighborhood of $(0,e_1)$, our assumptions that
%
\[ (D_\eta \nabla_x \phi)(0, e_1) = I \quad\text{and}\quad (D_\eta \partial_t \phi)(0,e_1) = 0 \]
%
imply that
%
\[ |\nabla_\eta \phi_{\zeta,\zeta'}(\eta)| \geq 0.5 |x - x'| - 0.1 |t - t'|. \]
%
Thus we conclude using this formula that if $|x - x'| \geq 0.5 |t - t'|$, then we have a non-stationary phase, and we can integrate by parts to conclude that for all $N \leq M$,
%
\[ \langle S_\zeta, S_{\zeta'} \rangle \lesssim_N R^{-N}. \]
%
On the other hand, suppose that $|x - x'| \leq 0.5 |t - t'|$. Set $\zeta(s) = \zeta + s(\zeta' - \zeta)$, and write
%
\[ \frac{\phi(\zeta',\eta) - \phi(\zeta, \eta)}{t' - t} = \int_0^1 \left[ \partial_t \phi(\zeta_s, \eta) + \left( \frac{x' - x}{t' - t} \right) \cdot \nabla_x \phi(\zeta_s, \eta) \right]\; ds. \]
%
It follows that the left hand side is a pertubation of $\partial_t \phi(0, \eta)$. The phase
%
\[ (t,t') \mapsto (t' - t) \cdot \partial_t \phi(0,\eta) \]
%
has a stationary point at $(0,e_1)$ by assumption that $(D_\eta \partial_t) \phi(0,e_1) = 0$. But the Hessian $H_\eta(\partial_t \phi)$ has rank $l$ at $(0,e_1)$, so the principle of stationary phase `with parameters' (to account for the pertubation in phase, see H\"{o}rmander's FIO I paper for details) implies that
%
\[ \langle S_\zeta, S_{\zeta'} \rangle \lesssim R^d \langle R |t - t'| \rangle^{-l/2} \quad\text{if}\ |x - x'| \leq 0.5 |t - t'|. \]
%
Putting this bound together with the previous bound, we conclude that for any two $\zeta$ and $\zeta'$, we have
%
\[ \langle S_\zeta, S_{\zeta'} \rangle \lesssim \frac{R^d}{\langle R |\zeta - \zeta'| \rangle^{l/2}}. \]
%
Together with a form of the Calder\'{o}n-Zygmund decomposition introduced by Heo, Nasarov and Seeger, this bound will be sufficient to prove we get the bound required. TODO.

\section{Kim: $L^2$-Based Multiplier Bounds Using Sogge's Eigenfunction Bounds}

This chapter discusses Jongchon Kim's 2017 paper \emph{Endpoint Bounds for Quasiradial Fourier Multipliers} \cite{KimQuasiradial}, and his 2018 paper \emph{Endpoint Bounds for a Class of Spectral Multipliers on Compact Manifolds} \cite{KimSpectral}. These two papers introduce some useful techniques which can be used to generalize some of the results of Heo-Nasarov-Seeger to variable-coefficient settings.

Lets begin with the results of \cite{KimQuasiradial}. Given a homogeneous function
%
\[ a: \RR^d \to (0,\infty) \]
%
of degree one, smooth away from the origin, the paper discusses the problem of bounding `quasiradial' Fourier multipliers with symbols of the form
%
\[ m(\xi) = h(a(\xi). \]
%
The homogeneity and non-negativity condition implies that the `cosphere'
%
\[ \Sigma = \{ \xi : a(\xi) = 1 \} \]
%
is a smooth hypersurface in $\RR^d$. Under the additional assumption that this surface has everywhere non-vanishing Gaussian curvature, Kim proves that for $d \geq 4$, and $1 < p < 2(d-1)/(d+1)$, if $h$ is a unit scale multiplier, then
%
\[ \| h \circ a \|_{M^p(\RR^d)} \lesssim C_p(h). \]
%
This is analogous to the result of Heo, Nasarov, and Seeger, but applied to multipliers that are now only \emph{quasi-radial} rather than radial.

Similar results are obtained in \cite{KimSpectral}, but in the setting of compact manifolds. Given a smooth, compact manifold $M$, and a first-order classical, elliptic, formally positive, self-adjoint pseudodifferential operator $P$, such that the cospheres
%
\[ \Sigma_x = \{ \xi \in T_x M : p(x,\xi) = 1 \} \]
%
have non-vanishing Gaussian curvature, Kim proves that, for a unit-scale function $h$ and for $1 < p < 2(d+1)/(d+3)$,
%
\[ \| h \|_{M^{p,q}_{\text{Dil}}(M,P)} \lesssim \| h \|_{B^2_{d(1/p - 1/2),q}(\RR)}, \]
%
This improves results of Seeger and Sogge (1989), who proved the result with $B^{2,q}_{\alpha_p}$ replaced with $L^2_\alpha$ for $\alpha > \alpha_p$, and a result of Seeger (1991), who replaced $B^{2,q}_{\alpha_p}$ with a subspace $R^{2,q}_{\alpha_p}$ consisting of functions which have decompositions similar to the Bochner-Riesz multipliers. TODO: Understand the relation between these two results.

\subsection{Quasi-Radial Multipliers}

Let's begin by describing the new ideas introduced in \cite{KimQuasiradial}. Let $h$ and $a$ be as above. If $\eta$ is supported on $\{ 1/4 \leq |\xi| \leq 4 \}$, and equal to one on $\{ 1/2 \leq |\xi| \leq 2 \}$, then
%
\begin{align*}
    m(D)f(x) &= \int h(a(\xi)) e^{2 \pi i \xi \cdot (x - y)} f(y)\; dy\; d\xi\\
    &= \int h(a(\xi)) \eta(a(\xi)) e^{2 \pi i \xi \cdot (x - y)} f(y)\; dy\; d\xi\\
    &= \int \widehat{h}(t) \eta(a(\xi)) e^{2 \pi i [ \xi \cdot (x - y) + t a(\xi) ]} f(y)\; dy\; d\xi\; dt\\
    &= \int \widehat{h}(t) (K_t * f)(x)\; dx\; dt,
\end{align*}
%
where $K_t$ is the kernel with
%,
\[ \widehat{K}_t(\xi) = \eta(a(\xi)) e^{2 \pi i t a(\xi)}. \]
%
Let's start with some $L^1$ estimates for the kernels $\{ K_t \}$, which do not even use the curvature properties of $\Sigma$.

\begin{theorem}
    \[ \| K_t \|_{L^1(\RR^d)} \lesssim \langle t \rangle^{\frac{d-1}{2}}. \]
\end{theorem}
\begin{proof}
    Let $\psi$ be the inverse Fourier transform of the function $\eta(a(\cdot))$. Then the Fourier transform of $\text{Dil}_{1/t} \psi$ is equal to $t^{-d} \text{Dil}_t (\eta \circ a)$. We calculate that
    %
    \begin{align*}
        K_t(tx) &= \int (\eta \circ a)(\xi) e^{2 \pi i [t a(\xi) + \xi \cdot tx]}\; d\xi\\
        &= \int t^{-d} \text{Dil}_t (\eta \circ a)(\xi) e^{2 \pi i [ a(\xi) + \xi \cdot x ]}\; d\xi\\
        &= e^{2 \pi i a(D)} (\text{Dil}_{1/t} \psi) (x).
    \end{align*}
    %
    Thus
    %
    \[ \| K_t \|_{L^1(\RR^d)} = t^{-d} \| e^{2 \pi i a(D)} ( \text{Dil}_{1/t} \psi ) \|_{L^1(\RR^d)}. \]
    %
    The operator $e^{2 \pi i a(D)}$ is a Fourier integral operator on $\RR^d$, whose canonical relation is the conormal bundle to the surface $\Sigma$, which is, in particular, a canonical graph. Thus Theorem 2.2 of \cite{SeegerSoggeStein} implies that
    %
    \[ \| e^{2 \pi i a(D)} ( \text{Dil}_{1/t} f ) \|_{L^1(\RR^d)} \lesssim \| (1 + \Delta)^{\frac{d-1}{2}} \{ \text{Dil}_{1/t} f \} \|_{H^1(\RR^d)} \lesssim t^{\frac{d-1}{2}}. \]
    %
    TODO: Finish off.
    % Int e^{2 \pi i [ a(xi) + xi * (x - y) ]}
    % Integral is stationary when Nabla_xi a = (x-y)
    % Then the x derivative is xi
    % And  the y derivative is -xi
    % Canonical relation is the conromal 
\end{proof}

By the curvature hypothesis, for each $z \in \RR^d - \{ 0 \}$, there are exactly two points $\xi_+$ and $\xi_-$ on $\Sigma$ such that $z$ is normal to $\Sigma$ at these points\footnote{To prove at least two such points exist, simply take the maxima and minima of the function $f(\xi) = z \cdot \xi$. The curvature condition on $\Sigma$ implies that any extremal point $\xi^*$ on $\Sigma$ is either a global maxima or a global minima, since $\Sigma$ must, by the curvature condition, always curve away from the hyperplane normal to $z$ at each point.}. Moreover, we can globally parameterize these normal points, such that $z \mapsto \xi_+(z)$ and $z \mapsto \xi_-(z)$ are smooth functions for $z \in \RR^d - \{ 0 \}$. We let
%
\[ \psi_+(z) = \xi_+ \cdot z \quad\text{and}\quad \psi_-(z) = \xi_- \cdot z. \]
%
Now consider the coordinate system $(0,\infty) \times \Sigma \to \RR^d$ given by $(\rho,\omega) \mapsto \rho \omega$. In coordinates, we have
%
\[ d\xi = \rho^{d-1} (\omega \cdot n(\omega)) d\rho d\sigma(\omega), \]
%
We can thus write
%
\begin{align*}
    K_t(z) &= \int \eta(a(\xi)) e^{2 \pi i [ t a(\xi) + \xi \cdot z ]}\; d\xi\\
    &= \int_0^\infty \int_\Sigma \rho^{d-1} \eta(\rho) \left( \langle \omega, n(\omega) \rangle e^{2 \pi i \rho [ t + \omega \cdot z ]} \right)\; d\sigma(\omega)\; d\rho\\
    &= \int_0^\infty \rho^{d-1} \eta(\rho) \Big( b_+(\rho z) e^{2 \pi i \rho (t + \psi_+(z))} + b_-(\rho z) e^{- 2 \pi i \rho (t + \psi_-(z))} \Big)\; d\rho,
\end{align*}
%
where $b_+$ and $b_-$ are symbols of order $-(d-1)/2$, and where $b_+$ and $b_-$ have principal symbols which are scalar multiples of
%
\[ z \mapsto |z|^{- \frac{d-1}{2}} \langle \xi_+, n(\xi_+) \rangle\; \quad\text{and}\quad z \mapsto |z|^{-\frac{d-1}{2}} \langle \xi_-, n(\xi_-) \rangle \]
%
respectively. Exploiting the oscillation of $e^{2 \pi i \rho (t + \psi_+(z))}$ and $e^{-2 \pi i \rho (t + \psi_-(z))}$, integrating by parts, we conclude that for all $N \geq 0$,
%
\begin{equation} \label{KtDecayEquation}
    |K_t(z)| \lesssim_N (1 + |t| + |x|)^{- \frac{d-1}{2}} \sum_{\pm} \langle t + \psi_{\pm}(x) \rangle^{-N}.
\end{equation}
%
We note that $\psi_+(x) \geq 0 \geq \psi_-(x)$, that both functions are homogeneous of degree one, and that in the basic situation where $\Sigma$ is the unit sphere, i.e. when $a(\xi) = |\xi|$,
%
\[ \psi_+(x) = x \quad\text{and}\quad \psi_-(x) = -x. \]
%
For $t > 0$, we thus see that \eqref{KtDecayEquation} reads
%
\[ |K_t(z)| \lesssim_N (1 + |t| + |x|)^{- \frac{d-1}{2}} \langle t + \psi_-(x) \rangle^{-N}. \]
%
Similarily, we can apply Parseval's identity, together with the fact that

\[ \widehat{K}_t(\zeta) = \eta(a(\zeta)) e^{2 \pi i t a(\zeta)}, \]
%
to conclude that
%
\begin{align*}
    &\int K_{t_0}(x - x_0) \overline{K_{t_1}(x - x_1)}\; dx\\
    &\quad\quad\quad= \int |\eta(a(\zeta))|^2 e^{2 \pi i [ (x_0 - x_1) \cdot \zeta + (t_0 - t_1) a(\zeta) ]},
\end{align*}
%
and similar stationary phase estimates to above show that for $N \geq 0$,
%
\begin{align*}
    & \left| \int K_{t_0}(x - x_0) \overline{K_{t_1}(x - x_1)}\; dx \right|\\
    &\quad\quad\quad \lesssim_N (1 + |x_0 - x_1| + |t_0 - t_1|)^{- \frac{d-1}{2}} \sum_{\pm} \langle (t_0 - t_1) + \phi_{\pm}(x_0 - x_1) \rangle^{-N}.
\end{align*}

\begin{remark}
    Using \eqref{KtDecayEquation}, one sees the kernels $K_t$ decay rapidly away from
    %
    \[ \{ x : |t + \psi_-(x)| \leq 1 \}, \]
    %
    a set well approximated by an annulus of thickness $O(1)$ and radius $\sim t$, in particular having measure $O(t^{d-1})$. On this set, $K_t$ has magnitude $O(t^{-\frac{d-1}{2}})$. This gives an alternate method to obtain the $L^1$ bound
    %
    \[ \| K_t \|_{L^1(\RR^d)} \lesssim t^{d-1} t^{- \frac{d-1}{2}} = t^{\frac{d-1}{2}}, \]
    %
    that was obtained without the use of the curvature hypothesis.
\end{remark}

Together with a discretization argument, and the density decomposition arguments introduced in Heo-Nazarov-Seeger, the result follows. One slight difference is that, since we are using the half-wave equation formalism, our discretized estimates have slightly different weights, corresponding to the definition of $C_p(h)$. Indeed, it is proved that
%
\[ \left\| \sum_{n \geq 2} K_{n + u} * f(n, \cdot) \right\|_{L^p(\RR^d)} \lesssim \Big\| (1 + n + u)^{s_p} f(n + u,y) \Big\|_{l^p_n(\NN) L^p_y(\RR^d)}, \]
%
uniformly for $0 \leq u \leq 1$. Applying Minkowski's inequality, and the fact that $L^1[0,1] \leq L^p[0,1]$,
%
\begin{align*}
    \left\| \int_2^\infty (K_t * f)(t, \cdot)\; dt \right\|_{L^p(\RR^d)} & \lesssim \Big\| \sum_{n \geq 2} K_{n + u} * f(n+u,\cdot) \Big\|_{L^1_u[0,1] L^p(\RR^d)}\\
    &\leq \Big\| \sum_{n \geq 2} K_{n + u} * f(n+u,\cdot) \Big\|_{L^p_u[0,1] L^p(\RR^d)}\\
    &\lesssim \Big\| (1 + n+u)^{s_p} f(n + u,y) \|_{L^p_u[0,1] l^p_n(\NN) L^p_y(\RR^d)}\\
    &= \left( \iint_{t \geq 1} (1 + t)^{(d-1)(1 - p/2)} |f(t,y)|^p\; dt\; dy \right)^{1/p}.
\end{align*}
%
The right hand side of this inequality is comparable to $C_p(h)$ times $\| g \|_{L^p(\RR^d)}$ if $f(t,y) = \widehat{m}(t) g(y)$, in which case the computed quantity is essentially the $L^p$ norm of $m(a(D)) \{ g \}$. This inequality is equivalent to the boundedness of the convolution kernel in terms of $C_p(h)$.

Next, discretization is done in the $y$-variable. To prove the inequality above, it suffices to show that for functions $b_{n,z}$ concentrated on neighborhoods of $z \in \ZZ^d$,
%
\[ \left\| \sum_{n \geq 2} \sum_{z \in \ZZ^d} c(n,z) (K_n * b_{n,z}) \right\|_{L^p(\RR^d)} \lesssim \Big\| c(n,z) (1 + n)^{s_p} \Big\|_{l^p_n l^p_z}. \]
%
For interpolation purposes, it is better to reweight this inequality as
%
\[ \left\| \sum_{n \geq 2} \sum_{z \in \ZZ^d} (1 + n)^{\frac{d-1}{2}} c(n,z) (K_n * b_{n,z}) \right\|_{L^p(\RR^d)} \lesssim \left( \sum_n \sum_n |c(n,z)|^p (1 + n)^{d-1} \right)^{1/p}. \]

\subsection{Spectral Multipliers}

The result of \cite{KimSpectral} uses similar techniques, combined with the Lax parametrix to substitute for the convolution kernel decomposition possible in \cite{KimQuasiradial} because of the translation invariance of the operators studied.

In the proof, by a partition of unity argument, it suffices to prove the result assuming that our inputs functions $f$ lie in $L^p(\Omega_0)$, where $\Omega_0$ is a compact subset of $M$ contained in a single coordinate chart $\Omega$ of $M$. It will help to fix a compact set $\Omega_1$ whose interior contains $\Omega_0$.

To prove the result, we write
%
\[ m(P / R) f = \int [R \widehat{m}(R\xi) ] e^{2 \pi i t P} f. \]
%
Write $m = \sum_{j \geq 0} m_j$, where for $j > 0$, $m_j$ is a Littlewood-Paley cutoff on an annulus at a scale $2^j$, and $m_0$ is a Littlewood-Paley cutoff on the unit ball.

For $2^j \lesssim 1$, we can reduce the study of the operators $m_j(P/R)$ to symbol estimates. For $2^j \gtrsim R$, one can use the local constancy property of the multipliers, together with $L^p(M)$ to $L^2(M)$ by using the fact that $M$ is compact, and thus has finite volume, and then apply orthogonality, to obtain the boundedness of $m_j(P/R)$. We mainly concentrate on the new techniques for controlling the interactions between the multipliers $\{ m_j : 1 \lesssim j \lesssim \log R \}$.

We're going to need some even, smooth cutoff functions $\eta \prec \eta' \prec \eta''$:
%
\begin{itemize}
    \item $\eta$ is supported on $\{ |\lambda| \in [1/4,4] \}$, and equal to one on $\{ |\lambda| \in [1/2,2] \}$.

    \item $\eta'$ is supported on $\{ |\lambda| \in [1/6,6] \}$, and equal to one on $\{ |\lambda| \in [1/4,4] \}$.

    \item $\eta''$ is supported on $\{ |\lambda| \in [1/16,16] \}$, and equal to one on $\{ |\lambda| \in [1/8,8] \}$.
\end{itemize}
%
We let $\eta_j, \eta'_j$, and $\eta''_j$ denote the dilations of these cutoffs by $2^j$.

It will help us to localize the function $f$ to the eigenband $\lambda \sim R$. For $f \in L^p(\Omega_0)$, we can write
%
\[ \eta''(P/R) f = S_R f + A_R f, \]
%
where $\| A_R f \|_{L^p(M)} \lesssim_N R^{-N} \| f \|_{L^p(\Omega_0)}$ for all $f \in L^p(\Omega_0)$, and where $S_R f$ is supported on $\Omega_1$ for $f \in L^p(\Omega_0)$.

We could conclude our argument if we could show that
%
\[ \left\| \sum_{1 \lesssim j \lesssim \log R} m_j(P/R) f \right\|_{L^{p,q}(M)} \lesssim \| m \|_{B_{\alpha_p}^{2,q}(\RR)} \| f \|_{L^p(\Omega_0)}, \]
%
for all functions $f \in L^p(\Omega_0)$. We claim this result follows from the following proposition.

\begin{lemma}
    Fix a family of functions $\{ b_j: 1 \lesssim j \lesssim \log R \}$ such that:
    %
    \begin{itemize}
        \item $\| b_j \|_{L^2(\RR)} \lesssim 1$.

        \item $\widehat{b}_j$ was supported on $\{ |t| \sim 2^j \}$.

        \item For any $n$ and $M$, and $|\lambda| \not \in [1/8, 8]$,
        %
        \[ |\partial^n b_j(\lambda)| \lesssim_{n,M} 2^{-jM} \langle 2^j \lambda \rangle^{-M}. \]
    \end{itemize}
    %
    Then for functions $\{ f_j \}$ in $L^p(\Omega_0)$,
    %
    \[ \left\| \sum_j 2^{jd/2} b_j(P/R) \{ S_R f_j \} \right\|_{L^p(\Omega)} \lesssim \left( \sum_j 2^{jd} \| f_j \|_{L^p(M)}^p \right)^{1/p}. \]
    %
    It follows simply from this that
    %
    \[ \left\| \sum_j 2^{jd/2} b_j(P/R) f_j \right\|_{L^p(M)} \lesssim \left( \sum_j 2^{jd} \| f_j \|_{L^p(\Omega_0)}^p \right)^{1/p}. \]
    %
    An interpolation lemma (Lemma 2.4 of Lee, Rogers, and Seeger) yields a square function estimate of the form.
    %
    \[ \left\| \sum_j 2^{-jd(1/p - 1/2)} b_j(P/R) f_j \right\|_{L^{p,q}(M)} \lesssim \left\| \left( |f_j|^q \right)^{1/q} \right\|_{L^p(\Omega_0)} \]
\end{lemma}

How does this result imply the required inequality? We can write
%
\[ m_j = \eta_j(D) m_j = \eta_j(D) \left\{ m_j \eta' \right\} + \eta_j(D) \left\{ m_j (1 - \eta') \right\}. \]
%
The multipliers $\eta_j(D) \{ m_j (1 - \eta') \}$ are rapidly decaying (TODO: I Can't see intuitively why this should be the case), so it's easy to bound the corresponding multiplier operators. We therefore reduce our required estimate to
%
\[ \left\| \sum_j [ \eta_j(D) \{ m_j \eta' \} ](P/R) f \right\|_{L^{p,q}(M)} \lesssim \| m \|_{B^{2,q}_{\alpha_p}(\RR)} \| f \|_{L^p(\Omega)}, \]
%
for $f$ supported on $\Omega$. If $b_j = \| m_j \|_{L^2(\RR)}^{-1} \eta_j(D) \{ m_j \eta' \}$, then $\{ b_j \}$ satisfies the assumptions of the lemma above, and the interpolated result, with $f_j = 2^{j \alpha_p} \| m_j \|_{L^2(\RR)} f$.

The Lemma will be proved by a reduction to a restricted weak-type inequality, namely, that for any family of finite subsets $\mathcal{E}_j$ of $(\ZZ / R)^d$, 
%
\[ \left\| \sum_j 2^{jd/2} \sum_{n \in \mathcal{E}_j} b_j(P/R) \chi_{j,n} \right\|_{L^{p,\infty}(\Omega)} \lesssim R^{-d/p} \left( \sum_j 2^{jd} \# \mathcal{E}_j \right)^{1/p}. \]
%
We will mainly focus now on the $L^2$ estimates that imply this restricted weak-type inequality.

For a fixed $\alpha > 0$, we cover $\RR^d$ by essentially disjoint cubes with sidelength $2^j / R$, and denote the collection of cubes in $\mathcal{Q}_j$. Let $\mathcal{Q}_j(\lambda)$ be the collection of all $Q \in \mathcal{Q}_j$ such that $Q \cap \# \mathcal{E}_j > \lambda^p$. This allows us to write
%
\[ \mathcal{E}_j = \mathcal{E}^{\text{High}}_j(\lambda) \cup \mathcal{E}^{\text{Low}}_j(\lambda), \]
%
where $\mathcal{E}^{\text{High}}_j(\lambda)$ is the collection of points in $\mathcal{E}_j$ that lie in a cube $Q$ lying in $\mathcal{Q}_j(\lambda)$, i.e. the set of `high density points'. Fix $C > 0$ suitably large, and for each cube $Q$, let $Q^*$ denote the cube with the same center as $Q$, but $C$ times the sidelength. Then
%
\[ \sum_j \sum_{Q \in \mathcal{Q}_j(\lambda)} |Q| \lesssim \sum_j \sum_{Q \in \mathcal{Q}_j} (2^{jd} / R) \alpha^{-p} \#( \mathcal{E}_j \cap Q ) \leq \alpha^{-p} \sum_j 2^{jd} t^{-d} \#(\mathcal{E}_j). \]
%
This inequality shows that it is fair to throw out the union of the cubes in $\mathcal{Q}_j(\lambda)$ in the analysis of the measure of the set
%
\[ \left\{ x \in \Omega : \left| \sum_j 2^{jd/2} \sum_{Q \in \mathcal{Q}_j(\lambda)} b_j(P/R) \chi_{j,n} \right| > \alpha \right\}. \]
%
The required bound would follow if we could show that
%
\[ \lambda^{-1} \sum_j 2^{jd/2} \left\| \sum_{Q \in \mathcal{Q}_j(\lambda)} [ b_j(P/R) \chi_{j,Q} ] \mathbf{I}_{(Q^*)^c} \right\|_{L^1(\Omega)} \lesssim \lambda^{-p} \sum_j 2^{jd} R^{-d} \# \mathcal{E}_j. \]
%
But (see end of Section 5.1 for more details), this follows from the fact that the Lax parametrix is supported on a small neighborhood of the diagonal.

We are now left with the analysis of $\mathcal{E}_j^{\text{Low}}(\lambda)$. To simplify notation, we will assume that $\mathcal{E}_j = \mathcal{E}_j^{\text{Low}}(\lambda)$ for all $j$. We will try and control
%
\[ \left\| \sum_j 2^{jd/2} \sum_{n \in \mathcal{E}_j} b_j(P/R) \chi_{j,n} \right\|_{L^2(\Omega)}^2 \lesssim \alpha^{2 - p} \log \alpha \sum_j 2^{jd} R^{-d} \# \mathcal{E}_j. \]
%
Define $G_j = \sum_{n \in \mathcal{E}_j} b_j(P/R) f_{j,n}$. Then, applying the triangle inequality for $j \lesssim \log \alpha$, we conclude that
%
\[ \| \sum_j 2^{jd/2} G_j \|_{L^2(\Omega)}^2 \lesssim \log \alpha \left( \sum_j 2^{jd} \| G_j \|_{L^2(\Omega)}^2 + \sum_j \sum_{\log \alpha \lesssim k \lesssim j} 2^{ \frac{k+j}{2} } |\langle G_j, G_k \rangle| \right). \]
%
We will bound each of these quantities separately.

First, let's bounded $\| G_j \|_{L^2(\Omega)}$.







Let us suppose the Lax parametrix applies on times $|t| \leq \varepsilon$. Times larger than this are dealt with fairly simply

We can apply the Lax parametrix for times $|t| \leq \varepsilon$. The large times are dealt with easily, using compactness to reduce $L^p(M)$ estimates to $L^2(M)$ estimates, and then applying orthogonality. Similarily, small times can also be dealt with by reducing to the study of pseudodifferential operators.


% E_j Finite subset of Z^d_t = (Z / t)^d
% For z in Z^d_t, q_z = X cap prod [z_i, z_i + 1/t].












%\chapter{Large Times}

%Write
%
%\[ m(P/R) f = \int [ R \widehat{m}(Rt) ] e^{2 \pi i t P} f. \]
%
%Parametrix methods can only understand the wave equation $\{ e^{2 \pi i t P} f \}$ for times $|t| \lesssim 1$. When studying multipliers on manifolds. But this is often fine, because our assumptions normally allow for greater control on $\widehat{m}$ for large times. Here are some approaches we can do:
%
%\begin{itemize}
%    \item Since $M$ is compact, it has finite volume, so we can reduce our $L^p$ estimates to $L^2$ estimates if $p \leq 1$ and then use orthogonality, concluding that
    %
%    \begin{align*}
%        \| m(P/R) f \|_{L^p(M)} &\lesssim \| m(P/R) f \|_{L^2(M)}\\
%        &= \left\| \sum_\lambda m(\lambda/R) P_\lambda f \right\|_{L^2(M)}\\
%        &= \left( \sum_\lambda |m(\lambda/R)|^2 \| P_\lambda f \|_{L^2(M)}^2 \right)^{1/2}
%    \end{align*}
    %
%    TODO: See Jongchon's paper, i.e. Section 2.2

%    \item We can reduce our analysis to local smoothing.
%\end{itemize}