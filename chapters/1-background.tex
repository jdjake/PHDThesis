%!TEX root = ../main.tex

\chapter{Multipliers of an Elliptic Operator} \label{cha:multipliers_of_an_elliptic_operator}

Let us now more precisely describe the problem of this thesis. Let $X$ be a compact manifold equipped with a volume density, and let $P$ be a classical elliptic operator on $X$ of order $s > 0$, formally positive-definite in the sense that
%
\[ \langle Pf, g \rangle = \langle f, Pg \rangle \quad\text{and}\quad \langle Pf, f \rangle \geq 0 \quad\quad\text{for all $f,g \in C^\infty(M)$}. \]
%
By general properties of elliptic operators, $1 + P$ is an isomorphism between the Sobolev space $H^s(X)$ and $L^2(X)$,
%, and because $P$ is positive-definite, we have an estimate of the form
%
%\[ \big\| (1 + P) f \big\|_{L^2(X)} \geq \| f \|_{L^2(X)}, \]
%
%which implies $1 + P$ is a bijection between $H^a(X)$ and $L^2(X)$.
and so the inverse $(1 + P)^{-1}$, viewed as a map from $L^2(X)$ to itself, is compact by a form of the Rellich-Kondrachov embedding theorem. By the spectral theorem for compact operators, there exists a discrete set $\Lambda \subset [0,\infty)$, and a decomposition
%
\[ L^2(X) = \bigoplus\nolimits_{\lambda \in \Lambda} \mathcal{V}_\lambda, \]
%
where $\mathcal{V}_\lambda$ is a finite dimensional subspace of $C^\infty(X)$, such that $Pf = \lambda f$ for all $f \in \mathcal{V}_\lambda$. We use this decomposition to define a functional calculus for the operator $P$; given a bounded function $a: \Lambda \to \CC$, we can define an operator $a(P)$ on $L^2(X)$ so that $a(P) f = a(\lambda) f$ for all $f \in \mathcal{V}_\lambda$. Our goal is to study the regularity of such operators, in terms of properties of the function $a$.

In the sequel, we will assume the operator $P$ is an elliptic operator \emph{of order one}. This does not restrict the scope of our analysis; given a formally positive elliptic operator $P$ of order $s$, the operator $P^{1/s}$ defined by the functional calculus above is a formally positive elliptic operator of order one\footnote{See Theorem 3.3.1 of \cite{Sogge} for a proof of this fact, based on a technique of \cite{Seeley} initially developed for elliptic differential operators, but which can be straightforwardly adapted to pseudodifferential operators.}, and the spectral theory of $P$ is identical with the spectral theory of $P^{1/s}$. Fixing the order of our operator reflects the fact that the theory of Fourier integral operators we will eventually employ is most elegant when the resulting oscillatory integrals have homogeneous phases of order one. To prevent being overly wordy, in all that follows by an `elliptic operator' we will mean a classical elliptic operator of order one which is formally positive-definite in the sense above.

%The Weyl Law\footnote{See \cite{Hormander1} for details} for such operators tells us that $\# ( \Lambda \cap [0,R] ) = C \cdot R^d + O(R^{d-1})$, where
%
%\[ C = \int_{T^* M} \II[ p(x,\xi) \leq 1 ]\; d\xi\; dx, \]
%
%the function $p: T^* M \to [0,\infty)$ being the principal symbol of $P$.

The primary example of elliptic operators for our purposes are obtained from a Laplace-Beltrami operator on a compact Riemannian manifold $M$; the operator $-\Delta$ is self-adjoint and positive-definite with respect to the volume form $dV$ on $M$, because of the `integration by parts' identity $\langle \Delta f, g \rangle = - \langle \nabla f, \nabla g \rangle$. Since we restrict our attention to elliptic operators of order one, the archetypical object of study is the operator $P = \sqrt{-\Delta}$.

To study the regularity of $a(P)$, we introduce the spaces $M^p(X)$, consisting of all functions $a: \Lambda \to \CC$ for which the operator $a(P)$ is  bounded on $L^p(X)$. The space $M^p(X)$ is then equipped with a norm
%
\[ \| a \|_{M^p(X)} = \sup \left\{ \frac{\| a(P) f \|_{L^p(X)}}{\| f \|_{L^p(X)}} : f \in C^\infty(X) \right\}. \]
%
%For notational convenience, we write $M^p(X) = M^{p,p}(X)$.
Orthogonality implies $M^2(X)$ is isometrically equal to $l^\infty(\Lambda)$, but the structure of the other spaces $M^p(X)$ is unclear.

The discrete spectrum of the operator $P$ makes it difficult to apply methods of oscillatory integrals to the problem. Fortunately, certain semiclassical heuristics tell us that these problems disappear when we restrict the problem to 'high frequency inputs'. In order to take advantage of this fact, rather than studying what conditions ensure the operators $a(P)$ are bounded, we study conditions that ensure the operators $a_R(P) = a(P/R)$ are uniformly bounded, for functions $a: [0,\infty) \to \CC$. %, since these operators depend more and more on 'high frequency behaviour' in the limit.
To prevent pathological examples from arising under dilation, we restrict ourselves to \emph{regulated functions} $a$, i.e. functions such that
%
\[ a(\lambda_0) = \lim_{\delta \to 0} \fint_{|\lambda - \lambda_0| \leq \delta} a(\lambda)\; d\lambda \quad \text{for all $\lambda_0 \in [0,\infty)$}. \]
%
%For the manifolds, elliptic operators, and exponents that we focus on in this thesis, for any regulated function $a$,
%
%\[ \sup\nolimits_R \| a_R \|_{M^{p,q}(X)} \sim \limsup\nolimits_{R \to \infty} \| a_R \|_{M^{p,q}(X)}. \]  
%
%Thus uniform boundedness as $R \to \infty$ gives uniform boundedness for all $R > 0$,
Define the set $M^p_{\text{Dil}}(X)$ of all regulated functions $a$ such that $\| a \|_{M^p_{\text{Dil}}(X)} = \sup_R \| a_R \|_{M^p(X)}$ is finite. With notation introduced, we now state precisely the problem studied in this thesis:
\begin{changemargin}{2cm}{2cm}
\begin{center}
  \emph{Can one find simple conditions on a function $a$,\\
  necessary and sufficient to be contained in $M^p_{\text{Dil}}(X)$ for $p \neq 2$.}
\end{center}
\end{changemargin}
%
For $p = 2$, orthogonality gives the isometric equivalence $M^2_{\text{Dil}}(X) = L^\infty[0,\infty)$.

\begin{lemma}
  For any regulated $a$ and any compact manifold $X$, $\| a \|_{M^2_{\text{Dil}}(X)} = \| a \|_{L^\infty[0,\infty)}$.
\end{lemma}
\begin{proof}
  Given $f \in L^2(X)$, if we write $f = \sum f_\lambda$ with $f_\lambda \in \mathcal{V}_\lambda$, then for any $R > 0$,
  %
  \[ \| a_R(P) f \|_{L^2(X)} = \big\| \sum\nolimits_\lambda a_R(\lambda) f_\lambda \big\|_{L^2(X)} = \left( \sum\nolimits_\lambda |a_R(\lambda)|^2 \| f_\lambda \|_{L^2(X)}^2 \right)^{1/2} \]
  %
  Conversely,
  %
  \[ \| f \|_{L^2(X)} = \left( \sum\nolimits_\lambda \| f_\lambda \|_{L^2(X)}^2 \right)^{1/2}. \]
  %
  Setting $\| f_\lambda \|_{L^2(X)} = c_\lambda$, the quantity $\| a_R \|_{M^2(X)}$ is thus the smallest constant such that
  %
  \[ \left( \sum\nolimits_\lambda |a_R(\lambda)|^2 c_\lambda^2 \right)^{1/2} \leq \| a_R \|_{M^2(X)} \left( \sum\nolimits_\lambda c_\lambda^2 \right)^{1/2}. \]
  %
  It is now clear that $\| a_R \|_{M^2(X)} = \| a_R \|_{l^\infty(\Lambda)}$, and thus (now using the fact that $a$ is regulated),
  %
  \[ \| a \|_{M^2_{\text{Dil}}(X)} = \sup\nolimits_R \| a_R \|_{l^\infty(\Lambda)} = \| a \|_{L^\infty[0,\infty)} \qedhere. \]
\end{proof}

Before the results of this thesis, no conditions had been proved necessary and sufficient for boundedness for $p \neq 2$ and $X \neq \TT^d$, aside from the isometric , which holds by orthogonality. We will obtain results under two main assumptions, which we call Assumption A and Assumption B, the first being a curvature condition on the principal symbol of the $P$, and the second an assumption about the operator's eigenvalues.

\vspace{0.5em}
\noindent \fbox{\parbox{\textwidth}{\textbf{Assumption A}: For each $x_0 \in M$, the cosphere
%
\[ S_{\!\! x_0} = \{ \xi \in T^*_{\! \! x_0} M : p(x_0,\xi) = 1 \} \]
%
is a hypersurface in $T^*_{\!\!x} M$ with non-vanishing Gauss curvature.}}

\vspace{0.5em}

\noindent \fbox{\parbox{\textwidth}{\textbf{Assumption B}: The spectrum of the operator $P$ is contained in an arithmetic progression.}}

\vspace{0.4em}

When $P = \sqrt{-\Delta}$ on a Riemannian manifold $M$, Assumption A always holds, because the cospheres defined above are ellipsoids. For more general operators, Assumption A implies a geometry on the manifold $M$, and Assumption B is closely related to this geometry, in particular, that geodesics with respect to this geometry are all closed and have common length. 

Under Assumption A and Assumption B, we find necessary and sufficient conditions for a function to be contained in $M^{p,q}_{\text{Dil}}(M)$ for $1/p - 1/2 > 1/(d+1)$ and for $p \leq q < 2$. Fix $\chi \in C_c^\infty(\RR^d)$. For $s \geq 0$ and $1 \leq p \leq \infty$, define the norm
%
\[ \| a \|_{R^{s,p}[0,\infty)} = \sup\nolimits_{R > 0} \left( \int_0^\infty \left|\;\! \widehat{a}_R(t) \langle t \rangle^s \right|^p\; dt \right)^{1/p} \quad\text{where}\quad a_R(\lambda) = \chi(\lambda) a(R \lambda). \]
%
Here $\widehat{a}_R(t) = \int_0^\infty a_R(\lambda) \cos(2 \pi \lambda t)$ is the cosine transform of $a_\rho$, i.e. the Fourier transform of the even extension of $a$ to a function on $\RR$. The main result of this thesis is that if $s = (d-1)(1/p - 1/2)$, then for $1/p - 1/2 > 1/(d+1)$, $M^p_{\text{Dil}}(X)$ is isomorphically equal to $R^{s,p}[0,\infty)$. This is the first such characterization for $p \neq 2$ and $X \neq \TT^d$.

\begin{theorem} \label{maintheorem}
  Suppose $X$ is a manifold, and $P$ is an elliptic operator on $X$ satisfying Assumption A and Assumption B. Then for $1/p - 1/2 > 1/(d+1)$, if $s = (d-1)(1/p - 1/2)$, then for any regulated $a$, $\| a \|_{M^p_{\text{Dil}}(X)} \sim \| a \|_{R^{s,p}[0,\infty)}$.
\end{theorem}

% f_R^(xi) = R^d chi(R xi)
% f_R(xi) = chi^(xi / R)
%
% T f_R(x) = R^d int a(xi) chi(R xi) e^{2 pi i xi * x}
%     = int a(xi / R) chi(xi) e^{2 pi i xi * x / R}
%     = R^{-d} int a_R(xi)
%

One may view control on the $R^{s,p}$ norm as a smoothness condition on the multiplier $a$; indeed the Hausdorff-Young inequality implies a Sobolev space estimate
%
\begin{equation} \label{equation12980u21u89eqiwjqwiou}
  \| a_R \|_{W^{s,p'}[0,\infty)} \lesssim \| a \|_{R^{s,p}[0,\infty)}.
\end{equation}
%
However, the space $R^{s,p}[0,\infty)$ is not equal to a Sobolev or Besov space, though we do have the Besov space estimate
%
\begin{equation} \label{aiwodjawoij334242}
  \| a \|_{R^{s,p}[0,\infty)} \lesssim \sup\nolimits_R \| a_R \|_{B^{s+1/p - 1/2,2}_p[0,\infty)}.
\end{equation}
%
The result is somewhat intuitive. Inequality \eqref{equation12980u21u89eqiwjqwiou} tells us that elements of $R^{s,p}$ have $s$ derivatives in $L^{p'}$, though with `some extra control' occuring on the other side of the Fourier transform. Sobolev embedding heuristics tell us that having $s + 1/p - 1/2$ derivatives in $L^2$ is sufficient to have $s$ derivatives in $L^{p'}$. The presence of $L^2$ norms on the right hand side of \eqref{aiwodjawoij334242} allows us to losslessly convert between estimates on the Fourier transform side, allowing us to recover the information on the Fourier side and obtain \eqref{aiwodjawoij334242}.

% (d-1)(1/p - 1/2) derivatives in L^{p'}
% vs. d(1/p - 1/2) derivatives in L^2
% lose precisely (1/2 - 1/p') = (1/p - 1/2) deriavtives to move up to L^{p'}
% 

For most manifolds $X$ and $P$, elements of $\mathcal{V}_\lambda$ are difficult to describe explicitly; even for the relatively simple case of the sphere $S^d$ many questions about the geometric behaviour of eigenfunctions remain open. Many arguments in harmonic analysis involve an interplay between spatial and frequential control, and without explicit descriptions of eigenfunctions, spatial control becomes difficult. Nonetheless, we will find we can obtain some spatial control by utilizing the wave equation on $M$, which carries geometric information via the behaviour of wave propogation. Under the curvature assumptions we make, wave propogation has sufficient smoothing properties to match certain necessary conditions that multipliers need in order to be bounded. And Assumption B implies that the wave equation associated with the operator $P$ is periodic, which simplifies the large time analysis of the wave equation.

%Up to lower order pertubations, the Laplace-Beltrami operator is the unique elliptic operator of order two whose principal symbol is equal to the square of the norm induced on $T^* M$ from the Riemannian metric on $M$. In Chapter BLAH, we will see that this property is true of \emph{all} elliptic operators that satisfy the first assumption of our problem, provided that we are willing to generalize our study to \emph{Finsler manifolds} rather than just Riemannian manifolds. The second assumption is then closely related to the behaviour of geodesics on the manifold, in particular, that all geodesics and closed and have common length.

In the next chapter, we will begin our study by describing what is currently known for the boundedness problem on $\TT^d$, where eigenfunctions are explicit, and one can study the behaviour of spectral multipliers via the Fourier transform and radial convolution operators on $\RR^d$. Our analysis here will help gain intuition and motivate potential hypotheses in the more general setting.

%Under this assumption, the function $p$ defines a \emph{Finsler metric} on $M$, and thus we can define a theory of geodesics on $M$. The second assumption we make is that with respect to this metric, all geodesics on $M$ are closed, and have common length.

%
%\begin{itemize}
%    \item Consider what seems a relatively simple case, the sphere $S^d$ equipped with the Laplacian. Then $\Lambda = \{ k(k+d-1) : k \geq 0 \}$, and for $\lambda = k(k+d-1)$, the space $\mathcal{V}_\lambda$ consists of the spherical harmonics of degree $k$, restrictions of harmonic, homogeneous, degree $k$ polynomials on $\RR^{d+1}$ to $S^d$. There is a basis for such a space defined in terms of the associated Legendre polynomials, but this is still relatively non-explicit.

%    \item On the other hand, on $\TT^d$ with it's Laplacian $-\Delta = - \partial_1^2 - \cdots - \partial_d^2$, for each $\lambda \in \Lambda$, $\mathcal{V}_\lambda$ has an orthogonal basis consisting of linear combinations of the exponentials $x \mapsto e^{2 \pi i k \cdot x}$, and thus multipliers of the Laplacian can be understood using the Fourier series.
%\end{itemize}
%

\chapter{Radial Multipliers on Euclidean Space}

Consider the elliptic operator $P = \sqrt{-\Delta}$ on $\TT^d$, where $\Delta = \partial_1^2 + \cdots + \partial_d^2$ is the usual Laplacian. In such a setting, we have an explicit basis for the eigenfunctions of $\Delta$: for a given eigenvalue $\lambda > 0$, the space $\mathcal{V}_\lambda$ has an orthonormal basis consisting of the exponentials $e^{2 \pi i n \cdot x}$, where $n \in \ZZ^d$ and $|n| = \lambda$. Since the Fourier series of a function gives the expansion of the function in this basis, it follows that we can expand the spectral multiplier operator $T = a(P)$ using a Fourier series, i.e. writing
%
\[ Tf(x) = \sum\nolimits_{n \in \ZZ^d} a( |n| ) \widehat{f}(n) e^{2 \pi i n \cdot x}. \]
%
Thus a multiplier of $\Delta$ on $\TT^d$ is nothing more than a Fourier multiplier operator on $\TT^d$ whose symbol is radial. Methods of transference\footnote{See Section 3.6.2 of Grafakos \cite{Grafakos}, based on methods of de Leeuw \cite{deLeeuw}} show that $\| a \|_{M^p(\RR^d)} \sim \| a \|_{M^p_{\text{Dil}}(\TT^d)}$, where $\| a \|_{M^p(\RR^d)}$ is the operator norm of the Fourier multiplier on $\RR^d$ given by
%
\[ Tf(x) = \int_{\RR^d} a \big(|\xi| \big) \widehat{f}(\xi) e^{2 \pi i \xi \cdot x}\; d\xi \]
%
on $L^p(\RR^d)$. In this section we will focus on operators of this type, which we call \emph{radial Fourier multiplier operators}.

The study of the regularity of Fourier multiplier operators has proved central to the development of modern harmonic analysis and the theory of linear partial differential operators. This is because essentially any translation invariant operator $T$ on $\RR^d$ is a Fourier multiplier operator, i.e. we can find a tempered distribution $m$ on $\RR^d$, the \emph{symbol} of $T$, such that for any Schwartz function $f$,
%
\[ Tf(x) = \int_{\RR^d} m(\xi) \widehat{f}(\xi) e^{2 \pi i \xi \cdot x}\; d\xi. \]
%
%Applying the notation of spectral calculus, one might also write this operator as $m(D)$, where $D = (2 \pi i)^{-1} \nabla$ is a self-adjoint normalization of the gradient operator. Thus the study of the boundedness of translation invariant operators is closely connected to the study of the interactions of the operators
%
%\[ E_\xi f(x) = \widehat{f}(\xi) e^{2 \pi i \xi \cdot x}, \]
%
%which act as projections onto the common eigenspaces of the components of $D$, since we can write $m(D)$ as a vector-valued integral of the form
%
%\[ m(D) = \int_{\RR^d} m(\xi) E_\xi\; d\xi. \]
%
%Thus $m(D)$ is represented as a weighted average of the operators $\{ E_\xi \}$.
The study of translation invariant operators emerges from classical questions in analysis, such as the convergence of Fourier series, and problems in mathematical physics related to the study of the heat, wave, and Schr\"{o}dinger equations. These physical equations also often have \emph{rotational} symmetry, so it is natural to restrict our attention to translation-invariant operators which are also rotation-invariant. These operators are precisely the family of radial multiplier operators. If $m(\xi) = a(|\xi|)$ for a tempered distribution on $[0,\infty)$ we will write $T = a(P)$, where $P = \sqrt{-\Delta}$, since $Tf = a(\lambda) f$ whenever $\Delta f = - \lambda^2 f$.
%Such operators are precisely those operators associated with \emph{radial} symbols $m: \RR^d \to \CC$, i.e. symbols for which there exists a function $h: [0,\infty) \to \RR$ such that
%
%\[ m(\xi) = h( |\xi|) \]
%
%for some function $h: [0,\infty) \to \CC$. This is the class of \emph{radial Fourier multipliers}. The study of radial multipliers is closely connected to interactions between the operators
%
%\[ E_\lambda f(x) = \int_{|\xi| = \lambda} \widehat{f}(\xi) e^{2 \pi i \xi \cdot x}\; d\xi, \]
%
%for $0 < \lambda < \infty$, which are the projection operators onto the eigenspaces of $\Delta$.  Similar to the study of $m(D)$, we then have
%
%\[ h \left( P \right) = \int_0^\infty h(\lambda) E_\lambda\; d\lambda. \]
%
%Thus studying the regularity of radial Fourier multipliers allows us to understand the interactions between the operators $\{ E_\lambda \}$.

\section{Convolution Kernels of Fourier Multipliers}

It is often useful to study spatial representations of these operators, since one can often exploit certain geometric information about the behaviour of operators. Given any translation invariant operator $T$ on $\RR^d$, we can associate a tempered distribution $k$, the \emph{convolution kernel} of $T$, such that
%
\[ Tf(y) = \int_{\RR^d} k(x) f(y-x)\; dx \quad\quad\text{for any Schwartz $f \in \mathcal{S}(\RR^d)$}. \]
%
If $T$ is radial, then so is $k$, and so we can write $k(x) = b(|x|)$ for some distribution $b$ on $[0,\infty)$, and then we have a representation
%
\[ T = \int_0^\infty b(r) S_r\; dr,\quad\text{where}\quad S_rf(x) = \int_{|y| = r} f(x + y)\; dy \]
%
are the \emph{spherical averaging operators}.

With the notation as above, the function $k$ is the Fourier transform of $m$, and the function $b$ is a Bessel transform of $a$, i.e. $b = \mathcal{B}_d a$, where
%
\[ \mathcal{B}_d a(r) = \int_0^\infty s^{\frac{d-1}{2}} J_{\frac{d-1}{2}}(2 \pi s) a(s)\; ds, \]
% b(r) = int_0^infty s^{d-1} a(s) int_{S^{d-1}} e^{2 pi i s xi * x}
%      = V_{d-2} int_0^infty s^{d-1} a(s) int_{-1}^1 (1-s^2)^{d/2-1} e^{2piist}
%      = 2 V_{d-2} int_0^infty s^{d-1} a(s) int_0^1 (1-s^2)^{d/2 - 1} cos(2 pi s t)
%      = V_{d-2} Gamma(nu + 1/2) pi^{1 - d/2} int_0^infty s^{(d-1)/2} a(s) J_{(d-1)/2}(2 pi s)
%      = int_0^infty
%
% nu = (d-1)/2
% z = 2 pi s
% J_nu(z) Gamma(nu + 1/2) pi^{1/2 - nu} / 2 s^nu
%
and where
%
\[ J_\alpha(\lambda) = \frac{(\lambda / 2)^\alpha}{\Gamma(\alpha + 1/2)} \int_{-1}^1 e^{i \lambda s} (1 - s^2)^{\alpha - 1/2}\; ds. \]
%
Using the theory of stationary phase, for each $d$ we can write
%
\[ J_d(\lambda) = e^{2 \pi i \lambda} s_1(\lambda) + e^{-2 \pi i \lambda} s_2(\lambda). \]
%
for symbols $s_1$ and $s_2$ of order $-1/2$. The presence of $e^{2 \pi i \lambda}$ and $e^{-2 \pi i \lambda}$ allows one to relate the Bessel transform of a function to it's Fourier transform, to a certain extent. In particular, we record the following result of Garrigos and Seeger \cite{GarrigosandSeeger}.

\begin{theorem} \label{GarrigosSeegerTheorem}
    Suppose $d > 1$ and $1 < p < 2d/(d+1)$, and suppose $a: [0,\infty) \to \CC$ has compact support away from the origin. Then, with implicit constants depending on the support of $a$,
    %
    \[ \left( \int_0^\infty |\mathcal{B}_d a (t)|^p t^{d-1} \right)^{1/p} \sim_{p,d,\phi} \left( \int_0^\infty |\;\!\widehat{a}(t)|^p \langle t \rangle^{(d-1)(1 - p/2)}\; dt \right)^{1/p}, \]
    %
    where $\widehat{a}(t) = \int_0^\infty a(\lambda) \cos(2 \pi \lambda t)\; dt$ is the cosine transform of $a$.
\end{theorem}

%Later on, we will also need to consider analogues of this result for \emph{quasi-radial} multipliers, i.e. multipliers of the form $a \circ r$, where $r: \RR^d \to [0,\infty)$ is a homogeneous function of order one with $r(\xi) \neq 0$ for $\xi \neq 0$, and such that the cosphere $\Sigma = \{ \xi : r(\xi) = 1 \}$ has non-vanishing Gaussian curvature. We calculate using the Fourier transform that
%
%\begin{align*}
%  \int (a \circ r)(\xi) e^{2 \pi i \xi \cdot x}\; d\xi &= C \int_0^\infty a(t) \int_\Sigma e^{2 \pi i t \xi \cdot x}\; d\xi = (\mathcal{A}_r a)(x).
%\end{align*}
%
%Using stationary phase and the curvature of $\Sigma$, we can write
%
%\[ \int_\Sigma e^{2 \pi i t \xi \cdot x}\; d\xi = s_1(x,t) e^{2 \pi i t \xi_+(x) \cdot x} + s_2(x,t) e^{2 \pi i t \xi_-(x) \cdot x} \]
%
%and thus $(\mathcal{A}_r a)(x)$ is connected to the Fourier transform of $a$ near $\xi_+(x) \cdot x$ and $\xi_-(x) \cdot x$.

\section{The Radial Multiplier Conjecture}

The general study of the boundedness properties of Fourier multiplier operators in multiple variables was initiated in the 1950s, as connections of the theory to partial differential equations became more fully realized\footnote{See \cite{Hormander1} for a more detailed overview of what was known at this time.}. It was quickly realized that the most fundamental estimates were $L^p \to L^q$ estimates for such operators.
%
%for $1 \leq p \leq 2$, and $q \geq p$, which by duality are equivalent to bounds
%
%\[ \| Tf \|_{L^{p^*}(\RR^d)} \lesssim \| f \|_{L^{q^*}(\RR^d)}, \]
%
It is therefore natural to introduce the space $M^p(\RR^d)$, consisting of all symbols $m$ which induce a Fourier multiplier operator $T$ bounded on $L^p(\RR^d)$. Duality implies that $M^p(\RR^d)$ is isometric to $M^{p'}(\RR^d)$, where $p$ and $p'$ are conjugates, so it suffices to study the spaces $M^p(\RR^d)$ where $1 \leq p \leq 2$ or when $2 \leq p \leq \infty$. We only know simple characterizations of $M^p(\RR^d)$ for very particular $p$:
%
\begin{itemize}
    \item The spaces $M^1(\RR^d) = M^\infty(\RR^d)$ can be characterized, by virtue of the fact that the boundedness of operators with domain $L^1(\RR^d)$ or range $L^\infty(\RR^d)$ is often simple; we have
    %
    \[ M^1(\RR^d) = \widehat{M}(\RR^d), \]
    %
    Here $M(\RR^d)$ is the space of all finite signed Borel measures, equipped with the total variation norm. The proof follows from Schur's Lemma for integral operators, which often gives tight estimates to bound operators with domain $L^1$ or range $L^\infty$.

    \item The unitary nature of the Fourier transform also allows for the characterization $M^2(\RR^d) = L^\infty(\RR^d)$. The proof follows from Parseval's identity.
\end{itemize}
%
It is perhaps surprising that these are the \emph{only} known characterizations of the spaces $M^p(\RR^d)$. No necessary and simple conditions for boundedness are known for any other values of $p$, and perhaps no simple characterization exists.

Despite the lack of a characterization of the classes $M^p(\RR^d)$, it is surprising that we \emph{can} conjecture a characterization of the subspace of $M^p(\RR^d)$ consisting of \emph{radial symbols}, for an appropriate range of exponents. The conjectured range of estimates was first suggested by a result of \cite{GarrigosandSeeger}, concerning the boundedness of a radial Fourier multiplier $T$ with symbol $a(|\cdot|)$ \emph{restricted to radial functions}, i.e. such that the norm
%
\[ \| a \|_{M^p_{\text{Rad}}(\RR^d)} = \sup \left\{ \frac{\| a(P)f \|_{L^q(\RR^d)}}{\| f \|_{L^p(\RR^d)}} : \text{$f$ is radial} \right\} \]
%
is finite. For any $\chi \in C_c^\infty(\RR)$, if we define $k_R$ to be the Fourier transform of $\chi(|\cdot|) a(R |\cdot|)$, then the identity $k_R = a(RP) \{\;\! \widehat{\chi}\;\! \}$ and dilation symmetry imply that
%
\begin{equation}
  \| k_R \|_{L^p(\RR^d)} \lesssim \| a \|_{M^p_{\text{Rad}}(\RR^d)},
\end{equation}
%
with implicit constants depending on $\chi$. For $d > 1$, and for $1/p - 1/2 > 1/2d$, Garrigos and Seeger proved \cite{GarrigosandSeeger} the converse bound
%
\begin{equation} \label{RadialMultiplierRadialBound}
    \| a \|_{M^p_{\text{Rad}}(\RR^d)} \lesssim \sup\nolimits_{R > 0} \| k_R \|_{L^p(\RR^d)},
\end{equation}
%
Theorem \ref{GarrigosSeegerTheorem} implies that
%
\begin{equation}
  \sup\nolimits_{R > 0} \| k_R \|_{L^p(\RR^d)} \sim \| a \|_{R^{s,p}[0,\infty)},
\end{equation}
%
where $s = (d-1)(1/p - 1/2)$. Thus Garrigos and Seeger have proved that the isomorphism $M^p_{\text{Rad}}(\RR^d) = R^{s,p}[0,\infty)$ holds in the range above.

%Lower bounding the left hand side of \eqref{RadialMultiplierRadialBound} by the right hand side follows from rescaling the convolution identity $k_j * \psi = k_j$, where $\psi$ is a radial Schwartz function whose Fourier transform is equal to one on the Fourier support of $\chi$. The hard part of the result of \cite{GarrigosandSeeger} is upper bounding the left hand side by the right.

For any other value of $p$, $M^p_{\text{Rad}}(\RR^d)$ is a proper subset of $R^{s,p}[0,\infty)$, as the following counterexamples show:
%
\begin{itemize}
  \item For $p = 1$, the inclusion `fails by a logarithm', as we can see by the characterization of $M^(\RR^d)$ in the last section; if $a$ is supported on an interval $I$ we have the converse inequality $\| a \|_{M^1(\RR^d)} \lesssim \log |I|\; \| a \|_{R^{s,p}[0,\infty)}$ where dependence on $I$ is in general sharp, and cannot be removed.

  \item Suppose $p \geq 2d/(d+1)$. Then $R^{s,p}[0,\infty)$ contains all elements of the Besov space $B^{{\tiny 1/2},2}_p[0,\infty)$ supported on $[1/2,2]$, and thus, in particular, must contain unbounded functions for $p > 1$ (because the Sobolev embedding theorem fails when used at the endpoint to embed into $L^\infty$). Since $M^p(\RR^d) \subset M^2(\RR^d) = L^\infty[0,\infty)$, $M^p(\RR^d)$ cannot contain unbounded functions, and thus $M^p(\RR^d)$ is a proper subset of $R^{s,p}(\RR^d)$.

%  But any function in $M^{p,q}_{\text{Rad}}(\RR^d)$ supported on $[1/2,2]$ would have to be in $M^{2d/(d+1),2}_{\text{Rad}}(\RR^d)$ by Bernstein's inequality,

  % Find a so that it's Fourier transform decays on the order of
  % |t|^{-(d-1)/2d}, and then define a_N = a * chi_{1/N}. Then
  %
  % the R^{s,q} norm should be (log N)^{1/q}
  %
  % |t|^{-}
  % (d-1)/2d = 1 - alpha
  % alpha = 1 - (d-1)/2d = (2d - d + 1) / 2d = (d + 1)/2d
  % So take a = |t|^{-(d+1)/2d}
  %
  % On the other hand, if f = I_{[1,N]} then it's L^p(R^d) norm
  % is O(N^{1/p}) but the L^2 norm of a(P)f is >> log N) N^{-1}



  % a_N = I_{[1,1+1/N]} * chi_{1/N}
  % Likely has B^{0.5,2}_p norm O( N^{1/2} )
  %
  % Fourier transform of a_N is [sin((1 + 1/N) t) - sin(t)] / t
  % int_1^infty t^{(s-1)q}

  % So if we cut off the Fourier transform at a frequency 2^j for j > 0
  % then the function is smooth on [2^j, 2^{j+1}] with amplitude 2^{-j}
  % and so the Fourier transform should be concentrated near [0,2^{-j}] with
  % amplitude 1, and thus have L^2 norm 2^{-j/2}. But for half derivatives
  % we need to multiply by 2^{j/2}. So the B^{0.5,2}_infty norm is finite
  % But the other ones are infinite.

  % Which is smooth and slowly oscillating away from the origin.
  % If we cut off the Fourier transform at a frequency 2^j for j > 0, then the
  % function has amplitude 2^{-j}, so when we convert back we get a function which
  % is 

%   $a_N(\lambda) = N^{1/2} \mathbb{I}_{[1,1 + 1/N]}$. Then
   % a_N = N^{1/2} I_{[1,1+1/N]}
   % f_N = |x|^{-(d+1)/2} e^{-is} I(1 <= |x| <= N)
   %
   % Then L^p norm of f_N is O(log N)^{1/p}
   % Then a_N(P) f_N >> log N for ||x| - 1| <= 1/N
   % So the L^2 norm of a_N(P) f_N is >> log N 
   %
   % So for p > 1, the operators a_N(P) are not uniformly
   % bounded from L^p to L^2
   %
   % But they are in R^{0,2} uniformly in N

   % q(s-1) < -1
   % s < 1 - 1/q
   % 1 - 1/q = 1 - (d+1)/2d = (d - 1)/2d

   % (d-1)/2d

   % O(1) over small time scales.
   % q(s-1) < -1
   % s < -1 - 1/q


%    and thus by interpolation and duality, in $M^2_{\text{Rad}}(\RR^d)$. But $M^2_{\text{Rad}}(\RR^d)$ is isometrically equal to $L^\infty(\RR^d)$, and so any function would have to be bounded. Thus there are functions in $R^{s,q}_\alpha(\RR^d)$ which are not in $M^{p,q}_{\text{Rad}}(\RR^d)$ when $p \geq 2d/(d+1)$.

   % If bounded on M^p, then bounded on M^{p,2}
   % But boundedness from L^p to L^2
   % 1/p - 1/2 >= 1/(d+1)

   % and if bounded on M^{p,infty} = M^{1,p'}, then the Fourier transform of your function must lie in p'
     % Can only be bounded from L^p to L^q if they are also bounded from L^1 to L^q
     % L^p to L^q boundedness implies L^1 to L^q, but the opposite is surprising.


%  consider the functions $a_\delta(\lambda) = (1 - \lambda^2)^\delta_+$, whose associated operators $a_\delta(P)$ give the class of Bochner-Riesz multipliers. One may represent the cosine transform of the functions $a_\delta$ using Bessel functions, and the asymptotics of such functions tells us $|\widehat{a}_\delta(t)| \lesssim t^{-\delta - 1}$ as $t \to \infty$, so that $a_\delta \in R^{q,s}_\alpha[0,\infty)$ for $s < \delta + 1 - 1/q$, and thus for $s = (d-1)(1/q - 1/2)$ when $\delta > d/q - (d+1)/2$.
\end{itemize}
  % (d-1)(1/q - 1/2) derivatives in L^{q'}
  %
  % Is implied if we have d(1/q - 1/2) derivatives in L^2
  %
  % But 

  % 1/2 derivatives in L^p for p > 1
  % (d-1)(1/p - 1/2) derivatives in L^{p'}

% For $p = 2d/(d+1)$, consider the  In particular, if $q > 2d/(d+1)$, then $a_0 \in R^{q,s}_\alpha[0,\infty)$.

% On the other hand, the kernel $k_\delta$ of $a_\delta(P)$ can also be expressed using Bessel functions, and we have $|k_\delta(t)| \lesssim |t|^{-(d+1)/2-\delta}$, and so we find using the characterization of $M^{1,q}(\RR^d)$ that $a_\delta \in M^{1,q}(\RR^d)$
% (-delta - 1 + s)q < -1
% -delta - 1 + s < -1/q
% s < delta + 1 - 1/q
% (d-1)(1/q - 1/2) < delta + 1 -1/q
% d/q - (d-1)/2 < delta + 1
% delta > d/q - - (d+1)/2


%
% J_{delta + 1/2}(z) = (z/2)^{delta + 1/2} / pi^{1/2} Gamma(delta + 1) int_{-1}^1 (1 - t^2)^delta e^{izt} dt
% So the Fourier transform of (1 - lambda^2)^delta is
%     J_{delta + 1/2}( 2 pi t ) t^{-delta - 1/2} pi^{-delta} Gamma(delta + 1)
%     Which is O( <t>^{-1/2} t^{-delta - 1/2} ) for t -> infinity
% (d-1)(1/q - 1/2) > -delta - 1/q

%
% int_0^infty [ <t>^delta <t>^s ]^q dt
% int_0^infty <t>^{(delta + s) q} dt
%
%     (delta + s) q < -1
%   delta + s < -1/q
%
% Integrability near 0: q (-d/2 - delta) >= -1
%           delta <= 1/q - d/2
% Integrability near infinity:   q (-d/2 - delta + s) > -1
%           delta < 1/q - d/2 + s

%and $q > 1$, we have
%
%\[ \| a \|_{M^{1,q}(\RR^d)} \gtrsim \| a \|_{R^{q,s}_\alpha[0,\infty)} \]
%
%where $s = \tfrac{d-1}{2}$ and $\alpha = 1 - 1/q$

%The result cannot hold for $p \geq 2d/(d+1)$, because, adapting the analysis of Fefferman \cite{Fefferman} to a family of multiplier operators whose symbols are smooth and adapted to a $\delta$ neighborhood of an annulus as $\delta \to 0$, such a result would imply all Kakeya sets have positive measure TODO I DON'T THINK THIS WORKS FOR RADIAL INPUTS.

%
% T chi = k gives the L^{q'} -> L^{q'} extreme values
%
% < T chi , k^{q' - 1} > = |k|_{L^{q'}}^{q'}
% < chi, T k^{q' - 1} >
%
% < T chi , g > = |Tf|_{L^q}
%
% If |m| ~ |k|_{L^{q'}}
%
% Then we get a counterexample if |k|_{L^q} is small but |k|_{L^{q'}} is big.
% That's legitimate because of Bernstein's inequality.
%
% | Tf |_{L^q} = |k|_{L^{q'}} |f|_{L^q} ~ |k|_{L^{q'}} |f|_{L^p}
% Thus |m| << |k|_{L^{q'}} rather than 
%
%
% To be bounded on L^q, need qth root cancellation
% But only have square root cancellation 

%hold. On the other hand, by Plancherel, we can find a radial function $f$ with compact Fourier support such that $\| \kappa_t * f \|_{L^2(\RR^d)} \gtrsim 1$ uniformly as $t \to \infty$ (consider such $f$ with Fourier support concentrated near the maximum value of the Fourier transform of $\kappa_0$). By Bernstein's inequality,
%
%\[ \| m_t \|_{M^{p,q}(\RR^d)} \| \kappa_t * f \|_{L^p(\RR^d)} \gtrsim 1 \]

%for each $t$ we can find a radial function $f_t$ with Fourier support near $1$, with $\| f_t \|_{L^2(\RR^d)} = 1$ and with $\| \kappa_t * f_t \|_{L^2(\RR^d)} \gtrsim 1$. By Sobolev embedding we must also have $\| \kappa_t * f_t \|_{L^p(\RR^d)} \gtrsim 1$ and $\| f_t \|_{L^p(\RR^d)} \lesssim 1$.

%And one can see that the condition is not sufficient for $q > \min \left(2, \tfrac{d-1}{d+1} p' \right)$ by testing the same multiplier operators above on `Knapp examples', i.e. smooth functions whose Fourier transforms are adapted to a $\delta$ cap on a $\delta$-neighborhood of the annulus upon which $m$ is supported.

It is natural to conjecture that the same constraint continues to hold when we remove the constraint that our inputs $f$ are radial, i.e so that for a radial function $m(\xi) = a(|\xi|)$, for $d > 1$, $1/2d < 1/p - 1/2 < 1/2$, and for $s = (d-1)(1/p - 1/2)$,
%
\begin{equation} \label{RadialMultiplierBound}
  \| m \|_{M^p(\RR^d)} \sim_{p,d} \| a \|_{R^{p,s}[0,\infty)}.
\end{equation}
%
%and for general locally integrable symbols $m$,
%
%\[ \| m \|_{M^{p,q}} \sim_{p,q,d} \| k \|_{\dot{B}_{-d/p^*}^{q,\infty}} \]
%
%However, for $1/2d < 1/p - 1/2 < 1/(d+1)$ an additional counterexample is obtained by considering Knapp type examples, i.e. considering the multiplier operators whose symbols are supported on $\delta$-annuli as above, and testing them against smooth, \emph{non-radial} functions whose Fourier transforms are adapted to $\delta$-caps on the annuli. These counterexamples show that the condition $q < (d-1)/(d+1) \cdot p'$ is necessary for \eqref{RadialMultiplierBound} to hold. We thus conjecture that \eqref{RadialMultiplierBound} holds for all radial $m$ when either $1/p - 1/2 > 1/(d+1)$ and $p \leq q < 2$, or when $1/2d < 1/p - 1/2 < 1/(d+1)$ and $p \leq q < (d-1)/(d+1) \cdot p'$, and we call this the \emph{radial multiplier conjecture} on $\RR^d$.
We call this the \emph{radial multiplier conjecture} on $\RR^d$.

We now know, by results of Heo, Nazarov, and Seeger \cite{HeoandNazarovandSeeger} that the radial multiplier conjecture is true when $d \geq 4$ and when $1/p - 1/2 > 1/(d+1)$. A summary of the proof strategies of this argument is provided in the following section, and a major part of our bounds for spectral multipliers follow by adapting this argument to the non-Euclidean setting. Partial improvements were obtained by Cladek \cite{Cladek} for symbols $m$ compactly supported away from the origin, obtaining results for compactly supported multipliers when $d = 4$ and $1/p - 1/2 > 11/36$, and establishing \emph{restricted weak type} bounds when $d = 3$ and $1 < p < 11/26$. %Cladek's argument is described in Section \ref{Cladek}.
But the radial multiplier conjecture has not been resolved fully in any dimension $d$, we do not have any strong type $L^p$ bounds when $d = 3$, and no bounds whatsoever are known when $d = 2$.

\section{Radial-Multiplier Bounds by Density Decompositions}

In this section and the following, we give an overview of the proof of the radial multiplier bounds obtained by Heo, Nazarov, and Seeger in \cite{HeoandNazarovandSeeger}. The main tool we will take away for applications to the proof of Theorem \ref{maintheorem} is the method of \emph{density decompositions}.

\begin{theorem} \label{HeoNazarovSeegerTheorem}
    Suppose $1 < p < 2(d-1)/(d+1)$, and $m(\xi) = a(|\xi|)$ is a radial function. Then $\| m \|_{M^p(\RR^d)} \lesssim \| a \|_{R^{p,s}[0,\infty)}$, where $s = (d-1)(1/p - 1/2)$.
\end{theorem}

In this section, we establish a single scale version of this result, as described in the following lemma.

\begin{lemma} \label{HeoNazarovSeegerSingleScaleInequality}
  Suppose $1 < p < 2(d-1)/(d+1)$, and $k$ is a radial function with Fourier transform supported on $1/2 \leq |\xi| \leq 2$. Then
%
\begin{equation}
    \| k * f \|_{L^p(\RR^d)} \lesssim \| k \|_{L^p(\RR^d)} \| f \|_{L^p(\RR^d)}
\end{equation}
%
\end{lemma}

We reduce Lemma \eqref{HeoNazarovSeegerSingleScaleInequality} to an inequality for sums of functions oscillating on spheres. Let $\sigma_r$ be the surface measure for the sphere of radius $r$ centered at the origin in $\RR^d$. Also fix a nonzero, radial, compactly supported function $\psi \in \mathcal{S}(\RR^d)$ whose Fourier transform is non-negative, and vanishes to high order at the origin. Given $x \in \RR^d$ and $r \geq 1$, define $\chi_{x, r} = \text{Trans}_x (\sigma_r * \psi)$. Then $\chi_{x,r}$ is a smooth function adapted to a thickness $O(1)$ annulus of radius $r$ centered at $x$, which is \emph{slightly oscillating}. We will verify the following lemma.

\begin{lemma} \label{lemma1}
    For any $a : \RR^d \times [1,\infty) \to \CC$, and if $1/p - 1/2 > 1/(d+1)$,
    %
    \[ \left\| \int_{\RR^d} \int_1^\infty a(x,r) \chi_{x, r}\; dx\; dr \right\|_{L^p(\RR^d)} \lesssim \left( \int_{\RR^d} \int_1^\infty |a(x,r)|^p r^{d-1} dr dx \right)^{1/p}. \]
    %
    The implicit constant here depends on $p$, $d$, and $\psi$.
\end{lemma}

\begin{proof}[Proof of Inequality \eqref{HeoNazarovSeegerSingleScaleInequality} from Lemma \ref{lemma1}] Suppose $k(\cdot) = b(|\cdot|)$ for some function $b: [0,\infty) \to \CC$. If we set $a(x,r) = f(x) b(r)$ for any function $f: \RR^d \to \CC$, then
%
\begin{equation}
  k * \psi * f = \int_{\RR^d} \int_1^\infty a(x,r) \chi_{x, r}\; dx\; dr,
\end{equation}
%
Lemma \ref{lemma1} thus says that
%
\begin{equation}
  \| k * \psi * f \|_{L^p(\RR^d)} \lesssim \| k \|_{L^p(\RR^d)} \| f \|_{L^p(\RR^d)}.
\end{equation}
%
If we choose $\psi$ so that $\widehat{\psi}$ is non-vanishing on the support of $k$, then the function $1/\widehat{\psi}(\cdot)$ is smooth on the support of $m$; if $T$ is a Fourier multiplier operator with a smooth, compactly supported symbol agreeing with $1/\widehat{\psi}(\cdot)$ on the support of $m$, then $T$ is bounded on $L^p(\RR^d)$, and $T(k * \psi * f) = k * f$, and so we conclude that
%
\begin{equation}
  \| k * \psi \|_{L^p(\RR^d)} = \| T(k * \psi * f) \|_{L^p(\RR^d)} \lesssim \| k * \psi * f \|_{L^p(\RR^d)} \lesssim \| k \|_{L^p(\RR^d)} \| f \|_{L^p(\RR^d)},
\end{equation}
%
which completes the argument.
\end{proof}

Next, we consider a discretization of Lemma \ref{lemma1}.

\begin{theorem} \label{lemma2}
    Fix a 1-separated set $\mathcal{E} \subset \RR^d \times [1,\infty)$. Then for any $a: \mathcal{E} \to \CC$ and if $1/p - 1/2 > 1/(d+1)$,
    %
    \[ \left\| \sum\nolimits_{(x,r) \in \mathcal{E}} a(x,r) \chi_{x, r} \right\|_{L^p(\RR^d)} \lesssim \left( \sum\nolimits_{(x,r) \in \mathcal{E}} |a(x,r)|^p r^{d-1} \right)^{1/p}, \]
    %
    where the implicit constant is independent of $\mathcal{E}$.
\end{theorem}

\begin{proof}[Proof of Lemma \ref{lemma1} from Lemma \ref{lemma2}]
    For any $a: \RR^d \times [1,\infty) \to \CC$, if we consider the vector-valued function $\mathbf{a}(x,r) = a(x,r) \chi_{x,r}$, then
    %
    \begin{equation}
      \int_{\RR^d} \int_1^\infty \mathbf{a}(x,r)\; dr\; dx = \int_{[0,1)^d} \int_0^1 \sum\nolimits_{n \in \ZZ^d} \sum\nolimits_{m > 0} \text{Trans}_{n,m} \mathbf{a}(x,r)\; dr\; dx
    \end{equation}
    %
    The triangle inequality and the increasing property of norms on $[0,1)^d \times [0,1]$ imply that
    %
    \begin{equation}
    \begin{split}
    &\left\| \int_{\RR^d} \int_1^\infty \mathbf{a}(x,r)\; dr\; dx \right\|_{L^p(\RR^d)}\\
    &\quad \leq \int_{[0,1)^d} \int_0^1 \left\| \sum\nolimits_{n \in \ZZ^d} \sum\nolimits_{m > 0} \text{Trans}_{n,m} \mathbf{a}(x,r) \right\|_{L^p(\RR^d)}\; dr\; dx\\
    &\quad \lesssim \int_{[0,1)^d} \int_0^1 \left( \sum\nolimits_{n \in \ZZ^d} \sum\nolimits_{m > 0} |a(x - n, r + m)|^p r^{d-1} \right)^{1/p}\; dr\; dx\\
    &\quad \leq \left( \int_{[0,1)^d} \int_0^1 \sum\nolimits_{n \in \ZZ^d} \sum\nolimits_{m > 0} |a(x - n, r + m)|^p r^{d-1}\; dr\; dx \right)^{1/p}\\
    &\quad = \left( \int_{\RR^d} \int_1^\infty |a(x,r)|^p r^{d-1} dr dx \right)^{1/p},
    \end{split}
    \end{equation}
    %
    which completes the proof.
\end{proof}

Lemma \ref{lemma2} can be further reduced by considering it as a bound on the operator
%
\[ a \mapsto \sum\nolimits_{(x,r) \in \mathcal{E}} a(x,r) \chi_{x,r}. \]
%
In particular, applying a real interpolation, since we are proving a result for an open range of $p$, it suffices for us to prove a restricted strong type bound. Given any discretized set $\mathcal{E}$, let $\mathcal{E}_k$ be the set of $(x,r) \in \mathcal{E}$ with $2^k \leq r < 2^{k+1}$. Then Lemma \ref{lemma2} is implied by the following Lemma.

\begin{lemma} \label{lemma3}
    For $1/p - 1/2 > 1/(d+1)$ and $k \geq 1$,
    %
    \[ \left\| \sum\nolimits_{(x,r) \in \mathcal{E}} \chi_{x,r} \right\|_{L^p(\RR^d)} \lesssim \left( \sum\nolimits_{k \geq 1} 2^{k(d-1)} \#(\mathcal{E}_k) \right)^{1/p}. \]
\end{lemma}

\begin{remark}
    Note that if $r \sim 2^k$, then $\| \chi_{x,r} \|_{L^p(\RR^d)} \sim 2^{k(d-1)/p}$, and so Lemma \ref{lemma3} says
    %
    \begin{equation}
      \left\| \sum\nolimits_{(x,r) \in \mathcal{E}} \chi_{x,r} \right\|_{L^p(\RR^d)} \lesssim_p \left( \sum\nolimits_{(x,r) \in \mathcal{E}} \| \chi_{x,r} \|_{L^p(\RR^d)}^p \right)^{1/p}.
    \end{equation}
    %
    Thus we are proving a $p$th root cancellation bound.
\end{remark}

\begin{comment}
\begin{proof}[Proof of Lemma \ref{lemma2} from Lemma \ref{lemma3}]
    Let
    %
    \[ F = \sum\nolimits_{(x,r) \in \mathcal{E}} \chi_{x,r} \]
    %
    and then for $k \geq 1$, let
    %
    \[ F_k = \sum\nolimits_{(x,r) \in \mathcal{E}_k} \chi_{x,r}. \]
    %
    Then $F = \sum\nolimits_k F_k$, and. Applying a dyadic interpolation result (Lemma 2.2 of that paper), the bound
    %
    \[ \| F_k \|_{L^r(\RR^d)} \lesssim 2^k (2^{k(d-r-1)} \#(\mathcal{E}_k)^{1/r}) \]
    %
    which holds for $r$ to the left and right of $p$, can be interpolated to yield that
    %
    \[ \| F \|_{L^p(\RR^d)} \lesssim \left( \sum\nolimits_k 2^{kp} ( 2^{k(d-r-1)} ) \right)^{1/p} \]


    Applying a dyadic interpolation result (Lemma 2.2 of the paper), Lemma \ref{lemma3} implies that
    %
    \[ \left\| \sum\nolimits_{(x,r) \in \mathcal{E}} \chi_{x,r} \right\| \]

    %
    \[ \left\| \sum\nolimits_{(x,r) \in \mathcal{E}} \chi_{x,r} \right\|_{L^p(\RR^d)} \lesssim \left( \sum 2^{kp} 2^{k(d-p-1)} \#(\mathcal{E}_k) \right)^{1/p} = \left( \sum 2^{k(d-1)} \#(\mathcal{E}_k) \right)^{1/p} \]
    %
    This is a restricted strong type bound for Lemma \ref{lemma2}, which we can then interpolate.
\end{proof}
\end{comment}

To control these sums, we apply a `density decomposition', somewhat analogous to a Calderon Zygmund decomposition, which will enable us to obtain $L^2$ bounds. We say a 1-separated set $\mathcal{E}$ in a metric space $X$ is of \emph{density type} $(u,r)$ if $\#(B \cap \mathcal{E}) \leq u \cdot \diam(B)$ for each ball $B$ in $X$ with diameter at most $r$.

%A covering argument then shows that for any ball $B$,
%
%\[ \#(B \cap \mathcal{E}) \lesssim_d u \cdot \left( 1 + \frac{\diam(B)}{R} \right)^d \cdot \diam(B). \]
%
%(NOTE: WE MIGHT BE ABLE TO DO BETTER USING THE FACT THAT $\mathcal{E} \subset \RR^d \times [R,2R)$, USING THE VALUE $R$).

\begin{theorem} \label{DecompositionTheorem}
    For any family of 1-separated sets $\mathcal{E}_k \subset \RR^d \times [2^k,2^{k+1})$, there exists a decomposition $\mathcal{E}_k = \bigcup_{m = 1}^\infty \mathcal{E}_k(2^m)$ with the following properties:
    %
    \begin{itemize}
        \item For each $m$, $\mathcal{E}_k(2^m)$ has density type $(2^m,2^k)$.

        \item If $B$ is a ball in $\RR^{d+1}$ of radius $r \leq 2^k$ containing at least $2^m \cdot r$ points of $\mathcal{E}_k$, then
        %
        \[ B \cap \mathcal{E}_k \subset \bigcup\nolimits_{m' \geq m} \mathcal{E}_k(2^{m'}). \]

        \item For each $m$, there are disjoint balls $\{ B_i \}$ in $\RR^{d+1}$ with radii $\{ r_i \}$, such that
        %
        \[ \sum\nolimits_i r_i \leq 2^{-m} \# \mathcal{E}_k, \]
        %
        such that $r_i \leq 2^k$ for all $i$, and such that $\bigcup B_i^*$ covers $\bigcup_{m' \geq m} \mathcal{E}_k(2^{m'})$, where $B_i^*$ denotes the ball with the same center as $B_i$ but 5 times the radius.
    \end{itemize}
\end{theorem}
\begin{proof}
    Define a function $M: \mathcal{E}_k \to [0,\infty)$ by setting
    %
    \[ M(x,r) = \sup \left\{ \frac{\#(\mathcal{E}_k \cap B)}{\text{rad}(B)} : (x,r) \in B\ \text{and}\ \text{rad}(B) \leq 2^k \right\}. \]
    %
    We can establish a kind of weak $L^1$ estimate for $M$ using a Vitali type argument. Let
    %
    \[ \widehat{\mathcal{E}}_k(2^m) = \{ (x,r) \in \mathcal{E}_k : M(x,r) \geq 2^m \}. \]
    %
    We can therefore cover $\widehat{\mathcal{E}}_k(2^m)$ by a family of balls $\{ B \}$ such that $\#(\mathcal{E}_k \cap B) \geq 2^m \text{rad}(B)$. The Vitali covering lemma allows us to find a disjoint subcollection of balls $B_1,\dots,B_N$ such that $B_1^* ,\dots, B_N^*$ covers $\widehat{\mathcal{E}}_k(2^m)$. We find that
    %
    \begin{equation}
      \#(\mathcal{E}_k) \geq \sum\nolimits_i \#(B_i \cap \mathcal{E}_k) \geq 2^m \sum\nolimits_i \text{rad}(B_i),
    \end{equation}
    %
    Setting $\mathcal{E}_k = \widehat{\mathcal{E}}_k(2^m) - \bigcup_{k' > k} \widehat{\mathcal{E}}_{k'}(2^m)$ thus gives the required result.
\end{proof}

\begin{remark}
  We will apply the Lemma with $\RR^d \times [0,\infty)$ replaced by $X \times [0,\infty)$, where $X$ is a compact, $d$-dimensional manifold equipped with a Finsler metric. A version of the Vitali covering lemma also holds for this metric space, so that the same proof allows one to perform a density decomposition in this setting.
\end{remark}

To prove Lemma \ref{lemma3}, we perform a decomposition of $\mathcal{E}_k$ for each $k$, into the sets $\mathcal{E}_k(2^m)$, and then define $\mathcal{E}^m = \bigcup_{k \geq 1} \mathcal{E}_k^m$. For appropriate exponents, we will prove $L^p$ bounds on the functions
%
\[ F^m = \sum\nolimits_{(x,r) \in \mathcal{E}^m} \chi_{x,r} \]
%
which are exponentially decaying in $m$, i.e. that
%
\[ \| F^m \|_{L^p(\RR^d)} \lessapprox 2^{m \big( \frac{1}{d+1} - (1/p - 1/2) \big)} \left( \sum\nolimits_k 2^{k(d-1)} \#(\mathcal{E}_k) \right)^{1/p}, \]
%
where the implicit constant in the inequality can depend polynomially on $m$. Thus, in the range $1/p - 1/2 > 1/(d+1)$, there is geometric decay in $m$, decaying much quicker than the polynomial implicit constant, and so we may sum in $m$ using the triangle inequality to conclude that
%
\[ \| F \|_{L^p(\RR^d)} \lesssim \left( \sum\nolimits_k 2^{k(d-1)} \#(\mathcal{E}_k) \right)^{1/p}, \]
%
proving Lemma \ref{lemma3}. To get the bound on $F^m$, we interpolate between an $L^2$ bound for $F^m$, and an $L^0$ bound (i.e. a bound on the measure of the support of $F^m$). First, we calculate the support of $F^m$.

\begin{lemma} \label{lemma5}
    For each $k$,
    %
    \[ |\text{supp}(F^m_k)| \lesssim 2^{-m} 2^{k(d-1)} \# \mathcal{E}_k. \]
    %
    Thus we have
    %
    \[ |\text{supp}(F^m)| \leq \sum\nolimits_k |\text{supp}(F^m_k)| \lesssim \sum\nolimits_k 2^{-m} 2^{k(d-1)} \# \mathcal{E}_k. \]
\end{lemma}
\begin{proof}
    We recall that for each $k$ and $m$, we can find disjoint balls $B_1,\dots,B_N$ with radii $r_1,\dots,r_N \leq 2^k$ such that
    %
    \[ \sum\nolimits_{i = 1}^N r_i \leq 2^{-m} \# \mathcal{E}_k, \]
    %
    where $\mathcal{E}_k(2^m)$ is covered by the expanded balls $B_1^* \cup \dots \cup B_N^*$. If we write
    %
    \[ F^m_{k,i} = \sum\nolimits_{(x,r) \in \mathcal{E}_k(2^m) \cap B_i^*} \chi_{x,r}, \]
    %
    then $\text{supp}(F^m_k) \subset \bigcup_i \text{supp}(F^m_{k,i})$. For each $(x,r) \in B_i^* \cap \mathcal{E}_k(2^m)$, the support of $\chi_{x,r}$, an annulus of thickness $O(1)$ and radius $r$, is contained in an annulus of thickness $O(r_i)$ and radius $O(2^k)$ with the same centre as $B_i$. Thus we conclude that
    %
    \[ |\text{supp}(F^m_{k,i})| \lesssim r_i 2^{k(d-1)}, \]
    %
    and it follows that
    %
    \[ |\text{supp}(F^m_k)| \leq \sum\nolimits_i r_i 2^{k(d-1)} \leq 2^{-m} 2^{k(d-1)} \# \mathcal{E}_k. \qedhere \]
\end{proof}

Interpolating, it suffices to prove the following $L^2$ estimate on the function $F^m$.

\begin{lemma} \label{lemma6}
    Suppose $\mathcal{E} = \bigcup_k \mathcal{E}_k$ is a 1-separated set, where $\mathcal{E}_k \subset \RR^d \times [2^k,2^{k+1})$ is a set of density type $(2^m, 2^k)$. Then
    %
    \[ \left\| \sum\nolimits_{(x,r) \in \mathcal{E}} \chi_{x,r} \right\|_{L^2(\RR^d)} \lesssim \sqrt{m} \cdot 2^{ \frac{m}{d-1} } \left( \sum\nolimits_k 2^{k(d-1)} \#(\mathcal{E}_k) \right)^{1/2}. \]
\end{lemma}

% The L2 norms of the chi_{x,r} are equal to 2^{k(d-1)/2}, so the
% triangle ienquality implies that the LHS is bounded by sum_k 2^{k(d-1)/2} \#(E_k)

\begin{comment}
\begin{proof}[Proof of Lemma \ref{lemma3} from Lemma \ref{lemma6}]
    Write $F = \sum\nolimits_{(x,r) \in \mathcal{E}_k} \chi_{x,r}$, and then perform a decomposition $\mathcal{E}_k = \bigcup_{m \geq 0} \mathcal{E}_k(2^m)$, and thus define $F = \sum\nolimits_{m \geq 0} F_m$, where
    %
    \[ F_m = \sum\nolimits_{(x,r) \in \mathcal{E}(2^m)} \chi_{x,r}. \]
    %
    We have
    %
    \[ \| F_m \|_{L^2(\RR^d)} \lesssim 2^{\frac{m}{d-1} + \frac{k(d-1)}{2}} \log(2 + 2^m)^{1/2} \cdot \#(\mathcal{E}_k)^{1/2}. \]
    %
    If we interpolate this bound with the support bound for $F_m$, a kind of $L^0$ norm estimate, we conclude that for $0 < p \leq 2$,
    % ( int |F_m|^p )^{1/p} <= |S|^{1/pq^*} int |F_m|^{pq} )^{1/pq}
    % pq = 2
    % Then q = 2/p so 1/q^* = 1 - 1/q = 1 - p/2 = (2 - p)/2
    % so q^* = 2/(2-p)
    % 1/pq^* = (2-p)/2p = (1/p - 1/2)
    \begin{align*}
        \| F_m \|_{L^p(\RR^d)} &\leq |\text{Supp}(F_m)|^{1/p - 1/2} \| F_m \|_{L^2(\RR^d)}\\
        &\lesssim ( 2^{k(d-1) - m})^{1/p - 1/2} 2^{\frac{m}{d-1} + \frac{k(d-1)}{2}} \log(2 + 2^m)^{1/2} \cdot \#(\mathcal{E}_k)^{1/p} \\
        &\lesssim 2^{m(1/p_d - 1/p)} \log(2 + 2^m)^{1/2} 2^{\frac{k(d-1)}{p}} \#(\mathcal{E}_k)^{1/p}.
    \end{align*}
    % int |F_m|^p <= |S|^{1/p-1/2} ( int |F_m|^2 )^{1/2}
    %
    where $p_d = 2(d-1)/(d+1)$. This bound is summable in $m$ for $p < p_d$, which enables us to conclude that
    %
    \[ \| F \|_{L^p(\RR^d)} \lesssim 2^{\frac{k(d-1)}{p}} \#(\mathcal{E}_k)^{1/p}. \]
    %
    Thus for $1 \leq p < p_d$, we obtain the bound stated in Lemma \ref{lemma3}.
\end{proof}
\end{comment}

The $L^2$ bound in Lemma \ref{lemma6} gets worse and worse as $m$ grows, whereas the $L^0$ bound in Lemma \ref{lemma5} gets better and better, since annuli are concentrating in a small set, which is bad from the perspective of constructive interference, but absolutely fine from the perspective of a support bound. To prove the $L^2$ bound, we require an analysis of the interference patterns of pairs of the functions $\chi_{x,r}$, as provided by the following lemma.

%If $\psi$ is compactly supported, and $r$ is sufficiently large depending on the size of this support, then $\chi_{x,r}$ is supported on an annulus with centre $x$, radius $r$, and thickness $O(1)$. Thus $\| \chi_{x,r} \|_{L^p(\RR^d)} \sim r^{(d-1)/p}$, which implies that
%
%\[ \left\| \sum\nolimits_{(x,r) \in \mathcal{E}_k} \chi_{x,r} \right\|_{L^p(\RR^d)} \gtrsim 2^{k(d-1)/p} \#(\mathcal{E}_k)^{1/p}. \]
%
%Thus this bound can only be true if $p \geq 1$, and becomes tight when $p = 1$, where we actually have
%
%\[ \left\| \sum\nolimits_{(x,r) \in \mathcal{E}_k} \chi_{x,r} \right\|_{L^1(\RR^d)} \sim 2^{k(d-1)} \#(\mathcal{E}_k) \]
%

\begin{lemma} \label{lemma4}
    For any $N > 0$, $x_1,x_2 \in \RR^d$ and $r_1,r_2 \geq 1$,
    %with $|x_1 - x_2| \geq 1$ or $x_1 = x_2$, and $r_1,r_2 > 1$,
    %
    \begin{align*}
        |\langle \chi_{x_1,r_1}, \chi_{x_2,r_2} \rangle| &\lesssim_N \left( \frac{r_1r_2}{\langle (x_1,r_1) - (x_2,r_2) \rangle} \right)^{\frac{d-1}{2}} \sum\nolimits_{\pm,\pm} \langle |x_1 - x_2| \pm r_1 \pm r_2 \rangle^{-N}.
    \end{align*}
\end{lemma}

\begin{remark}
    Suppose $r_1 \leq r_2$. Then Lemma \ref{lemma4} implies that $\chi_{x_1,r_1}$ and $\chi_{x_2,r_2}$ are roughly uncorrelated, except when they are supported on annuli that roughly have the same radii and centers, and in addition, one of the following two properties hold:
    %
    \begin{itemize}
        \item $r_1 + r_2 \approx |x_1 - x_2|$, which holds when the two annuli are `approximately' externally tangent to one another.

        \item $r_2 - r_1 \approx |x_1 - x_2|$, which holds when the two annuli are `approximately' internally tangent to one another.
    \end{itemize}
    %
    Heo, Nazarov, and Seeger do not exploit the tangency information, though utilizing the tangencies seems important to improve the results they obtain. Cladek exploits this tangency information further, to obtain improved results.
\end{remark}

%    Note that if we take too comparable radii $r_1,r_2 \sim 2^k$, then the functions $\{ 1_{x,r} \}$, which are the non-oscillating analogues of the functions $\{ \chi_{x,r} \}$ we are studying in this problem, if they correspond to tangent annuli, satisfy bounds  of the form
    %
%    \[ \langle 1_{x_1,r_1}, 1_{x_2,r_2} \rangle \lesssim 2^{- \left( \frac{d-1}{2} \right) k} \]
    %
%    whereas for internal tangencies we have
    %
%    \[ \langle \chi_{x_1,r_1}, \chi_{x_2,r_2} \rangle \lesssim 2^{(d-1)k} |x_1 - x_2|^{- \frac{d-1}{2}} \]
    %
%    In the Clamshell example, one has
    %
%    \[ \left\| \sum\nolimits_{k = 1}^N k^{d-1} 1_{k,k} \right\|_{L^2(\RR^d)} \]

    % delta = 2^{-k}
    % 3[d-1]/2
%    \[ 2^{(d-1)k} 2^{dk} 2^{-\frac{d+1}{2}} \]
    %
%    \[ \langle r_1^{d-1} 1_{x_1,r_1}, r_2^{d-1} 1_{x_2,r_2} \rangle \lesssim r_1^{d-1} r_2^{d-1}. \]
%    \end{comment}

\begin{proof}
%    We may assume $|x_1 - x_2| \geq 1$, for otherwise the inequality holds trivially since unless $|r_1 - r_2| \lesssim 1$, $f_{x_1r_1}$ and $f_{x_2r_2}$ have disjoint support, and if $|r_1 - r_2| \lesssim 1$ then Cauchy Schwartz implies that
    %
%    \begin{align*}
%        |\langle f_{x_1r_1}, f_{x_2r_2} \rangle| &\lesssim (r_1 r_2)^{(d-1)/2}\\
%        &\lesssim_{N,d} (r_1r_2)^{(d-1)/2} (1 + |r_1 - r_2| + |x_1 - x_2|)^{-(d-1)/2} \sum\nolimits_{\pm,\pm} (1 + ||x_1 - x_2| \pm r_1 \pm r_2|)^{-N}
%    \end{align*}
%
    We write
    %
    \begin{align*}
        \langle \chi_{x_1 r_1}, \chi_{x_2 r_2} \rangle &= \left\langle \widehat{\chi}_{x_1 r_1}, \widehat{\chi}_{x_2 r_2} \right\rangle\\
        &= \int_{\RR^d} \widehat{\sigma_{r_1} * \psi}(\xi) \cdot \overline{\widehat{\sigma_{r_2} * \psi}(\xi)} e^{2 \pi i (x_2 - x_1) \cdot \xi}\; d\xi\\
        &= (r_1 r_2)^{d-1} \int_{\RR^d} \widehat{\sigma}(r_1 \xi) \overline{\widehat{\sigma}(r_2 \xi)} |\widehat{\psi}(\xi)|^2 e^{2 \pi i (x_2 - x_1) \cdot \xi}\; d\xi.
    \end{align*}
    %
    Define functions $A$ and $B$ such that $B(|\xi|) = \widehat{\sigma}(\xi)$, and $A(|\xi|) = |\widehat{\psi}(\xi)|^2$. Then
    %
    \[ \langle \chi_{x_1, r_1}, \chi_{x_2, r_2} \rangle = C_d (r_1r_2)^{d-1} \int_0^\infty s^{d-1} A(s) B(r_1 s) B(r_2 s) B(|x_2 - x_1| s)\; ds. \]
    %
    Using well known asymptotics for $\widehat{\sigma}$, we have, for any $N > 0$,
    %
    \[ B(s) = s^{-(d-1)/2} \sum\nolimits_{n = 0}^{N-1} (c_{n,+} e^{2 \pi i s} + c_{n,-} e^{-2 \pi i s}) s^{-n} + O_N(s^{-N}). \]
    %
    Now, assuming $A(s)$ vanishes to order $\gtrsim N$ at the origin, we conclude that
    %
    \begin{align*}
        \langle \chi_{x_1 r_1}, \chi_{x_2 r_2} \rangle &= C_d \left( \frac{r_1r_2}{|x_1 - x_2|} \right)^{(d-1)/2} \sum\nolimits_{n,\tau} c_{n,\tau}  r_1^{-n_1} r_2^{-n_2} |x_2 - x_1|^{-n_3}\\
        &\quad\quad\quad \Bigg\{ \int_0^\infty A(s) s^{-(d-1)/2}  s^{-n_1-n_2-n_3} e^{2 \pi i (\tau_1 r_1 + \tau_2 r_2 + \tau_3 |x_2 - x_1|) s}\; ds \Bigg\}\\
        &\lesssim_N \left( \frac{r_1r_2}{|x_1 - x_2|} \right)^{\frac{d-1}{2}} \left(1 + \frac{1}{|x_1 - x_2|^N} \right) \sum\nolimits_{\tau} \left( 1 + |\tau_1 r_1 + \tau_2 r_2 + \tau_3 |x_2 - x_1|| \right)^{-5N}\\
        &\lesssim_N \left( \frac{r_1r_2}{|x_1 - x_2|} \right)^{\frac{d-1}{2}} \left(1 + \frac{1}{|x_1 - x_2|^N} \right) \sum\nolimits_\tau \left( 1 + |\tau_1 r_1 + \tau_2 r_2 + |x_2 - x_1|| \right)^{-5N}.
    \end{align*}
    %
    This gives the result provided that $1 + |x_1 - x_2| \geq |r_1 - r_2| / 10$ and $|x_1 - x_2| \geq 1$. If $1 + |x_1 - x_2| \leq |r_1 - r_2| / 10$, then the supports of $\chi_{x_1,r_1}$ and $\chi_{x_2,r_2}$ are disjoint, so the inequality is trivial. On the other hand, if $|x_1 - x_2| \leq 1$, then the bound is trivial by the last sentence unless $|r_1 - r_2| \leq 10$, and in this case the inequality reduces to the simple inequality
    %
    \[ \langle \chi_{x_1,r_1}, \chi_{x_2,r_2} \rangle \lesssim_N (r_1 r_2)^{(d-1)/2}. \] 
    %
    But this follows immediately from the Cauchy-Schwartz inequality.
\end{proof}

The exponent $(d-1)/2$ in Lemma \ref{lemma4} is too weak to apply almost orthogonality directly to obtain $L^2$ bounds on $\sum\nolimits_{(x,r) \in \mathcal{E}_k} \chi_{xr}$ on it's own, but together with the density decomposition assumption we will be able to obtain Lemma \ref{lemma6}.

\begin{proof}[Proof of Lemma \ref{lemma6}]
    Without loss of generality, we may assume that the set of $k$ such that $\mathcal{E}_k \neq \emptyset$ is $10$-separated. Write
    %
    \[ F = \sum\nolimits_{(x,r) \in \mathcal{E}} \chi_{x,r} \]
    %
    and $F_k = \sum\nolimits_{(x,r) \in \mathcal{E}_k} \chi_{x,r}$. First, we deal with $F_{\lesssim m} = \sum\nolimits_{k \leq 10 m} F_k$ trivially, i.e. writing
    %
    \begin{align*}
        \| F \|_{L^2(\RR^d)} &\lesssim m^{1/2} \left( \sum\nolimits_{k \leq 10m} \| F_k \|_{L^2(\RR^d)}^2 + \| \sum\nolimits_{k > 10m} F_k \|_{L^2(\RR^d)} \right)^{1/2}.
    \end{align*}
    %
    We then decompose
    %
    \[ \| \sum\nolimits_{k > 10 m} F_k \|_{L^2(\RR^d)}^2 \leq \sum\nolimits_{k > 10 m} \| F_k \|_{L^2(\RR^d)}^2 + 2 \sum\nolimits_{k' > k > 10m} |\langle F_k, F_{k'} \rangle|. \]
    %
    Let us analyze $\langle F_k, F_{k'} \rangle$. The term will become a sum of the form $\langle \chi_{x,r}, \chi_{y,s} \rangle$, where $r \sim 2^k$ and $s \sim 2^{k'}$. Because of our assumption of being 10-separated, we have $r \leq s / 2^{10}$. If $\langle \chi_{x,r}, \chi_{y,s} \rangle \neq 0$, then since the support of $\chi_{y,s}$ is an annulus of radius $s$ centered at $y$, with thickness $O(1)$, and $\chi_{x,r}$ has support on an annulus of radius $r$ centered at $x$, with thickness $O(1)$, the fact that $r$ is comparatively smaller than $s$ implies that $(x,r)$ must be contained in the annulus of radius $s$ centered at $y$, with thickness $O(2^k)$. Such an annulus is covered by $O( 2^{(k'-k)(d-1)} )$ balls of radius $2^k$. Each ball can only contain $2^{k + m}$ points $(x,r)$, and so there can be at most
    %
    \[ O(2^{k'(d-1)} 2^{-k(d-1)} 2^{k+m} ) = O( 2^{k'(d-1) - k(d-2) + m} ). \]
    %
    pairs $(x,r) \in \mathcal{E}_k$ for which $\langle \chi_{x,r}, \chi_{y,s} \rangle \neq 0$. For such pairs we have
    %
    \[ |\langle \chi_{x,r}, \chi_{y,s} \rangle| \lesssim \left( \frac{2^k 2^{k'}}{2^{k'}} \right)^{\frac{d-1}{2}} = 2^{\frac{k(d-1)}{2}}. \]
    %
    Thus we conclude that
    %
    \[ |\langle F_k, \chi_{y,s} \rangle| \lesssim 2^{-k ( \frac{d-3}{2} ) + k'(d-1) + m }. \]
    %
    Summing over $10m < k < k'$, we conclude that since $d \geq 4$,
    %
    \[ \sum\nolimits_{10m < k < k'} |\langle F_k, \chi_{y,s} \rangle| \lesssim 2^{k'(d-1) + m} \sum\nolimits_{10m < k < k'} 2^{-k \frac{d-3}{2}} \lesssim 2^{k'(d-1) + m} 2^{-5m} \lesssim 2^{k'(d-1)}. \]
    %
    But this means that
    %
    \[ \sum\nolimits_{10m < k < k'} |\langle F_k, F_{k'} \rangle| \lesssim 2^{k'(d-1)} \cdot \# ( \mathcal{E}_{k'} ). \]
    %
    This means that
    %
    \[ \| \sum\nolimits_{k > 10m} F_k \|_{L^2(\RR^d)}^2 \lesssim \sum\nolimits_{k > 10m} \| F_k \|_{L^2(\RR^d)}^2 + \sum\nolimits_{k'} 2^{k'(d-1)} \# (\mathcal{E}_{k'}), \]
    %
    and it now suffices to deal with estimates the $\| F_k \|_{L^2(\RR^d)}$, i.e. the interactions of functions supported on radii of comparable magnitude. To deal with these, we further decompose the radii, writing $[2^k,2^{k+1})$ as the disjoint union of intervals $I_{k,\mu} = [2^k + (\mu - 1) 2^{am}, 2^k + \mu 2^{am}]$, for some $a$ to be chosen later. These interval induces a decomposition $\mathcal{E}_k = \bigcup_\mu \mathcal{E}_{k,\mu}$. Again, incurring a constant loss at most, we may assume that the $\mu$ such that $\mathcal{E}_{k,\mu} \neq \emptyset$ are $10$ separated. We write $F_k = \sum F_{k,\mu}$, and we have
    %
    \[ \| F_k \|_{L^2(\RR^d)}^2 = \sum\nolimits_\mu \| F_{k,\mu} \|_{L^2(\RR^d)}^2 + \sum\nolimits_{\mu < \mu'} |\langle F_{k,\mu}, F_{k,\mu'} \rangle|. \]
    %
    We now consider $\chi_{x,r}$ and $\chi_{y,s}$ with $r \in I_{k,\mu}$ and $s \in I_{k,\mu'}$. Then we must have $|x - y| \lesssim 2^k$ and $2^{am} \leq |r - s| \lesssim 2^k$, and so we have
    %
    \begin{align*}
        |\sum\nolimits_{\mu < \mu'} \langle F_{k,\mu}, \chi_{y,s} \rangle| &\lesssim 2^{k(d-1)} \sum\nolimits_{\substack{(x,r) \in \mathcal{E}_k\\ 2^{am} \leq |(x,r) - (y,s)| \lesssim 2^k}} |(x,r) - (y,s)|^{- \frac{d-1}{2}}\\
        &\lesssim 2^{k(d-1)} \sum\nolimits_{am \leq l \leq k} 2^{-l(d-1)/2} \# \{ (x,r) \in \mathcal{E}_k: |(x,r) - (y,s)| \sim 2^l \}.
    \end{align*}
    %
    Using the density assumption,
    %
    \[ \# \{ (x,r) \in \mathcal{E}_k: |(x,r) - (y,s)| \sim 2^l \} \lesssim 2^{l + m} \]
    %
    and so we obtain that, again using the assumption that $d \geq 4$,
    %
    \[ |\sum\nolimits_{\mu < \mu'} \langle F_{k,\mu}, \chi_{y,s} \rangle| \lesssim 2^{k(d-1)} 2^{m(1-a(d-3)/2)}. \]
    %
    Now summing over all $(y,s)$, we obtain that
    %
    \[ |\sum\nolimits_{\mu < \mu'} \langle F_{k,\mu}, F_{k,\mu'} \rangle| \lesssim 2^{k(d-1)} 2^{m(1 - a(d-3)/2)} \#(\mathcal{E}_{k,\mu'}). \]
    %
    and now summing over $\mu'$ gives that
    %
    \[ \| F_k \|_{L^2(\RR^d)}^2 \lesssim \sum\nolimits_\mu \| F_{k,\mu} \|_{L^2(\RR^d)}^2 + 2^{k(d-1)} 2^{m(1 - a(d-3)/2)} \# \mathcal{E}_k, \]
    %
    which is a good enough bound if we pick $a$ to be large enough. Now we are left to analyze $\| F_{k,\mu} \|_{L^2(\RR^d)}$, i.e. analyzing interactions between annuli which have radii differing from one another by at most $O(2^{am})$. Since the family of all possible radii are discrete, the set $\mathcal{R}_{k,\mu}$ of all possible radii has cardinality $O(2^{am})$. We do not really have any orthogonality to play with here, so we just apply Cauchy-Schwartz, writing $F_{k,\mu} = \sum\nolimits_{r \in \mathcal{R}_{k,\mu}} F_{k,\mu,r}$, to write
    %
    \[ \| F_{k,\mu} \|_{L^2(\RR^d)}^2 \lesssim 2^{am} \sum\nolimits_r \| F_{k,\mu,r} \|_{L^2(\RR^d)}^2. \]
    %
    Recall that $\chi_{x,r} = \text{Trans}_x(\sigma_r * \psi)$, where $\psi$ is a compactly supported function whose Fourier transform is non-negative and vanishes to high order at the origin. In particular, we now make the additional assumption that $\psi = \psi_{\circ} * \psi_{\circ}$ for some other compactly function $\psi_{\circ}$ whose Fourier transform is non-negative and vanishes to high order at the origin. Then we find that $F_{k,\mu,r}$ is equal to the convolution of the function
    %
    \[ A_r = \sum\nolimits_{(x,r) \in \mathcal{E}} \text{Trans}_x \psi_{\circ} \]
    %
    with the function $\sigma_r * \psi_{\circ}$. Using the standard asymptotics for the Fourier transform of $\sigma_r$, i.e. that for $|\xi| \geq 1$,
    %
    \[ |\widehat{\sigma_r}(\xi)| \lesssim r^{d-1} (1 + r |\xi|)^{- \frac{d-1}{2}}, \]
    %
    and since $|\widehat{\psi_\circ}(\xi)| \lesssim_N |\xi|^N$, we get that if $r \geq 1$, then for $|\xi| \leq 1/r$,
    %
    \[ |\widehat{\sigma_r}(\xi) \widehat{\psi_\circ}(\xi)| \lesssim_N r^{d-1-N} \]
    %
    and for $|\xi| \geq 1/r$,
    %
    \[ |\widehat{\sigma_r}(\xi) \widehat{\psi_\circ}(\xi)| \lesssim_N r^{\frac{d-1}{2}} |\xi|^{-N}. \]
    %
    Thus in particular,the $L^\infty$ norm of the Fourier transform of $\sigma_r * \psi_\circ$ is $O(r^{(d-1)/2})$. Now the functions $\psi_{\circ}$ are compactly supported, so since the set of $x$ such that $(x,r) \in \mathcal{E}$ is one-separated, we find that
    %
    \[ \| A_r \|_{L^2(\RR^d)} \lesssim \# \{ x : (x,r) \in \mathcal{E} \}^{1/2}. \]
    %
    But this means that
    %
    \[ \| F_{k,\mu,r} \|_{L^2(\RR^d)} = \| A_r * (\sigma_r * \psi_{\circ}) \|_{L^2(\RR^d)} \lesssim r^{\frac{d-1}{2}} \# \{ x : (x,r) \in \mathcal{E} \}^{1/2}. \]
    %
    Thus we have that
    %
    \[ \| F_{k,\mu} \|_{L^2(\RR^d)}^2 = 2^{am} \cdot \# \mathcal{E}_{k,\mu} \cdot 2^{k(d-1)}. \]
    %
    Summing over $\mu$ gives that
    %
    \[ \| F_k \|_{L^2(\RR^d)}^2 = 2^{k(d-1)} \# \mathcal{E}_k (2^{am}  + 2^{m(1 - a(d-3)/2)}). \]
    %
    Picking $a = 2/(d-1)$ optimizes this bound, giving
    %
    \[ \| F_k \|_{L^2(\RR^d)} \lesssim 2^{m/(d-1)} 2^{k(d-1)/2} (\# \mathcal{E}_k)^{1/2}. \]
    %
    Plugging this into the estimates we got for $F$ gives the required bound.
\end{proof}

This completes a proof of the single scale estimates of the paper. The paper then uses an atomic decomposition method to combine these scales and thus complete the proof of Theorem \ref{HeoNazarovSeegerTheorem}. Rather than discuss these methods, we instead discuss an iteration of this method, developed by the same authors in a follow up paper \cite{HeoandNazarovandSeeger2}.

\section{Combining Scales with Atomic Decompositions} \label{sec:combining_scales_with_atomic_decompositions}

We now sketch a proof as to how we can complete the proof of Theorem \ref{HeoNazarovSeegerTheorem}, being deliberately vague, as our goal is to build intuition about the techniques, in order to carry out analogous methods more rigorously in the compact manifold setting later. If $T = m(P)$, then the last section proves that if we write $T = \sum T_j$, where $T_j$ is the radial Fourier multiplier operator with convolution kernel $k_j(\cdot/2^j)$, then the results of the last section (appropriately rescaled), imply that
%
\begin{equation} \label{individualscaleoperatorbound}
    \| T_j \|_{L^p(\RR^d) \to L^q(\RR^d)} \lesssim \| k_j \|_{L^q(\RR^d)}.
\end{equation}
%
The goal of this section is to prove that we can bound the sum of the operators $T_j$ by the suprema of the quantities on the right hand side in \eqref{individualscaleoperatorbound} which, morally speaking, is a bound of the form
%
\[ \left\| \sum\nolimits_j T_j \right\| \lesssim \sup\nolimits_j \| T_j \|. \]
%
We must thus show that the operators $\{ T_j \}$ do not constructively interfere with one another to a significant extent.

Atomic decompositions are a powerful way to control interactions between operators. The method involves decomposing elements of function spaces into more elementary components, which we call \emph{atoms}, that have controlled size, spatial support, and oscillatory properties. Such atoms are chosen via careful, non-linear selection processes, often related to stopping times, to ensure certain cancellation conditions hold. We use a variant of this method, involving a use of a Whitney decomposition rather than a stopping time argument, is used to obtain an atomic decomposition where atoms do not cluster together at each frequency scale, in the sense that the doubles of the cubes upon which the atoms are supported have the bounded overlap property.

Since we are controlling different operators supported on different dyadic frequency ranges on functions in $L^p(\RR^d)$ for $1 < p < \infty$, it is natural to consider the Littlewood-Paley inequality $\| Sf \|_{L^p(\RR^d)} \lesssim \| f \|_{L^p(\RR^d)}$, where
%
\[ Sf(x) = \left( \sum\nolimits_j |P_j f(x)|^2 \right)^{1/2}, \]
%
and the operators $P_j$ are Littlewood-Paley projections, Fourier multiplier operators with symbol $\psi(\cdot / 2^j)$, where $\psi \in C_c^\infty(\RR^d)$ is equal to one on the support of the function $\chi$ used to decompose $T$ into the operators $T_j$ above, so that $T_j f = T_j f_j$, where $f_j = P_j f$. Roughly speaking, our atomic decomposition will be of the following form: for each dyadic number $H$, we consider a family of dyadic cubes $\mathcal{W}_H$, whose union is the set $\{ x : |Sf(x)| \sim H \}$, and whose doubles have the bounded overlap property. We will then obtain a decomposition $f = \sum a_{j,H,W}$, where $a_{j,H,W}$ has Fourier support on $|\xi| \sim 2^j$, is supported on the cube $W$, and for each fixed $H$, $(\sum\nolimits_j |\sum\nolimits_W |a_{j,H,W}|\; |^2 )^{1/2} \sim H$.

The advantage of this decomposition is that it controls how many \emph{local interactions} the atoms $\{ a_{j,H,W} \}$ have. To see why this might be useful, suppose we were considering a Fourier multiplier operator $T$ with the property that $T a_{j,H,W}$ is essentially supported on the cube $W^*$ obtained by doubling the sidelengths of the cube $W$, but maintaining the same centre. The bounded overlap property of the dyadic cubes $\mathcal{W}_H$ would then imply an almost orthogonality bound for each $H$, that
%
\begin{align*}
  \left\| \sum\nolimits_{j,W} T_j a_{j,H,W} \right\|_{L^2(\RR^d)} &\lesssim \left( \sum\nolimits_{j,W} \| T_j a_{j,H,W} \|_{L^2(\RR^d)}^2 \right)^{1/2}\\
  &\lesssim \left( \sum\nolimits_{j,W} \| a_{j,H,W} \|_{L^2(\RR^d)}^2 \right)^{1/2}\\
  &= \left\| \left( \sum\nolimits_{j,W} |a_{j,H,W}|^2 \right)^{1/2} \right\|_{L^2(\RR^d)} \lesssim H |\Omega_H|^{1/2}
\end{align*}
%that for $1/r = 1/2 + 1/p - 1/q$,
%
%\begin{align*}
%  \left\| \sum\nolimits_{j,W} T_j a_{j,H,W} \right\|_{L^2(\RR^d)} &\lesssim \left( \sum\nolimits_{j,W} \| T_j a_{j,H,W} \|_{L^2(\RR^d)}^2 \right)^{1/2}\\
%  &\lesssim \left( \sum\nolimits_{j,W} \left[ 2^{-jd(1/p - 1/q)} \| a_{j,H,W} \|_{L^2(\RR^d)} \right]^2 \right)^{1/2}\\
%  &\lesssim \left( \sum\nolimits_{j,W} \| a_{j,H,W} \|_{L^r(\RR^d)}^2 \right)^{1/2}\\
%  &\lesssim \left\| \left( \sum |a_{j,H,W}|^2 \right)^{1/2} \right\|_{L^r(\RR^d)} \lesssim H |\Omega_H|^{1/2 + 1/p - 1/q}
%\end{align*}
%
%
H\"{o}lder's inequality then justifies that
%
\begin{align*}
  \left\| \sum\nolimits_{j,W} T_j a_{j,H,W} \right\|_{L^1(\RR^d)} &\lesssim |\Omega_H|^{1/2} \left\| \sum\nolimits_{j,W} T_j a_{j,H,W} \right\|_{L^2(\RR^d)} \lesssim H |\Omega_H|.
\end{align*}
%
Real interpolation allows us to sum in $H$, and conclude that for $1 < p < 2$,
%
\[ \| Tf \|_{L^p(\RR^d)} = \left\| \sum\nolimits_{H,j,W} T_j a_{j,H,W} \right\|_{L^p(\RR^d)} \lesssim \left( \sum \left[ H |\Omega_H|^{1/p} \right]^p \right)^{1/p} \lesssim \| S f \|_{L^p(\RR^d)} \lesssim \| f \|_{L^p(\RR^d)}. \]
%
and thus we have proven the boundedness of such operators.

Such a simple proof will not suffice for our analysis, since the class of multipliers we are considering is not pseudolocal at all; the kernels $k_j$ are only assumed to be uniformly bounded in $L^p(\RR^d)$, and need not satisfy any decay bound as $|x| \to \infty$ (though the atomic decomposition methods above can be used to prove bounds for other more pseudolocal multipliers, e.g. obtaining an alternate proof of the endpoint results of \cite{SeegerSingular}). We will, however, be able to exploit the above calculations to control \emph{close range interactions} of general multipliers, i.e. for any multiplier $T$, writing
%
\[ Tf = \sum T_j a_{j,H,W} = \sum T_{j,W,\text{Short}} a_{j,H,W} + T_{j,W,\text{Long}} a_{j,H,W}, \]
%
where
%
\[ T_{j,W,\text{Short}}(x,y) = \mathbb{I}_{W^*}(y)\; T_{j,W}(x,y) \quad\text{and}\quad T_{j,W,\text{Short}}(x,y) = (1 - \mathbb{I}_{W^*}(y))\; T_{j,W}(x,y). \]
%
The above analysis can be adapted to show that $\| \sum T_{j,W,\text{Short}} a_{j,H,W} \|_{L^q(\RR^d)} \lesssim \| f \|_{L^p(\RR^d)}$, and it remains to find a way to control the long range interactions between atoms.

There are several related methods to control these interactions, the most elegant formulation provided in the paper \cite{HeoandNazarovandSeeger2}. The idea is to take a singular scale estimate, and upgrade this to an estimate with a geometrically decaying constant term when our inputs have amplitudes that are locally constant on a much larger scale than is required by the uncertainty principle and the frequency support of the inputs, namely, that
%
\[ \left\| \sum\nolimits_H \sum\nolimits_{W \in \mathcal{W}_{H,l-j}} T_{j,W,\text{Long}} * a_{H,W} \right\|_{L^q(\RR^d)} \lesssim 2^{-l \varepsilon} \left( \sum\nolimits_H \sum\nolimits_{W \in \mathcal{W}_{H,l-j}} \| a_{H,W} \|_{L^\infty(\RR^d)}^p |W| \right)^{1/p}, \]
%
where $\mathcal{W}_{H,a}$ are the set of all cubes of sidelength $2^a$. Summing in frequencies using $L^p$ orthogonality, we find that
%
\[ \left\| \sum\nolimits_j \sum\nolimits_H \sum\nolimits_{W \in \mathcal{W}_{H,l-j}} T_{j,W,\text{Long}} a_{j,H,W} \right\|_{L^p(\RR^d)} \lesssim 2^{-l \varepsilon} \left( \sum\nolimits_j \sum\nolimits_H \sum\nolimits_{W \in \mathcal{W}_{H,l-j}} \| a_{j,H,W} \|_{L^\infty(\RR^d)}^p |W| \right)^{1/p}, \]
%
For a fixed $l$, and each $W \in \mathcal{W}_H$, there exists a unique $j$ such that $W \in \mathcal{W}_{H,l-j}$, which implies that the supports of the functions $\{ a_{j,H,W} \}$ in the sum above are almost disjoint, and thus the simple bound $\| a_{j,H,W} \|_{L^\infty(\RR^d)} \lesssim H$ gives that
%
\[ \sum\nolimits_j \sum\nolimits_{W \in \mathcal{W}_{H,l-j}} \| a_{j,H,W} \|_{L^\infty(\RR^d)}^p |W| \leq H^p |\Omega_H|, \]
%
and thus that
%
\[ \left\| \sum\nolimits_j \sum\nolimits_H \sum\nolimits_{W \in \mathcal{W}_{H,l-j}} T_{j,W,\text{Long}} a_{j,H,W} \right\|_{L^p(\RR^d)} \lesssim 2^{-l \varepsilon} \left( \sum\nolimits_H H^p |\Omega_H| \right)^{1/p} \lesssim 2^{-l\varepsilon} \| f \|_{L^p(\RR^d)}. \]
%
Summing in $l$ trivially using the triangle inequality and the geometric decay in $l$ gives that
%
\[ \left\| \sum\nolimits_j \sum\nolimits_H \sum\nolimits_{W \in \mathcal{W}_{H}} T_{j,W,\text{Long}} a_{j,H,W} \right\|_{L^p(\RR^d)} \lesssim \| f \|_{L^p(\RR^d)}, \]
%
which controls the long-range interactions in full.

%A related bound is a square function estimate of Peetre, which says that $\| \tilde{S} f \|_{L^p(\RR^d)} \lesssim \| f \|_{L^p(\RR^d)}$, where
%
%\[ \tilde{S} f(x) = \left( \sum\nolimits_j \sup\nolimits_{|h| \leq C 2^{-j}} |P_j f(x + h)|^2 \right)^{1/2} \]
%
%for a fixed $C > 0$. This bound is morally equivalent to the Littlewood-Paley inequality in light of uncertainty principle heuristics, which indicate that the function $P_j f$ is locally constant at a scale $2^{-j}$. TODO EDIT

%For each dyadic number $H$, we let $\Omega_H = \{ x : \tilde{S}f(x) \geq H \}$. For each dyadic cube $Q$, there exists a largest value $H(Q)$ such that $\Omega_{H(Q)}$ contains at least half the points in $Q$. We perform a wave packet decomposition $f_j = \sum f_{j,Q}$, where $Q$ ranges over all dyadic sidelength $2^{-j}$ cubes, and $f_{j,Q} = \mathbb{I}_Q f_j$. Our atoms will be given by $a_{j,H,W}$ TODO EDIT





\section{Boundedness of Quasi-Radial Multipliers and Local Smoothing Inequalities}

Kim \cite{KimQuasiradial} has extended 
the bounds of Heo-Nazarov-Seeger to quasiradial multipliers, i.e. multipliers of the form $(a \circ r)(\xi)$, where $r: \RR^d \to [0,\infty)$ is a smooth, homogeneous function of order one, such that the cosphere $S = \{ \xi : r(\xi) = 1 \}$ is a hypersurface with non-vanishing Gauss curvature. Such multipliers retain the translation and dilation symmetries of the family of radial Fourier multiplier operators, but are no longer \emph{rotation-invariant}.

Using the same identities as for radial Fourier multiplier operators,, one can verify that
%
\[ \| a \circ r \|_{M^p(\RR^d)} \gtrsim \sup\nolimits_R \| k_R \|_{L^q(\RR^d)}, \]
%
where $k_j$ is the Fourier transform of $\chi(\cdot) (a \circ r)(R \cdot)$, with $\chi$ as in that section. In an analysis of Bochner-Riesz type quasi-radial multipliers, Lee and Seeger \cite{LeeSeeger2} showed that
%
\[ \sup\nolimits_R \| k_R \|_{L^p(\RR^d)} \sim \| a \|_{R^{p,s}[0,\infty)}, \]
%
Kim \cite{KimQuasiradial} has verified that in the range $1/p - 1/2 > 1/(d-1)$,
%
\[ \| a \circ r \|_{M^{p,q}(\RR^d)} \lesssim \| a \|_{R^{q,s}_d[0,\infty)}, \]
%
thus generalizing the results of \cite{HeoandNazarovandSeeger} to the setting of quasi-radial multipliers.

The proof is an adaption of the proof of \cite{HeoandNazarovandSeeger2}. However, it is more difficult to work directly with convolution kernels $k$ in this problem, since, unlike for radial multipliers, whose kernels are also radial, the kernels $k$ corresponding to quasi-radial Fourier multiplier operators need not be quasi-radial. It is here that we introduce a technique that has proved essential to an analysis of multiplier operators in settings lacking the full symmetry of Euclidean space: a reduction of the study of multipliers to an analysis of wave equations. Using the Fourier inversion formula, we write
%
\begin{align*}
  (a \circ r) f(x) &= \int a(r(\xi)) e^{2 \pi i \xi \cdot (x - y)} f(y)\; dy\; dx\\
  &= \iint \widehat{a}(t) e^{2 \pi i [t r(\xi) + \xi \cdot (x - y)]} f(y)\; dy\; dx\; dt\\
  &= \int \widehat{a}(t) (w_t * f)(x)\; dt
\end{align*}
%
where $\widehat{w}_t(\xi) = e^{2 \pi i t r(\xi)}$. As $t$ varies, $u = w_t * f$ solves the wave equation $\partial_t u = 2 \pi i P u$, where $P$ is the Fourier multiplier operator whose symbol is the function $r$. Define $\chi_{x,t} = \text{Trans}_x \left( \int_{-\infty}^\infty \psi(s) w_{t - s}\; ds \right)$, where the Fourier transform of $\psi$ is non-negative and vanishing to high order at the origin. Then, using oscillatory integral techniques akin to Lemma \ref{lemma4}, one can obtain inner product estimates on the quantities $\langle \chi_{x_1,t_1}, \chi_{x_2,t_2} \rangle$ that show such terms are negligible unless $|x_1 - x_2| + |t_1 - t_2| \lesssim 1$. We can then adapt the proof of \cite{HeoandNazarovandSeeger2}, performing a density decomposition using the Euclidean metric on $\RR^{d+1}$, and thus obtain single scale bounds for the quasi-radial multipliers. An atomic decomposition analogous to that discussed in Section \ref{sec:combining_scales_with_atomic_decompositions} then yields Kim's result.

We note that the endpoint analysis of radial multipliers we have been discussing is very closely related to the regularity of solutions to wave equations. In particular, if $\widehat{w}_t = e^{2 \pi i t |\xi|}$, then for any function $f$, $u = w_t * f$ solves the half-wave equation $\partial_t u = 2 \pi i \sqrt{-\Delta} u$. The Littlewood-Paley pieces $(P_k u)(\cdot,t)$ are Fourier multipliers with symbol $\chi(\cdot / 2^k) e^{2 \pi i t |\xi|}$, which are radial, and thus can be written in terms of spherical averages. The analysis of Lemma \ref{lemma1} can then be used to show that for $1/2 - 1/q > 1/(d-1)$,
%
\[ \| P_k u \|_{L^q( \RR^d \times [1,2] )} \lesssim 2^{k s} \| f \|_{L^q(\RR^d)}\quad\text{where}\ s = (d-1)(1/2 - 1/q) - 1/q. \]
%
With some more work involving atomic decompositions, Heo, Nazarov, and Seeger are able to combine the frequency scales, and thus prove the local smoothing estimate
%
\[ \| u \|_{L^q(\RR^d \times [1,2])} \lesssim \| f \|_{W^{s,q}(\RR^d)}. \]
%
These are the current sharpest \emph{endpoint} local smoothing bounds (they do not lose any $\varepsilon$ in the smoothness parameter).

To end our discussion of radial multipliers, notice that the Fourier multiplier operator $P$ is above is an elliptic operator on $\RR^d$, satisfying the Assumption A we introduced in Chapter \ref{cha:multipliers_of_an_elliptic_operator}. For such an operator, the partial differential equation $\partial_t u = 2 \pi i P u$ is hyperbolic, and it's solutions have wavelike properties. We will rely on a similar decomposition method used for the quasi-radial multiplier with symbol $a \circ r$ on manifolds, together with a study of the geometry of the manifold upon which we study these multipliers, to extend the methods we have discussed in the previous three sections to compact manifolds. We will now return to discuss this setting in more detail.








\chapter{Wave Equations on Compact Manifolds}

\section{Geometries Induced by Elliptic Operators}

As alluded to in the last section, we now plan to study spectral multipliers of an elliptic operator $P$ on a compact manifold $X$ via the use of solutions of the wave equation $\partial_t u = 2 \pi i P u$ on $M$, which allow us to exploit geometric information about the manifold $M$ and thus extend the results of Heo, Nazarov, and Seeger to the setting of compact manifolds when Assumption A and Assumption B of Chapter \ref{cha:multipliers_of_an_elliptic_operator} hold.

But what is the geometry of the manifold we should be using? If the elliptic operator $P = \sqrt{-\Delta}$ is induced by a Riemannian metric on $X$, the geometric structure is clear. Given a more general operator $P$, we will use the principal symbol of $P$ to give $X$ a \emph{Finsler geometry}. In this section we describe the bare eessentials of Finsler geometry needed for the arguments which occur later on in the thesis.

Let $X$ be a $d$-dimensional manifold. We denote an element of the tangent bundle $TX$ by $(x,v)$, where $x \in X$ and $v \in T_x X$. A \emph{Finsler metric} on $X$ is a homogeneous function $F: T X \to [0,\infty)$, which is smooth on $TX - 0$, and strictly convex, in the sense that for $(x,v) \in TX - 0$, the Hessian matrix with entries
%
\begin{equation} \label{FinslerMetricCoefficients}
    g_{ij}(x,v) = \frac{1}{2} \frac{\partial^2 F^2}{\partial v^i \partial v^j}(x,v)
\end{equation}
%
is positive-definite. Then for each $x$ and $v$, the coefficients \eqref{FinslerMetricCoefficients} give rise to an inner product $\langle \cdot, \cdot \rangle_{(x,v)}$ on $T_x X$ which approximates the Finsler metric near $v$ up to second order.

%For a fixed tangent vector $T = (x,v)$, the coefficients $g_{jk}(x,v)$ then give rise to an inner product on $T_x M$ that approximates the Finsler metric up to second order at $(x,v)$, which we will denote by $g_T$. As alluded to before, Riemannian tools normally defined on the vector bundle $TM$ often have analogues in the setting of Finsler manifolds, but instead defined on the vector bundle $\mathscr{E} = \pi^*(TM)$ over $TM$. In particular, the $g_T$ form a smoothly varying inner product on the fibres of $\mathscr{E}$.

Given a metric $F$ on $X$, we define a dual metric $F_*: T^*X \to [0,\infty)$ by equipping $T_x^*X$ with the norm dual to the norm on $T_x X$, i.e. setting $F_*(x,\xi) = \sup \{ \xi(v) : F(x,v) = 1 \}$. Then $F_*$ is also strictly convex, in the sense that for an element $(x,\xi) \in T^* X - 0$, the Hessian matrix with entries
%
\begin{equation}
    g^{ij}(x,\xi) = \frac{1}{2} \frac{\partial^2 F_{\! \! *}^2}{\partial \xi_i \partial \xi_j}(x,\xi)
\end{equation}
%
is positive definite, and so the Finsler metric gives rise to an inner product on $T_x^* X$ for each $(x,\xi) \in T^* X$ which approximates $F_*$ up to second order.

If $X$ is a Finsler manifold, we define the \emph{Legendre transform} $\mathcal{L}: T X - 0 \to T^* X - 0$, by defining it's behaviour as a homogeneous diffeomorphism $\mathcal{L}_x$ from $T_x X - \{ 0 \} \to T^*_x X - \{ 0 \}$ on each fiber. We define the map so that for $v,w \in T_x X$, $\langle \mathcal{L}_x(v), w \rangle = \langle v, w \rangle_{(x,v)}$. In coordinates, $\mathcal{L}_x(v)_i = \sum\nolimits_j g_{ij}(x,v) v^j$. If $\xi = \mathcal{L}_x(v)$, then the matrix with entries $g_{ij}(x,v)$ is the inverse of the matrix with entries $g^{ij}(x,\xi)$, so that the inverse of $\mathcal{L}_x$ is given in coordinates by $\mathcal{L}_x^{-1}(\xi)^i = \sum\nolimits_j g^{ij}(x,\xi) \xi_j$. The Legendre transform is the Finsler variant of the musical isomorphism in Riemannian geometry, though the Legendre transform is in general only \emph{homogeneous}, rather than linear. % (TODO Special type of Finsler manifolds where the transform is linear).

Now let $P$ be an elliptic operator on a manifold $X$ satisfying Assumption A. The principal symbol of $P$ is a well-defined coordinate independent function $p: T^* M \to [0,\infty)$, and the following lemma applies.

\begin{lemma}
  Suppose $p: T^* M \to [0,\infty)$ is homogeneous, and for each $x \in M$, the cosphere $S_x^* = \{ \xi \in T_x^* M : p(x,\xi) = 1 \}$ has non-vanishing Gaussian curvature. Then the function
  %
  \[ F(x,v) = \{ \xi(v) : p(x,\xi) = 1 \} \]
  %
  is a Finsler metric on $M$, and the dual metric $F_*$ on $T^*M$ is equal to $p$.
\end{lemma}
\begin{proof}
  For each $x \in U_0$, the cosphere $S_x^* = \{ \xi \in T_x^* M : p(x,\xi) = 1 \}$ has non-vanishing Gaussian curvature. We claim that all principal curvatures of $S_x^*$ must actually be \emph{positive}. This follows from a simple modification of an argument found in Chapter 2 of \cite{HeinzHopf}. Indeed, if we fix an arbitrary point $v_0 \in T_x^*M$, and consider the smallest closed ball $B \subset T_x^* M$ centered at $v_0$ and containing $S_x^*$, then the sphere $\partial B$ must share the same tangent plane as $S_x^*$ at some point. All principal curvatures of $\partial B$ are positive, and at this point all principal curvatures of $S_x^*$ must be greater than the principal curvatures of $\partial B$, since $S_x^*$ curves away faster than $\partial B$ in all directions. By continuity, we conclude that the principal curvatures are everywhere positive.  Thus for each $x \in M$ and $\xi \in T_x^* M - \{ 0 \}$, the coefficients $g^{ij}(x,\xi) = (1/2) (\partial^2 p^2 / \partial \xi_i \partial \xi_j)$ form a positive-definite matrix. But inverting the procedure by which we constructed $F^*$ shows that the dual norm $F(x,v) = \sup\nolimits_{\xi \in S_x^*} \xi(v)$ is also strictly convex, and thus gives a Finsler metric on $M$ with $F_* = p$. %(Chapter 14 of \cite{BaoChern}). 
\end{proof}

 % and give rise to an inner product on $T_x M$ that best approximates the Finsler metric to second order at $(x,v)$.%, i.e. so that if $F(x,v) = F(x,w) = 1$, then
%
%\[ \left| F(x,w) - ( \sum g_{jk}(x,v) w^j w^k )^{1/2} \right| \lesssim |v - w|^3 \]

 %For each $v \in T_x M$, the coefficients $g_{jk}(x,v)$ define a Riemannian metric on $T_x M$ which approximates the Finsler metric to second order in a neighbourhood of $v$. % i.e. the functions $w \mapsto F(x,w)$ and $w \mapsto ( \sum g_{jk}(x,v) w^j w^k )^{1/2}$ agree up to second order in a neighborhood of $v$.

%The closest analogue to the Levi-Civita connection on Finsler manifolds is the \emph{Chern connection}. To discuss the connection, let $\bar{T}$ be the \emph{distinguished section} of $\mathscr{E}$, the section given by $\bar{T}(x,v) = F(v)^{-1} v$. For two sections $X$ and $Y$ of $\mathscr{E}$, the Chern connection gives a section $\nabla_X Y$ of $\mathscr{E}$ which is (a) $C^\infty(TM)$-linear in $X$ and $\RR$-linear in $Y$ (b) \emph{torsion free}, in the sense that $\nabla_X Y - \nabla_Y X = [X,Y]$, and (c) \emph{almost metric compatable} which is a somewhat technical to state fully, but for our purposes implies that for any sections $X$ and $Y$ and $Z$ of $\mathscr{E}$, when evaluating $X \{ g_{\bar{T}}(Y,Z) \}$ at a point in $TM$ where $X$, $Y$, or $Z$ is a multiple of $\bar{T}$, one has
%
% X { g_{Tbar}(Y,Z) } = g_{Tbar}( Nabla_X Y, Z ) + g_{Tbar} ( Y, Nabla_X Z ) + 2 C_{Tbar}(Nabla_X Tbar, Y, Z )
%
%\begin{equation} \label{almostmetriccompatibility}
%    X \{ g_{\bar{T}}(Y,Z) \} = g_{\bar{T}}(\nabla_X Y, Z) + g_{\bar{T}}(Y, \nabla_X Z).
%\end{equation}
%
%%Note that the Chern connection is \emph{not metric compatible}, i.e. \eqref{almostmetriccompatibility} does not hold for arbitrary $X$, $Y$, and $Z$. There does not exist a torsion free, metric compatible affine connection on the bundle $\mathscr{E}$ unless $M$ is a Riemannian manifold. As in Riemannian geometry, the value of $\nabla_X Y$ at some point $(x,v) \in TM$ depends only on the behaviour of $Y$ along some curve $c: I \to TM$ with $c(0) = (x,v)$ and $(\pi \circ c)'(0) = X(x,v)$, and we will sometime abuse notation by writing $\nabla_X Y$ if $Y$ is only defined along such a curve.

%The analogue of the Riemann curvature tensor is the \emph{first Chern curvature tensor} $R$, a $(1,3)$ tensor defined over $\mathscr{E}$ such for three sections $X$, $Y$, and $Z$ of $\mathscr{E}$, $R(X,Y) Z$ is the section defined by
%
%\begin{equation}
%    R(X,Y) Z = \nabla_X \nabla_Y Z - \nabla_Y \nabla_X Z - \nabla_{[X,Y]} Z.
%\end{equation}
%
%Given a section $X$ of $\mathscr{E}$, the \emph{flag curvature} $K(\bar{T},X)$ is defined by the formula
%
%\begin{equation}
%    K({\bar{T},X) = \frac{g_{\bar{T}}(R(X,\bar{T}) \bar{T}, X)}{g_{\bar{T}}(X,X)} - g_{\bar{T}}(X,\bar{T})^2},
%\end{equation}
%
%which generalizes the \emph{sectional curvature} from Riemannian geometry. By homogeneity, compactness, and continuity, on any compact Finsler manifold $M$, there exists constants $\delta$ and $\Delta$ such that for all sections $X$ of $\mathscr{E}$, $\delta \leq K(\bar{T},X) \leq \Delta$.

A Finsler metric gives a length to each tangent vector on the manifold, and can thus be used to define the lengths of curves $c: I \to U_0$ by the formula $L(c) = \int_I F(c,\dot{c})\; dt$. An analysis of length minimizing curves naturally leads to a theory of geodesics on a Finsler manifold. The theory of geodesics on Finsler manifolds is similar to the Riemannian case, except for the interesting quirk that a geodesic from a point $p$ to a point $q$ need not necesarily be a geodesic when considered as a curve from $q$ to $p$, and so we must consider \emph{forward} and \emph{backward} geodesics (the optimal path to descend a hill need not be the optimal path to ascend a hill, if your speed depends on the slope of the hill). We define the \emph{forward distance} $d_+: X \times X \to [0,\infty)$ by taking the infima of paths between points. This function is a \emph{quasi-metric}, as it satisfies the triangle inequality, but is not necessarily symmetric. We define the \emph{backward distance} $d_-: X \times X \to [0,\infty)$ by setting $d_-(p,q) = d_+(q,p)$. A \emph{constant speed speed forward geodesic} on a Finsler manifold is a curve $c: I \to U_0$ which, in coordinates, satisfies the geodesic equation
%$\nabla_{\bar{T}} \bar{T}$ vanishes when restricted the lifted curve $c': I \to TM$.
%
% Nabla_T T = T^a [partial_a T^b] e_b + T^a T^b N_b^c e_c
%           = T^a [ partial_a T^b + T^c N_c^b(T) ] e_b
%           = d^2c^a / dt^2 = - T^j T^k Gamma_{jk}^a
%
\begin{equation} \label{geodesicequation}
    \ddot{c}^a = - \sum\nolimits_{j,k} \gamma^a_{jk}(c,\dot{c}) \dot{c}^j \dot{c}^k \quad\text{for $1 \leq i \leq d$},
\end{equation}
%
where $\gamma^a_{jk} = \sum_i (g^{ai}/2) ( \partial_{x_k} g_{ij} + \partial_{x_j} g_{ik} - \partial_{x_i} g_{jk} )$ are the \emph{formal Christoffel symbols}. % defined by
%
%\[ \Gamma^a_{jk} = \sum\nolimits_i \frac{g^{ai}}{2} \left( \frac{\delta g_{ij}}{\delta x^k} + \frac{\delta g_{ik}}{\delta x^j} - \frac{\delta g_{jk}}{\delta x^i} \right), \]
%
%and where $\delta / \delta x^a = \partial / \partial x^a - \sum_b N^b_a \partial / \partial v^b$
%Such a curve is precisely one which is an extremal of the length functional $L$.
We call the reversal of such a geodesic a \emph{backward geodesic}. A curve $c$ is a geodesic if and only if it's Legendre transform $(x,\xi) = \mathcal{L}(c,\dot{c})$ satisfies
%
\begin{equation} \label{FStarHamilton}
    \dot{x} = (\partial_\xi F_*)(x,\xi) \quad\text{and}\quad \dot{\xi} = - (\partial_x F_*)(x,\xi),
\end{equation}
%
equations that will reoccur in our study of wave propogation, and will tell us that high frequency wave packet solutions to the equation $\partial_t u = 2 \pi i P$ travel along geodesics of the Finsler metric. Sufficiently short geodesics on a Finsler manifold are length minimizing. In particular, if $X$ is a compact Finsler manifold, then there exists $\varepsilon_X > 0$ such that if $c$ is a geodesic between two points $x_0$ and $x_1$ on $X$, and $L(c) < \varepsilon_X$, then $d_+(x_0,x_1) = L(c)$.

%The equations \eqref{FStarHamilton} are first order ordinary differential equations induced by the Hamiltonian vector field $( \partial_\xi F_*, - \partial_x F_* )$ on $T^* U_0$. Note that in our situation, the dual norm $F_*$ is the principal symbol $p$ of the operator $P$ we are studying, and so \eqref{FStarHamilton} is precisely the Hamiltonian flow alluded to in \eqref{qoiJIOJdoiwajfoiawjfoi}. Sufficiently short geodesics on a Finsler manifold are length minimizing. In particular, there exists $r > 0$ such that if $c$ is a geodesic between two points $x_0$ and $x_1$, and $L(c) < r$, then $d_M^+(x_0,x_1) = L(c)$.

%The analogue of the Riemann curvature tensor on a Finsler manifold is the \emph{$hh$-curvature tensor} $R$, a $(1,3)$ tensor defined on $E$ such that for three sections $X,Y,Z$ of $E$, $R(X,Y) Z$ is a section of $E$, which we define in the appendix. The analogue of sectional curvature in Finsler geometry is the \emph{flag curvature}. To define the curvature, let $T$ be the section of $E = \pi^*(TM)$ given by $T(x,v) = F(v)^{-1} v$, %and let $\omega$ be the section of $E^* = \pi^*(T^* M)$ given in coordinates by $\omega(x,v) = \sum (\partial F / \partial v^j)(x,v) dx^j$.
%This section is often called the \emph{distinguished section} in the literature. Given a section $X$ of $E$, we define the flag curvature $K(T,X)$ to be the function on $TM$ given by
%
%\[ K(T,X) = \frac{g(R(X,T) T, X)}{g(X,X) - g(X,T)^2} \]
%
%where $R$ is the \emph{$hh$-Curvature tensor}, a $(1,3)$ tensor defined on $E$ such that for three sections $X,Y,Z$ of $E$, $R(X,Y) Z$ is the section of $E$ given in coordinates by
%
%\[ R(X,Y) Z = \sum\nolimits_{i,j,k,l} Z^j X^k Y^l \left( \frac{\delta \Gamma^i_{jl}}{\delta x^k} - \frac{\delta \Gamma^i_{jk}}{\delta x^L} + \Gamma^i_{hk} \Gamma^h_{jl} - \Gamma^i_{hl} \Gamma^h_{jk} \right) \frac{\partial}{\partial x^i} \]
%
%

The last piece of technology we must consider before our analysis of critical points are \emph{variation formulas} for the length of curves on Finsler manifolds. Consider a function $A: (-\varepsilon,\varepsilon) \times [0,1] \to U_0$, such that for each $s$, the function $t \mapsto A(s,t)$ is a constant speed geodesic, and such that all such geodesics begin at the same point, i.e. so that $A(s,0)$ is independent of $s$. Define the vector field $V(s) = (\partial_s A)(s,1)$ and $T(s) = (\partial_t A)(s,1)$. Then if $L(s)$ gives the arclength of the geodesic $t \mapsto A(s,t)$, the first variation formula for energy tells us
%
\begin{equation} \label{FirstVariationFormula2}
    L'(s) = L(s)^{-1} \sum\nolimits_{i,j} g_{ij}( A(s,1), T(s) ) V^i(s) T^j(s).
\end{equation}
%
Details can be found in Chapter 5 of \cite{BaoChern}. We will consider \emph{second variations}, i.e. estimates on $L''$, but we will not need the second variation formula to obtain the estimates we need, relying only on differentiating the equation \eqref{FirstVariationFormula2} rather than explicitly working with Jacobi fields as is common in the analysis of second variations.

To conclude, we see that any elliptic operator $P$ on a manifold $M$ gives rise to a Finsler metric on $M$ which reflects the behaviour of the principal symbol of the operator. We will see this metric arises in the study of the wave equation $\partial_t u = 2 \pi i P u$ on $M$. In particular, high-frequency wave packet solutions to the wave equation travel along \emph{geodesics} in $M$, and so an analysis of the Finsler geometry will be necessary to understand the interactions of different wave packets, which will give us an analogue of Lemma \ref{lemma4} on compact manifolds for \emph{small time interactions} between the wave packets when Assumption A holds. We will control the large time behaviour for the wave equation via a reduction to the local smoothing inequality, which is sufficient to obtain an analogue of the results \cite{HeoandNazarovandSeeger} on manifolds, when Assumption B is true.

\section{An Analogue of the Radial Multiplier Conjecture on Compact Manifolds}

We now return to the study of spectral multipliers on compact manifolds. In a heuristic sense, one can think of the Fourier multipliers studied in the last Chapter as a limiting case of multipliers in the `high frequency limit', i.e. so that for any elliptic operator on a manifold $X$ with principal symbol $p(x_0,\xi)$ as $R \to \infty$, locally around $x_0 \in X$ the behaviour of the operators $P/R$ becomes more and more like the Fourier multiplier operator on $\RR^d$ with norm $p(x_0,\cdot)$ as $R \to \infty$. In particular, the following transference principle of Mitjagin \cite{Mitjagin} holds.

\begin{theorem}
  If $P$ is an elliptic operator on a compact manifold $X$ with principal symbol $p$, then for any regulated function $a$ and each $x_0 \in X$, $\| a(p(x_0,\cdot)) \|_{M^p(\RR^d)} \lesssim \| a \|_{M^p_{\text{Dil}}(X)}$.
\end{theorem}

Thus if $P$ satisfies Assumption A, it follows that
%
\[ M^p_{\text{Dil}}(X) \subset M^p(\RR^d) \subset R^{p,s}(\RR^d), \]
%
with $s = (d-1)(1/p - 1/2)$. We might conjecture that one can reverse this inclusion in the same range as for the radial multiplier conjecture, i.e. proving that $M^p_{\text{Dil}}(X) = R^{p,s}(\RR^d)$ for $1/p - 1/2 > 1/2d$. However, this is not possible on a general manifold, and in fact, fails in the full range whenever the geometry of $X$ induced by $P$ does not have constant sectional curvature. It is not clear what the range of the conjecture should be for an arbitrary manifold, though it is likely true that the conjecture holds on manifolds with constant sectional curvature. TODO EDIT THIS.

%Whether an analogous result remains true for more general Riemannian manifolds remains unclear, since the family of eigenfunctions to the Laplacian can take on various different forms on these manifolds, that can look quite different to the Euclidean case (TODO: Does the existence of low dimension Kakeya sets on certain manifolds show that the radial multiplier conjecture cannot be true in general). On general compact manifolds, there are difficulties arising from a generalization of the radial multiplier conjecture, connected to the fact that analogues of the Kakeya / Nikodym conjecture are false in this general setting \cite{Minicozzi}. But these problems do not arise for constant curvature manifolds, like the sphere. The sphere also has over special properties which make it especially amenable to analysis, such as the fact that solutions to the wave equation on spheres are periodic. Best of all, there are already results which achieve the analogue of \cite{GarrigosandSeeger} on the sphere. Thus it seems reasonable that current research techniques can obtain interesting results for radial multipliers on the sphere, at least in the ranges established in \cite{HeoandNazarovandSeeger} or even those results in \cite{Cladek}.

\section{Fourier Integral Operator Techniques}

For any operator $P$ to which we can establish a robust functional calculus, and for any regulated function $a$, the identity
%
\[ a(P) = \int_{-\infty}^\infty \widehat{a}(t) e^{2 \pi itP} \]
%
holds, obtained by plugging $P$ into the usual Fourier inversion formula. As $t$ varies, the operators $e^{2 \pi itP}$ solve the wave equation $\partial_t u = 2 \pi i P u$. Thus we can study multipliers of $P$ in terms of the wave equation $\partial_t u = 2 \pi i P u$. For a general operator $P$, it is difficult to understand the behaviour of the operator $e^{2 \pi i t P}$ given it's abstract definition via the functional calculus. However if $P$ is elliptic, then $e^{2 \pi i t P}$ is a \emph{Fourier integral operator}, which roughly means that the kernel of the operator is well approximated by oscillatory integral representation \emph{for high frequency inputs}. In this section, we briefly describe the relevant components we will need from the theory of Fourier integral operators for work in the latter part of the thesis.

To motivate the definition of Fourier integral operators, consider an operator $A$ from $\RR^m$ to $\RR^n$, whose kernel is a distribution given by an oscillatory integral of the form
%
\[ A(x,y) = \int_{\RR^p} s(x,y,\theta) e^{i \Phi(x,y,\theta)}\; d\theta, \]
%
where $s$ is a symbol of some order $\mu$, and $\Phi$ is smooth, and homogeneous of order one in the $\theta$ variable. If we localize this kernel in a small neighborhood of $x_0 \in \RR^n$ and $y_0 \in \RR^m$, defining $A_{x_0,y_0}(x,y) = \phi(x - x_0) A(x,y) \psi(y - y_0)$, then the principle of non-stationary phase tells us that for $|\xi_0| = |\eta_0| = 1$, we have rapid decay of the form
%
\[ |\widehat{A}_{x_0,y_0}(R\xi_0,R\eta_0)| \lesssim_N R^{-N}\quad \text{for all $N > 0$} \]
%
unless $(x_0,\xi_0,y_0,\eta_0)$ lies in the set
%
\[ \Lambda_\Phi = \Big\{ \big(x, \nabla_x \Phi(x,y,\theta) ,y, \nabla_y \Phi(x,y,\theta) \big) : \nabla_\theta \Phi(x,y,\theta) = 0 \Big\}. \]
%
We also consider the \emph{canonical relation}
%
\[ \mathcal{C}_\Phi = \Big\{ \big(x, \nabla_x \Phi(x,y,\theta) ,y, -\nabla_y \Phi(x,y,\theta) \big) : \nabla_\theta \Phi(x,y,\theta) = 0 \Big\}, \]
%
defined so that for wave packets $f_{y_0}(y) = \psi(y - y_0) e^{2 \pi i R \eta_0 \cdot y}$ and $g_{x_0}(x) = \phi(x - x_0) e^{2 \pi i R \xi_0 \cdot x}$,
%
\[ |\langle g_{x_0}, Af_{y_0} \rangle| \lesssim_N R^{-N}\quad \text{for all $N > 0$} \]
%
unless $(x_0,\xi_0,y_0,\eta_0) \in \mathcal{C}_\Phi$. The canonical relation of an oscillatory integral operator thus tells us the location where high-frequency wave packets are mapped in phase space. The main lesson of the theory of Fourier integral operators is that, to a large extent, the canonical relation \emph{determines} the behaviour of the operator $A$.

Under the assumption that the vectors $\partial_{\theta_1} \{ \nabla_{x,y,\xi} \Phi \}, \dots \partial_{\theta_N} \{ \nabla_{x,y,\xi} \Phi \}$ are everywhere linearly independent, $\mathcal{C}_\Phi$ is a smooth, $n + m$ dimensional manifold\footnote{It is actually a \emph{Lagrangian submanifold} of $T^* \RR^n \times T^* \RR^m$. Thus the canonical relation of any Fourier integral operator must be a Lagrangian submanifold, though we will only use this fact implicitly}, and then for $(x_0,\xi_0,y_0,\eta_0) \in \mathcal{C}_\Phi$ the principle of stationary phase implies that $|\langle g_{x_0}, Af_{y_0} \rangle| \lesssim R^{\mu + p/2 - (n+m)/2}$.
%
% R^{mu + p/2 - (n + m)/2} int chi(x,y,theta) e^{iR[ Phi(x,y,theta) + xi * x - y * eta ]}
% R^{mu +}
% Lambda = { (H'(xi), xi) }
% int chi(x,y,theta) e^{iR[ (x - y) * theta + xi * x - y * eta ]}
%
%
%
% Consider the submanifold of T^*R^2 defined by x = y and xi = eta
% Fix (0,0,1,0)
% The Lagrangian subspace tangent to the manifold is span( (1,1,0,0), (0,0,1,1) )
% Can we find a Lagrangian subspace transverse to this space and also span((0,0,1,0),(0,0,0,1))
% The graph of a linear map from R^2 -> (R^2)^*, i.e. span( (1,0,a,b), (0,1,b,c) )
%
% The intersection of this graph with the Lagrangian subspace above is trivial
% if a != c
% So we can take a = 1, b = c = d = 0
% span( (1,0,1,1), (0,1,0,0) )
%
%
% Define x = z + z^2/2
% and w = y
%
% Then dx = (1 + z) dz
% and dw = dy
%
% So x = y becomes (z + z^2/2) = w
% and dx = dy becomes (1 + z) dz = dw
% so z = A - 1 where A = dw / dz
% and so w = (A - 1) + (A - 1)^2 / 2 = A^2/2 - 1/2
%          = ((dw / dz)^2 - 1)/2
%
% dH = [(a/b) - 1] da + [(1/2)(a/b)^2 - 1/2] db
% H(a,b) = a^2/2b - a - b/2

% This is a section of T^* R^2, and hence can be written as df(x,y) = x (dx + dy)
% i.e. f(x,y) = x^2/2 + x
% for f(x,y) = x^2/2
% 
% Define z = x + x^2/2
% x = sqrt( 1 + 2z ) - 1
% and w = y
% Then dz = (1 + x) dx and dw = dy
% So dx = dy becomes dz = sqrt(1 + 2z) dw
% and x = y becomes 2z = w^2 + 2w
% So the manifold becomes { (z,w;zeta,omega) : zeta = sqrt(1 + 2z) omega, 2z = w^2 + 2w }
%    which is equal to  { (z,w;zeta,omega) : zeta = (w + 1) omega, 2z = w^2 + 2w }
% Near ( 0,0,1,0 ),
% w = zeta / omega - 1
% z = (zeta/omega)^2 / 2 + (zeta/omega)
% dH = (a/b - 1) da + [(a/b)^2 / 2 + (a/b)] db
% a^2/2b - a + cb

% So we can take Q(x,y) = x^2/2
%
% If we now define z = x + x^2/2
% and w = y
%
% Then x = sqrt( 1 + 2z ) - 1
% so x = y becomes z = w^2/2 + w 
% dz = (1 + x) dx and dw = dy
% dx = dy becomes dz = (1 + w) dw
% { z = w + w^2/2, dz = (1 + w) dw }
% 





% Given (0,0;xi,xi) -> (0,0;e_1,0)
% If x = w + (aw^2 + bwz + cz^2),
% then xi = omega + 2aw omega + bz omega + bw zeta + 2cz zeta
% C = { z = 0,  }

%To define Fourier integral operators, we must start by introducing some basic symplectic geometry. A \emph{symplectic manifold} is a manifold $Z$ whose tangent space $T Z$ is equipped with a smoothly varying non-degenerate alternating bilinear form $\omega$. For any manifold $X$, $T^*X$ is naturally a symplectic manifold; this is because for each $(x,\xi) \in T^* X$, $T_{(x,\xi)} T^* X$ can be naturally identified with $T_x X \times T^*_x X$, and we define $\omega = \sum d\xi_i \wedge dx^i$. A \emph{Lagrangian submanifold} of a symplectic manifold $Z$ is a submanifold $W$ of $Z$ such that, with respect to $\omega$, for each $z \in Z$, $(T_z W)^\perp = T_z W$.

%Lagrangian manifolds naturally occur when using stationary phase to analyze oscillatory integrals. Let $s$ be a symbol, and consider a kernel $K$ on $\RR^m \times \RR^n$ defined by an oscillatory integral of the form
%
%\[ I(x,y) = \int_{\RR^N} s(x,y,\theta) e^{i \Phi( x,y,\theta )}\; d\theta, \]
%
%where $\Phi$ is homogeneous in $\theta$ of order one. The principle of stationary phase tells us that the high-frequency behaviour of $I$ is determined by the stationary points of the integral $(x,\theta)$ where $\nabla_\theta \Phi(x,\theta) = 0$, and each such stationary point contributes oscillatory behaviour to $I$ near $x$ at a frequency $\nabla_x \Phi(x,\theta)$. Thus we should expect that if $\psi \in C_c^\infty(\RR^d)$ has suitably small enough support around a point $x_0$, then rapid decay estimates of the form $|\widehat{\psi I}(R \xi_0)| \lesssim_M R^{-M}$ should hold for all $M > 0$ unless $(x_0,\xi_0)$ lies in the set
%
%\[ \Lambda_\Phi = \{ (x,\xi): \nabla_\theta \Phi(x,\theta) = 0\ \text{and}\ \nabla_x \Phi(x,\theta) = \xi\ \text{for some $\theta$} \}, \]
%
%which is a Lagrangian submanifold of $T^* \RR^d$. This manifold is `where the high frequency parts of $I$ live'.

A Fourier integral operator is precisely an operator whose kernel is microlocally expressible in oscillatory integrals of the form above. Let $X$ be an $n$ dimensional manifold, and $Y$ an $m$ dimensional manifold. If $\mathcal{C}$ is a $n + m$ `Lagrangian' submanifold of $T^* X \times T^* Y$, we say an operator $A$ is a \emph{Fourier integral operator} with canonical relation $\mathcal{C}$ and order $\nu$ if, locally, the operator $A$ can, be expressed as a finite sum of oscillatory integral operators of the form above, with an amplitude given by a symbol $\mu$, a phase $\Phi$ in $N$ phase variables, with $\mathcal{C}_\Phi \subset \mathcal{C}$ and with $\nu = \mu - N/2 + (n+m)/4$.

The main examples of 

The advantage of this perspective is that the canonical relation is truely an `invariant' of the operator $A$, rather than the phase which is used to define the operator. The \emph{equivalence of phase theorem} says that for any phase $\Phi$ such that locally, $\mathcal{C}$ and $\mathcal{C}_\Phi$ agree around a pair of points $(x_0,\xi_0)$ and $(y_0,\eta_0)$

%The main lesson of the theory of Fourier integral operators is that to understand the properties of an oscillatory integral operator, one should not directly focus on properties of the phase of an oscillatory integral, but instead study properties of a geometric object derived from the phase of the integral, called the \emph{canonical relation} of the operator. A canonical relation is a `Lagrangian submanifold' $\mathcal{C}$ of $T^* X \times T^* Y$. A Fourier integral operator $A$ from a manifold $X$ to a manifold $Y$ associated with a canonical relation $\mathcal{C} \subset T^* X \times T^* Y$ is an operator such that for each $(x,\xi) \in T^* X$ and $(y,\eta) \in T^* Y$, the 

% locally, we can write the kernel of $A$ as a finite sum of distributions, each either expressible as an oscillatory integral of the form
%
%\[ \int_{\RR^N} s(x,y,\theta) e^{i \Phi(x,y,\theta)}, \]
%
%where $\Lambda_\Phi \subset \mathcal{C}$, or an element of $C^\infty(X \times Y)$.


\section{BLEH}

We will be able to control the behaviour of the operators $e^{2 \pi i t P}$ with respect to the geometry of $X$ by using the theory of Fourier integral operators, which allows us to approximate the high-frequency behaviour of the kernel of the operator $e^{2 \pi i t P}$ to a high degree of precision by an oscillatory integral.

Prove local regularity theorems for a class of Fourier integral operators in $I^\mu(Z,Y;\mathcal{C})$, where $Y$ is a manifold of dimension $n \geq 2$, and $Z$ is a manifold of dimension $n+1$, which naturally arise from the study of wave equations. A consequence of this result will be a local smoothing result for solutions to the wave equation, i.e. that if $2 < p < \infty$, then there is $\delta$ depending on $p$ and $n$, such that if $T: Y \to Y \times \RR$ is the solution operator to the wave equation, and $Y$ is a compact manifold whose geodesics are periodic, then $T$ is continuous from from $L^p_c(Y)$ to $L^p_{\alpha,\text{loc}}(Y \times \RR)$ for $\alpha \leq -(n-1)|1/2 - 1/p| + \delta$. Such a result is called local smoothing, since if we define $Tf(t,x) = T_tf(x)$, then the operator $T_t$ is, for each $t$, a Fourier integral operator of order zero, with canonical relation
%
\[ \mathcal{C}_t = \{ (x,y;\xi,\xi) : x = y + t \widehat{\xi} \}, \]
%
where $\widehat{\xi} = \xi / |\xi|$ is the normalization of $\xi$. Standard results about the regularity of hyperbolic partial differential equations show that each of the operators $T_t$ is continuous from $L^p_c(Y)$ To $L^p_{\alpha,\text{loc}}(Y \times \RR)$ for $\alpha \leq -(n-1)|1/2 - 1/p|$, and that this bound is sharp. Thus $T$ is \emph{smoothing} in the $t$ variable, so that for any $f \in L^p$, the functions $T_t f$ `on average' gain a regularity of $\delta$ over the worst case regularity at each time. The local smoothing conjecture states that this result is true for any $\delta < 1/p$.

The class of Fourier integral operators studied are those satisfying the following condition: as is standard, the canonical relation $\mathcal{C}$ is a conic Lagrangian manifold of dimension $2n + 1$. The fact that $\mathcal{C}$ is Lagrangian implies $\mathcal{C}$ is locally parameterized by $(\nabla_\zeta H(\zeta, \eta), \nabla_\eta H(\zeta, \eta),\zeta,\eta)$, where $H$ is a smooth, real homogeneous function of order one. If we assume $\mathcal{C} \to T^* Y$ is a submersion, then $D_\xi [\nabla_\eta H(\zeta,\eta)]$ has full rank, which implies $D_\eta [\nabla \xi H(\zeta, \eta)] = (D_\xi [\nabla_\eta H(\zeta, \eta)])^t$ has full rank, and thus the projection $\mathcal{C} \to T^* Z$ is an immersion. We make the further assumption that the projection $\mathcal{C} \to Z$ is a submersion, from which it follows that for each $z$ in the image of this projection, the projection of points in $\mathcal{C}$ onto $T^*_z Z$ is a conic hypersurface $\Gamma_z$ of dimension $n$. The final assumption we make is that all principal curvatures of $\Gamma_z$ are non-vanishing.

\section{Periodic Geodesics and Assumption B}

Using the Fourier integral operator methods above, we can prove that TODO

\section{Relation to Previous Work}

TODO: Lee Seeger exploited density decomposition to obtain local smoothing estimates for FIOs with cinematic curvature.

TODO: Kim exploited Sogge's Eigenfunction bounds to obtain optimal $L^2$ based multiplier bounds.

Large time control of the wave equation proved difficult.