%!TEX root = ../main.tex

Let $P$ be an elliptic linear operator on a compact manifold $M$, such that we can associate a functional calculus $a \mapsto a(P)$ for functions $a$ on the real line. In this thesis, we study the following question:  
%
\begin{changemargin}{2cm}{2cm}
\begin{center}
  \emph{What conditions on $a$ guarantee an operator $a(P)$ to be bounded.}
\end{center}
\end{changemargin}
%
Aside from testing our ability to understand the operator $P$ and the relations of it's eigenfunctions, the question has applications in the study of various partial differential equations associated with the operator $P$, and is closely related to the geometry of $M$, in particular, the way waves propogate and interact on the manifold.

Moreover, the study of multipliers on a manifold provides a useful setting with which to test and develop methods of harmonic analysis in a `variable-coefficient setting', where a lack of symmetry and translation invariance forces us to introduce more robust methods than are required in the Euclidean setting, i.e. as compared to studying multipliers of the Laplace operator on $\RR^d$ using the Fourier transform.

At present, there are many obstacles related to the global geometry of manifolds, which prevent an understanding of spectral multiplier operators on a general manifold. Over the years, sufficient conditions for boundedness had been established, but no characterizations of boundedness were known on any compact manifold, aside from the translation-invariant case when studying the Laplace operator on the torus. This thesis describes the first such characterizations beyond this setting, for a limited range of Lebesgue spaces, and on manifolds all of whose geodesics are closed and have common length. In particular, we find characterizations of functions $a$ whose dilates induce uniformly bounded multiplier operators for spherical harmonic expansions on $S^d$.

We obtain these results by expanding on techniques for understanding Fourier integral operators on manifolds. In Chapter TODO, we discuss the general theory of Fourier integral operators, as well as a discussion of the best results currently known for multipliers of the Laplacian $\Delta$ on $\RR^d$, which can also be described as \emph{radial Fourier multipliers on Euclidean Space}. In Chapter TODO, we give an exposition of methods more closely associated with the problem. Chapters TODO and TODO discuss our main contribution to the problem (TODO More Here). Chapter TODO discusses methods we hope to develop in the future.