%!TEX root = ../main.tex

\chapter*{Notation}

\begin{itemize}
    \item We use the normalization of the Fourier transform given by the formula
    %
    \[ \widehat{f}(\xi) = \int f(x) e^{- 2 \pi i \xi \cdot x}\; dx. \]
    %
    Overloading notation, for a function $f: [0,\infty) \to \CC$, for $\lambda \geq 0$ we write
    %
    \[ \widehat{f}(\lambda) = \int_0^\infty f(t) \cos(2 \pi \lambda t)\; dt, \]
    %
    for the cosine transform of the function $f$.

    \item For $p \in [1,\infty]$, we let $p' \in [1,\infty]$ be the dual exponent, with $1/p + 1/p' = 1$.

    \item We use the translation operators $\text{Trans}_y f(x) = f(x - y)$, and $L^\infty$-normalized dilations $\text{Dil}_t f(x) = f(x/t)$, chosen so that if a function $f$ is supported on a set $A$, then $\text{Trans}_y f$ is supported on $A + y$, and $\text{Dil}_t f$ is supported on $tA$. %Overloading notation, we also consider dyadic dilations of the form $\text{Dil}_j f(x) = f(x/2^j)$. The dyadic dilation operator will only be used along with symbols that conventionally stand for integers, like $n$, $m$, $j$, or $k$, whereas the other dilation operator will be used in all other cases, so the operator used should be clear from context.

    \item On $\RR^d$, we use $\partial_j$ to denote the usual partial derivative operators by the $j$th coordinate, and use $D_j$ to denote the self-adjoint normalization $D_j f = (2 \pi i)^{-1} \partial_j$. This normalization has the convenience that for a $d$-variate polynomial $P: \RR^d \to \CC$,
    %
    \[ P(D_j) \{ f \} = \int P(\xi) \widehat{f}(\xi) e^{2 \pi i \xi \cdot x}\; d\xi, \]
    %
    and simplifies many formulas associated with Fourier integral operators. We also consider the compositions of these operators $\partial^\alpha$ and $D^\alpha$ given by multi-indices $\alpha$.

    \item We will often use the Japanese bracket $\langle x \rangle = (1 + |x|^2)^{1/2}$ for $x \in \RR^d$.

    \item We let $\dot{\RR}^p = \RR^p - \{ 0 \}$. A function $f: {\RR^n} \times {\dot{\RR}^p} \to \CC$ is a \emph{symbol of order $s$} if it satisfies bounds of the form
    %
    \[ | \partial_x^\alpha \partial_\theta^\beta f (x,\theta) | \lesssim_{\alpha,\beta} \langle \theta \rangle^{s - |\beta|} \]
    %
    for all multi-indices $\alpha$ and $\beta$. More generally, symbols $s: \Gamma \to \CC$ can be defined for \emph{conical subsets} $\Gamma$ of $\RR^n \times \dot{\RR}^p$, i.e. sets $\Gamma$ with the property that if $(x,\theta) \in \Gamma$ and $\rho > 0$, then $(x,\rho \theta) \in \Gamma$. For a sequence of symbols $\{ a_k : k \geq 0 \}$, we write $a \sim \sum_{k = 0}^\infty a_k$ if, for any $l > 0$, there exists $N_0$ such that for $N \geq N_0$, $a - \sum_{k = 0}^N a_k$ is a symbol of order $s - l$. We view this relation as an asymptotic expansion of the function $a$.

    \item We let $\mathcal{S}(\RR^d)$ denote the Schwartz space of rapidly decreasing smooth functions on $\RR^d$, i.e. the functions $f: \RR^d \to \CC$ such that for any multi-index $\alpha$ and any $N > 0$, $|\partial^\alpha f(x)| \lesssim_{\alpha,N} \langle x \rangle^{-N}$.

    \item For a measure space $X$, $L^\infty(X)$ is the Banach space of essentially-bounded functions, defined almost everywhere. For a set $X$, $l^\infty(X)$ is the Banach space of bounded functions on $X$, defined everywhere. We also used mixed norm spaces; for a function $f(x,y)$ defined on the Cartesian product of two measure spaces $X$ and $Y$, we define
    %
    \[ \| f \|_{L^p(X) L^q(Y)} = \big\| \| f \|_{L^q(Y)} \big\|_{L^p(X)} = \left( \int\nolimits_X \left( \int\nolimits_Y |f(x,y)|^q\; dy \right)^{p/q}\; dx \right)^{1/p}. \]
    %
    Similarily, we can define the interchanged norm $L^q(Y) L^p(X)$ by swapping the order of application of the norms. It follows from Minkowski's inequality that if $p \geq q$ then $\| f \|_{L^p(X) L^q(Y)} \leq \| f \|_{L^q(Y) L^p(X)}$ and if $p \leq q$, then $\| f \|_{L^p(X) L^q(Y)} \geq \| f \|_{L^q(Y) L^p(X)}$.

    \item If $T$ is a Schwartz operator from a manifold $Y$ to a manifold $X$, we let $K_T$ denote the Schwartz distribution on $Y \times X$ which is the integral kernel for $T$, i.e. such that for any $f \in C_c^\infty(X)$ and $g \in C_c^\infty(Y)$,
    %
    \[ \int_X f(x) (Tg)(x)\; dx = \int_{X \times Y} K_T(x,y) f(x) g(y)\; dx\; dy. \]

%    All operators we consider are Schwartz operators between manifolds equipped with a canonical measure. Abusing notation, we thus identify an operator with the distribution that gives it's Schwartz kernel. Thus, for a Schwartz operator $A$ from a manifold $Y$ to a manifold $X$, we might write
    %
%    \[ A f(x) = \int_Y A(x,y) f(y)\; dy, \]
    %
%    where $A$ on the left hand side denotes the operator $A$, and the $A$ on the right hand side denotes the kernel.

    \item For $1 \leq p \leq \infty$ and $s \in \RR$, we consider the Sobolev spaces $W^{s,p}(\RR^d)$ with norm
    %
    \[ \| f \|_{W^{s,p}(\RR^d)} = \| (1-\Delta)^{s/2} f \|_{L^p(\RR^d)}, \]
    %
    and for $1 \leq r \leq \infty$, we consider the (homogeneous) Besov spaces
    %
    \[ \| f \|_{\dot{B}^{s,p}_r(\RR^d)} = \left( \sum\nolimits_{k \in \ZZ} \left[ 2^{ks} \| P_k f \|_{L^p(\RR^d)} \right]^r \right)^{1/r}, \]
    %
    where $\{ P_k \}$ are Littlewood-Paley projection operators, i.e. Fourier multiplier operators with compactly supported symbols $\chi( |\cdot| / 2^k )$, where $\chi$ is compactly supported and $\sum \chi(t/2^k) = 1$ for all $t > 0$. By working in coordinate systems, these definitions can also be used to define Sobolev and Besov spaces $W^{s,p}(X)$ and $\dot{B}^{s,p}_r(X)$ of functions on a compact manifold $X$.
\end{itemize}