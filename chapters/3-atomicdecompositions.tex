%!TEX root = ../main.tex

In this chapter, we discuss a prospective method of combining frequency scales for spectral multipliers, proving Theorem \ref{atomicscalestheorem}. We take an operator $P$ satisfying Assumptions A, B, and C, and show we can combine estimates between frequency scales under an assumption which is \emph{slightly stronger} than controlling each part of the operator separately. We do this by adapting the techniques of Heo, Nazarov, and Seeger from \cite{HeoandNazarovandSeeger2} to the case of compact manifolds, which we previous discussed in Section \ref{sec:combiningscaleswithatomicdecompositions}. We now restate the setup to Theorem \ref{atomicscalestheorem}.
    
Fix $k \geq 0$, and consider a pair of maximal $2^{-k}$ separated subsets $\mathcal{X}_k$ and $\mathcal{T}_k$ of $X$ and of $[0,\Pi]$, where $1/\Pi$ is the spacing between points in the arithmetic progression that the eigenvalues of the operator $P$ are contained in. Fix a family of $L^1$-normalized functions $\mathfrak{b} = \{ b_{t_0} \}$ and $\mathfrak{u} = \{ u_{x_0} \}$, with $u_{x_0}$ supported on a $2^{1-k}$ neighborhood of $x_0$, and $b_{t_0}$ on a $2^{1-k}$ interval centered at $t_0$. Also fix a bump function $q \in C_c^\infty(\RR)$ with $\supp(q) \subset [1/4,4]$ and $q(\lambda) = 1$ for $\lambda \in [1/2,2]$, and define $Q_k = q(P/2^k)$, which we view as `frequency localizations' at a scale $2^k$. For each $(x_0,t_0) \in \mathcal{X}_k \times \mathcal{T}_k$, define $S\!_{x_0,t_0} = \int b_{t_0}(t) (\cos(2 \pi i t P) \circ Q_k) \{ u_{x_0} \}$, a time average of a frequency localized solution to the wave equation $\partial_t^2 u = - P^2 u$. Finally, define an operator $A_k$ from functions on $\mathcal{X}_k \times \mathcal{T}_k$ to functions on $X$ by $A_k \{ c \} = \sum\nolimits_{(x_0,t_0) \in \mathcal{X}_k \times \mathcal{T}_k} c(x_0,t_0) S\!_{x_0,t_0}$. We assume uniform bounds on such operators.

\vspace{0.5em}

\noindent \fbox{\parbox{\textwidth}{\textbf{Assumption} $\text{Wave-Bound}(p)$:
There exists a constant $C_0 > 0$ such that
%
\begin{align*}
    \left\| A_k \{ c \} \right\|_{L^p(X)} \leq C_0\; 2^{k d/p'} \left( \sum\nolimits_{(x_0,t_0) \in \mathcal{X}_k \times \mathcal{T}_k} \left[ |c(x_0,t_0)| \langle 2^k t_0 \rangle^{s} \right]^p \right)^{1/p}.
\end{align*}
% but with 2^{kd/p'}          < 2^k t_0 >^{ (d-1)/2 }
%
uniformly in $k$, $\mathfrak{b}$, and $\mathfrak{u}$, with $s = (d-1)(1/p - 1/2)$.
%
%\begin{align*}
%    \left\| A_k \{ c \} \right\|_{L^{p_*}(X)} \leq C\; 2^{kd} \left( 2^{-k(2d+1)} \sum\nolimits_{(x_0,t_0) \in \mathcal{X}_k \times \mathcal{T}_k} |c(x_0,t_0)|^{p_*} \langle 2^k t_0 \rangle^{d-1} \right)^{1/p_*}.
%\end{align*}
% 2^{kd/p'} BUT PICK UP 2^{-k(d+1)}
% SO 2^{- k(1 + d/p)}
}}
\vspace{0.4em}

As we have seen in the previous chapter, such a bound naturally arises in the study of averages of the wave equation. In particular, $\text{Wave-Bound}(p,d)$ implies that for any function $a$, if $a_k = \chi(t) a(2^k t)$ for some $\chi \in C_c^\infty(\RR)$ with $1 = \sum \chi(\lambda/2^k)$ for $\lambda \neq 0$, then
%
\begin{equation}
    \| a_k(P/2^k) \|_{L^p(X) \to L^p(X)} \lesssim \| a \|_{R^{s,p}[0,\infty)} \quad\text{for $s = (d-1)(1/p - 1/2)$},
\end{equation}
%
uniformly in $k$. The main result of this section is that one can also `sum these bounds' to bound $a(P) = \sum a_k(P/2^k)$.

\thmatomicscalestheorem*

%Such a theorem implies a characterization of $L^p$ boundedness for rescaled spectral multipliers. Since $\text{Wave-Bound}(p,d)$ has been proved for $1/p - 1/2 > 1/2d$, in this range, this paper completes the proof of the characterization of $L^p$ boundedness for general multipliers.

%In order to study the behaviour of the multipliers $T_k = m_k(P/2^k)$, it is natural to represent the operator in terms of the wave equation on $M$, so that we can exploit geometric information about the behaviour of waves on $M$. We thus write
%
%\[ T_k = \int_0^\infty a_k(t) \cos(2 \pi t P), \]
%
%where $2^{-k} a_k(\cdot / 2^k)$ is the cosine transform of $m_k$. Given that $T_k$ is supported on the union of eigenspaces in the eigenband $[2^{k-1}, 2^{k+1}]$, so that the operator, and all inputs are `frequency localized' at a scale $2^k$, uncertainty principle heuristics suggest that it might be profitable to consider a further decomposition in spacetime at a scale $2^{-k}$; consider maximal $2^{-k}$ separated discrete subsets $\mathcal{X}_k$ and $\mathcal{T}_k$ of $M$ and $[0,\infty)$ respectively,  and consider an associated pair of partitions of unity $\{ \chi_{x_0} \}$ and $\{ \eta_{t_0} \}$ adapted to the balls of radius $2^{1-k}$ centered at the points in $\mathcal{X}_k$ and $\mathcal{T}_k$. Then given $u \in L^p(M)$, we can write
%
%\[ T_k u = \sum\nolimits_{(x_0,t_0) \in \mathcal{X}_k \times \mathcal{T}_k} c(x_0,t_0) f_{x_0,t_0}, \]
%
%where
%
%\[ f_{x_0,t_0} = \int_0^\infty a_{t_0} \cos(2 \pi t P) \{ u_{x_0} \}\; dt, \]
%
%where $a_{t_0}$ and $u_{x_0}$ are $L^1$-normalized multiples of $\eta_{t_0} a_k$ and $\chi_{x_0} u$ respectively, and $c(x_0,t_0) = \| \chi_{x_0} u \|_{L^1(M)} \| \eta_{t_0} a_k \|_{L^1(\RR)}$. One can verify that uniform bounds of the form
%
%\[ \| T_k \|_{L^p(M) \to L^p(M)} \lesssim C_p(m) \]
%
%follow if we could show an `$L^p$-cancellation' inequality of the form
%
%\[ \left\| \sum\nolimits_{x_0,t_0} \langle 2^k t_0 \rangle^{\frac{d-1}{2}} c(x_0,t_0) f_{x_0,t_0} \right\|_{L^p(M)} \lesssim 2^{k \beta(p,d)} \left( \sum\nolimits_{x_0,t_0} |c(x_0,t_0)|^p \langle 2^k t_0 \rangle^{d-1} \right)^{1/p}, \]
% ( Sum |chi_{x_0} u|_{L^1}^p )^{1/p} << 2^{-kd/p^*} |u|_{L^p}
% 
% ( Sum H^p W )^{1/p}
% ( (H W)^p W^{1 - p} )^{1/p}
%
% We have bounds on
%   int |m_k^(t)|^p <t>^{(d-1)(1 - p/2)}
%     = int |a_k(t)|^p <2^k t>^{(d-1)(1 - p/2)}
%     = Sum int |a_{t_0}(t)|^p <2^k t>^{(d-1)(1 - p/2)}
%     = Sum < 2^k t_0 >^{(d-1)(1 - p/2)} |a_{t_0}|_{L^p}^p
%     = Sum < 2^k t_0 >^{(d-1)(1 - p/2)} |a_{t_0}|_{L^1}^p 2^{k(p - 1)}
%   Thus (Sum |a_{t_0}|_{L^1}^p < 2^k t_0 >^{(d-1)(1 - p/2)} )^{1/p} << 2^{-k/p^*} C_p(m)
%
%
% ( Sum |a_{t_0}|^p < 2^k t_0 >^{d-1} )^{1/p}
%
%where $\beta(p,d) = (d + 1)/p^*$. In this paper, we will establish general bounds for multipliers, under an assumption that such a bound holds uniformly in $k$.

%\[ Tu = \sum\nolimits_k m_k( P / 2^k ) \{ u \}, \]
%
%where $\sup_k C_p(m_k) < \infty$, and $\text{supp}(m_k) \subset (1/2,2)$. It is conjectured that under these assumptions, for $|1/p - 1/2| > 1/2d$, the operator $T$ is bounded on $L^p(S^d)$. The goal of this paper is to obtain such bounds, under the assumption that certain bounds associated with the wave equation on $S^d$ hold uniformly at each frequency scale. The novelty in this paper is thus in obtaining a general method to efficiently combine bounds on operators at each frequency scale together.

% The finiteness of $C_p(m)$ is also necessary to control `the high frequency behaviour' of the function, in a certain sense. Namely, if the functions $\{ \chi_k \}$ are adapted to $(1/2,2)$, uniformly in $k$, then one can show that for $1/p - 1/2 > 1/2d$,
%
%\[ C_p(\text{Dil}_\rho m) \lesssim C_p(m). \]
%
%It is a result of Mitjagin that
%
%\[ \limsup\nolimits_{\rho \to \infty} \| (\text{Dil}_\rho m)(P) \|_{L^p(M) \to L^p(M)} \gtrsim C_p(m). \]
%
%Thus we conclude that
%
%\[ \limsup\nolimits_{\rho \to \infty} \| (\text{Dil}_\rho m)(P) \|_{L^p(M) \to L^p(M)} \sim \sup C_p(m). \]
%
%Thus we obtain necessary and sufficient conditions for $L^p$ boundedness `in the high frequency regime' (as $\rho \to \infty$).

Since the right hand side of the inequality in Theorem \ref{atomicscalestheorem} is invariant under dilations of $a$, it suffices to prove a bound of the form $\| a(P) \|_{L^p(X) \to L^p(X)} \lesssim \| a \|_{R^{s,q}[0,\infty)}$. To prove Theorem \ref{atomicscalestheorem}, we use an analogue of the technique of atomic decompositions introduced in Section \ref{sec:combiningscaleswithatomicdecompositions}. The following lemma describes the properties of this atomic decomposition useful to us, and is proved in an appendix.

\begin{lemma} \label{atomicdecompositionlemma}
    Consider coordinate charts $\{ U_\alpha \}$ covering $X$. Then, for any measurable function $u: X \to \CC$, if we consider the dyadic decomposition $u = \sum u_k$, where $u_k = Q_k u$ is a quasi-mode with eigenvalue $2^k$, then we have a further decomposition
    %
    \[ u_k = \sum\nolimits_\alpha \sum\nolimits_H \sum\nolimits_{W \in \mathcal{W}_{\alpha,H}} A_{\alpha,k,H,W}, \]
    %
    where $H$ ranges over powers of $2$. For each $\alpha$ and $H$, $\mathcal{W}_{\alpha,H}$ is a family of almost disjoint dyadic cubes in the coordinate system $U_\alpha$ whose union is a set $\Omega_{\alpha,H}$, such that the following properties hold:
    %
    \begin{itemize}
        \item The 10-fold dilates $\{ W^* : W \in \mathcal{W}_{\alpha,H} \}$ have the bounded overlap property.

        \item Let $l(W)$ denote the side-length of the cube $W$. If $l(W) = 2^l$, then $A_{\alpha,k,H,W} = 0$ for $k < -l$.

        \item For each $W$, $\text{supp}(A_{\alpha,k,H,W}) \subset W$, but as $H$ varies, the functions $\{ A_{\alpha,k,H,W} \}$ have almost disjoint support.

        \item For each $H$,
        %
        \[ \left( \sum\nolimits_k \sum\nolimits_\alpha \sum\nolimits_{W \in \mathcal{W}_{\alpha,H}} \| A_{\alpha,k,H,W} \|_{L^2(X)}^2 \right)^{1/2} \lesssim H |\Omega_{\alpha,H}|, \]

        \item For any choice of indices $k(\alpha,W)$ for each $\alpha$ and $W$, we have
        %
        \[ \left( \sum\nolimits_\alpha \sum\nolimits_{W \in \mathcal{W}_{\alpha,H}} |W| \| A_{\alpha,k(\alpha,W),H,W} \|_{L^\infty(X)}^p \right)^{1/p} \lesssim H |\Omega_{\alpha,H}|. \]

        \item For each $\alpha$,
        %
        \[ \left( \sum\nolimits_H H^p |\Omega_{\alpha,H}| \right)^{1/p} \lesssim \| u \|_{L^p(X)}. \]
    \end{itemize}
\end{lemma}
\begin{proof}
    We mostly adapt the approach of \cite{HeoandNazarovandSeeger2} to the compact manifold setting, using a decomposition phrased in terms of a square function of Peetre \cite{Peetre}. Define
    %
    \begin{equation}
        \mathfrak{S}u(x) = \left( \sum\nolimits_k \sup_{d(x,x') \leq 100 d 2^{-k} } |u(x')|^2 \right)^{1/2}.
    \end{equation}
    %
    Then, adapting a Euclidean result of Peetre \cite{Peetre}, Seeger \cite{SeegerEndpointEstimatesMultipliers} proved that $\mathfrak{S}$ is bounded on $L^p(X)$ for all $1 < p < \infty$.

    Consider a partition of unity $\{ \eta_\alpha \}$, with $\eta_\alpha$ supported on a pre-compact subset $V_\alpha$ of $U_\alpha$. For each dyadic $H$, define $\Omega_{\alpha,H} = \{ x \in V_\alpha : \mathfrak{S} u(x) > H \}$. In the coordinate system $U_\alpha$, define the set $\mathcal{Q}_{\alpha, H,k}$ to be the set of all dyadic cubes $Q$ of side-length $2^{-k}$ such that $|Q \cap \Omega_{\alpha,H}| \geq |Q| / 2$ but $|Q \cap \Omega_{\alpha,2H}| < |Q|/2$. Define
    %
    \begin{equation}
        \Omega_{\alpha,H}^* = \{ x \in V_\alpha : M \mathbb{I}_{\Omega_{\alpha,H}}(x) > 100^{-d} \}.
    \end{equation}
    %
    Then $\Omega_{\alpha,H}^*$ is open, and by the weak $L^1$ boundedness of the Hardy-Littlewood maximal function, $|\Omega_{\alpha,H}^*| \lesssim |\Omega_{\alpha,H}|$. We perform a Whitney decomposition of the  set $\Omega_{\alpha,H}^*$. Let $\mathcal{W}_{\alpha,H}$ be the set of all dyadic cubes $W$ which are maximal among cubes whose $50$-fold dilate of $W$ is contained in $\Omega_{\alpha,H}^*$. If we let $W^*$ denote the 10-fold dilate of $W$, then the sets $\{ W^* : W \in \mathcal{W}_{\alpha,H} \}$ have the bounded overlap property.
    
    For each $W \in \mathcal{W}_{\alpha,H}$, define
    %
    \begin{equation}
        A_{\alpha,k,H,W} = \eta_\alpha u_k \sum\nolimits_{\substack{Q \in \mathcal{Q}_{\alpha,H,k}\\Q \subset W}} \mathbb{I}_Q.
    \end{equation}
    %
    Each dyadic cube $Q \in \mathcal{Q}_{\alpha,H,k}$ is contained in some $W$, because the 50-fold dilate of $Q$ is contained in $\Omega_{\alpha,H}^*$. Since all dyadic side-length $2^{-k}$ cubes are contained in $\mathcal{Q}_{\alpha,H,k}$ for a unique $H$, modulo sets of measure zero, we have
    %
    \begin{equation}
        \eta_\alpha u_k = \sum \nolimits_H \sum\nolimits_{W \in \mathcal{W}_{\alpha,H}} A_{\alpha,k,H,W}.
    \end{equation}
    %
    We now verify each of the properties of the decomposition one by one, in the order stated:
    %
    \begin{itemize}
        \item The sets $\{ W^* \}$ have the bounded overlap property.

        \item This property follows immediately from the fact that $A_{\alpha,k,H,W}$ is a sum over side-length $2^{-k}$ cubes contained in $W$.

        \item The sets $\mathcal{Q}_{\alpha,H,k}$ are disjoint as $H$ varies, which implies the supports of the atoms $A_{\alpha,k,H,W}$ are disjoint as $H$ varies.

        \item We have a pointwise bound
    %
    \begin{equation}
        \sum\nolimits_\alpha \sum\nolimits_k \sum\nolimits_W |A_{\alpha,k,H,W}|^2 \lesssim H^2.
    \end{equation}
    %
    Indeed, if $x \in Q$ for some $Q \in \mathcal{Q}_{\alpha,H,k_0}$, then there exists a point $x' \in \Omega_{\alpha,H}$ with $d(x,x') \leq 100 d 2^{-k_0}$, and thus $\sum\nolimits_{k \leq k_0} |u_k(x)|^2 \leq \mathfrak{S} f(x') \leq H$. But if $k_0 \leq \infty$ is the suprema of such $k_0$ for which we can find such a cube $Q$, then because $x$ is supported on $O(1)$ of the functions $A_{\alpha,k,H,W}$ for each fixed $k$, we obtain that
    %
    \begin{equation}
        \sum\nolimits_\alpha \sum\nolimits_k \sum\nolimits_W |A_{\alpha,k,H,W}|^2 \lesssim \sum\nolimits_{k \leq k_0} |u_k(x)|^2 \leq H^2.
    \end{equation}
    %
    Integrating this pointwise bound over $\Omega_{\alpha,H}^*$ gives the relevant $L^2$ estimates.

    \item The pointwise estimates of the last section imply that $|A_{\alpha,k,H,W}| \lesssim H$, and since the supports of the cubes in $\mathcal{W}_{\alpha,H}$ are disjoint for each $H$ and have $\Omega_{\alpha,H}$ as their union, the required estimate follows immediately.

    \item The last property automatically follows from the $L^p$ boundedness of $\mathfrak{S}$.
    \end{itemize}
    %
    We have verified all properties stated by the lemma, which completes the proof.
%    Without loss of generality, by performing a partition of unity and then working in one of the coordinate systems $U_\alpha$, we may assume $u$ is a function on $\RR^d$. But then the decomposition is precisely that of Section 7 of \cite{HeoandNazarovandSeeger}, i.e. using the $L^p$ boundedness of a square function of Peetre \cite{Peetre}, and the level sets of that square function applied to $u$, in order to form the atoms.
\end{proof}

To exploit this atomic decomposition, we write
%
\begin{equation}
    a(P) u = \sum\nolimits_k m_k(P/2^k) \{ u_k \} = \sum\nolimits_\alpha \sum\nolimits_k \sum\nolimits_H \sum\nolimits_{W \in \mathcal{W}_{\alpha,H}} m_k(P/2^k) \{ A_{\alpha,k,H,W} \}.
\end{equation}
%
We regroup this sum as
%
\begin{equation}
    \sum\nolimits_\alpha \sum\nolimits_k \sum\nolimits_H \sum\nolimits_{l \geq 0} \sum\nolimits_{\substack{W \in \mathcal{W}_{\alpha,H}\\l(W) = l - k}} a_k(P/2^k) \{ A_{\alpha,k,H,W} \}.
\end{equation}
%
For each $k$ and $l$, we write $a_k(P/2^k) = T_{k,l,\text{Short}} + T_{k,l,\text{Long}}$, where
%
\begin{equation}
    T_{k,l,\text{Short}} = \int_0^\infty \chi( 2^{k-l} t ) 2^k\;\! \widehat{a}_k(2^k t) \cos(2 \pi i t P)
\end{equation}
%
and
%
\begin{equation}
    T_{k,l,\text{Long}} = \int_0^\infty (1 - \chi(2^{k-l} t )) 2^k\;\! \widehat{a}_k(2^k t) \cos(2 \pi i t P). 
\end{equation}
%
For $f = Tu$, we thus write $f = f_{\text{Short}} + f_{\text{Long}}$, where
%
\begin{equation}
    f_{\text{Short}} = \sum\nolimits_\alpha \sum\nolimits_k \sum\nolimits_H \sum\nolimits_{l \geq 0} \sum\nolimits_{\substack{W \in \mathcal{W}_{\alpha,H}\\l(W) = l - k}} T_{k,l,\text{Short}} \{ A_{\alpha,k,H,W} \} = \sum f_{\alpha,k,H,W,\text{Short}}. 
\end{equation}
%
and
%
\begin{equation}
    f_{\text{Long}} = \sum\nolimits_\alpha \sum\nolimits_k \sum\nolimits_H \sum\nolimits_{l \geq 0} \sum\nolimits_{\substack{W \in \mathcal{W}_{\alpha,H}\\l(W) = l - k}} T_{k,l,\text{Long}} \{ A_{\alpha,k,H,W} \} = \sum f_{\alpha,k,H,W,\text{Long}},
\end{equation}
%
and analyze each part using separate techniques.

\section{Short Range Bounds}

To obtain short range bounds, we exploit the propagation speed of the operators $\cos(2 \pi i t P)$, and the bounded overlap of the sets $W_{\alpha,H}$ for a fixed $H$. To obtain an $L^q$ bound for $f_{\text{Short}}$, we interpolate between an $L^2$ bound and an $L^1$ bound, at a fixed quantity $H$. So write $f_{\text{Short}} = \sum_\alpha \sum_k \sum_H f_{\alpha,k,H}$.
%To obtain $L^2$ estimates, it becomes necessary to estimate the inner products $\langle f_{\alpha,k,H,W}, f_{\alpha,k,H,W'} \rangle$.
%
%\[ \sum\nolimits_{W' \cap W^* = \emptyset} \langle f_{\alpha,k,H,W}, f_{\alpha,k,H,W'} \rangle \]
In $L^2$, different frequencies are orthogonal, so that
%
\begin{equation}
\begin{split}
    \| f_{H,\text{Short}} \|_{L^2(X)} \lesssim \left( \sum\nolimits_k \left\| \sum\nolimits_{W \in \mathcal{W}_{\alpha,H}} f_{\alpha,k,H,W,\text{Short}} \right\|_{L^2(X)}^2 \right)^{1/2}.
\end{split}
\end{equation}
%
We note that by the finite propagation speed of $\cos(2 \pi t P)$, $\langle f_{\alpha,k,H,W,\text{Short}}, f_{\alpha,k,H,W',\text{Short}} \rangle = 0$ if $W^* \cap (W')^* = \emptyset$. By the bounded overlap property of the sets $\{ W^* \}$, these functions are almost orthogonal, and so we conclude that
%
\begin{equation}
    \| f_{H,\text{Short}} \|_{L^2(X)} \lesssim \left( \sum\nolimits_k \sum\nolimits_{W \in \mathcal{W}_{\alpha,H}} \left\| f_{\alpha,k,H,W,\text{Short}} \right\|_{L^2(X)}^2 \right)^{1/2}.
\end{equation}
%
We can write $T_{k,l,\text{Short}} = a_{k,l}(P / 2^k)$, where
%
\begin{equation}
    a_{k,l}(t) = [2^l \widehat{\chi}(2^l \cdot) * a_k].
\end{equation}
%
Now
%
\begin{equation}
    \| a_{k,l}(P/2^k) \|_{L^2(X) \to L^2(X)} = \| a_{k,l} \|_{L^\infty(\RR)} \lesssim \| a_k \|_{L^\infty(\RR)} \lesssim \| a \|_{R^{s,p}[0,\infty)}, 
\end{equation}
%
and so
%
\begin{equation}
    \| f_{\alpha,k,H,W} \|_{L^2(X)} = \| T_{k,l,\text{Short}} \{ A_{\alpha,k,H,W} \} \|_{L^2(X)} \lesssim \| a \|_{R^{s,p}[0,\infty)} \| A_{\alpha,k,H,W} \|_{L^2(X)}.
\end{equation}
%
So
%
\begin{equation}
    \| f_{H,\text{Short}} \|_{L^2(X)} \lesssim \| a \|_{R^{s,p}[0,\infty)} H |\Omega_H|^{1/2}.
\end{equation}
%
By the finite propagation speed of the wave equation, $f_{H,\text{Short}}$ is supported on the set $\{ x: (M \chi_{\Omega_H})(x) \geq 1/10^d \}$, where $M$ is the Hardy-Littlewood maximal function; by the weak $L^1$ boundedness of $M$, this set has measure $O(|\Omega_H|)$. Thus we can use H\"{o}lder's inequality to conclude that for any $r \in [1,2]$,
%
\begin{equation}
    \| f_{H,\text{Short}} \|_{L^1(X)} \lesssim \| a \|_{R^{s,p}[0,\infty)} |\Omega_H|^{1/2} H |\Omega_H|^{1/2} = \| a \|_{R^{s,p}[0,\infty)} H |\Omega_H|.
\end{equation}
%
Performing a real interpolation between $r = 1$ and $r = 2$, which allows us to sum over the dyadic height scales $H$, we conclude that
%
\begin{equation}
    \| f_{\text{Short}} \|_{L^p(X)} \lesssim  \| a \|_{R^{s,p}[0,\infty)} \left( \sum H^p |\Omega_H| \right)^{1/p} \lesssim \| a \|_{R^{s,p}[0,\infty)} \| u \|_{L^p(X)}.
\end{equation}
%
This completes the analysis of the short range interactions.

\section{Long Range Bounds}

The long range interactions require slightly more work. Define an operator
%
\begin{equation}
\begin{split}
    S_{l,k} C = \sum\nolimits_{z_0} \sum\nolimits_{t_0 \geq 2^{l-k}} C(z_0,t_0) \sum\nolimits_{x_0 \in Q(z_0)} F_{x_0,t_0},
\end{split}
\end{equation}
%
where $U_{z_0} = \sum_{x_0 \in Q(z_0)} u_{x_0}$ and $F_{x_0,t_0} = \int b_{t_0}(t) \cos(2 \pi t P) \{ U_{z_0} \}\; dt$ for $L^1$ normalized functions $\{ b_{t_0} \}$. The bound $\text{Wave-Bound}(q)$ implies an exponential decay in $l$ on the $L^p$ operator norm of $S_{l,k}$.

\begin{lemma} \label{lemma:scaleupbound}
    Suppose $1 \leq p \leq q$, and that $\text{Wave-Bound}(q)$ is true. Then
    % |b_{t_0}|_{L^1} << 2^{-k/q}
    \begin{align*}
        &\| S_{l,k} \{ C \} \|_{L^p(X)}  \lesssim 2^{- l \varepsilon} \left( \sum\nolimits_{z_0} \sum\nolimits_{t_0 \geq 2^{l-k}} \left[ \langle 2^k t_0 \rangle^s |C(z_0,t_0)| 2^{(l-k)d/p} \| U_{z_0,t_0} \|_{L^\infty(X)} \right]^p \right)^{1/p},
    \end{align*}
    %
    where $s = (d-1)(1/p - 1/2)$, and
    %
    \[ \varepsilon = - \left( \frac{d-1}{2} \right) \left( \frac{1/p - 1/q}{1 - 1/q} \right), \]
    %
    which, in particular, is positive for $p < q$.
\end{lemma}
\begin{proof}
    By applying the triangle inequality, we may assume that the support of $C$ is $10$-separated in the $z_0$ variable. Thus any point $x_0$ is contained in a unique side-length cube $Q(z(x_0))$ with $z(x_0)$ in $\text{supp}_z(C)$, or is not contained in any such cube. If we define $c(x_0,t_0) = C(z(x_0),t_0) \| u_{x_0} \|_{L^1(X)}$, then we can apply $\text{Wave-Bound}(q)$ and H\"{o}lder's inequality to conclude that
    %
    \begin{equation}
    \begin{split}
      \| S_{l,k} C \|_{L^q(X)} \lesssim \left( \sum\nolimits_{x_0} \sum\nolimits_{t_0 \geq 2^{l-k}} \left[ |C(z(x_0),t_0)| \| u_{x_0} \|_{L^q(X)} \langle 2^k t_0 \rangle^{s(q)} \right]^q \right)^{1/q},
    \end{split}
    \end{equation}
    %
    % H^qW^q
    % vs H^q W
    % << 1/R^{1-1/q}
    where $s(q) = (d-1)(1/q - 1/2)$. Since the supports of the functions $\{ u_{x_0} \}$ are almost disjoint, and since $U_{z_0}$ is supported on a set of measure $2^{(l-k)d}$,
    %
    \begin{equation}
        \sum\nolimits_{z(x_0) = z_0} \| u_{x_0} \|_{L^q(X)}^q \lesssim \| U_{z_0} \|_{L^q(X)}^q \lesssim 2^{(l-k)d} \| U_{z_0} \|_{L^\infty(X)}^q.
    \end{equation}
    %
    %
    Substituting this bound into the prior bound yields that
    %
    \begin{equation}
    \begin{split}
        \| S_{l,k} C \|_{L^q(X)} &\lesssim \left( \sum\nolimits_{z_0} \sum\nolimits_{t_0 \geq 2^{l-k}} \left[ |C(z_0,t_0)| \big[ 2^{(l-k)d/q} \| U_{z_0} \|_{L^\infty(X)} \big] \langle 2^k t_0 \rangle^{s(q)} \right]^q \right)^{1/q}.
    \end{split}
    \end{equation}
    % H^p W^{p/2}
    % vs. H^p W
    % 2^{k(d - (2d+1)/p)}
    % Provided that epsilon(q) <= -1/q' we're fine
    %       epsilon(1) = 0 WHICH IS GOOD
    %       epsilon(p_*) = d - (2d+1)/p_*
    % d - (2d+1)/p_* <= -1/p_*'
    % Holds iff
    %   d - (2d+1)/p_* <= -1 + 1/p_*
    % p_* <= 2
    We will interpolate this bound with an $L^1$ bound with exponential decay, which will yield the result. We should expect $F_{t_0,z_0}$ to be concentrated on a set of measure $O( 2^{l-k} t_0^{d-1} )$, namely, the annulus $\text{Ann}_{t_0,z_0}$ of width $O(2^{l-k})$ upon a sphere of radius $t_0$ centered at $z_0$. By H\"{o}lder's inequality, we find that
    % Xaybe use F = T^I_{x_0} U_{z_0} = \sum f_{x_0,t_0}
    \begin{equation}
    \begin{split}
        \| F_{t_0,z_0} \|_{L^1(\text{Ann}_{t_0,z_0})} &\lesssim \left( 2^{l-k} t_0^{d-1} \right)^{1/2} \| F_{t_0,z_0} \|_{L^2(X)}\\
        &\lesssim 2^{\frac{l-k}{2}} t_0^{\frac{d-1}{2}} \| U_{z_0} \|_{L^2(X)}\\
        &\lesssim 2^{(l-k) \left( \frac{d+1}{2}\right)} t_0^{\frac{d-1}{2}} \| U_{z_0} \|_{L^\infty(X)}.
    \end{split}
    \end{equation}
    %
    Here we used the fact that $F_{t_0,z_0} = \int b_{t_0}(t) ( \cos(2 \pi t P) \circ Q_k ) \{ U_{z_0} \}$, that $\| b_{t_0} \|_{L^1(\RR)} \leq 1$,  and that $\| \cos(2 \pi t P) \circ Q_k \|_{L^2(X) \to L^2(X)} \lesssim 1$. so that we can apply the triangle inequality to conclude that $\| T_{t_0}^I \|_{L^2(X) \to L^2(X)} \lesssim \| b_{t_0} \|_{L^1(X)} \leq 1$. On the other hand, we can use Lemmas \ref{PseudoOsicllatoryLemma} and \ref{lemma:WaveOscillatoryLemmaddw} to prove that $(\cos(2 \pi t P) \circ Q_k)(x,y) \lesssim_N [2^k |d(x,y) - t|]^{-N}$, uniformly for $|t| \lesssim 1$, and thus that
    %
    \begin{equation}
        \| F_{t_0,z_0} \|_{L^1(\text{Ann}_{t_0,z_0}^c)} \lesssim_N 2^{-lN} 2^{-k} t_0^{d-1} \| U_{z_0} \|_{L^1(X)} \lesssim 2^{-lN} 2^{-k(d+1)} t_0^{d-1} \| U_{z_0} \|_{L^\infty(X)}.
    \end{equation}
    %
    Since $t_0 \geq 2^{l-k}$, we have
    %
    \begin{equation}
        \| F_{t_0,z_0} \|_{L^1(\text{Ann}_{t_0,z_0}^c)} \lesssim_N 2^{-lN} 2^{-k \left( \frac{d+3}{2} \right)} t_0^{\frac{d-1}{2}} \| U_{z_0} \|_{L^\infty(X)},
    \end{equation}
    %
    which is smaller than the $L^1$ norm bound on $\text{Ann}_{t_0,z_0}$.
    %
    Summing in $t_0$ and $z_0$ using the triangle inequality gives that
    %
    \begin{equation}
    \begin{split}
        &\| S_{l,k} C \|_{L^1(X)}\\
        &\quad \lesssim 2^{(l-k) \left( \frac{d+1}{2}\right)} \sum\nolimits_{z_0} \sum\nolimits_{t_0 \geq 2^{l-k}} t_0^{\frac{d-1}{2}} |C(z_0,t_0)| \| U_{z_0} \|_{L^\infty(X)}\\
        &\quad \lesssim 2^{-l \left( \frac{d-1}{2} \right)} \sum\nolimits_{z_0} \sum\nolimits_{t_0 \geq 2^{l-k}} |C(z_0,t_0)| \left[  2^{(l-k)d} \| U_{z_0} \|_{L^\infty(X)}\right] \langle 2^k t_0 \rangle^{\frac{d-1}{2}}.
    \end{split}
    \end{equation}
    %
    % 2^{k (d-1)(2)}
    %
    The argument is concluded by interpolation.
    % -l(d-1)/2
    %
\end{proof}

We now use this lemma to control the function $f_{\text{Long}}$. Because of the exponential decay in $l$ given by the lemma above, we may sum in $l$ trivially using the triangle inequality for $1 \leq q < p$. Using $L^2$ orthogonality, if $f_{\text{Long}} = \sum_k f_{\text{Long},k}$, then we find that
%
\begin{equation}
    \left\| \sum\nolimits_k f_{\text{Long},k} \right\|_{L^p(X)} = \left( \sum \| f_{\text{Long},k} \|_{L^p(X)}^p \right)^{1/p}.
\end{equation}
%
Now write $f_{\text{Long},k} = \sum f_{\text{Long},\alpha,l,k}$ with
%
\begin{equation}
    f_{\text{Long},\alpha,l,k} = \sum\nolimits_H \sum\nolimits_{\substack{W \in \mathcal{W}_{\alpha,H}\\l(W) = l - k}} T_{k,l,\text{Long}} \{ A_{\alpha,k,H,W} \}.
\end{equation}
%
Applying Lemma \ref{lemma:scaleupbound} with $l$ and $k$ as above, $C(z_0,t_0) =  \| b_{t_0} \|_{L^1(X)}$, and with $U_{z_0} = \sum\nolimits_H A_{\alpha,k,H,Q(z_0)}$, we conclude that %where $W = [z_0, z_0 + 2^{l-k}]$ we conclude that
% W^{1/p - 1}
\begin{equation}
\begin{split}
    \| f_{\text{Long},\alpha,l,k} \|_{L^p(X)} &\lesssim 2^{-l \varepsilon} \left( \sum\nolimits_{z_0} \sum\nolimits_{t_0 \geq 2^{l-k}} [ \langle 2^k t_0 \rangle^{s} \| b_{t_0} \|_{L^1(X)} ]^p \big[ 2^{(l-k)d} \| U_{z_0} \|_{L^\infty(X)}^p \big] \right)^{1/p}\\
    &\lesssim \| a \|_{R^{s,p}[0,\infty)} 2^{-l \varepsilon} \left(  \sum\nolimits_H \sum\nolimits_{\substack{W \in \mathcal{W}_{\alpha,H}\\l(W) = l-k}} |W| \| A_{\alpha,k,H,W} \|_{L^\infty(X)}^p \right)^{1/p}.
\end{split}
\end{equation}
% L^1 = H W
% L^p = 
Summing in $k$ using $L^p$ orthogonality, and summing over $\alpha$ trivially, we find that
%
\begin{equation}
    \| f_{\text{Long},l} \|_{L^p(X)} \lesssim 2^{-l \varepsilon}  \| a \|_{R^{s,p}[0,\infty)} \left( \sum\nolimits_H H^p |\Omega_H| \right) \lesssim 2^{-l \varepsilon}  \| a \|_{R^{s,p}[0,\infty)} \| u \|_{L^p(X)}. 
\end{equation}
%
Summing in $l$ trivially gives $\| f_{\text{Long}} \|_{L^p(X)} \lesssim \| a \|_{R^{s,p}[0,\infty)} \| u \|_{L^p(X)}$, completing the proof of the long range estimates, and thus the proof.

\section{Wave Bounds on the Sphere}

We end this chapter with a proof that $\text{Wave-Bound}(p)$ holds on $S^d$ for $1/p - 1/2 > 1/(d-1)$, which completes the proof of Theorem \ref{maintheoremsphere}.

\begin{lemma}
    The operator $P_{\text{SH}}$ satisfies $\text{Wave-Bound}(p)$ for $1/p - 1/2 > (d-1)^{-1}$.
\end{lemma}
\begin{proof}
    Here $\Pi = 1$. If $\varepsilon > 0$, and a given function $c$ satisfies $c(x_0,t_0) = 0$ for $t_0 \in [1/2 - \varepsilon, 1/2 + \varepsilon]$, then the bound has already been proven in this paper; indeed, it follows from Lemma \ref{LpBoundLemma}, where $R = 2^k$, since for $t_0 \leq 1/2$ we can write $2 \cos(2 \pi t P) = e^{2 \pi i t P} + e^{-2 \pi i t P}$, and for $t_0 \geq 1/2$ we can write $2 \cos(2 \pi t P) = e^{2 \pi i (t - 1) P} + e^{-2 \pi i (t - 1) P}$. Applying Lemma \ref{LpBoundLemma} to each piece of the exponential, we have that, if $f_{x_0,t_0}$ is as in the definition of the assumption $\text{Wave-Bound}(p)$,
    %
    \begin{equation}
    \begin{split}
        &\left\| \sum\nolimits_{\substack{(x_0,t_0) \in \mathcal{X}_k \times \mathcal{T}_k\\|t_0| \leq 1/2 - \varepsilon}} c(x_0,t_0) f_{x_0,t_0} \right\|_{L^p(S^d)} \lesssim 2^{kd/p'} \left( \sum\nolimits_{(x_0,t_0) \in \mathcal{X}_k \times \mathcal{T}_k} \Big[ |c(x_0,t_0)| \langle R t_0 \rangle^{s} \Big]^p \right)^{1/p}
    \end{split}
    \end{equation}
    %
    and
    %
    \begin{equation}
    \begin{split}
        &\left\| \sum\nolimits_{\substack{(x_0,t_0) \in \mathcal{X}_k \times \mathcal{T}_k\\|t_0| \geq 1/2 + \varepsilon}} c(x_0,t_0) f_{x_0,t_0} \right\|_{L^p(S^d)} \\
        &\quad\quad \lesssim 2^{kd/p'} \left( \sum\nolimits_{\substack{(x_0,t_0) \in \mathcal{X}_k \times \mathcal{T}_k\\|t_0| \geq 1/2 + \varepsilon}} \Big[ |c(x_0,t_0)| \langle R (t_0 - 1) \rangle^{s} \Big]^p \right)^{1/p}\\
        &\quad\quad \lesssim 2^{kd/p'} \left( \sum\nolimits_{\substack{(x_0,t_0) \in \mathcal{X}_k \times \mathcal{T}_k\\|t_0| \geq 1/2 + \varepsilon}} \Big[ |c(x_0,t_0)| \langle R t_0 \rangle^{s} \Big]^p \right)^{1/p}.
    \end{split}
    \end{equation}
    %
    It remains to prove the result for $c$ supported on $[1/2 - \varepsilon, 1/2 + \varepsilon]$. To obtain this result, if $Ux = -x$ is the reflection operator on $S^d$, then for $t \in [1/2 - \varepsilon, 1/2 + \varepsilon]$, for small $t$ the operators $U \circ e^{2 \pi i (t + 1/2) P}$ have the same conical relation as $e^{2 \pi i t P}$, and so writing out these operators using oscillatory integrals following the proof of Proposition \ref{theMainEstimatesForWave}, one can prove that the operators $S\!_{x_0,t_0} = U \circ \int b_{1/2 + t_0} e^{2 \pi i t P} \{ u_{x_0} \}$ satisfy the pointwise and orthogonality estimates of Proposition \ref{theMainEstimatesForWave}. Thus we can apply Lemma \ref{LpBoundLemma} to obtain bounds of the form
    %
    \begin{equation}
    \begin{split}
        &\left\| \sum\nolimits_{\substack{(x_0,t_0) \in \mathcal{X}_k \times \mathcal{T}_k\\t_0 \in [1/2 - \varepsilon, 1/2 + \varepsilon]}} \langle 2^k t_0 \rangle^{\frac{d-1}{2}} c(x_0,t_0) f_{x_0,t_0} \right\|_{L^p(S^d)} \\
        &\quad\quad \lesssim 2^{kd/p'} \left( \sum\nolimits_{\substack{(x_0,t_0) \in \mathcal{X}_k \times \mathcal{T}_k\\|t_0| \geq 1/2 + \varepsilon}} \Big[ |c(x_0,t_0)| \langle R (t_0 - 1/2) \rangle^{s} \Big]^p \right)^{1/p}\\
        &\quad\quad \lesssim 2^{kd/p'} \left( \sum\nolimits_{\substack{(x_0,t_0) \in \mathcal{X}_k \times \mathcal{T}_k\\|t_0| \geq 1/2 + \varepsilon}} \Big[ |c(x_0,t_0)| \langle R t_0 \rangle^{s} \Big]^p \right)^{1/p}.
    \end{split}
    \end{equation}
    %
    Combining these three bounds proves that $\text{Wave-Bound}(p)$ holds.
\end{proof}

\begin{remark}
    We note that this analysis requires an understanding of the global geometric of the geodesic flow on $S^d$, in particular, that points flow into a conjugate point exactly as they exit a starting point. A similar analysis may yield $\text{Wave-Bound}(p)$ for operators on the rank one symmetric spaces, but is unlikely to be easy to obtain on an arbitrary Zoll manifold with periodic geodesic flow, due to the absence of information about the geometric properties of conjugate points.
\end{remark}