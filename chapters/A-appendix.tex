%!TEX root = ../main.tex

%\renewcommand{\thechapter}{A}
\chapter{Elementary Theorems} \label{cha:elementary_theorems}

In this appendix, we list several results that occur frequently in the harmonic analysis literature. The first result, Euler's homogeneous function theorem, provides a way to relate a homogeneous function and it's derivatives.

\begin{theorem}[Euler's Homogeneous Function Theorem] \label{thm:Euler}
    Suppose $f: {\dot{\RR}{\vphantom{\RR}}^p} \to \CC$ is a smooth, homogeneous function of order $s$. Then for any $x \in {\dot{\RR}{\vphantom{\RR}}^p}$, $x \cdot \nabla f(x) = s f(x)$.
\end{theorem}
\begin{proof}
    Differentiate the relation $f(ax) = a^s f(x)$ with respect to $a$, and then set $a = 1$.
\end{proof}

The Sobolev embedding theorem allows us to trade smoothness for integrability.

\begin{theorem}[The Sobolev Embedding Theorem] \label{Chp:Sobolev}
    Suppose $X$ is a $d$-dimensional manifold, or $X = \RR^d$. Fix $1 \leq p \leq q < \infty$ and let $r = d(1/p - 1/q)$. Then for any $s \in \RR$, $W^{s,q}(X) \subset W^{s-r,p}(X)$, and if $s > d/p$, $W^{s,p}(X) \subset C_b(X)$, the space of continuous, bounded functions, where these inclusions are continuous.
\end{theorem}
\begin{proof}
    For the case where $X = \RR^d$, see Theorem 6.2.4 of \cite{GrafakosModern}. To obtain the result on a general compact manifold, apply the Sobolev embedding theorem in each coordinate system.
\end{proof}

\begin{comment}
\begin{theorem}[Bernstein's Inequality]
    Fix $1 \leq p \leq \infty$, $s \in \RR$, and $R > 0$. If the Fourier transform of a function $f: \RR^d \to \CC$ is supported on $\{ \xi : 0 \leq |\xi| \leq R \}$, then
        %
    \[ \| f \|_{W^{s,p}(\RR^d)} \sim \langle R \rangle^s \| f \|_{L^p(\RR^d)}. \]% \quad\text{and}\quad \| f \|_{\dot{B}^{s,p}_r(\RR^d)} \sim R^s \| f \|_{L^p(\RR^d)}. \]
    Similarily, suppose $X$ is a compact manifold, and $P$ is a classical, elliptic, self-adjoint operator of order one on $X$. If $f: \RR^d \to \CC$ can be written as a linear combination of eigenfunctions of $P$, whose eigenvalues are contained in the interval $[0,R]$, then
        %
        \[ \| f \|_{W^{s,p}(X)} \sim \langle R \rangle^s \| f \|_{L^p(\RR^d)} \]
\end{theorem}
\begin{proof}
    For the result on $\RR^d$, see Proposition 5.3 of \cite{Wolff}.
\end{proof}
\end{comment}

Interpolation is essential for reducing bounds to Lebesgue norms in which the boundedness of the operators is simpler to understand. If $X$ and $Y$ are measure spaces, recall that an operator $T$ mapping elements of $L^p(X)$ to measurable functions on $Y$ is said to be of \emph{restricted weak type} $(p,q)$ if it has the property that for any measurable set $E \subset X$,
%
\begin{equation}
    \| T \mathbb{I}_E \|_{L^{q,\infty}(Y)} \lesssim \| \mathbb{I}_E \|_{L^p(X)} = |E|^{1/p}.
\end{equation}
%
where $\mathbb{I}_E$ is the indicator function of $E$, and the implicit constant is uniform in $E$.

\begin{theorem}[The Marcinkiewicz Interpolation Theorem] \label{thm:marci}
    Let $X$ and $Y$ be measure spaces, and fix exponents $1 \leq p_0,p_1 < \infty$ and $1 \leq q_0,q_1 < \infty$. For $\theta \in [0,1]$, define $p_\theta$ and $q_\theta$ so that
    %
    \[ 1/p_\theta = \theta/p_1 + (1-\theta)/p_0 \quad\text{and}\quad 1/q_\theta = \theta/q_1 + (1-\theta)/q_0. \]
    %
    Let $T$ be a linear operator mapping elements of $L^{p_0}(X) + L^{p_1}(X)$ to measurable functions on $Y$. Then if $T$ is of restricted weak type $(p_0,q_0)$ and $(p_1,q_1)$, then for any $\theta \in (0,1)$, the operator $T$ is bounded from $L^{p_\theta}(X)$ to $L^{q_\theta}(Y)$.
\end{theorem}
\begin{proof}
    See Theorem 1.4.19 of \cite{Grafakos}.
\end{proof}

Schur's Test for integral kernels is a simple, but often optimal method, for obtaining the boundedness of integral operators from an $L^1$ norm, or into an $L^\infty$ norm. Interpolation can then be used to obtain certain bounds for operators with respect to other norms.

\begin{theorem}[Schur's Test for Integral Kernels] \label{thm:Schurl}
    Let $X$ and $Y$ be measure spaces.
    %
    \begin{itemize}
        \item If $K_T \in L^\infty(Y) L^p(X)$ and $f \in L^1(Y)$, then for almost every $x$, the integral
        %
        \[ Tf(x) = \int K_T(x,y) f(y)\; dy \]
        %
        is absolutely integrable, and defines an operator from $L^1(Y)$ to $L^p(X)$ such that
        %
        \[ \| T \|_{L^1(Y) \to L^p(X)} \leq \| K_T \|_{L^\infty(X) L^p(Y)}. \]

        \item If $K_T \in L^\infty(X) L^{p'}(Y)$, then for each $f \in L^p(Y)$, and for almost every $x$, the integral
        %
        \[ Tf(x) = \int K_T(x,y) f(y)\; dy \]
        %
        is absolutely integrable, defining an operator $T$ from $L^p(Y)$ to $L^\infty(X)$, such that
        %
        \[ \| T \|_{L^p(Y) \to L^\infty(X)} \leq \| K_T \|_{L^\infty(Y) L^{p'}(X)}. \]
    \end{itemize}
\end{theorem}
\begin{proof}
    For any measurable $f: Y \to \CC$, define $T^\dagger f: X \to [0,\infty]$ by setting
    %
    \begin{equation}
        T^\dagger f(x) = \int |K_T(x,y)| f(y)|\; dy.
    \end{equation}
    %
    Applying Minkowski's inequality and H\"{o}lder's inequality gives the bounds
    %
    \begin{equation}
        \| T^\dagger f \|_{L^p(X)} \leq \| K_T \|_{L^\infty(X) L^p(Y)} \| f \|_{L^1(Y)}
    \end{equation}
    %
    and
    \begin{equation}
        \| T^\dagger f \|_{L^\infty(X)} \leq \| K_T \|_{L^\infty(Y) L^{p'}(X)} \| f \|_{L^p(Y)}.
    \end{equation}
    %
    These bounds immediately imply the required results for the integral operator $T$.
\begin{comment}
    If $K_T \in L^\infty(Y) L^p(X)$, then for any measurable function $f$ on $Y$, the integral
    %
    \begin{equation} \label{equation189203u190248231094583129}
        \left( \int \left( \int |K_T(x,y)| |f(y)|\; dy \right)^p\; dx \right)^{1/p}
    \end{equation}
    %
    is well defined, though possibly infinite. Applying Minkowski's inequality, we find that
    %
    \begin{equation} \label{equatyion12890u3091284587u93125tuy98324hfv9i8euwhf}
    \begin{split}
        \left( \int \left( \int |K_T(x,y)| |f(y)|\; dy \right)^p\; dx \right)^{1/p} &\leq \int \left( \int |K_T(x,y)|^p |f(y)|^p\; dx \right)^{1/p}\; dy\\
        &\leq \int \| K_T \|_{L^p(X)} |f(y)|\; dy\\
        &\leq \| K_T \|_{L^\infty(X) L^p(Y)} \| f \|_{L^1(Y)}.
    \end{split}
    \end{equation}
    %
    Thus if $f \in L^1(Y)$, \eqref{equation189203u190248231094583129} is finite, and so $\int |K_T(x,y)| |f(y)|\; dy$ is finite for almost every $x \in X$, so that $Tf$ is well defined as a measurable function on $X$. But the triangle inequality and \eqref{equatyion12890u3091284587u93125tuy98324hfv9i8euwhf} imply the bound $\| Tf \|_{L^\infty(X)} \leq \| K_T \|_{L^\infty(X) L^p(Y)} \| f \|_{L^1(Y)}$.
\end{comment}
\end{proof}

We will rely on the following `$L^p$ orthogonality' result.

\begin{lemma} \label{lem:LpOrthogonalityEasyLemma}
    Let $X$ be a measure space. Suppose $\{ T_j \}$ are operators on $X$ which satisfy
    %
    \[ \| T_j \|_{L^1(X) \to L^{1,\infty}(X)} \lesssim 1, \]
    %
    uniformly in $j$, and suppose that the bound
    %
    \[ \left\| \sum T_j f_j \right\|_{L^2(X) \to L^2(X)} \lesssim \left( \sum\nolimits_j \| f_j \|_{L^2(X)}^2 \right)^{1/2} \]
    %
    holds. Then for $1 < p \leq 2$,
    %
    \[ \left\| \sum T_j f_j \right\|_{L^p(X)} \lesssim \left( \sum \| f_j \|_{L^p(X)}^p \right)^{1/p}. \]
\end{lemma}
\begin{proof}
    Define a vector-valued operator $T$ from sequences of functions on $X$ to functions on $X$, such that for a sequence $f = \{ f_j \}$, $Tf = \sum T_j f_j$. Then by assumption,
    %
    \begin{equation}
        \| Tf \|_{L^2(X)} \lesssim \| f \|_{l^2(\NN) L^2(X)},
    \end{equation}
    %
    and by the triangle inequality,
    %
    \begin{equation}
        \| Tf \|_{L^{1,\infty}(X)} \lesssim \| f \|_{l^1(\NN) L^1(X)}.
    \end{equation}
    %
    Real interpolation (i.e. the Marcinkiewicz interpolation theorem) thus tells us that for $1 < p < \infty$,
    %
    \begin{equation}
        \| Tf \|_{L^p(X)} \lesssim \| f \|_{l^p(\NN) L^p(X)},
    \end{equation}
    %
    which implies the required result.
\end{proof}

Finally, we will often use a helpful interpolation lemma.

\begin{lemma}[A Real Interpolation Lemma] \label{theoremrealinterpolation}
    Consider a family of functions $\{ f_H \}$, where $H$ ranges over the dyadic numbers, and fix $0 < p_0 < p_1 < \infty$. If for each $H$,
    %
    \[ \| f_H \|_{L^{p_0}(X)} \lesssim H W_H^{1/p_0} \quad\text{and}\quad \| f_H \|_{L^{p_1}(X)} \lesssim H W_H^{1/p_1}, \]
    %
    with implicit constants uniform in $H$, then for any $p_0 < p < p_1$,
    %
    \[ \left\| \sum\nolimits_H f_H \right\|_{L^p(X)} \lesssim \left( \sum\nolimits_H H^p W_H \right)^{1/p}. \]
\end{lemma}
\begin{proof}
    See Lemma 2.2 of \cite{HeoandNazarovandSeeger}.
\end{proof}

In an application of Lemma \ref{theoremrealinterpolation}, the quantities $H$ normally stand for the `height' of a given input, and the quantities $\{ W_H \}$ the `width', and so this interpolation result shows that if we are not proving results at an endpoint, it suffices to prove bounds for `a fixed height scale'.









%\renewcommand{\thechapter}{B}
\chapter{Distributions Defined by Oscillatory Distributions} \label{cha:distributions_defined_by_oscillatory_distributions}

The kernels of many of the Schwartz operators $T$ we study in this thesis have kernels which are defined by certain formal expression of the form
%
\begin{equation}
    K_T(x,y) = \int_{\RR^p} a(x,y,\theta) e^{i \phi(x,y,\theta)}\; d\theta,
\end{equation}
%
where $a$ is a symbol, and $\phi$ is smooth and homogeneous of order one.  In this appendix we show how these formal expressions can be interpreted in a way that ensures these expressions are well defined as distributions, provided that $\nabla_{x,y} \phi$ and $\nabla_\theta \phi$ have no common zeros. This ensures that, when $K_T$ is integrated against a pair of test functions on $\RR^m$ and $\RR^n$, the phase becomes rapidly oscillatory at high frequencies, which ensures convergence of appropriate integrals. We call a smooth, homogeneous function $\phi$ such that $\nabla_{x,y,\theta} \phi$ is non-vanishing a \emph{phase function}.

\begin{theorem}
    Let $\Omega$ be an open subset of $\RR^n$. Let $a: {\Omega} \times\; {\dot{\RR}{\vphantom{\RR}}^p} \to \CC$ be a symbol, and let $\phi: W\times\; \dot{\RR}{\vphantom{\RR}}^p \to \RR$ be a phase function. Fix $\chi \in C_c^\infty(\RR^p)$, equal to one in a neighborhood of the origin. The smooth functions
    %
    \[ I^{a,\theta}_R = \int_{\RR^p} a(x,\theta) \chi(\theta / R) e^{2 \pi i \phi(x,\theta)}\; d\theta \]
    %
    converge in the weak $*$ topology to a distribution $I$, independent of the choice of $\chi$. 
\end{theorem}
\begin{proof}
    If $\mu < -p$, then
    %
    \begin{equation}
        \int_{\RR^p} a(x,\theta) e^{2 \pi i \phi(x,\theta)}\; d\theta
    \end{equation}
    %
    is absolutely integrable, and the result follows from the dominated convergence theorem. For $\mu \geq -p$, fix $f \in C_c^\infty(\RR^d)$ supported on a compact set $K \subset \Omega$. Our goal is to show the quantities
    %
    \begin{equation}
        \langle I^{a,\theta}_R, f \rangle = \int a(x,\theta) \chi(\theta / R) e^{2 \pi i \phi(x,\theta)}\; d\theta
    \end{equation}
    %
    converges as $R \to \infty$ to a quantity independent of the choice of $\chi$, and that moreover, there exists $N > 0$ such that
    %
    \begin{equation}
        |\langle I^{a,\theta}_R, f \rangle| \lesssim \| f \|_{C^M(\RR^d)},
    \end{equation}
    %
    uniformly in $R$. Write
    %
    \begin{equation}
        a_R(x,\theta) = a(x, \theta) \chi(\theta / R).
    \end{equation}
    %
    We will assume without loss of generality that $a(x,\theta) = 0$ for $|\theta| \leq 1$, since the quantity
    %
    \begin{equation}
        \int a(x,\theta) \mathbb{I} \big(|\theta| \leq 1 \big) e^{2 \pi i \phi(x,\theta)}\; d\theta
    \end{equation}
    %
    defines a function in $C^\infty(\RR^d)$. We now apply the principle of non-stationary phase, i.e. integrating by parts. Consider the \emph{homogenized gradient}
    %
    \begin{equation}
        (\nabla^H\! \phi)(x,\theta) = ( \nabla_x \phi, |\theta| \nabla_\theta \phi(x,\theta) ).
    \end{equation}
    %
    Then $\nabla_{x,\theta}^H \phi$ is homogeneous of degree one in the $\theta$ variable, and $|\nabla_{x,\theta}^H \phi| > 0$. We note that
    %
    \begin{equation}
        (\nabla_{x,\theta}^H e^{2 \pi i \phi} ) = (2 \pi i) e^{2 \pi i \phi} \nabla_{x,\theta}^H \phi.
    \end{equation}
    %
    Note also that the \emph{formal transpose} of $\nabla_{x,\theta}^H$ is the differential operator $L$ given for pairs of functions $F_1: \RR^d \times \RR^p \to \RR^d$ and $F_2: \RR^d \times \RR^p \to \RR^p$ by the formula
    %
    \begin{equation}
    \begin{split}
        L(F_1, F_2) &= - ( \nabla_x \cdot F_1 + \nabla_\theta \cdot ( |\theta| F_2 ))\\
        &= - \left( \nabla_x \cdot F_1 + |\theta|^{-1} (\theta \cdot F_2) + |\theta| (\nabla_\theta \cdot F_2) \right),
    \end{split}
    \end{equation}
    i.e. so that for smooth, compactly supported functions $F_1$, $F_2$, and $G$,
    %
    \begin{equation}
        \int_{\RR^d \times \RR^p} (F_1,F_2) \cdot \nabla^H_{x,\theta} G = \int_{\RR^d \times \RR^p} L(F_1,F_2) \cdot G.
    \end{equation}
    %
    Thus we conclude that
    %
    \begin{equation}
    \begin{split}
        \langle I^{a,\theta}_R, f \rangle &= \frac{1}{2 \pi i} \int \frac{a \cdot f}{|\nabla_{x,\theta}^H \phi|^2} \left( \nabla_{x,\theta}^H e^{2 \pi i \phi} \right) \cdot (\nabla_{x,\theta}^H \phi)\\
        &= \frac{1}{2 \pi i} \int L \left\{ \frac{a \cdot f}{|\nabla_{x,\theta}^H \phi|^2} \nabla_{x,\theta}^H \phi \right\} e^{2 \pi i \phi}.
    \end{split}
    \end{equation}
    %
    Expanding out the differential operator $L$, we see we can write
    %
    \begin{equation}
        \langle I^{a,\phi}_R, f \rangle = \sum\nolimits_{|\alpha| \leq 1} \langle I^{a_\alpha, \phi}_R, \partial^\alpha \! f \rangle,
    \end{equation}
    %
    where $a_\alpha$ is a symbol of order at most $\mu - 1$. Iterating this argument, we find that for any $N > 0$, we can write
    %
    \begin{equation}
        \langle I^{a,\theta}_R, f \rangle = \sum\nolimits_{|\alpha| \leq N} \langle I^{a_\alpha, \phi}_{R}, \partial^\alpha\! f \rangle,
    \end{equation}
    %
    where $a_{\alpha}$ is a symbol of order at most $\mu - N$. If $N > \mu + p$, then we can apply the dominated convergence theorem to each term on the right hand side, from which the result follows.
\end{proof}









%\renewcommand{\thechapter}{C}
\chapter{Pseudo-Differential Operators} \label{appendixpsueiodjaweiodj}

In this appendix we briefly introduce the theory of pseudo-differential operators relevant to this thesis. Recall that a \emph{pseudo-differential operator} on $\RR^d$ of order $s \in \RR$ is a bounded Schwartz operator $T$ whose kernel is an oscillatory integral distribution of the form
%
\begin{equation}
    K_T(x,y) = \int_{\RR^d \times \RR^d} a(x,\xi) e^{2 \pi i \xi \cdot (x - y)}\; d\xi,
\end{equation}
%
where $a: {\RR^d} \times {\dot{\RR}{\vphantom{\RR}}^d} \to \CC$ is a fixed symbol of order $s$. The operator $T$ is often written $a(x,D)$, because if $a$ is a polynomial in $\xi$, i.e.
%
\begin{equation}
    a(x,\xi) = \sum\nolimits_\alpha c_\alpha(x) \xi^\alpha,
\end{equation}
%
then the pseudo-differential operator $a(x,D)$ is really the differential operator $\sum c_\alpha(x) D^\alpha$, where $D_j = (2\pi i)^{-1} \partial_j$ is a self-adjoint normalization of the partial derivative operator.

An important property of pseudo-differential operators is that they are \emph{pseudo-local}, in the sense that the Schwartz kernel $K_T$ is a smooth function away from the diagonal $\Delta_{\RR^d} = \{ (x,x): x \in \RR^d \}$, and $K_T$ and all of it's derivatives rapidly decay away from the diagonal\footnote{See e.g. Proposition 1 of Chapter 5, Section 4 of \cite{BigStein} for a proof of this rapid decay.}, i.e. if we write $k_T(x,z) = K_T(x,x+z)$, then
%
\begin{equation} \label{eq1029312094312093821094u29th}
    | \partial_x^\alpha \partial_z^\beta k_T(x,z)| \lesssim_{\alpha,\beta,N} |z|^{-d-s-|\beta| - N} \quad\text{for all $N > 0$}.
\end{equation}
%
Using duality, one can extend the definition of the operator $T$ to the space of tempered distributions, and the estimates in \eqref{eq1029312094312093821094u29th} imply that if $f$ is a distribution smooth in a neighborhood of a given point, then $Tf$ is smooth in a neighborhood of this point.
 % and satisfies, for $|x - y| \geq 1$, estimates of the form
%
%\[ |\partial_x^\alpha \partial_y^\beta K_T(x,y)| \lesssim_{\alpha,\beta,N} \langle x - y \rangle^{-N} \]
%
%for arbitrarily large $N > 0$.

A symbol $a$ of order $s$ is \emph{classical} if it has an expansion of the form
%
\begin{equation}
    a \sim \sum\nolimits_{k = 0}^\infty a_{s - k},
\end{equation}
%
where $a_{s-k}: {\RR^d} \times  {\dot{\RR}{\vphantom{\RR}}^d} \to \CC$ is a smooth function which is homogeneous of order $s - k$ in the second variable. A pseudo-differential operator $T = a(x,D)$ is then \emph{classical} if the symbol $a$ is classical. We then call $a_s$ the \emph{principal symbol} of the operator $T$. A classical pseudo-differential operator is \emph{elliptic} if it's principal symbol is non-vanishing.

All this generalizes to the setting of compact manifolds by working in coordinates. Let $X$ be a compact manifold, and consider a cover of $X$ by open sets $\{ U_\alpha \}$, where the closure of $U_\alpha$ is contained in an open set $V_\alpha$ which is diffeomorphic to a pre-compact subset $W_\alpha$ of $\RR^d$ by a diffeomorphism $F_\alpha: V_\alpha \to W_\alpha$. Consider a partition of unity $\{ \psi_\alpha \}$ of $X$, subordinate to the cover $\{ U_\alpha \}$, and also consider a smooth function $\phi_\alpha$ on $\RR^d$, equal to one on $F_\alpha(U_\alpha)$ but vanishing outside of $W_\alpha$. A Schwartz operator $T$ on a compact $d$-dimensional manifold $X$ is a \emph{pseudo-differential operator} of order $s > 0$ if it's Schwartz kernel $K_T$, defined with respect to any fixed smooth volume density on $X$, agrees with a smooth function away from the diagonal $\Delta_X = \{ (x,x): x \in X \}$, and if for each $\alpha$, the operators $T_\alpha$ on $\RR^d$ given by
%
\begin{equation}
    (T_\alpha f)(x) = \phi_\alpha(x)\; T \{ \psi_\alpha \cdot (f \circ F_\alpha) \} (F_\alpha^{-1}(x))
\end{equation}
%
are pseudo-differential operators on $\RR^d$. The operator $T$ is classical if each of the operators $T_\alpha$ is classical, and if $T_\alpha$ has principal symbol $p_\alpha(x,\xi)$ we can then define the principal symbol of $T$ to be the function $p: T^* X - \{ 0 \} \to \CC$ defined by
%
\begin{equation}
    p(x,\xi) = \sum \psi_\alpha(x) p_\alpha(x, DF_\alpha(x)^{-T} \xi).
\end{equation}
%
One verifies using the change-of-variable formulas for pseudo-differential operators\footnote{See Theorem 18.1.17 of \cite{Hormander3}.} that if $T$ is a pseudo-differential operator for one choice of cover $\{ U_\alpha \}$, then it is a pseudo-differential operator for \emph{all} choices of covers, and that for a classical pseudo-differential operator the function $p$ is invariant of this choice when it is interpreted as defined on the cotangent bundle of $X$. A classical pseudo-differential operator $T$ on a compact manifold is then \emph{elliptic} if it's principal symbol is non-vanishing.