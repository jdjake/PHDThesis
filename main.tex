\documentclass[12pt]{report}
\usepackage[utf8]{inputenc}
\usepackage{graphicx}
\graphicspath{ {images/} }
\usepackage{caption}
\usepackage{subcaption}
\usepackage{float}
\usepackage[width=150mm,top=35mm,bottom=25mm,bindingoffset=6mm]{geometry}
\usepackage{fancyhdr}
\usepackage{setspace}
\pagestyle{fancyplain}% <- use fancyplain instead fancy
\fancyhf{}
\fancyhead[R]{\thepage}
\renewcommand{\headrulewidth}{0pt}
\setlength{\headheight}{14.5pt}

\usepackage{enumitem}

\usepackage{xfrac}

\usepackage{amsmath}
\usepackage{amssymb}
\usepackage{amsthm}
\usepackage{amsopn}
\usepackage{mathtools}
\usepackage{txfonts}
\usepackage{tensor}
\usepackage{stmaryrd}
\usepackage{thmtools}
\usepackage{thm-restate}

\def\changemargin#1#2{\list{}{\rightmargin#2\leftmargin#1}\item[]}
\let\endchangemargin=\endlist 

\usepackage{amsthm}
\theoremstyle{plain}
\declaretheorem[name=Theorem,numberwithin=chapter]{theorem}

%\newtheorem{theorem}{Theorem}[chapter]
\newtheorem{lemma}[theorem]{Lemma}
\newtheorem{corollary}[theorem]{Corollary}
\newtheorem{prop}[theorem]{Proposition}

\newtheorem*{example}{Example}

\theoremstyle{remark}
\newtheorem*{exposition}{Exposition}
\newtheorem*{remark}{Remark}
\newtheorem*{remarks}{Remarks}

\usepackage{comment}

%\usepackage[style=authoryear,sorting=none]{biblatex}
\usepackage[style=numeric]{biblatex}
\addbibresource{references.bib}

\usepackage[perpage, symbol*]{footmisc}

\usepackage{tikz}
\usetikzlibrary{matrix}
\usetikzlibrary{calc}


\DeclareMathOperator{\QQ}{\mathbb{Q}}
\DeclareMathOperator{\ZZ}{\mathbb{Z}}
\DeclareMathOperator{\RR}{\mathbb{R}}
\DeclareMathOperator{\HH}{\mathbb{H}}
\DeclareMathOperator{\BB}{\mathbb{B}}
\DeclareMathOperator{\CC}{\mathbb{C}}
\DeclareMathOperator{\AB}{\mathbb{A}}
\DeclareMathOperator{\PP}{\mathbb{P}}
\DeclareMathOperator{\MM}{\mathbb{M}}
\DeclareMathOperator{\VV}{\mathbb{V}}
\DeclareMathOperator{\TT}{\mathbb{T}}
\DeclareMathOperator{\LL}{\mathcal{L}}
\DeclareMathOperator{\DD}{\mathcal{D}}
\DeclareMathOperator{\SW}{\mathcal{S}}
\DeclareMathOperator{\EC}{\mathcal{E}}
\DeclareMathOperator{\AC}{\mathcal{A}}

\newcommand\COP[1]{{\text{COP}^{+}_{{#1}}}}
\newcommand\Sd[1]{{S}^{\!{#1}}}

\DeclareMathOperator{\EE}{\mathbb{E}}
\DeclareMathOperator{\NN}{\mathbb{N}}

\DeclareMathOperator{\II}{\mathbb{I}}

\DeclareMathOperator{\DQ}{\mathcal{Q}}

\DeclareMathOperator{\Ind}{\mathbb{I}}

\DeclareMathOperator{\diam}{\text{diam}}
\DeclareMathOperator{\supp}{\text{supp}}


\numberwithin{equation}{section}

\title{Multipliers of Spherical Harmonic Expansions}
\author{Jacob Denson}
\date{Day Month 2025}

\begin{document}

\pagenumbering{roman} 

\begin{titlepage}
    \begin{center}
        \vspace*{1cm}
        
        \Huge
        \textbf{Multipliers of Elliptic Operators on Compact Manifolds}
        
        \vspace{0.5cm}
        \LARGE
        {Necessary and Sufficient Conditions For Boundedness}%Multipliers of Spherical Harmonic Expansions}
        \vspace{1em}
        \\
        by\\
        \vspace{1em}
    
        \textbf{Jacob Denson}
        
        \vfill
        
        A dissertation submitted in partial fulfillment\\
        of the requirements for the degree of\\
        Doctor of Philosophy\\
        (Mathematics)\\
    
        
        \vspace{1.8cm}

        
        \Large
        at the\\University of Wisconsin-Madison\\
        2025\\
        \vspace{1.0cm}
        %\begin{flushleft}
        %\large
        %Date of Final Oral Exam: TODO\\
        %The dissertation is approved by the following members of the Final Oral Committee: \\
        %\setlength{\parindent}{10ex}
        %Im A FacultyMember, Professor, Mathematics\\Im A FacultyMember, Associate Professor, Mathematics\\ Im A FacultyMember, Assistant Professor, Statistics\\ Im A FacultyMember, Professor, Computer Science\\
        %\end{flushleft}
        
    \end{center}
    
\end{titlepage}

%!TEX root = main.tex

\fancyhf{} % clear all header and footer fields
\fancyhead[RO,R]{\thepage} %RO=right odd, RE=right even
\renewcommand{\headrulewidth}{0pt}

\begin{center}
    \Large
    \textbf{Multipliers of Elliptic Operators on Compact Manifolds}
    
    \vspace{0.4cm}
    \large
    Necessary and Sufficient Conditions For Boundedness
    
    \vspace{0.4cm}
    \textbf{Jacob Denson}
    
    \vspace{0.9cm}
    \textbf{Abstract}
\end{center}

Let $T$ be a Schwartz operator on $S^d$ commuting with rotations. Then there exists a function $a: \ZZ \to \CC$, the symbol of $T$, such that for any spherical harmonic $f: S^d \to \mathbb{C}$ of degree $k$, $Tf = a \big( k \big) f$. This thesis discusses the work of the author on obtaining new results about the boundedness of such operators on $L^p(S^d)$ for $p \neq 2$ and $d \geq 4$.

Given any regulated function $a: [0,\infty) \to \mathbb{C}$, define operators $T_a$ and $\{ T_\rho : \rho > 0 \}$ on $S^d$, commuting with rotations, with symbols $a(k)$ and $a_\rho(k) = a \big( \rho \big( k + \tfrac{d-1}{2} \big) \big)$ respectively. For $d \geq 4$, and $1 < p < 2(d-1)/(d+1)$, we obtain sufficient conditions on $a$ in order for $T_a$ to be bounded on $L^p(S^d)$, and necessary and sufficient conditions for the operators $\{ T_\rho \}$ to be uniformly bounded on $L^p(S^d)$. One consequence of these results are new transplantation principles, which control the $L^p$ boundedness of the multiplier operators $\{ T_\rho \}$ in terms of the radial Fourier multiplier operator on $\RR^d$ with symbol $a(|\cdot|): \mathbb{R}^d \to \mathbb{C}$. This is the first nontrivial transplantation result of it's kind.

%, for a compact manifold $X$, and a restricted family of self-adjoint, first order, elliptic, pseudo-differential operator of order one, we obtain results giving necessary and sufficient conditions for the uniform boundedness of a family of rescaled spectral multiplier operators for eigenfunction expansions.

%
% Given a(P_delta) = a(P_{SH} + delta - (d-1)/2)
% Trans_{delta + (d-1)/2 - delta} a

% If P_delta f = (k + delta) f
% P_{SH} f = (k + (d-1)/2) f
% So a

%
% 
%
% 1/p - 1/2 > 1/(d-1)
% 1/p = (d-1)/2(d+1)
% P = P_{SH} - (d-1)/2
% P_{SH}/R = (d-1)/2R + P

%For any bounded, regulated function $a: [0,\infty) \to \mathbb{C}$, consider the family of operators $\{ T_R \}$ on the sphere $S^d$ such that $T_R f = a(k/R) f$ for any spherical harmonic $f$ of degree $k$. This thesis discusses the work of the author on characterizing the functions $a$ for which the operators $\{ T_R \}$ are uniformly bounded on $L^p(S^d)$ in the range $1/(d-1) < |1/p - 1/2| < 1/2$. More generally, we obtain analogous results for spectral multiplier operators of eigenfunction expansions of a first order elliptic pseudo-differential operator $P$ on a compact manifold $X$. Under curvature assumptions on the principal symbol of the operator $P$, and assuming the eigenvalues of $P$ are contained in an arithmetic progression, we characterize all regulated and compactly supported functions $a$ whose associated spectral multiplier operators $\{ a(P/R) \}$ are uniformly bounded on $L^p(X)$, under curvature assumptions on the principal symbol of $P$, and assuming the eigenvalues of $P$ are contained in an arithmetic progression. If in addition, we assume that solutions to the wave equation $\partial_t^2 u = P^2 u$ have finite propagation speed, and that certain estimates for the wave equation hold, then we characterize all regulated functions whose associated operators $\{ T_R \}$ are uniformly bounded on $L^p(X)$, without the assumption of compact support.

Our methods do not take advantage of the rotational symmetries of $S^d$, and can thus be applied in greater generality. In particular, we obtain results for a restricted family of spectral multipliers of a more general self-adjoint elliptic pseudo-differential operator $P$, under  assumptions related to the periodicity and curvature properties of a Hamiltonian flow on $T^* X$ associated with $P$.

In order to prove these results, we obtain new quasi-orthogonality estimates for averages of solutions to the half-wave equation $\partial_t - i P = 0$, via a connection between pseudo-differential operators satisfying an appropriate curvature condition and Finsler geometry, and combine estimates for different frequency scales via arguments based on a theory of atomic decompositions in $L^p$ for $p > 1$.

In Part I of the thesis, we review prior work on endpoint bounds relevant to the new results stated, in particular, endpoint bounds for radial and quasi-radial Fourier multiplier operators on Euclidean space, as well as Finsler geometry and Fourier integral operators on compact manifolds. In Part II, we prove the results stated above. The characterizations of $L^p$ boundedness for compactly supported and regulated functions are adapted from a paper of the author \cite{Denson}, with the methods of combining frequency scales taken from forthcoming work.

%\chapter*{Dedication}

%To mum and dad

%\chapter*{Declaration}

%I declare that..

%\chapter*{Acknowledgements}

%I want to thank...

%Supported by grant numbers...

\tableofcontents

%\listoffigures
%\listoftables

%!TEX root = ../main.tex

\chapter*{Notation} \label{not:not}

\vspace{-1em}

\begin{itemize}
    \item We use the normalization of the Fourier transform given by the formula
    %
    \[ \widehat{f}(\xi) = \int f(x) e^{- 2 \pi i \xi \cdot x}\; dx. \]
    %
    Overloading notation, for a function $f: [0,\infty) \to \CC$ and $\lambda \geq 0$ we write the cosine transform of $f$ as $\widehat{f}(\lambda) = \int_0^\infty f(t) \cos(2 \pi \lambda t)\; dt$.

    \item For $p \in [1,\infty]$, we let $p' \in [1,\infty]$ be it's dual exponent, satisfying $1/p + 1/p' = 1$.

    \item We use the translation and $L^\infty$-normalized dilation operators
    %
    \[ \text{Trans}_y f(x) = f(x - y) \quad\text{and}\quad \text{Dil}_t f(x) = f(x/t), \]
    %
    defined so that if a function $f$ is supported on a closed set $A$, then $\text{Trans}_y f$ is supported on $A + y$, and $\text{Dil}_t f$ is supported on $tA$. %Overloading notation, we also consider dyadic dilations of the form $\text{Dil}_j f(x) = f(x/2^j)$. The dyadic dilation operator will only be used along with symbols that conventionally stand for integers, like $n$, $m$, $j$, or $k$, whereas the other dilation operator will be used in all other cases, so the operator used should be clear from context.

    \item On $\RR^d$, we let $\partial_i$ denote the usual partial derivative operators in the $i$th coordinate direction (if we are using a variable $x = (x_1,\dots,x_n)$ to denote these coordinates, we might also write this derivative as $\partial_{x_i}$), and use $D_j$ to denote the self-adjoint normalization $D_j f = (2 \pi i)^{-1} \partial_j$. This normalization has the convenience that for a $d$-variate polynomial $P: \RR^d \to \CC$,
    %
    \[ P(D_j) \{ f \} = \int P(\xi) \widehat{f}(\xi) e^{2 \pi i \xi \cdot x}\; d\xi, \]
    %
    and simplifies many formulas associated with Fourier integral operators. We also consider the compositions of these operators $\partial^\alpha$ and $D^\alpha$ given by multi-indices $\alpha$.

    \item For a differentiable function $F: \RR^m \to \RR^n$ and $x \in \RR^m$, we let $F_{\!*}(x) = \{ \partial_j F_k(x) \}$ denote the $n \times m$ matrix of partial derivatives of the components of $F$. If $F: X \to Y$ is a smooth map between two manifolds, we also let $F_*: TX \to TY$ denote the map, which is a linear map from $T_x X$ to $T_{f(x)} Y$ for each $x \in X$, given by the matrix of partial derivatives of $F$ in any particular pair of coordinate systems.

    \item We will often use the Japanese bracket $\langle x \rangle = (1 + |x|^2)^{1/2}$ for $x \in \RR^d$.

    \item Given a topological space $X$ and two sets $U$ and $V$, we write $U \Subset V$ if the closure of $U$ is a compact subset of $V$.

    \item Let $\Omega \subset \RR^d$ be a bounded open set. For each affine transformation $T: \RR^d \to \RR^d$, consider a smooth function $\phi_T$ supported on $T(\Omega)$. We say these functions are \emph{adapted} if $|\partial_x^\alpha \{ \phi_T \circ T \}| \lesssim_\alpha 1$, uniformly in $T$. We will apply this definition when $\Omega$ is a cube (to obtain functions adapted to certain rectangles), and when $\Omega$ is an annulus (to obtain functions adapted to various annuli).

    \item We let ${\dot{\RR}{\vphantom{\RR}}^p} = \RR^p - \{ 0 \}$. A smooth function $f: {\RR^n} \times {\dot{\RR}{\vphantom{\RR}}^p} \to \CC$ is a \emph{symbol of order $s$} if it satisfies bounds of the form $| \partial_x^\alpha \partial_\theta^\lambda f (x,\theta) | \lesssim_{\alpha,\beta} \langle \theta \rangle^{s - |\lambda|}$ for all multi-indices $\alpha$ and $\lambda$, and all $x \in \RR^n$ and $\theta \in {\dot{\RR}{\vphantom{\RR}}^p}$, where $\langle \cdot \rangle$ is the Japanese bracket defined above. More generally, symbols $s: \Gamma \to \CC$ can be defined for \emph{conical subsets} $\Gamma$ of ${\RR^n} \times {\dot{\RR}{\vphantom{\RR}}^p}$, i.e. sets $\Gamma$ with the property that if $(x,\theta) \in \Gamma$ and $\rho > 0$, then $(x,\rho \theta) \in \Gamma$. For a sequence of symbols $\{ a_k : k \geq 0 \}$, we write
    %
    \[ a \sim \sum\nolimits_k a_k \]
    %
    if, for any $l > 0$, there exists $N_0$ such that for $N \geq N_0$, $a - \sum_{k = 0}^N a_k$ is a symbol of order $s - l$. We view this relation as an asymptotic expansion of the function $a$

    \item We let $\mathcal{S}(\RR^d)$ denote the Schwartz space of rapidly decreasing smooth functions on $\RR^d$, i.e. the functions $f: \RR^d \to \CC$ such that for any multi-index $\alpha$ and any $N \geq 0$, the inequality $|\partial^\alpha f(x)| \lesssim_{\alpha,N} \langle x \rangle^{-N}$ holds for all $x \in \RR^d$.

    \item For a measure space $X$, $L^\infty(X)$ is the Banach space of essentially-bounded functions, defined almost everywhere. For a set $X$, $l^\infty(X)$ is the Banach space of bounded functions on $X$, defined everywhere. We also used mixed norm spaces; for a function $f(x,y)$ defined on the Cartesian product of two measure spaces $X$ and $Y$, we define
    %
    \[ \| f \|_{L^p(X) L^q(Y)} = \big\| \| f \|_{L^q(Y)} \big\|_{L^p(X)} = \left( \int\nolimits_X \left( \int\nolimits_Y |f(x,y)|^q\; dy \right)^{p/q}\; dx \right)^{1/p}. \]
    %
    Similarly, we can define the interchanged norm $L^q(Y) L^p(X)$ by swapping the order of application of the norms. It follows from Minkowski's inequality that if $p \geq q$ then $\| f \|_{L^p(X) L^q(Y)} \leq \| f \|_{L^q(Y) L^p(X)}$ and if $p \leq q$, then $\| f \|_{L^p(X) L^q(Y)} \geq \| f \|_{L^q(Y) L^p(X)}$.

    \item If $T$ is a Schwartz operator from a manifold $Y$ to a manifold $X$, each equipped with some smooth density, we let $K_T$ denote the Schwartz distribution on $X \times Y$ which is the integral kernel for $T$, i.e. such that for any $f \in C_c^\infty(X)$ and $g \in C_c^\infty(Y)$,
    %
    \[ \int_X f(x) (Tg)(x)\; dx = \int_{X \times Y} K_T(x,y) f(x) g(y)\; dx\; dy. \]
    %
    We define the \emph{support} of a Schwartz operator $T$ from a manifold $Y$ to a manifold $X$ to be the smallest closed set $\text{supp}(T) \subset X \times Y$ such that $\langle Tf, g \rangle = 0$ whenever $f \in C_c^\infty(X)$ and $g \in C_c^\infty(Y)$ are such that $\text{supp}(g) \times \text{supp}(f)$ is disjoint from $\text{supp}(T)$.

%    All operators we consider are Schwartz operators between manifolds equipped with a canonical measure. Abusing notation, we thus identify an operator with the distribution that gives it's Schwartz kernel. Thus, for a Schwartz operator $A$ from a manifold $Y$ to a manifold $X$, we might write
    %
%    \[ A f(x) = \int_Y A(x,y) f(y)\; dy, \]
    %
%    where $A$ on the left hand side denotes the operator $A$, and the $A$ on the right hand side denotes the kernel.

    \item For $1 \leq p \leq \infty$ and $s \in \RR$, we consider the Sobolev spaces $W^{s,p}(\RR^d)$ with norm
    %
    \[ \| f \|_{W^{s,p}(\RR^d)} = \| (1-\Delta)^{s/2} f \|_{L^p(\RR^d)}, \]
    %
    and for $1 \leq r \leq \infty$, we consider the (homogeneous) Besov spaces
    %
    \[ \| f \|_{\dot{B}^{s,p}_r(\RR^d)} = \left( \sum\nolimits_{k \in \ZZ} \left[ 2^{ks} \| P_k f \|_{L^p(\RR^d)} \right]^r \right)^{1/r}. \]
    %
    Here $(1 - \Delta)^{s/2}$ is the Fourier multiplier operator with symbol $(1 + |\xi|^2)^{s/2}$, and the operators $\{ P_k \}$ are Littlewood-Paley projection operators, i.e. Fourier multiplier operators with compactly supported symbols $\chi( |\cdot| / 2^k )$, where $\chi$ is compactly supported and $\sum \chi(t/2^k) = 1$ for all $t > 0$. By working in coordinate systems, these definitions can also be used to define Sobolev spaces $W^{s,p}(X)$ of functions on a compact manifold $X$. We write $H^s(X)$ for $W^{s,2}(X)$.

    \item If $X$ is a compact manifold equipped with a smooth volume density, we let $\COP{s}(X)$ denote the family of all classical, elliptic pseudo-differential operators of order $s$ on $X$ which are positive-semidefinite in the sense that $\langle Pf, g \rangle = \langle f, Pg \rangle$ and $\langle Pf, f \rangle \geq 0$ hold for all $f,g \in C^\infty(X)$.
\end{itemize}

%\doublespacing
\chapter*{Introduction}
\pagenumbering{arabic} 

%!TEX root = ../main.tex

Let $P$ be an elliptic linear operator on a compact manifold $M$, such that we can associate a functional calculus $a \mapsto a(P)$ for functions $a$ on the real line. In this thesis, we study the following question:  
%
\begin{changemargin}{2cm}{2cm}
\begin{center}
  \emph{What conditions on $a$ guarantee an operator $a(P)$ to be bounded.}
\end{center}
\end{changemargin}
%
Aside from testing our ability to understand the operator $P$ and the relations of it's eigenfunctions, the question has applications in the study of various partial differential equations associated with the operator $P$, and is closely related to the geometry of $M$, in particular, the way waves propogate and interact on the manifold.

Moreover, the study of multipliers on a manifold provides a useful setting with which to test and develop methods of harmonic analysis in a `variable-coefficient setting', where a lack of symmetry and translation invariance forces us to introduce more robust methods than are required in the Euclidean setting, i.e. as compared to studying multipliers of the Laplace operator on $\RR^d$ using the Fourier transform.

At present, there are many obstacles related to the global geometry of manifolds, which prevent an understanding of spectral multiplier operators on a general manifold. Over the years, sufficient conditions for boundedness had been established, but no characterizations of boundedness were known on any compact manifold, aside from the translation-invariant case when studying the Laplace operator on the torus. This thesis describes the first such characterizations beyond this setting, for a limited range of Lebesgue spaces, and on manifolds all of whose geodesics are closed and have common length. In particular, we find characterizations of functions $a$ whose dilates induce uniformly bounded multiplier operators for spherical harmonic expansions on $S^d$. We obtain these results by expanding on techniques for understanding Fourier integral operators on manifolds.

In Chapter \ref{cha:multipliers_of_an_elliptic_operator}, we describe the setup to problem we are to discuss in more detail. In Chapter \ref{sec:radmult}, we discuss results known for radial and quasi-radial multipliers which are analogoue to the endpoint bounds we will establish for spectral multipliers on manifolds. In Chapter \ref{chap:waveequation}, we return to the study of spectral multipliers on manifolds, introducing the background on Finsler geometry and Fourier integral operators we will need in the proofs to come. In Chapter \ref{chap:boundedsinglefrequencyscale}, we discuss our new results for spectral multipliers with a compactly supported symbol, and in Chapter \ref{chap:spectralatomicdchapter}, we discuss new results about combining different frequency scales to bound more general spectral multiplier operators.

\part{Background}
%!TEX root = ../main.tex

\chapter{Statement of Main Results} \label{cha:multipliers_of_an_elliptic_operator}

\section{Multipliers of Spherical Harmonic Expansions}

Consider a bounded operator $T$ on $L^2(S^d)$ commuting with rotations, i.e. such that
%
\begin{equation}
  T \circ R^* = R^* \circ T \quad\text{for all $R \in \text{O}(d+1)$},
\end{equation}
%
where $(R^*\! f)(x) = f(Rx)$. We have an orthogonal decomposition $L^2(S^d) = \bigoplus \mathcal{H}_k$, where $\mathcal{H}_k$ is the space of \emph{spherical harmonics of degree $k$}, i.e. the restrictions of harmonic, homogeneous polynomials of degree $k$ on $\RR^{d+1}$ to $S^d$. And there exists a bounded function $a: \NN \to \CC$ such that $Tf = a(k) f$ for all $f \in \mathcal{H}_k(S^d)$, so that $T$ is diagonalized by the spherical harmonics\footnote{This follows from Schur's Lemma, and that the vector spaces $\mathcal{H}_k$ are non-isomorphic irreducible representations of $\text{O}(d+1)$. See Theorem 2.33 of \cite{Sepanski} for a proof of irreducibility, and Theorem 3.5 of \cite{Folland} for a discussion of Schur's Lemma for unitary representations.}. Conversely, any bounded function $a$ defines an operator $T_a$ on $L^2(S^d)$ commuting with rotations. We will call such an operator a \emph{multiplier operator for spherical harmonic expansions}.

To study the regularity of such operators, we introduce the space $M^p(S^d)$, consisting of all functions $a: \NN \to \CC$ for which the operator $T_a$ is bounded on $L^p(S^d)$. The space $M^p(S^d)$ is then equipped with the norm
%
\begin{equation}
  \| a \|_{M^p(S^d)} = \sup \left\{ \frac{\| T_a f \|_{L^p(S^d)}}{\| f \|_{L^p(S^d)}} : f \in C^\infty(S^d) \right\}.
\end{equation}
%
$M^2(S^d)$ is isometric to $l^\infty(\NN)$ (as the calculations below in Lemma \ref{L2M2Lemma} show), but the structure of $M^p(S^d)$ for $p \neq 2$ is unclear. Duality implies that $M^p(S^d)$ is isometric to $M^{p'}(S^d)$, so attention may be restricted to the case $1 \leq p \leq 2$.

We wish to study operators commuting with rotations analogously to how we might study Fourier integral operators on $\RR^d$, i.e. using oscillatory integrals. However, the discrete decomposition of $L^2(S^d)$ into spherical harmonics makes it difficult to apply methods of oscillatory integrals to the problem. Fortunately, certain `semi-classical' heuristics tell us that these problems disappear when we restrict the problem to 'high frequency inputs'. In order to take advantage of this fact, rather than studying what conditions ensure the operators $T_a$ are bounded, we study conditions that ensure the operators $T_{a_\rho}$ are uniformly bounded on $L^p(X)$, where $a_\rho(\lambda) = a \big( \rho \big(\lambda + \tfrac{d-1}{2} \big) \big)$. As $\rho \to 0$, the function becomes `concentrated at high frequency'. %, since these operators depend more and more on 'high frequency behaviour' in the limit.
To prevent pathological examples from arising, we restrict ourselves to \emph{regulated functions} $a$, i.e. functions such that
%
% P_{SH} = 
% P_{SH} f = (k + (d-1)/2) f
% a(rho P_{SH}) = a(rho (P + (d-1)/2))
% a(rho(k + (d-1)/2))
%
\begin{equation}
  a(\lambda_0) = \lim_{\delta \to 0} \fint_{|\lambda - \lambda_0| \leq \delta} a(\lambda)\; d\lambda \quad \text{for all $\lambda_0 \in [0,\infty)$}.
\end{equation}
%
%For the manifolds, elliptic operators, and exponents that we focus on in this thesis, for any regulated function $a$,
%
%\[ \sup\nolimits_R \| a_R \|_{M^{p,q}(X)} \sim \limsup\nolimits_{R \to \infty} \| a_R \|_{M^{p,q}(X)}. \]  
%
%Thus uniform boundedness as $R \to \infty$ gives uniform boundedness for all $R > 0$,
Define the set $M^p_{\text{Dil}}(S^d)$ of all regulated functions $a$ such that $\| a \|_{M^p_{\text{Dil}}(S^d)} = \sup_\rho \| a_\rho \|_{M^p(S^d)}$ is finite. With notation introduced, we now state precisely the problem studied in this thesis:
\begin{changemargin}{2cm}{2cm}
\begin{center}
  \emph{Can one find simple conditions on a function $a$,\\
  necessary and sufficient to be contained in $M^p_{\text{Dil}}(S^d)$ for $p \neq 2$.}
\end{center}
\end{changemargin}
%
For $p = 2$, orthogonality gives the isometric equivalence $M^2_{\text{Dil}}(X) = L^\infty[0,\infty)$.

\begin{lemma} \label{L2M2Lemma}
  For any regulated $a$, $\| a \|_{M^2_{\text{Dil}}(S^d)} = \| a \|_{L^\infty[0,\infty)}$.
\end{lemma}
\begin{proof}
  Given $f \in L^2(S^d)$, if we write $f = \sum f_k$ with $f_k \in \mathcal{H}_k$, then for any $R > 0$,
  %
  \begin{equation} \label{awiodjawoidjoi}
    \| T_{a,\rho} f \|_{L^2(S^d)} = \left\| \sum\nolimits_k a_\rho(k) f_k \right\|_{L^2(S^d)} = \left( \sum\nolimits_k |a_\rho(k)|^2 \| f_k \|_{L^2(S^d)}^2 \right)^{1/2}.
  \end{equation}
  %
  Conversely,
  %
  \begin{equation} \label{dawoidjaw}
    \| f \|_{L^2(S^d)} = \left( \sum\nolimits_k \| f_k \|_{L^2(S^d)}^2 \right)^{1/2}.
  \end{equation}
  %
  Setting $c_k = \| f_k \|_{L^2(S^d)}$, \eqref{awiodjawoidjoi} and \eqref{dawoidjaw} show $\| a_R \|_{M^2(S^d)}$ is the smallest constant so that
  %
  \begin{equation}
    \left( \sum\nolimits_k |a_\rho(k)|^2 c_k^2 \right)^{1/2} \leq \| a_\rho \|_{M^2(S^d)} \left( \sum\nolimits_k c_k^2 \right)^{1/2}.
  \end{equation}
  %
  for all sequences $\{ c_k \}$. It is then clear that $\| a_\rho \|_{M^2(S^d)} = \| a_\rho \|_{l^\infty(\NN)}$, and thus (now using the fact that $a$ is regulated),
  %
  \begin{equation}
    \| a \|_{M^2_{\text{Dil}}(S^d)} = \sup\nolimits_\rho \| a_\rho \|_{l^\infty(\NN)} = \| a \|_{L^\infty[0,\infty)},
  \end{equation}
  %
  which completes the proof.
\end{proof}

The main result of this thesis, proved in Chapters \ref{chap:boundedsinglefrequencyscale} and \ref{chap:spectralatomicdchapter}, are results implying necessary and sufficient conditions for a function to be contained in $M^p_{\text{Dil}}(S^d)$ for $1 < p < (d-1)/2(d+1)$. To introduce the results, fix a non-zero $\chi \in C_c^\infty(0,\infty)$. For $s \geq 0$ and $1 \leq p \leq \infty$, define the norm
%
\begin{equation}
  \| a \|_{R^{s,p}[0,\infty)} = \sup\nolimits_{R > 0} \left( \int_0^\infty \left|\;\! \widehat{a}_R(t) \langle t \rangle^s \right|^p\; dt \right)^{1/p} \quad\text{where}\quad a_R(\lambda) = \chi(\lambda) a(R \lambda).
\end{equation}
%
Here $\widehat{a}_R(t) = \int_0^\infty a_R(\lambda) \cos(2 \pi \lambda t)$ is the cosine transform of $a_R$, i.e. the Fourier transform of the even extension of $a$ to a function on $\RR$. The particular choice of $\chi$ is irrelevant, as the resulting norms will all be equivalent. %The main result of this thesis is that the space $M^p_{\text{Dil}}(X)$s compactly supported away from $0$ are equal to the elements of $R^{s,p}[0,\infty)$ compactly supported away from $0$ for appropriate $s$. This is the first such characterization for $p \neq 2$ and $X \neq \TT^d$.

\begin{restatable}{theorem}{thmmaintheoremsphere} \label{maintheoremsphere}
  Suppose $1 < p < 2(d-1)/(d+1)$. If $s = (d-1)(1/p - 1/2)$, and $a: [0,\infty) \to \CC$ is a regulated function, then
  %
  \[ \| a \|_{M^p(S^d)} \lesssim \| a \|_{R^{s,p}[0,\infty)} \quad\text{and}\quad \| a \|_{M^p_{\text{Dil}}(S^d)} \sim \| a \|_{R^{s,p}[0,\infty)}. \]
\end{restatable}

One may view control on the $R^{s,p}$ norm as a smoothness condition on the multiplier $a$, and so the theorem fits the natural heuristic that one can control the regularity of a multiplier operator through control on the smoothness of it's symbol; indeed the Hausdorff-Young inequality implies a Sobolev space estimate
%
\begin{equation} \label{equation12980u21u89eqiwjqwiou}
  \| a_R \|_{W^{s,p'}[0,\infty)} \lesssim \| a \|_{R^{s,p}[0,\infty)}.
\end{equation}
%
The space $R^{s,p}[0,\infty)$ cannot be characterized by Sobolev, Besov, or Triebel-Lizorkin estimates, though we do have the Besov space estimate
%
\begin{equation} \label{aiwodjawoij334242}
  \| a \|_{R^{s,p}[0,\infty)} \lesssim \sup\nolimits_R \| a_R \|_{\dot{B}^{s+1/p - 1/2,2}_p[0,\infty)}.
\end{equation}
%
The result is somewhat intuitive. Inequality \eqref{equation12980u21u89eqiwjqwiou} tells us that elements of $R^{s,p}$ have $s$ derivatives in $L^{p'}$, though with `some extra control' occurring on the other side of the Fourier transform. Sobolev embedding heuristics tell us that having $s + 1/p - 1/2$ derivatives in $L^2$ is sufficient to have $s$ derivatives in $L^{p'}$. The presence of $L^2$ norms on the right hand side of \eqref{aiwodjawoij334242} allows us to losslessly convert between estimates on the Fourier transform side and obtain \eqref{aiwodjawoij334242}.

Interest in bounding such operators has a long history, being studied throughout the 20th century. Classical methods involving the analysis of special functions and orthogonal polynomials culminated in Bonami and Clerc's extension of the Marcinkiewicz multiplier theorem to multipliers for spherical harmonic expansions \cite{BonamiClerc}.

The methods used to attack the problem changed in the 1960s, when H\"{o}rmander introduced the powerful method of Fourier integral operators, which he used to obtain $L^p$ bounds for Bochner-Riesz multiplier expansions for spherical harmonics \cite{HormanderRiesz}, with later developments on $S^d$ and for spectral multipliers on other compact manifolds by Sogge \cite{SoggeSpectralClusters,SoggeRieszMeans,SoggeSphericalHarmonics}, Christ and Sogge \cite{ChristandSogge}, Seeger and Sogge \cite{SeegerSoggeBochnerRiesz}, Seeger \cite{SeegerEndpointEstimatesMultipliers}, Terence Tao \cite{Tao}, and Jongchon Kim \cite{KimSpectral}. None of these results were able to \emph{completely} characterize the uniform $L^p$ boundedness of dilated multipliers, making Theorem \ref{maintheoremsphere} the first result of that kind. The state of art results of \cite{KimSpectral} obtain boundedness under the weakest possible Besov assumptions on the function $a$, i.e. proving that
%
\begin{equation}
  \| a \|_{M^p_{\text{Dil}}(S^d)} \lesssim \| a \|_{\dot{B}^{s+1/p - 1/2,2}_p[0,\infty)}.
\end{equation}
%
The methods we use to prove Theorem \ref{maintheoremsphere} also follow the line of attack introduced by H\"{o}rmanbder, but are also heavily inspired by recent results on characterization of $L^p$ boundedness for radial Fourier multiplier operators on $\RR^d$ \cite{Cladek,GarrigosandSeeger,HeoandNazarovandSeeger,KimQuasiradial}, especially \cite{HeoandNazarovandSeeger}, which might be expected since transplantation results of \cite{Mitjagin} shows the results of \cite{HeoandNazarovandSeeger} are implied by the results of this paper.

To prove Theorem \ref{maintheoremsphere}, it will be essential to switch our point of view to the spectral multiplier theorem so we may apply the Fourier integral operator methods mentioned above. We now introduce this perspective.

\section{Spectral Multipliers of an Elliptic Operator}

We now restate our results in the more general setting of the study of spectral multipliers of an elliptic pseudo-differential operator. Appendix \ref{appendixpsueiodjaweiodj} gives a very brief summary of the relevant theory of pseudo-differential operators for our purposes. We suppose a smooth volume density on $X$ has been fixed, and then let $\COP{s}(X)$ denote the family of all classical, elliptic pseudo-differential operators of order $s$ on a compact manifold $X$ which are formally positive-semidefinite, in the sense that $\langle Pf, g \rangle = \langle f, Pg \rangle$ and $\langle Pf, f \rangle \geq 0$ hold for all $f,g \in C^\infty(X)$.

Let $X$ be a compact manifold equipped with a smooth volume density, and consider $P \in \COP{s}(X)$ for $s > 0$. By general properties of elliptic operators\footnote{See Theorem 18.1.29 of \cite{Hormander3}.}, $1 + P$ is an isomorphism between the Sobolev space $H^s(X)$ and $L^2(X)$,
%, and because $P$ is positive-definite, we have an estimate of the form
%
%\[ \big\| (1 + P) f \big\|_{L^2(X)} \geq \| f \|_{L^2(X)}, \]
%
%which implies $1 + P$ is a bijection between $H^a(X)$ and $L^2(X)$.
and so the inverse $(1 + P)^{-1}$, viewed as a map from $L^2(X)$ to itself, is compact by a form of the Rellich-Kondrachov embedding theorem\footnote{See Proposition 4.4 of \cite{Taylor}.}. By the spectral theorem for bounded, self-adjoint compact operators, there exists a discrete set $\Lambda \subset [0,\infty)$, and a decomposition $L^2(X) = \bigoplus\nolimits_{\lambda \in \Lambda} \mathcal{V}_\lambda$, where $\mathcal{V}_\lambda$ is a finite dimensional subspace of $C^\infty(X)$, such that $Pf = \lambda f$ for all $f \in \mathcal{V}_\lambda$. We use this decomposition to define a functional calculus for the operator $P$; given a bounded function $a: \Lambda \to \CC$, we can define an operator $a(P)$ on $L^2(X)$ so that $a(P) f = a(\lambda) f$ for all $f \in \mathcal{V}_\lambda$. Our goal is to study the regularity of $a(P)$, in terms of properties of the function $a$.

In what follows, we will assume the operator $P$ is an elliptic operator \emph{of order one}. This does not restrict the scope of our analysis; given a formally positive operator $P \in \COP{s}(X)$, a standard argument\footnote{See Theorem 3.3.1 of \cite{Sogge} for a proof of this fact, based on a technique of \cite{Seeley} initially developed for elliptic differential operators, but which can be straightforwardly adapted to pseudo-differential operators.} shows that the operator $P^{1/s}$ defined by the functional calculus above is in $\COP{1}(X)$, and the spectral theory of $P$ is identical with the spectral theory of $P^{1/s}$. Fixing the order of our operator reflects the fact that the theory of Fourier integral operators we will eventually employ is most elegant when the resulting oscillatory integrals have homogeneous phases of order one.

%The Weyl Law\footnote{See \cite{Hormander1} for details} for such operators tells us that $\# ( \Lambda \cap [0,R] ) = C \cdot R^d + O(R^{d-1})$, where
%
%\[ C = \int_{T^* M} \II[ p(x,\xi) \leq 1 ]\; d\xi\; dx, \]
%
%the function $p: T^* M \to [0,\infty)$ being the principal symbol of $P$.

The primary example for our purposes is obtained from a Laplace-Beltrami operator on a compact Riemannian manifold $X$; the operator $-\Delta$ is classsical (since it is a differential operator), and formally positive-semidefinite with respect to the volume form $dV$ on $X$, because of the `integration by parts' identity $\smash{\int_X (\Delta f_1) f_2\; dV = - \int_X g( \nabla_g f_1(x), \nabla_g f_2 )\; dV}$. Since we restrict our attention to elliptic operators of order one, the archetypal object of study is the operator $P = \sqrt{-\Delta}$.

We claim that the study of spectral multipliers for eigenfunctions is a generalization of the study of multipliers for spherical harmonic expansions introduced in the previous section. Define an operator $P_{\text{SH}}$ on $S^d$ by setting $P_{\text{SH}} f = \big( k + \tfrac{d-1}{2} \big)f$ for $f \in \mathcal{H}_k$. Then $T_{a_\rho} = a(\rho P_{\text{SH}})$. In Lemma \ref{lemmaoijioawjdioawjdw} we will see that $P_{\text{SH}}$ is an element of $\COP{s}(S^d)$ with respect to the Riemannian volume density on $S^d$.

To study the regularity of $a(P)$, we introduce the spaces $M^p(X)$, consisting of all functions $a: \Lambda \to \CC$ for which the operator $a(P)$ is  bounded on $L^p(X)$. The space $M^p(X)$ is then equipped with a norm
%
\begin{equation}
  \| a \|_{M^p(X)} = \sup \left\{ \frac{\| a(P) f \|_{L^p(X)}}{\| f \|_{L^p(X)}} : f \in C^\infty(X) \right\}.
\end{equation}
%
%For notational convenience, we write $M^p(X) = M^{p,p}(X)$.
$M^2(X)$ is isometrically equal to $l^\infty(\Lambda)$, and duality implies $M^p(X)$ is isometric to $M^{p'}\!(X)$. We then define $\| a \|_{M^p_{\text{Dil}}(X)} = \sup \| a_\rho \|_{M^p(X)}$, where $a_\rho(\lambda) = a(\rho \lambda)$.

Before the results of this thesis, no simple conditions had been proved necessary and sufficient for inclusion in $M^p_{\text{Dil}}(X)$ for $p \neq 2$ and $X \neq \TT^d$. The main result of this thesis is that under suitable assumptions on $X$ and the operator $P$, which we call Assumption A and Assumption B, we can find and prove the sufficiency of such a necessary condition. The first assumption is a curvature condition on the principal symbol of the $P$, and the second is an assumption about the operator's eigenvalues.

\vspace{0.5em}
\noindent \fbox{\parbox{\textwidth}{\textbf{Assumption A}: For each $x_0 \in M$, the co-sphere
%
\[ S_{\!\! x_0} = \{ \xi \in T^*_{\! \! x_0} M : p(x_0,\xi) = 1 \} \]
%
is a hypersurface in $T^*_{\!\!x} M$ with non-vanishing Gauss curvature, where $p$ is the principal symbol of the operator $P$.}}

\vspace{0.5em}

\noindent \fbox{\parbox{\textwidth}{\textbf{Assumption B}: The spectrum of the operator $P$ is contained in an arithmetic progression.}}

\vspace{0.4em}

\noindent \fbox{\parbox{\textwidth}{\textbf{Assumption C}: The operators $\cos(2 \pi t P)$ on $X$ have `finite propagation speed', in the sense that for one (and thus all) metrics on $X$ which are locally bi-Lipschitz equivalent to the Euclidean metric in coordinates, there exists $C > 0$ such that the kernel of the spectral multiplier operators $\cos(2 \pi t P)$ is supported on $\{ (x,y): d(x,y) \leq C t \}\}$ for all $t \in \RR$.}}

\vspace{0.4em}

\begin{lemma} \label{lemmaoijioawjdioawjdw}
  The operator $P_{\text{SH}}$ is an element of $\COP{1}(S^d)$ with respect to the Riemannian volume form on $S^d$, and satisfies Assumptions A, B, and C.
\end{lemma}
\begin{proof}
  Let $\Delta$ denote the Laplace-Beltrami operator on $S^d$. Then $P = \sqrt{-\Delta}$ is an element of $\COP{1}(S^d)$ satisfying Assumption A. If $f$ is a spherical harmonic of degree $k$, then
  %
  \begin{equation}
    \Delta f = - \sqrt{k(k+d-1)} f.
  \end{equation}
  %
  Thus we can write $P_{\text{SH}} = \alpha(P)$, where $\alpha(\lambda) = \sqrt{ \lambda^2 + c^2 }$ with $c = \tfrac{d-1}{2}$. Since $\alpha$ is a symbol of order $1$ with principal symbol $|\lambda|$, it follows that $P_{\text{SH}} \in \COP{1}(S^d)$, and $P_{\text{SH}}$ and $P$ share the same principal symbol\footnote{See Theorem 4.3.1 of \cite{Sogge} for details.}. Thus $P_{\text{SH}}$ satisfies Assumption A. The operator also satisfies Assumption B by definition since it is diagonalized by the spherical harmonics, and it's eigenvalues form the arithmetic progression $\NN + \tfrac{d-1}{2}$.

  Now we prove Assumption C. Fix $c \in \RR$, and let $Lu = (\Delta + c) u$. Then as $t$ varies, for each input $f$ the functions $u = \cos(2 \pi t P_{\text{SH}}) f$ solve the differential equation $\partial_t^2 u = Lu$, and so it suffices to show solutions to the partial differential equation have finite propagation speed. This follows if we can prove that for sufficiently small $t$, if $u(x,0) = 0$ for $d(x,x_0) \leq t$, then $u(x,s) = 0$ for $d(x,x_0) \leq t - s$ and $s \leq t$, since the general case follows by repeatedly composing the result by time invariance of the equation. We prove this result by using standard energy methods for partial differential equations. See Theorem 2.4.2 of \cite{SoggeHangzhou} for another example of this approach.

  Fix $x_0 \in X$ and $t < 1/2$. Define a geodesic normal coordinate ball\footnote{See Theorem 31 of \cite{spivakvol4} for a proof that the exponential maps a ball of radius $t$ at the origin in $T_{x_0} S^d$ diffeomorphically to a ball in $\RR^d$.} $B_t(x_0)$ in $S^d$, and set $e = u^2 + (\partial_t u)^2 + |\nabla_g u|_g^2$. Then define $E(s) = \int_{B_{t-s}(x_0)} e\; dV$. For solutions to $\partial_t^2 u = Lu$, we have $\partial_t e = 2[ (c + 1) u \partial_t u + \text{div}_g \{ \partial_t u\; \nabla_g u \} ]$.
  %
  Substituting this identity into the equation
  %
  \begin{equation}
    E'(s) = \int_{B_{t-s}(x_0)} \partial_t e\; dV - \int_{\partial B_{t-s}} e\; dS,
  \end{equation}
  %
  we find that
  %
  \begin{equation}
  \begin{split}
    E'(s) &= 2 \int_{B_{t-s}(x_0)} [(c+1) u \partial_t u + \text{div}_g \{ \partial_t u\; \nabla_g u \}]\; dV - \int_{\partial B_{t-s}(x_0)} e\; dS.
  \end{split}
  \end{equation}
  %
  The divergence theorem\footnote{See Proposition 59 of \cite{spivakvol4}} implies that if $\nu$ is the outward pointing unit normal vector field to $\partial B_{t-s}$, then
  %
  \begin{equation}
  \begin{split}
    2 \left| \int_{B_{t-s}(x_0)} \text{div}_g \{ \partial_t u \nabla_g u \}\; dV \right| &= 2 \left| \int_{\partial B_{t-s}(x_0)} \partial_t u\;\! \langle \nabla_g u, \nu \rangle_g\; dS \right|\\
    &\leq \int_{\partial B_{t-s}(x_0)} [|\partial_t u|^2 + |\nabla_g u|_g^2]\; dS\\
    &\leq \int_{\partial B_{t-s}(x_0)} e\; dS,
  \end{split}
  \end{equation}
  %
  and thus that
  %
  \begin{equation}
    E'(s) \leq \int_{B_{t-s}(x_0)} 2(c+1) u \partial_t u \leq |c + 1| E(s).
  \end{equation}
  %
  Thus by Gronwall's inequality\footnote{See Theorem 1.10 of \cite{TaoDispersive}}, $E(s) \leq E(0) e^{|c+1|s}$. Since $E(0) = 0$, we have $E(s) = 0$ for $s \leq t$, which implies that $u(x,s) = 0$ for $d(x,x_0) \leq t - s$.
\end{proof}

To obtain more general examples of operators satisfying Assumptions A and B, we will see in Section \ref{sec:geometriesinduced} that if an operator $P$ satisfies Assumption A then it gives $X$ a geometry, and Assumption B is closely related to this geometry. Assumption B implies that geodesics on $X$ with respect to this geometry are all closed and have commensurable lengths. Conversely, for most naturally occuring operators in $\COP{1}(X)$ with this geometrical property, including the Riemannian case where $P = \sqrt{-\Delta}$, there exists\footnote{See Lemma 29.2.1 of \cite{Hormander4}, discussed in slightly more detail (but not proved) in Section \ref{sec:PeriodicGeodesics}.} there exists $\tilde{P} \in \COP{1}(X)$ satisfying Assumptions A and B, and such that $P - \tilde{P}$ is a pseudo-differential operator of order $0$, so we can view $\tilde{P}$ as a perturbation of $P$. We can study the regularity of a subfamily of multipliers of $P$ by representing them as multipliers of $\tilde{P}$. Thus we can study multipliers of an appropriate pertubation of the Laplacian on any compact rank one symmetric space, or any Zoll manifold\footnote{The rank one $d$-dimensional symmetric spaces are $S^d$, $\RR \PP^d$, $\CC \PP^{d/2}$ (if $d$ is even), $\mathbb{H} \PP^{d/4}$ (if $d$ is a multiple of $4$), and the Cayley plane $\mathbb{O} \PP^2$ (if $d = 16$). Zoll manifolds are Riemannian manifolds diffeomorphic to $S^d$. See Chapter 3 of \cite{Besse} for a definition and classification of the rank one symmetric spaces, and Chapter 4 for an analysis of Zoll manifolds.}.

In order for Assumption C to hold, it is necessary that the operator $P^2$ be \emph{local} rather than merely \emph{pseudo-local}, i.e. it must be true that $\text{supp}(P^2u) \subset \text{supp}(u)$ for all inputs $u$ rather than that $P^2u$ is smooth and rapidly decaying away from the support of $u$. A result of Peetre \cite{PeetreDiff} implies that $P^2$ must be a \emph{differential operator}, and thus that $P$ be a Dirac operator. Thus a use of Assumption C restricts our assumption to Dirac operators, and thus explains the presence of the shift $\tfrac{d-1}{2}$ that occurs in our definition of dilations in our discussion of multipliers of spherical harmonic expansions in the previous section (since the square of the operatore $P_{\text{SH}} - \tfrac{d-1}{2}$ is not a differential operator). This assumption is only needed when combining frequency scales together in our theorem, and thus does not occur when we restrict our attention to the study of spectral multipliers of the form $a(P)$, where $a$ is compactly supported away from the origin.

Under Assumption A and Assumption B, in Chapter \ref{chap:boundedsinglefrequencyscale} we prove necessary and sufficient conditions hold for a compactly supported function to be contained in $M^p_{\text{Dil}}(X)$ for $1 < p < (d-1)/2(d+1)$.

\begin{restatable}{theorem}{thmmaintheorem} \label{maintheorem}
  Suppose $X$ is a compact $d$-dimensional manifold, and $P \in \COP{1}(X)$ satisfies Assumptions A and B. Then for $1 < p < 2(d-1)/(d+1)$, if $s = (d-1)(1/p - 1/2)$, then for any regulated $a$ with $\text{supp}(a) \subset [1/2,2]$,
  %
  \[ \| a \|_{M^p_{\text{Dil}}(X)} \sim \| a \|_{R^{s,p}[0,\infty)} \sim \left( \int_0^\infty |\;\!\widehat{a}(t)\;\!|^p \langle t \rangle^s\; dt \right)^{1/p}. \]
\end{restatable}

For most manifolds $X$ and $P$, elements of $\mathcal{V}_\lambda$ are difficult to describe explicitly; even for the relatively simple case of spherical harmonics on the sphere $S^d$, many questions about the geometric behavior of eigenfunctions remain open. Many arguments in harmonic analysis involve an interplay between spatial and frequential control, and without explicit descriptions of eigenfunctions, spatial control becomes difficult. Nonetheless, we will find we can obtain some spatial control by utilizing the wave equation on $M$, which carries geometric information via the behavior of wave propagation. Under the curvature assumptions we make, wave propagation has sufficient smoothing properties to match certain necessary conditions that multipliers need in order to be bounded. And Assumption B implies that the wave equation associated with the operator $P$ is periodic, which simplifies the large time analysis of the wave equation.

Under Assumption C, and the additional assumption of certain wave estimates associated with $P$ at each frequency scale, in Chapter \ref{chap:spectralatomicdchapter} we are able to characterize all elements of $M^p_{\text{Dil}}(X)$ rather than just the compactly supported ones. The statement of the estimates is somewhat technical, but relates to control over discrete averages of a wave equation on $X$. Fix $k \geq 0$, and consider a pair of maximal $2^{-k}$ separated subsets $\mathcal{X}_k$ and $\mathcal{T}_k$ of $X$ and of $[0,1]$. Fix a family of $L^1$-normalized functions $\mathfrak{b} = \{ b_{t_0}: \RR \to \CC \}$ and $\mathfrak{u} = \{ u_{x_0}: X \to \CC \}$, with $u_{x_0}$ supported on a $2^{1-k}$ neighborhood of $x_0$, and $b_{t_0}$ on a $2^{1-k}$ interval centered at $t_0$. Also fix a bump function $q \in C_c^\infty(\RR)$ with $\supp(q) \subset [1/4,4]$ and $q(\lambda) = 1$ for $\lambda \in [1/2,2]$, and define $Q_k = q(P/2^k)$, which we view as `frequency localizations' at a scale $2^k$. For each $(x_0,t_0) \in \mathcal{X}_k \times \mathcal{T}_k$, define $f_{x_0,t_0} = \int b_{t_0}(t) (\cos(2 \pi i t P) \circ Q_k) \{ u_{x_0} \}$, a time average of a frequency localized solution to the wave equation $\partial_t^2 u = - P^2 u$. Finally, define an operator $A_k$ from functions on $\mathcal{X}_k \times \mathcal{T}_k$ to functions on $X$ by $A_k \{ c \} = \sum\nolimits_{(x_0,t_0) \in \mathcal{X}_k \times \mathcal{T}_k} c(x_0,t_0) f_{x_0,t_0}$. We assume uniform bounds on such operators.

\vspace{0.5em}

\noindent \fbox{\parbox{\textwidth}{\textbf{Assumption} $\text{Wave-Bound}(p)$:
There exists a constant $C_0 > 0$ such that
%
\begin{align*}
    \left\| A_k \{ c \} \right\|_{L^p(X)} \leq C_0\; 2^{kd/p'} \left( \sum\nolimits_{(x_0,t_0) \in \mathcal{X}_k \times \mathcal{T}_k} \left[ |c(x_0,t_0)| \langle 2^k t_0 \rangle^s \right]^p \right)^{1/p}.
\end{align*}
%
uniformly in $k$, $\mathfrak{b}$, and $\mathfrak{u}$, where $s = (d-1)(1/p - 1/2)$.
%
%\begin{align*}
%    \left\| A_k \{ c \} \right\|_{L^{p_*}(X)} \leq C\; 2^{kd} \left( 2^{-k(2d+1)} \sum\nolimits_{(x_0,t_0) \in \mathcal{X}_k \times \mathcal{T}_k} |c(x_0,t_0)|^{p_*} \langle 2^k t_0 \rangle^{d-1} \right)^{1/p_*}.
%\end{align*}
% 2^{kd/p'} BUT PICK UP 2^{-k(d+1)}
% SO 2^{- k(1 + d/p)}
}}

\vspace{0.4em}

\begin{restatable}{theorem}{thmatomicscalestheorem} \label{atomicscalestheorem}
  Let $X$ be a compact manifold, and let $P \in \COP{1}(X)$ satisfy Assumptions A, B, and C. Fix $1 < p < q < 2$ with $1 < q < 2d/(d+1)$, and suppose that $\text{Wave-Bound}(q)$ holds for $P$. Then for any regulated function $a$, if $s = (d-1)(1/p - 1/2)$,
  %
  \[ \| a \|_{M^p_{\text{Dil}}(X)} \sim \| a \|_{R^{s,p}[0,\infty)}. \]
\end{restatable}

In Chapter \ref{chap:spectralatomicdchapter}, we verify that $\text{Wave-Bound}(p)$ holds for $1/p - 1/2 > 1/(d-1)$ when $P = P_{\text{SH}}$, which justifies why Theorem \ref{atomicscalestheorem} implies Theorem \ref{maintheoremsphere}.% We thus obtain the following characterization of boundedness.

%\begin{restatable}{theorem}{thmmaintheoremsphere} \label{maintheoremsphere}
%  Consider $S^d$, equipped with the operator $P_{\text{SH}}$. Then if $1/p - 1/2 > 1/(d-1)$, $s = (d-1)(1/p - 1/2)$, and $a$ is any regulated function, then $\| a \|_{M^p_{\text{Dil}}(S^d)} \sim \| a \|_{R^{s,p}[0,\infty)}$.
%\end{restatable}

%Up to lower order pertubations, the Laplace-Beltrami operator is the unique elliptic operator of order two whose principal symbol is equal to the square of the norm induced on $T^* M$ from the Riemannian metric on $M$. In Chapter BLAH, we will see that this property is true of \emph{all} elliptic operators that satisfy the first assumption of our problem, provided that we are willing to generalize our study to \emph{Finsler manifolds} rather than just Riemannian manifolds. The second assumption is then closely related to the behaviour of geodesics on the manifold, in particular, that all geodesics and closed and have common length.

In the next chapter, we will begin our study by describing what is currently known for the boundedness problem on the compact manifold $\TT^d$, where eigenfunctions are explicit, and, using a simple transplantation theorem, one can study the behavior of spectral multipliers via the Fourier transform and radial convolution operators on $\RR^d$. Our analysis here will help gain intuition and motivate potential hypotheses in the more general setting.

%Under this assumption, the function $p$ defines a \emph{Finsler metric} on $M$, and thus we can define a theory of geodesics on $M$. The second assumption we make is that with respect to this metric, all geodesics on $M$ are closed, and have common length.

%
%\begin{itemize}
%    \item Consider what seems a relatively simple case, the sphere $S^d$ equipped with the Laplacian. Then $\Lambda = \{ k(k+d-1) : k \geq 0 \}$, and for $\lambda = k(k+d-1)$, the space $\mathcal{V}_\lambda$ consists of the spherical harmonics of degree $k$, restrictions of harmonic, homogeneous, degree $k$ polynomials on $\RR^{d+1}$ to $S^d$. There is a basis for such a space defined in terms of the associated Legendre polynomials, but this is still relatively non-explicit.

%    \item On the other hand, on $\TT^d$ with it's Laplacian $-\Delta = - \partial_1^2 - \cdots - \partial_d^2$, for each $\lambda \in \Lambda$, $\mathcal{V}_\lambda$ has an orthogonal basis consisting of linear combinations of the exponentials $x \mapsto e^{2 \pi i k \cdot x}$, and thus multipliers of the Laplacian can be understood using the Fourier series.
%\end{itemize}
%

\chapter{Radial Multipliers on Euclidean Space} \label{sec:radmult}

Consider the elliptic operator $P = \sqrt{-\Delta}$ on $\TT^d$, where $\Delta = \partial_1^2 + \cdots + \partial_d^2$ is the usual Laplacian. In such a setting, we have an explicit basis for the eigenfunctions of $\Delta$: for a given eigenvalue $\lambda > 0$, the space $\mathcal{V}_\lambda$ has an orthonormal basis consisting of the exponentials $e^{2 \pi i n \cdot x}$, where $n \in \ZZ^d$ and $|n| = \lambda$. Since the Fourier series of a function gives the expansion of the function in this basis, it follows that we can expand the spectral multiplier operator $T = a(P)$ using a Fourier series, i.e. writing
%
\begin{equation}
  Tf(x) = \sum\nolimits_{n \in \ZZ^d} a( |n| ) \widehat{f}(n) e^{2 \pi i n \cdot x}.
\end{equation}
%
Thus a multiplier of $\Delta$ on $\TT^d$ is nothing more than a Fourier multiplier operator on $\TT^d$ whose symbol is radial. Methods of transplantation\footnote{See Section 3.6.2 of Grafakos \cite{Grafakos}, based on methods of de Leeuw \cite{deLeeuw}} show that $\| a \|_{M^p(\RR^d)} \sim \| a \|_{M^p_{\text{Dil}}(\TT^d)}$, where $\| a \|_{M^p(\RR^d)}$ is the operator norm of the Fourier multiplier on $\RR^d$ given by
%
\begin{equation}
  Tf(x) = \int_{\RR^d} a \big(|\xi| \big) \widehat{f}(\xi) e^{2 \pi i \xi \cdot x}\; d\xi 
\end{equation}
%
on $L^p(\RR^d)$. In this section we will focus on operators of this type, which we call \emph{radial Fourier multiplier operators}.

The study of the regularity of Fourier multiplier operators has proved central to the development of modern harmonic analysis and the theory of linear partial differential operators. This is because essentially any translation invariant operator $T$ on $\RR^d$ is a Fourier multiplier operator, i.e. we can find a tempered distribution $m$ on $\RR^d$, the \emph{symbol} of $T$, such that for any Schwartz function $f$,
%
\begin{equation}
  Tf(x) = \int_{\RR^d} m(\xi) \widehat{f}(\xi) e^{2 \pi i \xi \cdot x}\; d\xi.
\end{equation}
%
%Applying the notation of spectral calculus, one might also write this operator as $m(D)$, where $D = (2 \pi i)^{-1} \nabla$ is a self-adjoint normalization of the gradient operator. Thus the study of the boundedness of translation invariant operators is closely connected to the study of the interactions of the operators
%
%\[ E_\xi f(x) = \widehat{f}(\xi) e^{2 \pi i \xi \cdot x}, \]
%
%which act as projections onto the common eigenspaces of the components of $D$, since we can write $m(D)$ as a vector-valued integral of the form
%
%\[ m(D) = \int_{\RR^d} m(\xi) E_\xi\; d\xi. \]
%
%Thus $m(D)$ is represented as a weighted average of the operators $\{ E_\xi \}$.
The study of translation invariant operators emerges from classical questions in analysis, such as the convergence of Fourier series, and problems in mathematical physics related to the study of the heat, wave, and Schr\"{o}dinger equations. These physical equations also often have \emph{rotational} symmetry, so it is natural to restrict our attention to translation-invariant operators which are also rotation-invariant. These operators are precisely the family of radial multiplier operators. If $m(\xi) = a(|\xi|)$ for a tempered distribution on $[0,\infty)$ we will write $T = a(P)$, where $P = \sqrt{-\Delta}$, since $Tf = a(\lambda) f$ whenever $\Delta f = - \lambda^2 f$.
%Such operators are precisely those operators associated with \emph{radial} symbols $m: \RR^d \to \CC$, i.e. symbols for which there exists a function $h: [0,\infty) \to \RR$ such that
%
%\[ m(\xi) = h( |\xi|) \]
%
%for some function $h: [0,\infty) \to \CC$. This is the class of \emph{radial Fourier multipliers}. The study of radial multipliers is closely connected to interactions between the operators
%
%\[ E_\lambda f(x) = \int_{|\xi| = \lambda} \widehat{f}(\xi) e^{2 \pi i \xi \cdot x}\; d\xi, \]
%
%for $0 < \lambda < \infty$, which are the projection operators onto the eigenspaces of $\Delta$.  Similar to the study of $m(D)$, we then have
%
%\[ h \left( P \right) = \int_0^\infty h(\lambda) E_\lambda\; d\lambda. \]
%
%Thus studying the regularity of radial Fourier multipliers allows us to understand the interactions between the operators $\{ E_\lambda \}$.

\section{Convolution Kernels of Fourier Multipliers}

It is often useful to study spatial representations of these operators, since one can often exploit certain geometric information about the behavior of operators. Given any translation invariant operator $T$ on $\RR^d$, we can associate a tempered distribution $k$, the \emph{convolution kernel} of $T$, such that
%
\begin{equation}
  Tf(y) = \int_{\RR^d} k(x) f(y-x)\; dx \quad\quad\text{for all $f \in \mathcal{S}(\RR^d)$}.
\end{equation}
%
If $T$ is radial, then so is $k$, and so we can write $k(x) = b(|x|)$ for some distribution $b$ on $[0,\infty)$, and then we have a representation
%
\begin{equation}
  T = \int_0^\infty b(r) S_r\; dr,\quad\text{where}\quad S_rf(x) = \int_{|y| = r} f(x + y)\; dy 
\end{equation}
%
are the \emph{spherical averaging operators}.

With the notation as above, the function $k$ is the Fourier transform of $m$, and the function $b$ is a Bessel transform of $a$, i.e. $b = \mathcal{B}_d a$, where
%
\begin{equation}
  \mathcal{B}_d a(r) = \int_0^\infty s^{\frac{d-1}{2}} J_{\frac{d-1}{2}}(2 \pi s) a(s)\; ds,
\end{equation}
% b(r) = int_0^infty s^{d-1} a(s) int_{S^{d-1}} e^{2 pi i s xi * x}
%      = V_{d-2} int_0^infty s^{d-1} a(s) int_{-1}^1 (1-s^2)^{d/2-1} e^{2piist}
%      = 2 V_{d-2} int_0^infty s^{d-1} a(s) int_0^1 (1-s^2)^{d/2 - 1} cos(2 pi s t)
%      = V_{d-2} Gamma(nu + 1/2) pi^{1 - d/2} int_0^infty s^{(d-1)/2} a(s) J_{(d-1)/2}(2 pi s)
%      = int_0^infty
%
% nu = (d-1)/2
% z = 2 pi s
% J_nu(z) Gamma(nu + 1/2) pi^{1/2 - nu} / 2 s^nu
%
and where
%
\begin{equation}
  J_\alpha(\lambda) = \frac{(\lambda / 2)^\alpha}{\Gamma(\alpha + 1/2)} \int_{-1}^1 e^{i \lambda s} (1 - s^2)^{\alpha - 1/2}\; ds.
\end{equation}
%
Using the theory of stationary phase, for each $d$ we can write
%
\begin{equation}
  J_d(\lambda) = e^{2 \pi i \lambda} s_1(\lambda) + e^{-2 \pi i \lambda} s_2(\lambda).
\end{equation}
%
for symbols $s_1$ and $s_2$ of order $-1/2$. The presence of $e^{2 \pi i \lambda}$ and $e^{-2 \pi i \lambda}$ allows one to relate the Bessel transform of a function to it's Fourier transform, to a certain extent. In particular, we record the following result of Garrigos and Seeger \cite{GarrigosandSeeger}.

\begin{theorem} \label{GarrigosSeegerTheorem}
    Suppose $d > 1$ and $1/2d < 1/p - 1/2 < 1/2$, and suppose $a: [0,\infty) \to \CC$ has compact support away from the origin. Then, with implicit constants depending on the support of $a$,
    %
    \[ \left( \int_0^\infty |\mathcal{B}_d a (t)|^p t^{d-1} \right)^{1/p} \sim_{p,d,\phi} \left( \int_0^\infty |\;\!\widehat{a}(t)|^p \langle t \rangle^{(d-1)(1 - p/2)}\; dt \right)^{1/p}, \]
    %
    where $\widehat{a}(t) = \int_0^\infty a(\lambda) \cos(2 \pi \lambda t)\; dt$ is the cosine transform of $a$.
\end{theorem}

%Later on, we will also need to consider analogues of this result for \emph{quasi-radial} multipliers, i.e. multipliers of the form $a \circ r$, where $r: \RR^d \to [0,\infty)$ is a homogeneous function of order one with $r(\xi) \neq 0$ for $\xi \neq 0$, and such that the cosphere $\Sigma = \{ \xi : r(\xi) = 1 \}$ has non-vanishing Gaussian curvature. We calculate using the Fourier transform that
%
%\begin{align*}
%  \int (a \circ r)(\xi) e^{2 \pi i \xi \cdot x}\; d\xi &= C \int_0^\infty a(t) \int_\Sigma e^{2 \pi i t \xi \cdot x}\; d\xi = (\mathcal{A}_r a)(x).
%\end{align*}
%
%Using stationary phase and the curvature of $\Sigma$, we can write
%
%\[ \int_\Sigma e^{2 \pi i t \xi \cdot x}\; d\xi = s_1(x,t) e^{2 \pi i t \xi_+(x) \cdot x} + s_2(x,t) e^{2 \pi i t \xi_-(x) \cdot x} \]
%
%and thus $(\mathcal{A}_r a)(x)$ is connected to the Fourier transform of $a$ near $\xi_+(x) \cdot x$ and $\xi_-(x) \cdot x$.

\section{The Radial Multiplier Conjecture}

The general study of the boundedness properties of Fourier multiplier operators in multiple variables was initiated in the 1950s, as connections of the theory to partial differential equations became more fully realized\footnote{See \cite{Hormander1} for a more detailed overview of what was known at this time.}. It was quickly realized that the most fundamental estimates were $L^p \to L^q$ estimates for such operators.
%
%for $1 \leq p \leq 2$, and $q \geq p$, which by duality are equivalent to bounds
%
%\[ \| Tf \|_{L^{p^*}(\RR^d)} \lesssim \| f \|_{L^{q^*}(\RR^d)}, \]
%
It is therefore natural to introduce the space $M^p(\RR^d)$, consisting of all symbols $m$ which induce a Fourier multiplier operator $T$ bounded on $L^p(\RR^d)$. Duality implies that $M^p(\RR^d)$ is isometric to $M^{p'}(\RR^d)$, where $p$ and $p'$ are conjugates, so it suffices to study the spaces $M^p(\RR^d)$ where $1 \leq p \leq 2$ or when $2 \leq p \leq \infty$. We only know simple characterizations of $M^p(\RR^d)$ for very particular $p$:
%
\begin{itemize}
    \item The spaces $M^1(\RR^d) = M^\infty(\RR^d)$ can be characterized, by virtue of the fact that the boundedness of operators with domain $L^1(\RR^d)$ or range $L^\infty(\RR^d)$ is often simple; we have $M^1(\RR^d) = \widehat{M}(\RR^d)$, where  $M(\RR^d)$ is the space of all finite signed Borel measures, equipped with the total variation norm. The proof follows from Schur's test for integral kernels, which often gives tight estimates to bound operators with domain $L^1$ or range $L^\infty$.

    \item The unitary nature of the Fourier transform also allows for the characterization $M^2(\RR^d) = L^\infty(\RR^d)$. The proof follows from Parseval's identity.
\end{itemize}
%
It is perhaps surprising that these are the \emph{only} known characterizations of the spaces $M^p(\RR^d)$. No necessary and simple conditions for boundedness are known for any other values of $p$, and perhaps no simple characterization exists.

Despite the lack of a characterization of the classes $M^p(\RR^d)$, it is surprising that we \emph{can} conjecture a characterization of the subspace of $M^p(\RR^d)$ consisting of \emph{radial symbols}, for an appropriate range of exponents. The conjectured range of estimates was first suggested by a result of \cite{GarrigosandSeeger}, concerning the boundedness of a radial Fourier multiplier $T$ with symbol $a(|\cdot|)$ \emph{restricted to radial functions}, i.e. such that the norm
%
\begin{equation}
  \| a \|_{M^p_{\text{Rad}}(\RR^d)} = \sup \left\{ \frac{\| a(P)f \|_{L^q(\RR^d)}}{\| f \|_{L^p(\RR^d)}} : \text{$f$ is radial} \right\}
\end{equation}
%
is finite. For any $\chi \in C_c^\infty(\RR)$, if we define $k_R$ to be the Fourier transform of $\chi(|\cdot|) a(R |\cdot|)$, then the identity $k_R = a(RP) \{\;\! \widehat{\chi}\;\! \}$ and dilation symmetry imply that
%
\begin{equation}
  \| k_R \|_{L^p(\RR^d)} \lesssim \| a \|_{M^p_{\text{Rad}}(\RR^d)},
\end{equation}
%
with implicit constants depending on $\chi$. For $d > 1$, and for $1/p - 1/2 > 1/2d$, Garrigos and Seeger proved \cite{GarrigosandSeeger} the converse bound
%
\begin{equation} \label{RadialMultiplierRadialBound}
    \| a \|_{M^p_{\text{Rad}}(\RR^d)} \lesssim \sup\nolimits_{R > 0} \| k_R \|_{L^p(\RR^d)},
\end{equation}
%
Theorem \ref{GarrigosSeegerTheorem} implies that
%
\begin{equation}
  \sup\nolimits_{R > 0} \| k_R \|_{L^p(\RR^d)} \sim \| a \|_{R^{s,p}[0,\infty)},
\end{equation}
%
where $s = (d-1)(1/p - 1/2)$. Thus Garrigos and Seeger have proved that the isomorphism $M^p_{\text{Rad}}(\RR^d) = R^{s,p}[0,\infty)$ holds in the range above.

%Lower bounding the left hand side of \eqref{RadialMultiplierRadialBound} by the right hand side follows from rescaling the convolution identity $k_j * \psi = k_j$, where $\psi$ is a radial Schwartz function whose Fourier transform is equal to one on the Fourier support of $\chi$. The hard part of the result of \cite{GarrigosandSeeger} is upper bounding the left hand side by the right.

For any other value of $p$, $M^p_{\text{Rad}}(\RR^d)$ is a proper subset of $R^{s,p}[0,\infty)$, as the following counterexamples show:
%
\begin{itemize}
  \item For $p = 1$, the inclusion `fails by a logarithm', as we can see by the characterization of $M^1(\RR^d)$ in the last section; if $a$ is supported on an interval $I$ we have the converse inequality $\| a \|_{M^1(\RR^d)} \lesssim \log |I|\; \| a \|_{R^{s,p}[0,\infty)}$ where dependence on $I$ is in general sharp, and cannot be removed.

  \item Suppose $p \geq 2d/(d+1)$. Then $R^{s,p}[0,\infty)$ contains all elements of the Besov space $\dot{B}^{{1/2},2}_p[0,\infty)$ supported on $[1/2,2]$, and thus, in particular, must contain unbounded functions for $p > 1$ (because the Sobolev embedding theorem fails when used at the endpoint to embed into $L^\infty$). Since $M^p(\RR^d) \subset M^2(\RR^d) = L^\infty[0,\infty)$, $M^p(\RR^d)$ cannot contain unbounded functions, and thus $M^p(\RR^d)$ is a proper subset of $R^{s,p}(\RR^d)$.
  % 1/p - 1/2 = 1/2d, have 1/2 derivatives in L^p
  %

%  But any function in $M^{p,q}_{\text{Rad}}(\RR^d)$ supported on $[1/2,2]$ would have to be in $M^{2d/(d+1),2}_{\text{Rad}}(\RR^d)$ by Bernstein's inequality,

  % Find a so that it's Fourier transform decays on the order of
  % |t|^{-(d-1)/2d}, and then define a_N = a * chi_{1/N}. Then
  %
  % the R^{s,q} norm should be (log N)^{1/q}
  %
  % |t|^{-}
  % (d-1)/2d = 1 - alpha
  % alpha = 1 - (d-1)/2d = (2d - d + 1) / 2d = (d + 1)/2d
  % So take a = |t|^{-(d+1)/2d}
  %
  % On the other hand, if f = I_{[1,N]} then it's L^p(R^d) norm
  % is O(N^{1/p}) but the L^2 norm of a(P)f is >> log N) N^{-1}



  % a_N = I_{[1,1+1/N]} * chi_{1/N}
  % Likely has B^{0.5,2}_p norm O( N^{1/2} )
  %
  % Fourier transform of a_N is [sin((1 + 1/N) t) - sin(t)] / t
  % int_1^infty t^{(s-1)q}

  % So if we cut off the Fourier transform at a frequency 2^j for j > 0
  % then the function is smooth on [2^j, 2^{j+1}] with amplitude 2^{-j}
  % and so the Fourier transform should be concentrated near [0,2^{-j}] with
  % amplitude 1, and thus have L^2 norm 2^{-j/2}. But for half derivatives
  % we need to multiply by 2^{j/2}. So the B^{0.5,2}_infty norm is finite
  % But the other ones are infinite.

  % Which is smooth and slowly oscillating away from the origin.
  % If we cut off the Fourier transform at a frequency 2^j for j > 0, then the
  % function has amplitude 2^{-j}, so when we convert back we get a function which
  % is 

%   $a_N(\lambda) = N^{1/2} \mathbb{I}_{[1,1 + 1/N]}$. Then
   % a_N = N^{1/2} I_{[1,1+1/N]}
   % f_N = |x|^{-(d+1)/2} e^{-is} I(1 <= |x| <= N)
   %
   % Then L^p norm of f_N is O(log N)^{1/p}
   % Then a_N(P) f_N >> log N for ||x| - 1| <= 1/N
   % So the L^2 norm of a_N(P) f_N is >> log N 
   %
   % So for p > 1, the operators a_N(P) are not uniformly
   % bounded from L^p to L^2
   %
   % But they are in R^{0,2} uniformly in N

   % q(s-1) < -1
   % s < 1 - 1/q
   % 1 - 1/q = 1 - (d+1)/2d = (d - 1)/2d

   % (d-1)/2d

   % O(1) over small time scales.
   % q(s-1) < -1
   % s < -1 - 1/q


%    and thus by interpolation and duality, in $M^2_{\text{Rad}}(\RR^d)$. But $M^2_{\text{Rad}}(\RR^d)$ is isometrically equal to $L^\infty(\RR^d)$, and so any function would have to be bounded. Thus there are functions in $R^{s,q}_\alpha(\RR^d)$ which are not in $M^{p,q}_{\text{Rad}}(\RR^d)$ when $p \geq 2d/(d+1)$.

   % If bounded on M^p, then bounded on M^{p,2}
   % But boundedness from L^p to L^2
   % 1/p - 1/2 >= 1/(d+1)

   % and if bounded on M^{p,infty} = M^{1,p'}, then the Fourier transform of your function must lie in p'
     % Can only be bounded from L^p to L^q if they are also bounded from L^1 to L^q
     % L^p to L^q boundedness implies L^1 to L^q, but the opposite is surprising.


%  consider the functions $a_\delta(\lambda) = (1 - \lambda^2)^\delta_+$, whose associated operators $a_\delta(P)$ give the class of Bochner-Riesz multipliers. One may represent the cosine transform of the functions $a_\delta$ using Bessel functions, and the asymptotics of such functions tells us $|\widehat{a}_\delta(t)| \lesssim t^{-\delta - 1}$ as $t \to \infty$, so that $a_\delta \in R^{q,s}_\alpha[0,\infty)$ for $s < \delta + 1 - 1/q$, and thus for $s = (d-1)(1/q - 1/2)$ when $\delta > d/q - (d+1)/2$.
\end{itemize}
  % (d-1)(1/q - 1/2) derivatives in L^{q'}
  %
  % Is implied if we have d(1/q - 1/2) derivatives in L^2
  %
  % But 

  % 1/2 derivatives in L^p for p > 1
  % (d-1)(1/p - 1/2) derivatives in L^{p'}

% For $p = 2d/(d+1)$, consider the  In particular, if $q > 2d/(d+1)$, then $a_0 \in R^{q,s}_\alpha[0,\infty)$.

% On the other hand, the kernel $k_\delta$ of $a_\delta(P)$ can also be expressed using Bessel functions, and we have $|k_\delta(t)| \lesssim |t|^{-(d+1)/2-\delta}$, and so we find using the characterization of $M^{1,q}(\RR^d)$ that $a_\delta \in M^{1,q}(\RR^d)$
% (-delta - 1 + s)q < -1
% -delta - 1 + s < -1/q
% s < delta + 1 - 1/q
% (d-1)(1/q - 1/2) < delta + 1 -1/q
% d/q - (d-1)/2 < delta + 1
% delta > d/q - - (d+1)/2


%
% J_{delta + 1/2}(z) = (z/2)^{delta + 1/2} / pi^{1/2} Gamma(delta + 1) int_{-1}^1 (1 - t^2)^delta e^{izt} dt
% So the Fourier transform of (1 - lambda^2)^delta is
%     J_{delta + 1/2}( 2 pi t ) t^{-delta - 1/2} pi^{-delta} Gamma(delta + 1)
%     Which is O( <t>^{-1/2} t^{-delta - 1/2} ) for t -> infinity
% (d-1)(1/q - 1/2) > -delta - 1/q

%
% int_0^infty [ <t>^delta <t>^s ]^q dt
% int_0^infty <t>^{(delta + s) q} dt
%
%     (delta + s) q < -1
%   delta + s < -1/q
%
% Integrability near 0: q (-d/2 - delta) >= -1
%           delta <= 1/q - d/2
% Integrability near infinity:   q (-d/2 - delta + s) > -1
%           delta < 1/q - d/2 + s

%and $q > 1$, we have
%
%\[ \| a \|_{M^{1,q}(\RR^d)} \gtrsim \| a \|_{R^{q,s}_\alpha[0,\infty)} \]
%
%where $s = \tfrac{d-1}{2}$ and $\alpha = 1 - 1/q$

%The result cannot hold for $p \geq 2d/(d+1)$, because, adapting the analysis of Fefferman \cite{Fefferman} to a family of multiplier operators whose symbols are smooth and adapted to a $\delta$ neighborhood of an annulus as $\delta \to 0$, such a result would imply all Kakeya sets have positive measure TODO I DON'T THINK THIS WORKS FOR RADIAL INPUTS.

%
% T chi = k gives the L^{q'} -> L^{q'} extreme values
%
% < T chi , k^{q' - 1} > = |k|_{L^{q'}}^{q'}
% < chi, T k^{q' - 1} >
%
% < T chi , g > = |Tf|_{L^q}
%
% If |m| ~ |k|_{L^{q'}}
%
% Then we get a counterexample if |k|_{L^q} is small but |k|_{L^{q'}} is big.
% That's legitimate because of Bernstein's inequality.
%
% | Tf |_{L^q} = |k|_{L^{q'}} |f|_{L^q} ~ |k|_{L^{q'}} |f|_{L^p}
% Thus |m| << |k|_{L^{q'}} rather than 
%
%
% To be bounded on L^q, need qth root cancellation
% But only have square root cancellation 

%hold. On the other hand, by Plancherel, we can find a radial function $f$ with compact Fourier support such that $\| \kappa_t * f \|_{L^2(\RR^d)} \gtrsim 1$ uniformly as $t \to \infty$ (consider such $f$ with Fourier support concentrated near the maximum value of the Fourier transform of $\kappa_0$). By Bernstein's inequality,
%
%\[ \| m_t \|_{M^{p,q}(\RR^d)} \| \kappa_t * f \|_{L^p(\RR^d)} \gtrsim 1 \]

%for each $t$ we can find a radial function $f_t$ with Fourier support near $1$, with $\| f_t \|_{L^2(\RR^d)} = 1$ and with $\| \kappa_t * f_t \|_{L^2(\RR^d)} \gtrsim 1$. By Sobolev embedding we must also have $\| \kappa_t * f_t \|_{L^p(\RR^d)} \gtrsim 1$ and $\| f_t \|_{L^p(\RR^d)} \lesssim 1$.

%And one can see that the condition is not sufficient for $q > \min \left(2, \tfrac{d-1}{d+1} p' \right)$ by testing the same multiplier operators above on `Knapp examples', i.e. smooth functions whose Fourier transforms are adapted to a $\delta$ cap on a $\delta$-neighborhood of the annulus upon which $m$ is supported.

It is natural to conjecture the same bounds hold when we remove the constraint that the inputs are radial, i.e so that for a radial function $m(\xi) = a(|\xi|)$ and $1/2d < 1/p - 1/2 < 1/2$,
%
\begin{equation} \label{RadialMultiplierBound}
  \| m \|_{M^p(\RR^d)} \sim_{p,d} \| a \|_{R^{p,s}[0,\infty)} \quad\text{where $s = (d-1)(1/p - 1/2)$}.
\end{equation}
%
%and for general locally integrable symbols $m$,
%
%\[ \| m \|_{M^{p,q}} \sim_{p,q,d} \| k \|_{\dot{B}_{-d/p^*}^{q,\infty}} \]
%
%However, for $1/2d < 1/p - 1/2 < 1/(d+1)$ an additional counterexample is obtained by considering Knapp type examples, i.e. considering the multiplier operators whose symbols are supported on $\delta$-annuli as above, and testing them against smooth, \emph{non-radial} functions whose Fourier transforms are adapted to $\delta$-caps on the annuli. These counterexamples show that the condition $q < (d-1)/(d+1) \cdot p'$ is necessary for \eqref{RadialMultiplierBound} to hold. We thus conjecture that \eqref{RadialMultiplierBound} holds for all radial $m$ when either $1/p - 1/2 > 1/(d+1)$ and $p \leq q < 2$, or when $1/2d < 1/p - 1/2 < 1/(d+1)$ and $p \leq q < (d-1)/(d+1) \cdot p'$, and we call this the \emph{radial multiplier conjecture} on $\RR^d$.
We call this the \emph{radial multiplier conjecture} on $\RR^d$.

We now know, by results of Yaryong Heo, Fedor Nazarov, and Andreas Seeger \cite{HeoandNazarovandSeeger} that the radial multiplier conjecture is true when $d \geq 4$ and when $1/(d-1) < 1/p - 1/2 < 1/2$. A summary of the proof strategies of this argument is provided in the following two sections, and a major part of our bounds for spectral multipliers follow by adapting this argument to the non-Euclidean setting. Partial improvements were obtained by Cladek \cite{Cladek} for symbols $m$ compactly supported away from the origin, obtaining results for multipliers with a compactly supported symbol when $d = 4$ and $1/p - 1/2 > 11/36$, and establishing \emph{restricted weak type} bounds when $d = 3$ and $1/p - 1/2 > 41/22$. But the radial multiplier conjecture has not been resolved fully in any dimension $d$, we do not have any strong type $L^p$ bounds when $d = 3$, and no bounds whatsoever are known when $d = 2$.

\section{Radial-Multiplier Bounds by Density Decompositions} \label{sec:densitydecompositions}

In this section and the following, we give an overview of the proof of the radial multiplier bounds obtained by Heo, Nazarov, and Seeger in \cite{HeoandNazarovandSeeger}.

\begin{theorem} \label{HeoNazarovSeegerTheorem}
    Suppose $1/2 > 1/p - 1/2 > 1/(d-1)$, and $m(\xi) = a(|\xi|)$ is radial. Then
    %
    \[ \| m \|_{M^p(\RR^d)} \lesssim \| a \|_{R^{s,p}[0,\infty)}, \]
    %
    where $s = (d-1)(1/p - 1/2)$.
\end{theorem}

The main tool we will take away for application to the proof of Theorem \ref{maintheorem} is the method of \emph{density decompositions}, and a method of \emph{atomic decompositions} used to combine frequency scales in the problem. We give an overview of the density decomposition argument in this section to establish a single scale version of Theorem \ref{HeoNazarovSeegerTheorem}, as described in the following lemma, and discuss the particular atomic decomposition method in the following section.

\begin{prop} \label{HeoNazarovSeegerSingleScaleInequality}
  Suppose $1/2 > 1/p - 1/2 > 1/(d-1)$, and $k$ is a radial function with Fourier transform supported on $1/2 \leq |\xi| \leq 2$. Then
%
\[ \| k * f \|_{L^p(\RR^d)} \lesssim \| k \|_{L^p(\RR^d)} \| f \|_{L^p(\RR^d)} \]
%
\end{prop}

We reduce Proposition \ref{HeoNazarovSeegerSingleScaleInequality} to an inequality for sums of functions oscillating on spheres. Let $\sigma_r$ be the surface measure for the sphere of radius $r$ centered at the origin in $\RR^d$. Also fix a nonzero, radial, compactly supported function $\psi \in \mathcal{S}(\RR^d)$ whose Fourier transform is non-negative, and vanishes to high order at the origin. Given $x \in \RR^d$ and $r \geq 1$, define $\chi_{x, r} = \text{Trans}_x (\sigma_r * \psi)$, where $\text{Trans}_x g(y) = g(y - x)$. Then $\chi_{x,r}$ is a smooth function adapted to a thickness $O(1)$ annulus of radius $r$ centered at $x$, which is \emph{slightly oscillating}. We will verify the following proposition.

\begin{prop} \label{lemma1}
    For any $a : \RR^d \times [1,\infty) \to \CC$, if $1/2 > 1/p - 1/2 > 1/(d-1)$,
    %
    \[ \left\| \int_{\RR^d} \int_1^\infty a(x,r) \chi_{x, r}\; dx\; dr \right\|_{L^p(\RR^d)} \lesssim \left( \int_{\RR^d} \int_1^\infty |a(x,r)|^p r^{d-1} dr dx \right)^{1/p}. \]
    %
    The implicit constant here depends on $p$, $d$, and $\psi$.
\end{prop}

\begin{proof}[Proof of Proposition \ref{HeoNazarovSeegerSingleScaleInequality} from Proposition \ref{lemma1}] Suppose $k(\cdot) = b(|\cdot|)$ for $b: [0,\infty) \to \CC$. If we set $a(x,r) = f(x) b(r)$ for any function $f: \RR^d \to \CC$, then
%
\begin{equation} \label{awoidjawiodjwaioj13131314135}
  (k * \psi * f)(y) = \int_{\RR^d} \int_1^\infty a(x,r) \chi_{x, r}(y)\; dx\; dr,
\end{equation}
%
Proposition \ref{lemma1}, applied to \eqref{awoidjawiodjwaioj13131314135} tells us that
%
\begin{equation}
  \| k * \psi * f \|_{L^p(\RR^d)} \lesssim \| k \|_{L^p(\RR^d)} \| f \|_{L^p(\RR^d)}.
\end{equation}
%
If we choose $\psi$ so that $\widehat{\psi}$ is non-vanishing on the support of $\widehat{k}$, then the function $1/\widehat{\psi}(\cdot)$ is smooth on the support of $\widehat{k}$; if $T$ is a Fourier multiplier operator with a smooth, compactly supported symbol agreeing with $1/\widehat{\psi}(\cdot)$ on the support of $\widehat{k}$, then the convolution kernel of $T$ is Schwartz, and so $T$ is bounded on $L^p(\RR^d)$ by Schur's test. Since $T(k * \psi * f) = k * f$, we conclude
%
\begin{equation}
  \| k * f \|_{L^p(\RR^d)} = \| T(k * \psi * f) \|_{L^p(\RR^d)} \lesssim \| k * \psi * f \|_{L^p(\RR^d)} \lesssim \| k \|_{L^p(\RR^d)} \| f \|_{L^p(\RR^d)},
\end{equation}
%
which completes the argument.
\end{proof}

Next, we consider a discretization of Proposition \ref{lemma1}.

\begin{prop} \label{lemma2}
    Fix a 1-separated set $\mathcal{E} \subset \RR^d \times [1,\infty)$. Then for any $a: \mathcal{E} \to \CC$, and for $1/2 > 1/p - 1/2 > 1/(d-1)$,
    %
    \[ \left\| \sum\nolimits_{(x,r) \in \mathcal{E}} a(x,r) \chi_{x, r} \right\|_{L^p(\RR^d)} \lesssim \left( \sum\nolimits_{(x,r) \in \mathcal{E}} |a(x,r)|^p r^{d-1} \right)^{1/p}, \]
    %
    where the implicit constant is independent of $\mathcal{E}$.
\end{prop}

\begin{proof}[Proof of Proposition \ref{lemma1} from Proposition \ref{lemma2}]
    For any $a: \RR^d \times [1,\infty) \to \CC$, if we consider the vector-valued function $A(x,r) = a(x,r) \chi_{x,r}$, then
    %
    \begin{equation}
      \int_{\RR^d} \int_1^\infty A(x,r)\; dr\; dx = \int_{[0,1)^d} \int_0^1 \sum\nolimits_{n \in \ZZ^d} \sum\nolimits_{m > 0} A(x + n,r + m)\; dr\; dx
    \end{equation}
    %
    The triangle inequality and the increasing property of norms on $[0,1)^d \times [0,1]$ imply that
    %
    \begin{equation}
    \begin{split}
    &\left\| \int_{\RR^d} \int_1^\infty A(x,r)\; dr\; dx \right\|_{L^p(\RR^d)}\\
    &\quad \leq \int_{[0,1)^d} \int_0^1 \left\| \sum\nolimits_{n \in \ZZ^d} \sum\nolimits_{m > 0} A(x + n,r + m) \right\|_{L^p(\RR^d)}\; dr\; dx\\
    &\quad \lesssim \int_{[0,1)^d} \int_0^1 \left( \sum\nolimits_{n \in \ZZ^d} \sum\nolimits_{m > 0} |a(x + n, r + m)|^p r^{d-1} \right)^{1/p}\; dr\; dx\\
    &\quad \leq \left( \int_{[0,1)^d} \int_0^1 \sum\nolimits_{n \in \ZZ^d} \sum\nolimits_{m > 0} |a(x + n, r + m)|^p r^{d-1}\; dr\; dx \right)^{1/p}\\
    &\quad = \left( \int_{\RR^d} \int_1^\infty |a(x,r)|^p r^{d-1} dr dx \right)^{1/p},
    \end{split}
    \end{equation}
    %
    which completes the proof.
\end{proof}

Proposition \ref{lemma2} can be further reduced by considering it as a bound on the operator
%
\begin{equation}
  a \mapsto \sum\nolimits_{(x,r) \in \mathcal{E}} a(x,r) \chi_{x,r}.
\end{equation}
%
In particular, since Proposition \ref{lemma2} is an estimate for an open interval of $L^p$ spaces, by using real interpolation methods it suffices to prove a restricted strong type bound $L^p$ bound for $1 < p < 2(d-1)/(d+1)$. Given any discretized set $\mathcal{E}$, let $\mathcal{E}_k$ be the set of $(x,r) \in \mathcal{E}$ with $2^k \leq r < 2^{k+1}$. Then Proposition \ref{lemma2} is implied by the following proposition.

\begin{prop} \label{lemma3}
    For $1/2 > 1/p - 1/2 > 1/(d-1)$, and $k \geq 1$,
    %
    \[ \left\| \sum\nolimits_{(x,r) \in \mathcal{E}} \chi_{x,r} \right\|_{L^p(\RR^d)} \lesssim \left( \sum\nolimits_{k \geq 1} 2^{k(d-1)} \#(\mathcal{E}_k) \right)^{1/p}. \]
\end{prop}

\begin{remark}
    Note that if $r \sim 2^k$, then $\| \chi_{x,r} \|_{L^p(\RR^d)} \sim 2^{k(d-1)/p}$, and so Proposition \ref{lemma3} says
    %
    \begin{equation}
      \left\| \sum\nolimits_{(x,r) \in \mathcal{E}} \chi_{x,r} \right\|_{L^p(\RR^d)} \lesssim_p \left( \sum\nolimits_{(x,r) \in \mathcal{E}} \| \chi_{x,r} \|_{L^p(\RR^d)}^p \right)^{1/p}.
    \end{equation}
    %
    Thus we are proving a $p$th root cancellation bound for the functions $\{ \chi_{x,r} \}$.
\end{remark}

\begin{comment}
\begin{proof}[Proof of Lemma \ref{lemma2} from Lemma \ref{lemma3}]
    Let
    %
    \[ F = \sum\nolimits_{(x,r) \in \mathcal{E}} \chi_{x,r} \]
    %
    and then for $k \geq 1$, let
    %
    \[ F_k = \sum\nolimits_{(x,r) \in \mathcal{E}_k} \chi_{x,r}. \]
    %
    Then $F = \sum\nolimits_k F_k$, and. Applying a dyadic interpolation result (Lemma 2.2 of that paper), the bound
    %
    \[ \| F_k \|_{L^r(\RR^d)} \lesssim 2^k (2^{k(d-r-1)} \#(\mathcal{E}_k)^{1/r}) \]
    %
    which holds for $r$ to the left and right of $p$, can be interpolated to yield that
    %
    \[ \| F \|_{L^p(\RR^d)} \lesssim \left( \sum\nolimits_k 2^{kp} ( 2^{k(d-r-1)} ) \right)^{1/p} \]


    Applying a dyadic interpolation result (Lemma 2.2 of the paper), Lemma \ref{lemma3} implies that
    %
    \[ \left\| \sum\nolimits_{(x,r) \in \mathcal{E}} \chi_{x,r} \right\| \]

    %
    \[ \left\| \sum\nolimits_{(x,r) \in \mathcal{E}} \chi_{x,r} \right\|_{L^p(\RR^d)} \lesssim \left( \sum 2^{kp} 2^{k(d-p-1)} \#(\mathcal{E}_k) \right)^{1/p} = \left( \sum 2^{k(d-1)} \#(\mathcal{E}_k) \right)^{1/p} \]
    %
    This is a restricted strong type bound for Lemma \ref{lemma2}, which we can then interpolate.
\end{proof}
\end{comment}

To control these sums, we apply a `density decomposition', which splits a discrete set into different parts which are either spread out or clustered on small sets. The density decomposition will enable us to obtain $L^p$ bounds from $L^2$ bounds. We say a 1-separated set $\mathcal{E}$ in a metric space $X$ is of \emph{density type} $(u,r)$ if $\#(B \cap \mathcal{E}) \leq u \cdot \diam(B)$ for each ball $B$ in $X$ with diameter at most $r$.

%A covering argument then shows that for any ball $B$,
%
%\[ \#(B \cap \mathcal{E}) \lesssim_d u \cdot \left( 1 + \frac{\diam(B)}{R} \right)^d \cdot \diam(B). \]
%
%(NOTE: WE MIGHT BE ABLE TO DO BETTER USING THE FACT THAT $\mathcal{E} \subset \RR^d \times [R,2R)$, USING THE VALUE $R$).

\begin{lemma} \label{DecompositionTheorem}
    For any family of 1-separated sets $\mathcal{E}_k \subset \RR^d \times [2^k,2^{k+1})$, there exists a decomposition $\mathcal{E}_k = \bigcup_{m = 1}^\infty \mathcal{E}_k(2^m)$ with the following properties:
    %
    \begin{itemize}[leftmargin=30pt]
        \item For each $m$, $\mathcal{E}_k(2^m)$ has density type $(2^m,2^k)$.

        \item If $B$ is a ball in $\RR^{d+1}$ of radius $r \leq 2^k$ containing at least $2^m\;\! r$ points of $\mathcal{E}_k$, then
        %
        \[ B \cap \mathcal{E}_k \subset \bigcup\nolimits_{m' \geq m} \mathcal{E}_k(2^{m'}). \]

        \item For each $m$, there are disjoint balls $\{ B_i \}$ in $\RR^{d+1}$ with radii $\{ r_i \}$, such that
        %
        \[ \sum\nolimits_i r_i \leq 2^{-m} \# \mathcal{E}_k, \]
        %
        such that $r_i \leq 2^k$ for all $i$, and such that $\bigcup B_i^*$ covers $\bigcup_{m' \geq m} \mathcal{E}_k(2^{m'})$, where $B_i^*$ denotes the ball with the same center as $B_i$ but 5 times the radius.
    \end{itemize}
\end{lemma}
\begin{proof}
    Define a function $M: \mathcal{E}_k \to [0,\infty)$ by setting
    %
    \begin{equation}
      M(x,r) = \sup \left\{ \frac{\#(\mathcal{E}_k \cap B)}{\text{rad}(B)} : (x,r) \in B\ \text{and}\ \text{rad}(B) \leq 2^k \right\}.
    \end{equation}
    %
    We can establish a kind of weak $L^1$ estimate for $M$ using a Vitali type argument. Let
    %
    \begin{equation}
      \widehat{\mathcal{E}}_k(2^m) = \{ (x,r) \in \mathcal{E}_k : M(x,r) \geq 2^m \}.
    \end{equation}
    %
    We can therefore cover $\widehat{\mathcal{E}}_k(2^m)$ by a family of balls $\{ B \}$ such that $\#(\mathcal{E}_k \cap B) \geq 2^m \text{rad}(B)$. The Vitali covering lemma allows us to find a disjoint subcollection of balls $B_1,\dots,B_N$ such that $B_1^* ,\dots, B_N^*$ covers $\widehat{\mathcal{E}}_k(2^m)$. We find that
    %
    \begin{equation}
      \#(\mathcal{E}_k) \geq \sum\nolimits_i \#(B_i \cap \mathcal{E}_k) \geq 2^m \sum\nolimits_i \text{rad}(B_i),
    \end{equation}
    %
    Setting $\mathcal{E}_k = \widehat{\mathcal{E}}_k(2^m) - \bigcup_{k' > k} \widehat{\mathcal{E}}_{k'}(2^m)$ thus gives the required result.
\end{proof}

\begin{remark}
  We will later apply the Lemma with $\RR^d \times [0,\infty)$ replaced by $X \times [0,\infty)$, where $X$ is a compact, $d$-dimensional manifold equipped with a Finsler metric. A version of the Vitali covering lemma also holds for this metric space, so that the same proof allows one to perform a density decomposition in this setting.
\end{remark}

To prove Proposition \ref{lemma3}, we perform a decomposition of $\mathcal{E}_k$ for each $k$, into the sets $\mathcal{E}_k(2^m)$, and then define $\mathcal{E}^m = \bigcup_{k \geq 1} \mathcal{E}_k^m$. For appropriate exponents, we prove $L^p$ bounds on the functions $F^m = \sum\nolimits_{(x,r) \in \mathcal{E}^m} \chi_{x,r}$ which are exponentially decaying in $m$, i.e. that
%
\begin{equation}
  \| F^m \|_{L^p(\RR^d)} \lessapprox 2^{m \big( \frac{1}{d-1} - (1/p - 1/2) \big)} \left( \sum\nolimits_k 2^{k(d-1)} \#(\mathcal{E}_k) \right)^{1/p},
\end{equation}
where the implicit constant in the inequality can depend polynomially on $m$. Thus, in the range $1/p - 1/2 > 1/(d-1)$, i.e. for $1 < p < 2(d-1)/(d+1)$ there is geometric decay in $m$, decaying much quicker than the polynomial implicit constant, and so we may sum in $m$ using the triangle inequality to conclude that
%
\begin{equation}
  \| F \|_{L^p(\RR^d)} \lesssim \left( \sum\nolimits_k 2^{k(d-1)} \#(\mathcal{E}_k) \right)^{1/p},
\end{equation}
%
proving Proposition \ref{lemma3}. To get the bound on $F^m$, we interpolate between an $L^2$ bound for $F^m$, and an $L^0$ bound (i.e. a bound on the measure of the support of $F^m$). First, we calculate the support of $F^m$.

\begin{lemma} \label{lemma5}
    For each $k$,
    %
    \[ |\text{supp}(F^m_k)| \lesssim 2^{-m} 2^{k(d-1)} \# \mathcal{E}_k. \]
    %
    Thus we have
    %
    \[ |\text{supp}(F^m)| \leq \sum\nolimits_k |\text{supp}(F^m_k)| \lesssim \sum\nolimits_k 2^{-m} 2^{k(d-1)} \# \mathcal{E}_k. \]
\end{lemma}
\begin{proof}
    We recall that for each $k$ and $m$, we can find disjoint balls $B_1,\dots,B_N$ with radii $r_1,\dots,r_N \leq 2^k$ such that
    %
    \begin{equation}
      \sum\nolimits_{i = 1}^N r_i \leq 2^{-m} \# \mathcal{E}_k,
    \end{equation}
    %
    where $\mathcal{E}_k(2^m)$ is covered by the expanded balls $B_1^* \cup \dots \cup B_N^*$. If we write
    %
    \begin{equation}
      F^m_{k,i} = \sum\nolimits_{(x,r) \in \mathcal{E}_k(2^m) \cap B_i^*} \chi_{x,r},
    \end{equation}
    %
    then $\text{supp}(F^m_k) \subset \bigcup_i \text{supp}(F^m_{k,i})$. For each $(x,r) \in B_i^* \cap \mathcal{E}_k(2^m)$, the support of $\chi_{x,r}$, an annulus of thickness $O(1)$ and radius $r$, is contained in an annulus of thickness $O(r_i)$ and radius $O(2^k)$ with the same center as $B_i$. Thus we conclude that
    %
    \begin{equation}
      |\text{supp}(F^m_{k,i})| \lesssim r_i 2^{k(d-1)},
    \end{equation}
    %
    and it follows that
    %
    \begin{equation}
      |\text{supp}(F^m_k)| \leq \sum\nolimits_i r_i 2^{k(d-1)} \leq 2^{-m} 2^{k(d-1)} \# \mathcal{E}_k,
    \end{equation}
    %
    which completes the proof.
\end{proof}

Interpolating, it suffices to prove the following $L^2$ estimate on the function $F^m$.

\begin{lemma} \label{lemma6}
    Suppose $\mathcal{E} = \bigcup_k \mathcal{E}_k$ is a 1-separated set, where $\mathcal{E}_k \subset \RR^d \times [2^k,2^{k+1})$ is a set of density type $(2^m, 2^k)$. Then
    %
    \[ \left\| \sum\nolimits_{(x,r) \in \mathcal{E}} \chi_{x,r} \right\|_{L^2(\RR^d)} \lesssim \sqrt{m} \cdot 2^{ \frac{m}{d-1} } \left( \sum\nolimits_k 2^{k(d-1)} \#(\mathcal{E}_k) \right)^{1/2}. \]
\end{lemma}

% The L2 norms of the chi_{x,r} are equal to 2^{k(d-1)/2}, so the
% triangle ienquality implies that the LHS is bounded by sum_k 2^{k(d-1)/2} \#(E_k)

\begin{comment}
\begin{proof}[Proof of Lemma \ref{lemma3} from Lemma \ref{lemma6}]
    Write $F = \sum\nolimits_{(x,r) \in \mathcal{E}_k} \chi_{x,r}$, and then perform a decomposition $\mathcal{E}_k = \bigcup_{m \geq 0} \mathcal{E}_k(2^m)$, and thus define $F = \sum\nolimits_{m \geq 0} F_m$, where
    %
    \[ F_m = \sum\nolimits_{(x,r) \in \mathcal{E}(2^m)} \chi_{x,r}. \]
    %
    We have
    %
    \[ \| F_m \|_{L^2(\RR^d)} \lesssim 2^{\frac{m}{d-1} + \frac{k(d-1)}{2}} \log(2 + 2^m)^{1/2} \cdot \#(\mathcal{E}_k)^{1/2}. \]
    %
    If we interpolate this bound with the support bound for $F_m$, a kind of $L^0$ norm estimate, we conclude that for $0 < p \leq 2$,
    % ( int |F_m|^p )^{1/p} <= |S|^{1/pq^*} int |F_m|^{pq} )^{1/pq}
    % pq = 2
    % Then q = 2/p so 1/q^* = 1 - 1/q = 1 - p/2 = (2 - p)/2
    % so q^* = 2/(2-p)
    % 1/pq^* = (2-p)/2p = (1/p - 1/2)
    \begin{align*}
        \| F_m \|_{L^p(\RR^d)} &\leq |\text{Supp}(F_m)|^{1/p - 1/2} \| F_m \|_{L^2(\RR^d)}\\
        &\lesssim ( 2^{k(d-1) - m})^{1/p - 1/2} 2^{\frac{m}{d-1} + \frac{k(d-1)}{2}} \log(2 + 2^m)^{1/2} \cdot \#(\mathcal{E}_k)^{1/p} \\
        &\lesssim 2^{m(1/p_d - 1/p)} \log(2 + 2^m)^{1/2} 2^{\frac{k(d-1)}{p}} \#(\mathcal{E}_k)^{1/p}.
    \end{align*}
    % int |F_m|^p <= |S|^{1/p-1/2} ( int |F_m|^2 )^{1/2}
    %
    where $p_d = 2(d-1)/(d+1)$. This bound is summable in $m$ for $p < p_d$, which enables us to conclude that
    %
    \[ \| F \|_{L^p(\RR^d)} \lesssim 2^{\frac{k(d-1)}{p}} \#(\mathcal{E}_k)^{1/p}. \]
    %
    Thus for $1 \leq p < p_d$, we obtain the bound stated in Lemma \ref{lemma3}.
\end{proof}
\end{comment}

The $L^2$ bound in Lemma \ref{lemma6} gets worse and worse as $m$ grows, whereas the $L^0$ bound in Lemma \ref{lemma5} gets better and better, since annuli are concentrating in a small set, which is bad from the perspective of constructive interference, but absolutely fine from the perspective of a support bound. To prove the $L^2$ bound, we require an analysis of the interference patterns of pairs of the functions $\chi_{x,r}$, as provided by the following lemma.

%If $\psi$ is compactly supported, and $r$ is sufficiently large depending on the size of this support, then $\chi_{x,r}$ is supported on an annulus with centre $x$, radius $r$, and thickness $O(1)$. Thus $\| \chi_{x,r} \|_{L^p(\RR^d)} \sim r^{(d-1)/p}$, which implies that
%
%\[ \left\| \sum\nolimits_{(x,r) \in \mathcal{E}_k} \chi_{x,r} \right\|_{L^p(\RR^d)} \gtrsim 2^{k(d-1)/p} \#(\mathcal{E}_k)^{1/p}. \]
%
%Thus this bound can only be true if $p \geq 1$, and becomes tight when $p = 1$, where we actually have
%
%\[ \left\| \sum\nolimits_{(x,r) \in \mathcal{E}_k} \chi_{x,r} \right\|_{L^1(\RR^d)} \sim 2^{k(d-1)} \#(\mathcal{E}_k) \]
%

\begin{lemma} \label{lemma4}
    For any $N > 0$, $x_1,x_2 \in \RR^d$ and $r_1,r_2 \geq 1$,
    %with $|x_1 - x_2| \geq 1$ or $x_1 = x_2$, and $r_1,r_2 > 1$,
    %
    \begin{align*}
        |\langle \chi_{x_1,r_1}, \chi_{x_2,r_2} \rangle| &\lesssim_N \left( \frac{r_1r_2}{\langle (x_1,r_1) - (x_2,r_2) \rangle} \right)^{\frac{d-1}{2}} \sum\nolimits_{\pm,\pm} \langle |x_1 - x_2| \pm r_1 \pm r_2 \rangle^{-N}.
    \end{align*}
\end{lemma}

\begin{remark}
    Suppose $r_1 \leq r_2$. Then Lemma \ref{lemma4} implies that $\chi_{x_1,r_1}$ and $\chi_{x_2,r_2}$ are roughly uncorrelated, except when they are supported on annuli that roughly have the same radii and centers, and in addition, one of the following two properties hold:
    %
    \begin{itemize}
        \item $r_1 + r_2 \approx |x_1 - x_2|$, which holds when the two annuli are `approximately' externally tangent to one another.

        \item $r_2 - r_1 \approx |x_1 - x_2|$, which holds when the two annuli are `approximately' internally tangent to one another.
    \end{itemize}
    %
    Heo, Nazarov, and Seeger do not exploit the tangency information, though utilizing the tangencies seems important to improve the results they obtain. Cladek exploits this tangency information further, to obtain improved results.
\end{remark}

%    Note that if we take too comparable radii $r_1,r_2 \sim 2^k$, then the functions $\{ 1_{x,r} \}$, which are the non-oscillating analogues of the functions $\{ \chi_{x,r} \}$ we are studying in this problem, if they correspond to tangent annuli, satisfy bounds  of the form
    %
%    \[ \langle 1_{x_1,r_1}, 1_{x_2,r_2} \rangle \lesssim 2^{- \left( \frac{d-1}{2} \right) k} \]
    %
%    whereas for internal tangencies we have
    %
%    \[ \langle \chi_{x_1,r_1}, \chi_{x_2,r_2} \rangle \lesssim 2^{(d-1)k} |x_1 - x_2|^{- \frac{d-1}{2}} \]
    %
%    In the Clamshell example, one has
    %
%    \[ \left\| \sum\nolimits_{k = 1}^N k^{d-1} 1_{k,k} \right\|_{L^2(\RR^d)} \]

    % delta = 2^{-k}
    % 3[d-1]/2
%    \[ 2^{(d-1)k} 2^{dk} 2^{-\frac{d+1}{2}} \]
    %
%    \[ \langle r_1^{d-1} 1_{x_1,r_1}, r_2^{d-1} 1_{x_2,r_2} \rangle \lesssim r_1^{d-1} r_2^{d-1}. \]
%    \end{comment}

\begin{proof}
%    We may assume $|x_1 - x_2| \geq 1$, for otherwise the inequality holds trivially since unless $|r_1 - r_2| \lesssim 1$, $f_{x_1r_1}$ and $f_{x_2r_2}$ have disjoint support, and if $|r_1 - r_2| \lesssim 1$ then Cauchy Schwartz implies that
    %
%    \begin{align*}
%        |\langle f_{x_1r_1}, f_{x_2r_2} \rangle| &\lesssim (r_1 r_2)^{(d-1)/2}\\
%        &\lesssim_{N,d} (r_1r_2)^{(d-1)/2} (1 + |r_1 - r_2| + |x_1 - x_2|)^{-(d-1)/2} \sum\nolimits_{\pm,\pm} (1 + ||x_1 - x_2| \pm r_1 \pm r_2|)^{-N}
%    \end{align*}
%
    We write
    %
    \begin{equation}
    \begin{split}
        \langle \chi_{x_1 r_1}, \chi_{x_2 r_2} \rangle &= \left\langle \widehat{\chi}_{x_1 r_1}, \widehat{\chi}_{x_2 r_2} \right\rangle\\
        &= \int_{\RR^d} \widehat{\sigma_{r_1} * \psi}(\xi) \cdot \overline{\widehat{\sigma_{r_2} * \psi}(\xi)} e^{2 \pi i (x_2 - x_1) \cdot \xi}\; d\xi\\
        &= (r_1 r_2)^{d-1} \int_{\RR^d} \widehat{\sigma}(r_1 \xi) \overline{\widehat{\sigma}(r_2 \xi)} |\widehat{\psi}(\xi)|^2 e^{2 \pi i (x_2 - x_1) \cdot \xi}\; d\xi.
    \end{split}
    \end{equation}
    %
    Define functions $A$ and $B$ such that $B(|\xi|) = \widehat{\sigma}(\xi)$, and $A(|\xi|) = |\widehat{\psi}(\xi)|^2$. Then
    %
    \begin{equation}
      \langle \chi_{x_1, r_1}, \chi_{x_2, r_2} \rangle = C_d (r_1r_2)^{d-1} \int_0^\infty s^{d-1} A(s) B(r_1 s) B(r_2 s) B(|x_2 - x_1| s)\; ds.
    \end{equation}
    %
    Using well known asymptotics for $\widehat{\sigma}$, we have, for any $N > 0$,
    %
    \begin{equation}
      B(s) = s^{-(d-1)/2} \sum\nolimits_{n = 0}^{N-1} (c_{n,+} e^{2 \pi i s} + c_{n,-} e^{-2 \pi i s}) s^{-n} + O_N(s^{-N}).
    \end{equation}
    %
    Write
    %
    \begin{equation}
      a_{n,\tau} = \int_0^\infty A(s) s^{- \frac{d-1}{2} - n_1 - n_2 - n_3} e^{2 \pi i (\tau_1 r_1 + \tau_2 r_2 + \tau_3 |x_2 - x_1|) s}\; ds.
    \end{equation}
    %
    Assuming $A(s)$ vanishes to suitably high order at the origin, depending on $N$, we find that
    %
    \begin{equation}
    \begin{split}
        \langle \chi_{x_1 r_1}, \chi_{x_2 r_2} \rangle &= C_d \left( \frac{r_1r_2}{|x_1 - x_2|} \right)^{(d-1)/2} \sum\nolimits_{n,\tau} a_{n,\tau} c_{n,\tau} r_1^{-n_1} r_2^{-n_2} |x_2 - x_1|^{-n_3}\\
        &\lesssim \left( \frac{r_1r_2}{|x_1 - x_2|} \right)^{\frac{d-1}{2}} \left(1 + \frac{1}{|x_1 - x_2|^N} \right) \sum\nolimits_{\tau} \langle \tau_1 r_1 + \tau_2 r_2 + \tau_3 |x_2 - x_1| \rangle^{-N}.
    \end{split}
    \end{equation}
    %
    This gives the result provided that $1 + |x_1 - x_2| \geq |r_1 - r_2| / 10$ and $|x_1 - x_2| \geq 1$. If $1 + |x_1 - x_2| \leq |r_1 - r_2| / 10$, then the supports of $\chi_{x_1,r_1}$ and $\chi_{x_2,r_2}$ are disjoint, so the inequality is trivial. On the other hand, if $|x_1 - x_2| \leq 1$, then the bound is trivial by the last sentence unless $|r_1 - r_2| \leq 10$, and in this case the inequality reduces to the simple inequality
    %
    \begin{equation}
      \langle \chi_{x_1,r_1}, \chi_{x_2,r_2} \rangle \lesssim_N (r_1 r_2)^{(d-1)/2}. 
    \end{equation}
    %
    But this follows immediately from the Cauchy-Schwartz inequality.
\end{proof}

The exponent $\tfrac{d-1}{2}$ in Lemma \ref{lemma4} is too weak to apply almost orthogonality directly to obtain $L^2$ bounds on $\sum\nolimits_{(x,r) \in \mathcal{E}_k} \chi_{xr}$ on it's own, but together with the density decomposition assumption we will be able to obtain Lemma \ref{lemma6}.

\begin{proof}[Proof of Lemma \ref{lemma6}]
    Without loss of generality, we may assume that the set of $k$ such that $\mathcal{E}_k \neq \emptyset$ is $10$-separated. Write
    %
    \begin{equation}
      F = \sum\nolimits_{(x,r) \in \mathcal{E}} \chi_{x,r}
    \end{equation}
    %
    and $F_k = \sum\nolimits_{(x,r) \in \mathcal{E}_k} \chi_{x,r}$. First, we deal with $F_{\lesssim m} = \sum\nolimits_{k \leq 10 m} F_k$ trivially, i.e. writing
    %
    \begin{equation}
        \| F \|_{L^2(\RR^d)} \lesssim m^{1/2} \left( \sum\nolimits_{k \leq 10m} \| F_k \|_{L^2(\RR^d)}^2 + \| \sum\nolimits_{k > 10m} F_k \|_{L^2(\RR^d)} \right)^{1/2}.
    \end{equation}
    %
    We then decompose
    %
    \begin{equation}
      \left\| \sum\nolimits_{k > 10 m} F_k\;\! \right\|_{L^2(\RR^d)}^2 \leq \sum\nolimits_{k > 10 m} \| F_k \|_{L^2(\RR^d)}^2 + 2 \sum\nolimits_{k' > k > 10m} |\langle F_k, F_{k'} \rangle|.
    \end{equation}
    %
    Let us analyze $\langle F_k, F_{k'} \rangle$. The term will become a sum of the form $\langle \chi_{x,r}, \chi_{y,s} \rangle$, where $r \sim 2^k$ and $s \sim 2^{k'}$. Because of our assumption of being 10-separated, we have $r \leq s / 2^{10}$. If $\langle \chi_{x,r}, \chi_{y,s} \rangle \neq 0$, then since the support of $\chi_{y,s}$ is an annulus of radius $s$ centered at $y$, with thickness $O(1)$, and $\chi_{x,r}$ has support on an annulus of radius $r$ centered at $x$, with thickness $O(1)$, the fact that $r$ is comparatively smaller than $s$ implies that $(x,r)$ must be contained in the annulus of radius $s$ centered at $y$, with thickness $O(2^k)$. Such an annulus is covered by $O( 2^{(k'-k)(d-1)} )$ balls of radius $2^k$. Each ball can only contain $2^{k + m}$ points $(x,r)$, and so there can be at most $O( 2^{k'(d-1) - k(d-1) + k + m} ) = O( 2^{k'(d-1) - k(d-2) + m} )$ pairs $(x,r) \in \mathcal{E}_k$ for which $\langle \chi_{x,r}, \chi_{y,s} \rangle \neq 0$. For such pairs we have
    %
    \begin{equation}
      |\langle \chi_{x,r}, \chi_{y,s} \rangle| \lesssim \left( \frac{2^k 2^{k'}}{2^{k'}} \right)^{\frac{d-1}{2}} = 2^{\frac{k(d-1)}{2}}.
    \end{equation}
    %
    Thus we conclude that
    %
    \begin{equation}
      |\langle F_k, \chi_{y,s} \rangle| \lesssim 2^{-k ( \frac{d-3}{2} ) + k'(d-1) + m }.
    \end{equation}
    %
    Summing over $10m < k < k'$, we conclude that since $d \geq 4$,
    %
    \begin{equation}
      \sum\nolimits_{10m < k < k'} |\langle F_k, \chi_{y,s} \rangle| \lesssim 2^{k'(d-1) + m} \sum\nolimits_{10m < k < k'} 2^{-k \frac{d-3}{2}} \lesssim 2^{k'(d-1) + m} 2^{-5m} \lesssim 2^{k'(d-1)}.
    \end{equation}
    %
    But this means that
    %
    \begin{equation}
      \sum\nolimits_{10m < k < k'} |\langle F_k, F_{k'} \rangle| \lesssim 2^{k'(d-1)} \cdot \# ( \mathcal{E}_{k'} ).
    \end{equation}
    %
    This means that
    %
    \begin{equation} \label{awiodjwaoidfjawoid12343124134123453}
      \left\| \sum\nolimits_{k > 10m} F_k \right\|_{L^2(\RR^d)}^2 \lesssim \sum\nolimits_{k > 10m} \| F_k \|_{L^2(\RR^d)}^2 + \sum\nolimits_{k'} 2^{k'(d-1)} \# (\mathcal{E}_{k'}),
    \end{equation}
    %
    and it now suffices to deal with estimates the $\| F_k \|_{L^2(\RR^d)}$, i.e. the interactions of functions supported on radii of comparable magnitude. To deal with these, we further decompose the radii, writing $[2^k,2^{k+1})$ as the disjoint union of intervals $I_{k,\mu} = [2^k + (\mu - 1) 2^{um}, 2^k + \mu 2^{um}]$, for some $u$ to be chosen later. These interval induces a decomposition $\mathcal{E}_k = \bigcup_\mu \mathcal{E}_{k,\mu}$. Again, incurring a constant loss at most, we may assume that the $\mu$ such that $\mathcal{E}_{k,\mu} \neq \emptyset$ are $10$ separated. We write $F_k = \sum F_{k,\mu}$, and we have
    %
    \begin{equation}
      \| F_k \|_{L^2(\RR^d)}^2 = \sum\nolimits_\mu \| F_{k,\mu} \|_{L^2(\RR^d)}^2 + \sum\nolimits_{\mu < \mu'} |\langle F_{k,\mu}, F_{k,\mu'} \rangle|.
    \end{equation}
    %
    We now consider $\chi_{x,r}$ and $\chi_{y,s}$ with $r \in I_{k,\mu}$ and $s \in I_{k,\mu'}$. Then we must have $|x - y| \lesssim 2^k$ and $2^{am} \leq |r - s| \lesssim 2^k$, and so we have
    %
    \begin{equation}
    \begin{split}
        \left| \sum\nolimits_{\mu < \mu'} \langle F_{k,\mu}, \chi_{y,s} \rangle \right| &\lesssim 2^{k(d-1)} \sum\nolimits_{\substack{(x,r) \in \mathcal{E}_k\\ 2^{um} \leq |(x,r) - (y,s)| \lesssim 2^k}} |(x,r) - (y,s)|^{- \frac{d-1}{2}}\\
        &\lesssim 2^{k(d-1)} \sum\nolimits_{um \leq l \leq k} 2^{-l(d-1)/2} \# \{ (x,r) \in \mathcal{E}_k: |(x,r) - (y,s)| \sim 2^l \}.
    \end{split}
    \end{equation}
    %
    Using the density assumption,
    %
    \begin{equation}
      \# \{ (x,r) \in \mathcal{E}_k: |(x,r) - (y,s)| \sim 2^l \} \lesssim 2^{l + m}
    \end{equation}
    %
    and so we obtain that, again using the assumption that $d \geq 4$,
    %
    \begin{equation}
      |\sum\nolimits_{\mu < \mu'} \langle F_{k,\mu}, \chi_{y,s} \rangle| \lesssim 2^{k(d-1)} 2^{m(1-u(d-3)/2)}.
    \end{equation}
    %
    Now summing over all $(y,s)$, we obtain that
    %
    \begin{equation}
      \left| \sum\nolimits_{\mu < \mu'} \langle F_{k,\mu}, F_{k,\mu'} \rangle \right| \lesssim 2^{k(d-1)} 2^{m(1 - u(d-3)/2)} \#(\mathcal{E}_{k,\mu'}).
    \end{equation}
    %
    and now summing over $\mu'$ gives that
    %
    \begin{equation}
      \| F_k \|_{L^2(\RR^d)}^2 \lesssim \sum\nolimits_\mu \| F_{k,\mu} \|_{L^2(\RR^d)}^2 + 2^{k(d-1)} 2^{m(1 - u(d-3)/2)} \# \mathcal{E}_k,
    \end{equation}
    %
    Now we are left to analyze $\| F_{k,\mu} \|_{L^2(\RR^d)}$, i.e. analyzing interactions between annuli which have radii differing from one another by at most $O(2^{am})$. Since the family of all possible radii are discrete, the set $\mathcal{R}_{k,\mu}$ of all possible radii has cardinality $O(2^{um})$. We do not really have any orthogonality to play with here, so we just apply Cauchy-Schwartz, writing $F_{k,\mu} = \sum\nolimits_{r \in \mathcal{R}_{k,\mu}} F_{k,\mu,r}$, to write
    %
    \begin{equation}
      \| F_{k,\mu} \|_{L^2(\RR^d)}^2 \lesssim 2^{am} \sum\nolimits_r \| F_{k,\mu,r} \|_{L^2(\RR^d)}^2.
    \end{equation}
    %
    Recall that $\chi_{x,r} = \text{Trans}_x(\sigma_r * \psi)$, where $\psi$ is a compactly supported function whose Fourier transform is non-negative and vanishes to high order at the origin. In particular, we now make the additional assumption that $\psi = \psi_{\circ} * \psi_{\circ}$ for some other compactly function $\psi_{\circ}$ whose Fourier transform is non-negative and vanishes to high order at the origin. Then we find that $F_{k,\mu,r}$ is equal to the convolution of the function
    %
    \begin{equation}
      G_r = \sum\nolimits_{(x,r) \in \mathcal{E}} \text{Trans}_x \psi_{\circ}
    \end{equation}
    %
    with the function $\sigma_r * \psi_{\circ}$. Using the standard asymptotics for the Fourier transform of $\sigma_r$, i.e. that for $|\xi| \geq 1$,
    %
    \begin{equation}
      |\widehat{\sigma_r}(\xi)| \lesssim r^{d-1} (1 + r |\xi|)^{- \frac{d-1}{2}},
    \end{equation}
    %
    and since $|\widehat{\psi_\circ}(\xi)| \lesssim_N |\xi|^N$, we get that if $r \geq 1$, then for $|\xi| \leq 1/r$,
    %
    \begin{equation}
      |\widehat{\sigma_r}(\xi) \widehat{\psi_\circ}(\xi)| \lesssim_N r^{d-1-N}
    \end{equation}
    %
    and for $|\xi| \geq 1/r$,
    %
    \begin{equation}
      |\widehat{\sigma_r}(\xi) \widehat{\psi_\circ}(\xi)| \lesssim_N r^{\frac{d-1}{2}} |\xi|^{-N}.
    \end{equation}
    %
    Thus in particular,the $L^\infty$ norm of the Fourier transform of $\sigma_r * \psi_\circ$ is $O(r^{(d-1)/2})$. Now the functions $\psi_{\circ}$ are compactly supported, so since the set of $x$ such that $(x,r) \in \mathcal{E}$ is one-separated, we find that
    %
    \begin{equation}
      \| G_r \|_{L^2(\RR^d)} \lesssim \# \{ x : (x,r) \in \mathcal{E} \}^{1/2}.
    \end{equation}
    %
    But this means that
    %
    \begin{equation}
      \| F_{k,\mu,r} \|_{L^2(\RR^d)} = \| G_r * (\sigma_r * \psi_{\circ}) \|_{L^2(\RR^d)} \lesssim r^{\frac{d-1}{2}} \# \{ x : (x,r) \in \mathcal{E} \}^{1/2}.
    \end{equation}
    %
    Thus we have that
    %
    \begin{equation}
      \| F_{k,\mu} \|_{L^2(\RR^d)}^2 = 2^{um} \cdot \# \mathcal{E}_{k,\mu} \cdot 2^{k(d-1)}.
    \end{equation}
    %
    Summing over $\mu$ gives that
    %
    \begin{equation}
      \| F_k \|_{L^2(\RR^d)}^2 = 2^{k(d-1)} \# \mathcal{E}_k (2^{um}  + 2^{m(1 - u(d-3)/2)}).
    \end{equation}
    %
    Picking $u = 2/(d-1)$ optimizes this bound, giving
    %
    \begin{equation} \label{AOWIDFjaewiofjaqwiofdrj2342309854324}
      \| F_k \|_{L^2(\RR^d)} \lesssim 2^{m/(d-1)} 2^{k(d-1)/2} (\# \mathcal{E}_k)^{1/2}.
    \end{equation}
    %
    Combining \eqref{awiodjwaoidfjawoid12343124134123453} and \eqref{AOWIDFjaewiofjaqwiofdrj2342309854324} completes the proof.
\end{proof}

This completes a proof of the single scale estimates of the paper. The paper then uses an atomic decomposition method to combine these scales and thus complete the proof of Theorem \ref{HeoNazarovSeegerTheorem}. Rather than discuss these methods, we instead discuss an iteration of this method, developed by the same authors in a follow up paper \cite{HeoandNazarovandSeeger2}.

\section{Combining Scales with Atomic Decompositions} \label{sec:combiningscaleswithatomicdecompositions}

We now sketch a proof as to how we can complete the proof of Theorem \ref{HeoNazarovSeegerTheorem}. Our arguments are deliberately vague, as our goal is to build intuition about the techniques involved, in order to carry out analogous methods more rigorously in the compact manifold setting later. If $T = m(P)$, then Proposition \ref{HeoNazarovSeegerSingleScaleInequality}, appropriately rescaled, implies that if we write $T = \sum T_j$, where $T_j$ is the radial Fourier multiplier operator with convolution kernel $k_j(\cdot/2^j)$, then
%
\begin{equation} \label{individualscaleoperatorbound}
    \| T_j \|_{L^p(\RR^d) \to L^q(\RR^d)} \lesssim \| k_j \|_{L^q(\RR^d)}.
\end{equation}
%
The goal of this section is to prove that we can bound the sum of the operators $T_j$ by $\sup_j \| k_j \|_{L^q(\RR^d)}$, which morally speaking, is a bound of the form
%
\begin{equation}
  \left\| \sum\nolimits_j T_j \right\| \lesssim \sup\nolimits_j \| T_j \|.
\end{equation}
%
We must thus show that the operators $\{ T_j \}$ do not constructively interfere with one another to a significant extent.

Atomic decompositions are a powerful way to control interactions between operators. The method involves decomposing functions into more elementary components, which we call \emph{atoms}, that have controlled size, spatial support, and oscillatory properties. Such atoms are chosen via careful, non-linear selection processes, often related to stopping times, to ensure certain cancellation conditions hold. We use a variant of this method, involving a use of a Whitney decomposition rather than a stopping time argument, to obtain an atomic decomposition where atoms do not cluster, in an appropriate sense. The method employed was first introduced in \cite{SeegerRemarks}.

Since we are controlling parts of an operator supported on different dyadic frequency ranges on functions in $L^p(\RR^d)$ for $1 < p < \infty$, it is natural to consider the Littlewood-Paley inequality $\| Sf \|_{L^p(\RR^d)} \lesssim \| f \|_{L^p(\RR^d)}$, where
%
\begin{equation}
  Sf(x) = \left( \sum\nolimits_j |P_j f(x)|^2 \right)^{1/2},
\end{equation}
%
and the operators $P_j$ are Littlewood-Paley projections, Fourier multiplier operators with symbol $\psi(\cdot / 2^j)$, where $\psi \in C_c^\infty(\RR^d)$ is equal to one on the support of the function $\chi$ used to decompose $T$ into the operators $T_j$ above, so that $T_j f = T_j f_j$, where $f_j = P_j f$. Roughly speaking, our atomic decomposition will be of the following form: for each dyadic number $H$, we consider a family of dyadic cubes $\mathcal{W}_H$, whose union is the set $\{ x : |Sf(x)| \sim H \}$, and whose doubles have the bounded overlap property. We will then obtain a decomposition $f_j = \sum A_{j,H,W}$, where $A_{j,H,W}$ has Fourier support on $|\xi| \sim 2^j$, is supported on the cube $W$, and for each fixed $H$,
%
\begin{equation}
  \left( \sum\nolimits_j \left|\sum\nolimits_W |A_{j,H,W}(x)|\; \right|^2 \right)^{1/2} \sim H \quad\text{if $|Sf(x)| \sim H$}.
\end{equation}
%
The advantage of this decomposition is that it controls how many \emph{local interactions} the atoms $\{ A_{j,H,W} \}$ have. To see why this might be useful, suppose we were considering a Fourier multiplier operator $T$ with the property that $T A_{j,H,W}$ is supported on the cube $W^*$ obtained by doubling the side lengths of the cube $W$, but maintaining the same center. The bounded overlap property of the dyadic cubes $\mathcal{W}_H$ would then imply an almost orthogonality bound for each $H$, that
%
\begin{equation}
\begin{split}
  \left\| \sum\nolimits_{j,W} T_j A_{j,H,W} \right\|_{L^2(\RR^d)} &\lesssim \left( \sum\nolimits_{j,W} \| T_j A_{j,H,W} \|_{L^2(\RR^d)}^2 \right)^{1/2}\\
  &\lesssim \left( \sum\nolimits_{j,W} \| A_{j,H,W} \|_{L^2(\RR^d)}^2 \right)^{1/2}\\
  &= \left\| \left( \sum\nolimits_{j,W} |A_{j,H,W}|^2 \right)^{1/2} \right\|_{L^2(\RR^d)} \lesssim H |\Omega_H|^{1/2}
\end{split}
\end{equation}
%that for $1/r = 1/2 + 1/p - 1/q$,
%
%\begin{align*}
%  \left\| \sum\nolimits_{j,W} T_j a_{j,H,W} \right\|_{L^2(\RR^d)} &\lesssim \left( \sum\nolimits_{j,W} \| T_j a_{j,H,W} \|_{L^2(\RR^d)}^2 \right)^{1/2}\\
%  &\lesssim \left( \sum\nolimits_{j,W} \left[ 2^{-jd(1/p - 1/q)} \| a_{j,H,W} \|_{L^2(\RR^d)} \right]^2 \right)^{1/2}\\
%  &\lesssim \left( \sum\nolimits_{j,W} \| a_{j,H,W} \|_{L^r(\RR^d)}^2 \right)^{1/2}\\
%  &\lesssim \left\| \left( \sum |a_{j,H,W}|^2 \right)^{1/2} \right\|_{L^r(\RR^d)} \lesssim H |\Omega_H|^{1/2 + 1/p - 1/q}
%\end{align*}
%
%
Since the sum is supported on $\Omega_H$, H\"{o}lder's inequality then justifies that
%
\begin{equation}
\begin{split}
  \left\| \sum\nolimits_{j,W} T_j A_{j,H,W} \right\|_{L^1(\RR^d)} &\lesssim |\Omega_H|^{1/2} \left\| \sum\nolimits_{j,W} T_j A_{j,H,W} \right\|_{L^2(\RR^d)} \lesssim H |\Omega_H|.
\end{split}
\end{equation}
%
Real interpolation allows us to sum in $H$, so that for $1 < p < 2$,
%
\begin{equation}
\begin{split}
  \| Tf \|_{L^p(\RR^d)} &\lesssim \left( \sum \left[ H |\Omega_H|^{1/p} \right]^p \right)^{1/p} \lesssim \| S f \|_{L^p(\RR^d)} \lesssim \| f \|_{L^p(\RR^d)}.
\end{split}
\end{equation}
%
and thus we have proven the boundedness of such operators.

Such a simple proof will not suffice for our analysis, since the class of multipliers we are considering is not pseudo-local at all; the kernels $k_j$ are only assumed to be uniformly bounded in $L^p(\RR^d)$, and need not satisfy any decay bound as $|x| \to \infty$. Nonetheless, the argument above can be used to prove bounds for other more pseudo-local multipliers, for instance, obtaining an alternate proof of the endpoint results of \cite{SeegerSingular}. We will, however, be able to exploit the above calculations to control \emph{close range interactions} of general multipliers. Given $T$ and $f$, write
%
\begin{equation}
  Tf = \sum T_j \{ A_{j,H,W} \} = \sum T_{j,W,\text{Short}} \{ A_{j,H,W} \} + T_{j,W,\text{Long}} \{ A_{j,H,W} \},
\end{equation}
%
where
%
\begin{equation}
  T_{j,W,\text{Short}}(x,y) = \mathbb{I}_{W^*}(x)\; T_{j,W}(x,y) \quad\text{and}\quad T_{j,W,\text{Long}}(x,y) = \mathbb{I}_{(W^*)^c}(x)\; T_{j,W}(x,y).
\end{equation}
%
The calculations of the previous paragraph can be adapted to show that
%
\begin{equation}
  \left\| \sum\nolimits_{j,W,H} T_{j,W,\text{Short}} \{ A_{j,H,W} \} \right\|_{L^q(\RR^d)} \lesssim \| f \|_{L^p(\RR^d)},
\end{equation}
%
and it remains to find a way to control the long range interactions between atoms.

There are several related techniques to control the long range interactions, the most elegant formulation provided in the paper \cite{HeoandNazarovandSeeger2}. The idea is to take a singlef scale estimate, and upgrade this to an estimate with a geometrically decaying constant term when our inputs have amplitudes that are locally constant on a much larger scale than is required by the uncertainty principle and the frequency support of the inputs, namely, that
%
\begin{equation}
  \left\| {\textstyle \sum\nolimits_H} {\textstyle \sum\nolimits_{W \in \mathcal{W}_{H,l-j}}} T_{j,W,\text{Long}} \{ A_{j,H,W} \} \right\|_{L^q(\RR^d)} \lesssim 2^{-l \varepsilon} \left( {\textstyle \sum\nolimits_H} \| {\textstyle \sum\nolimits_{W \in \mathcal{W}_{H,l-j}}} A_{j,H,W} \|_{L^\infty(\RR^d)}^p |W| \right)^{1/p},
\end{equation}
%
where $\mathcal{W}_{H,a}$ are the set of all cubes of side-length $2^a$. Summing in $j$ using the $L^2$ orthogonality of different frequency scales, we find that
%
\begin{equation}
  \left\| {\textstyle \sum\nolimits_{j,H}} {\textstyle \sum\nolimits_{W \in \mathcal{W}_{H,l-j}}} T_{j,W,\text{Long}} A_{j,H,W} \right\|_{L^p(\RR^d)} \lesssim 2^{-l \varepsilon} \left( {\textstyle \sum\nolimits_{j,H}} \|  {\textstyle \sum\nolimits_{W \in \mathcal{W}_{H,l-j}}} A_{j,H,W} \|_{L^\infty(\RR^d)}^p |W| \right)^{1/p}.
\end{equation}
%
For a fixed $l$, and each $W \in \mathcal{W}_H$, there exists a unique $j$ such that $W \in \mathcal{W}_{H,l-j}$, which implies that the supports of the functions $\{ A_{j,H,W} \}$ in the sum above are almost disjoint, and thus the simple bound $\| A_{j,H,W} \|_{L^\infty(\RR^d)} \lesssim H$ gives that
%
\begin{equation}
  \sum\nolimits_j \| \sum\nolimits_{W \in \mathcal{W}_{H,l-j}} A_{j,H,W} \|_{L^\infty(\RR^d)}^p |W| \leq H^p |\Omega_H|,
\end{equation}
%
and thus that
%
\begin{equation}
  \left\| \sum\nolimits_{j,H} \sum\nolimits_{W \in \mathcal{W}_{H,l-j}} T_{j,W,\text{Long}} A_{j,H,W} \right\|_{L^p(\RR^d)} \lesssim 2^{-l \varepsilon} \left( \sum\nolimits_H H^p |\Omega_H| \right)^{1/p} \lesssim 2^{-l\varepsilon} \| f \|_{L^p(\RR^d)}.
\end{equation}
%
Summing in $l$ trivially using the triangle inequality and the geometric decay in $l$ gives that
%
\begin{equation}
  \left\| \sum\nolimits_j \sum\nolimits_H \sum\nolimits_{W \in \mathcal{W}_{H}} T_{j,W,\text{Long}} A_{j,H,W} \right\|_{L^p(\RR^d)} \lesssim \| f \|_{L^p(\RR^d)},
\end{equation}
%
which controls the long-range interactions in full, completing the proof of Theorem \ref{HeoNazarovSeegerTheorem}.

%A related bound is a square function estimate of Peetre, which says that $\| \tilde{S} f \|_{L^p(\RR^d)} \lesssim \| f \|_{L^p(\RR^d)}$, where
%
%\[ \tilde{S} f(x) = \left( \sum\nolimits_j \sup\nolimits_{|h| \leq C 2^{-j}} |P_j f(x + h)|^2 \right)^{1/2} \]
%
%for a fixed $C > 0$. This bound is morally equivalent to the Littlewood-Paley inequality in light of uncertainty principle heuristics, which indicate that the function $P_j f$ is locally constant at a scale $2^{-j}$. TODO EDIT

%For each dyadic number $H$, we let $\Omega_H = \{ x : \tilde{S}f(x) \geq H \}$. For each dyadic cube $Q$, there exists a largest value $H(Q)$ such that $\Omega_{H(Q)}$ contains at least half the points in $Q$. We perform a wave packet decomposition $f_j = \sum f_{j,Q}$, where $Q$ ranges over all dyadic sidelength $2^{-j}$ cubes, and $f_{j,Q} = \mathbb{I}_Q f_j$. Our atoms will be given by $a_{j,H,W}$ TODO EDIT





\section{Quasi-Radial Multipliers and Local Smoothing}

Jongchon Kim \cite{KimQuasiradial} has extended 
the bounds of Heo-Nazarov-Seeger to quasi-radial multipliers, i.e. multipliers of the form $(a \circ r)(\xi)$, where $r: \RR^d \to [0,\infty)$ is a smooth, homogeneous function of order one, such that the cosphere $S = \{ \xi : r(\xi) = 1 \}$ is a hypersurface with non-vanishing Gauss curvature. Such multipliers retain the translation and dilation symmetries of the family of radial Fourier multiplier operators, but are no longer \emph{rotation-invariant}.

Using the same identities as for radial Fourier multiplier operators, one can verify that
%
\begin{equation}
  \| a \circ r \|_{M^p(\RR^d)} \gtrsim \sup\nolimits_R \| k_R \|_{L^q(\RR^d)},
\end{equation}
%
where $k_R$ is the Fourier transform of $\chi(\cdot) (a \circ r)(R \cdot)$, for some fixed $\chi \in C_c^\infty(\RR)$. In a paper analyzing quasi-radial multipliers of Bochner-Riesz type, Sanghyuk Lee and Andreas Seeger \cite{LeeSeeger2} showed
%
\begin{equation}
  \sup\nolimits_R \| k_R \|_{L^p(\RR^d)} \sim \| a \|_{R^{p,s}[0,\infty)},
\end{equation}
%
Thus
%
\begin{equation} \label{auiwodjawiodjawoi2342342}
  \| a \circ r \|_{M^p(\RR^d)} \gtrsim \| a \|_{R^{p,s}[0,\infty)}.
\end{equation}
%
Jongchon Kim \cite{KimQuasiradial} has showed the converse, in the same range $1/p - 1/2 > 1/(d-1)$ as considered by Heo, Nazarov and Seeger in the last section.

\begin{theorem}
  For $1/p - 1/2 > 1/(d - 1)$, and $r: \RR^d \to [0,\infty)$ is smooth and homogeneous, then for any regulated function $a$,
  %
  \[ \| a \circ r \|_{M^{p,q}(\RR^d)} \lesssim \| a \|_{R^{q,s}_d[0,\infty)}. \]
\end{theorem}

The proof is an adaption of the proof of \cite{HeoandNazarovandSeeger2}. However, it is more difficult to work directly with convolution kernels $k$ in this problem, since, unlike for radial multipliers, whose kernels are also radial, the kernels $k$ corresponding to quasi-radial Fourier multiplier operators need not be quasi-radial. It is here that we introduce a technique that has proved essential to an analysis of multiplier operators in settings lacking the full symmetry properties of radial multipliers: a reduction of the study of multipliers to an analysis of wave equations. Using the Fourier inversion formula, we write
%
\begin{equation}
\begin{split}
  (a \circ r) f(x) &= \int a(r(\xi)) e^{2 \pi i \xi \cdot (x - y)} f(y)\; dy\; dx\\
  &= \iint \widehat{a}(t) e^{2 \pi i [t r(\xi) + \xi \cdot (x - y)]} f(y)\; dy\; dx\; dt\\
  &= \int \widehat{a}(t) (w_{2 \pi t} * f)(x)\; dt
\end{split}
\end{equation}
%
where $\widehat{w}_t(\xi) = e^{i t r(\xi)}$. As $t$ varies, $u = w_t * f$ solves the wave equation $\partial_t u = i P u$, where $P$ is the Fourier multiplier operator whose symbol is the function $r$. Define $\chi_{x,t} = \text{Trans}_x \left( \int_{-\infty}^\infty \psi(s) w_{t - s}\; ds \right)$, where the Fourier transform of $\psi$ is non-negative and vanishing to high order at the origin. Then, using oscillatory integral techniques akin to Lemma \ref{lemma4}, one can obtain inner product estimates on the quantities $\langle \chi_{x_1,t_1}, \chi_{x_2,t_2} \rangle$ that show such terms are negligible unless $|x_1 - x_2| + |t_1 - t_2| \lesssim 1$. We can then adapt the proof of \cite{HeoandNazarovandSeeger2}, performing a density decomposition using the Euclidean metric on $\RR^{d+1}$, and thus obtain single scale bounds for the quasi-radial multipliers. An atomic decomposition analogous to that discussed in Section \ref{sec:combiningscaleswithatomicdecompositions} then yields Jongchon Kim's result.

We note that the endpoint analysis of radial multipliers we have been discussing is very closely related to the regularity of solutions to wave equations. In particular, if $\widehat{w}_t = e^{i t |\xi|}$, then for any $f$, the function $u = w_t * f$ solves the half-wave equation $\partial_t u = i P u$ where $P = \sqrt{-\Delta}$. The Littlewood-Paley pieces $(P_k u)(\cdot,t)$ are Fourier multipliers with symbol $\chi(\cdot / 2^k) e^{i t |\xi|}$, which are radial, and thus can be written in terms of spherical averages. The analysis of Proposition \ref{lemma1} can then be used to show that for $1/2 - 1/q > 1/(d-1)$,
%
\begin{equation}
  \| P_k u \|_{L^q( \RR^d \times [1,2] )} \lesssim 2^{k s} \| f \|_{L^q(\RR^d)}\quad\text{where}\ s = (d-1)(1/2 - 1/q) - 1/q.
\end{equation}
%
With some more work, involving the atomic decomposition methods discussed in Section \ref{sec:combiningscaleswithatomicdecompositions}, Heo, Nazarov, and Seeger are able to combine the frequency scales, and thus prove the local smoothing estimate
%
\begin{equation}
  \| u \|_{L^q(\RR^d \times [1,2])} \lesssim \| f \|_{W^{s,q}(\RR^d)}.
\end{equation}
%
These are the current sharpest \emph{endpoint} local smoothing bounds for the wave equation in any dimension. Local smoothing is thus closely connected to radial multiplier bounds, and we will exploit this relation to bound spectral multipliers in Chapter \ref{chap:boundedsinglefrequencyscale}.

To end our discussion of radial multipliers, notice that the Fourier multiplier operator $P$ is above is an elliptic operator on $\RR^d$, satisfying the Assumption A we introduced in Chapter \ref{cha:multipliers_of_an_elliptic_operator}. For such an operator, the equation $\partial_t u = i P u$ is hyperbolic, and it's solutions have wavelike properties. We will rely on a similar decomposition method used for the quasi-radial multiplier with symbol $a \circ r$ on manifolds, together with a study of the geometry of the manifold upon which we study these multipliers, to extend the methods we have discussed in the previous three sections to compact manifolds. Such a method has been used by Jongchon Kim to study certain spectral multipliers on compact manifolds \cite{KimSpectral}, but the results here do not come close to a full characterization of boundedness. We will now return to discuss this setting in more detail.








\chapter{Wave Equations on Compact Manifolds} \label{chap:waveequation}

\section{An Analogue of the Radial Multiplier Conjecture on Compact Manifolds} \label{sec:AnAnalogueOf}

We now return to the study of spectral multipliers on compact manifolds. In a heuristic sense, one can think of the Fourier multipliers studied in the last Chapter as a `high frequency' limiting case of multipliers on manifolds, i.e. so that for any compact manifold $X$, and any operator $P \in \COP{1}(X)$ with principal symbol $p(x_0,\xi)$, locally around $x_0 \in X$ the operators $P/R$ behave more and more like the Fourier multiplier operator on $\RR^d$ with symbol $p(x_0,\cdot)$ as $R \to \infty$. In particular, the following transplantation result of Mitjagin \cite{Mitjagin} holds.

\begin{theorem} \label{Pawdiojwaoij123423423423423}
  Suppose $X$ is a $d$-dimensional compact manifold, and $P \in \COP{1}(X)$. For each $x_0 \in X$, if we choose a basis, identifying $T_{x_0}^* X$ with $\RR^d$, and let $p_{x_0}: \RR^d \to [0,\infty)$ be the principal symbol of $p$ restricted to $T_{x_0}^* X$, then for any regulated function $a$,
  %
  \[ \| a \circ p_{x_0} \|_{M^p(\RR^d)} \lesssim \| a \|_{M^p_{\text{Dil}}(X)}. \]
\end{theorem}

Thus if $P$ satisfies Assumption A, it follows from Theorem \ref{Pawdiojwaoij123423423423423} and \eqref{auiwodjawiodjawoi2342342} that
%
\begin{equation} \label{upperboundawdoiwjdoawi}
  \| a \|_{M^p_{\text{Dil}}(X)} \gtrsim \| a \|_{R^{p,s}(\RR^d)},
\end{equation}
%
with $s = (d-1)(1/p - 1/2)$.

We might conjecture that one can reverse \eqref{upperboundawdoiwjdoawi} in the same range as for the radial multiplier conjecture, i.e. proving that $\| a \|_{M^p_{\text{Dil}}(X)} \lesssim \| a \|_{R^{p,s}(\RR^d)}$ for $1/p - 1/2 > 1/2d$. However, this is unlikely to be possible on a general manifold. Such a result would imply the Bochner-Riesz conjecture in the full range, i.e. that if $m^\delta(\lambda) = (1 - \lambda^2)^{\delta/2}$, then $\| m^\delta \|_{M^p_{\text{Dil}}(X)} < \infty$ for $\delta > d(1/p - 1/2) - 1/2$. But results of Shaoming Guo, Hong Wang, and Ruixiang Zhang \cite{GuoWangZhang} and Song Dai, Liuwei Gong, Shaoming Guo, and Ruixiang Zhang \cite{DaiGongGuoZhang}, with related work of Minicozzi and Sogge \cite{Minicozzi}, prove that certain oscillatory integral operators associated with Bochner-Riesz operators on $3$-dimensional Riemannian manifolds fail to be bounded appropriately when the geometry of $X$ induced on $P$, which we introduce in the next section, does not have constant sectional curvature. This provides weak evidence that the Bochner-Riesz conjecture may fail on a general manifold in the full range. On the other hand, it is likely that the conjecture holds in the full range on manifolds with constant sectional curvature; results of \cite{Alladi}, which obtain the full range of the conjecture on $S^d$ for \emph{zonal inputs}, provide some evidence for the truth of the conjecture. Nonetheless, in this thesis we do describe positive results for this analogue of the radial multiplier conjecture, and our results can be applied on certaom manifolds, such as zonal manifolds and the rank one symmetric spaces, which do not have constant sectional curvature.

% It is unclear what the correct range the analogue of the radial multiplier conjecture should be in the worst case, nor the quantitative features of a manifold (aside from some aspect of curvature) which determine the range of exponents under which the conjecture should hold. 

%
%
% Oscillatory integral operators with phase d(x,y)
% Cannot be bounded in the full range
%
% This means that the Bochner-Riesz conjecture fails
%
%
% R^{(M,M')}_N f(x) = int_{M'} e^{iN d(x,y)} a(x) f(y) dy
% If M does not have constant sectional curvature, there
% exists a hyperplane M' and a smooth a

%Whether an analogous result remains true for more general Riemannian manifolds remains unclear, since the family of eigenfunctions to the Laplacian can take on various different forms on these manifolds, that can look quite different to the Euclidean case (TODO: Does the existence of low dimension Kakeya sets on certain manifolds show that the radial multiplier conjecture cannot be true in general). On general compact manifolds, there are difficulties arising from a generalization of the radial multiplier conjecture, connected to the fact that analogues of the Kakeya / Nikodym conjecture are false in this general setting \cite{Minicozzi}. But these problems do not arise for constant curvature manifolds, like the sphere. The sphere also has over special properties which make it especially amenable to analysis, such as the fact that solutions to the wave equation on spheres are periodic. Best of all, there are already results which achieve the analogue of \cite{GarrigosandSeeger} on the sphere. Thus it seems reasonable that current research techniques can obtain interesting results for radial multipliers on the sphere, at least in the ranges established in \cite{HeoandNazarovandSeeger} or even those results in \cite{Cladek}.

\section{Geometries Induced by Elliptic Operators} \label{sec:geometriesinduced}

We plan to study spectral multipliers of an operator $P \in \COP{1}(X)$ on a compact manifold $X$ via the use of solutions of the wave equation $\partial_t u = i P u$ on $M$, which allow us to exploit geometric information about the manifold $M$ and thus extend the results of Heo, Nazarov, and Seeger to the setting of compact manifolds when Assumption A and Assumption B of Chapter \ref{cha:multipliers_of_an_elliptic_operator} hold. But what is the geometry of the manifold we should be using? If the operator $P = \sqrt{-\Delta}$ is induced by a Riemannian metric on $X$, the geometric structure is clear, since $X$ already has a Riemannian geometry. Given a more general operator $P$, we will use the principal symbol of $P$ to give $X$ a \emph{Finsler geometry}. In this section we describe the bare essentials of Finsler geometry needed for the arguments which occur later on in the thesis. We mainly refer to a textbook on Finsler geometry written by David Dai-Wai Bao, Shiing-Shen Chern, and Zhongmin Shen \cite{BaoChern} for a reference to further details.

Let $X$ be a $d$-dimensional manifold. We denote an element of the tangent bundle $TX$ by $(x,v)$, where $x \in X$ and $v \in T_x X$, and an element of the cotangent bundle $T^*X$ by $(x,\xi)$, with $\xi \in T_x^* X$. A \emph{Finsler metric} on $X$ is a homogeneous function $F: T X \to [0,\infty)$, which is smooth on $TX - 0$, and such that for each $x \in X$, the function $F_x: T^*_x X \to [0,\infty)$ is a \emph{strictly convex} norm, in the sense that for $(x,v) \in TX - 0$, the Hessian of $F_x^2$ in the $v$ variable is positive-definite, i.e. the $(2,0)$ tensor $g(x,v)$ given in coordinates by
%
\begin{equation} \label{FinslerMetricCoefficients}
    g_{ij}(x,v) = \frac{1}{2} \frac{\partial^2 F^2}{\partial v^i \partial v^j}(x,v).
\end{equation}
%
These coefficients give rise to an inner product on $T_x X$ which approximates the Finsler metric near $v$ to second order. We record the \emph{fundamental inequality for Finsler metrics}.

\begin{lemma}
  For any $v,w \in T_x X$, $\sum g_{ij}(x,v) v^i w^j \leq F(x,v) F(x,w)$.
\end{lemma}
\begin{proof}
  By Euler's homogeneous function theorem, we can write
  %
  \begin{equation}
    \sum g_{ij}(x,v) v^i w^j = (1/2) \sum \partial_{v_i} F^2(x,v) w^i = F(x,v) \sum \partial_{v_i} F(x,v) w^i.
  \end{equation}
  %
  To complete the proof, we must thus show that $\sum \partial_{v_i} F(x,v) w^i \leq F(x,w)$. But this follows by the triangle inequality $F(x,v + tw) \leq F(x,v) + t F(x,w)$.
\end{proof}

A Finsler metric gives each of the tangent spaces of a manifold the structure of a norm space. Such a structure naturally gives the dual space a dual norm $F_*: T^* X \to [0,\infty)$. The strict convexity of $F$ gives rise to a \emph{Legendre transform} on $X$.

\begin{lemma}
  For a Finsler manifold $X$, define a homogeneous map from $TX$ to $T^*X$, defined in coordinates by setting
  %
  \[ \mathcal{L}(x,v)_i = \sum g_{ij}(x,v) v^j. \]
  %
  Then $\mathcal{L}$ is a diffeomorphism from $TX - 0$ to $T^*X - 0$, and for each $x \in X$ and each unit vector $v \in T_x X$, $\mathcal{L}(x,v)$ is the unique unit vector $\xi$ such that $\langle \xi, v \rangle = 1$.
\end{lemma}
\begin{proof}
  Begin by defining a map $\mathcal{L}^{-1}$, which is homogeneous, and for each unit vector $\xi \in T_x^* X$, is the unique unit vector $v \in T_x X$ such that $\langle \xi, v \rangle = 1$. The existence follows from the definition of the dual norm, and the uniqueness follows from the strict convexity of the Finsler norm. By the method of Lagrangian multipliers, if $\xi$ is a unit vector, and $(x,v) = \mathcal{L}^{-1}(x,\xi)$, it must be true in coordinates that $\xi_i = \sum g_{ij}(x,v) v^j$ for each $i$, which shows that the map $\mathcal{L}^{-1}$ is injective. Moreover, given any unit vector $v \in T_x X$, the vector $\xi \in T_x^* X$ given in coordinates by $\xi_i = \sum g_{ij}(x,v) v^j$ is a unit vector, and satisfies $\langle \xi, v \rangle = 1$, by Euler's homogeneous function theorem and the fundamental inequality. Thus we conclude that $\mathcal{L}^{-1}(x,\xi) = (x,v)$, which shows that the map $\mathcal{L}^{-1}$ is surjective, and the inverse is given by the map $\mathcal{L}$ defined above. By Euler's homogeneous function theorem, we calculate that $\partial_{v_i} \mathcal{L}_j = g_{ij}$, and since the values $g_{ij}$ define a positive-definite (and thus invertible) matrix, it follows that $\mathcal{L}$ is a diffeomorphism.
\end{proof}

The Legendre transform is the Finsler variant of the musical isomorphism in Riemannian geometry, though the musical isomorphism of Riemannian geometry is linear rather than just homogeneous.

\begin{corollary}
  The norm $F_*$ is strictly convex, and if $\mathcal{L}(x,v) = (x,\xi)$, and we define
  %
  \[ g^{ij}(x,\xi) = \frac{1}{2} \frac{\partial^2 F_*^2}{\partial \xi_i \partial \xi_j}(x,\xi). \]
  %
  then
  %
  \[ \sum\nolimits_j g^{ij}(x,\xi) g_{jk}(x,v) = \delta^i_k. \]
\end{corollary}
\begin{proof}
  The equation implies strict convexity, because it implies that the matrix with entries $g^{ij}(x,\xi)$ is the inverse of the matrix with entries $g_{ij}(x,v)$, and the inverse of a positive definite matrix is positive definite. If $v$ is a unit vector, then a form of the fundamental inequality (applied to $F_*$ rather than $F$) implies that the vector $w \in T_x X$ defined in coordinates by $w^i = g^{ij}(x,\xi) \xi_j$ is a unit vector, and Euler's homogeneous function implies that $\langle \xi, w \rangle = 1$. Thus $w = v$, and so we conclude by homogeneity that in general $\mathcal{L}^{-1}(x,\xi)^i = \sum g^{ij}(x,\xi) \xi_j$. We calculate that $\partial_i \mathcal{L}^{-1}(x,\xi)^j = g^{ij}(x,\xi)$, and then differentiating the identity $\mathcal{L}^{-1} \circ \mathcal{L} = I$ implies the required claim.
\end{proof}

Now let $P \in \COP{1}(X)$ satisfy Assumption A. The principal symbol of $P$ is a function $p: T^* M \to [0,\infty)$, and the following lemma applies.

\begin{lemma}
  Suppose $p: T^* M \to [0,\infty)$ is homogeneous, and for each $x \in M$, the cosphere $S_x^* = \{ \xi \in T_x^* M : p(x,\xi) = 1 \}$ has non-vanishing Gaussian curvature. Then
  %
  \[ F(x,v) = \{ \xi(v) : p(x,\xi) = 1 \} \]
  %
  is a Finsler metric on $M$, and the dual metric $F_*$ on $T^*M$ is equal to $p$.
\end{lemma}
\begin{proof}
  For each $x \in U_0$, the cosphere $S_x^* = \{ \xi \in T_x^* M : p(x,\xi) = 1 \}$ has non-vanishing Gaussian curvature. We claim that all principal curvatures of $S_x^*$ must actually be \emph{positive}. This follows from a simple modification of an argument found in Chapter 2 of \cite{HeinzHopf}. Indeed, if we fix an arbitrary point $v_0 \in T_x^*M$, and consider the smallest closed ball $B \subset T_x^* M$ centered at $v_0$ and containing $S_x^*$, then the sphere $\partial B$ must share the same tangent plane as $S_x^*$ at some point. All principal curvatures of $\partial B$ are positive, and at this point all principal curvatures of $S_x^*$ must be greater than the principal curvatures of $\partial B$, since $S_x^*$ curves away faster than $\partial B$ in all directions. By continuity, we conclude that the principal curvatures are everywhere positive.  Thus for each $x \in M$ and $\xi \in T_x^* M - \{ 0 \}$, the coefficients
  %
  \begin{equation}
    g^{ij}(x,\xi) = (1/2) (\partial^2 p^2 / \partial \xi_i \partial \xi_j)
  \end{equation}
  %
  form a positive-definite matrix. But inverting the procedure of the previous two lemmas shows that the dual norm $F(x,v) = \sup\nolimits_{\xi \in S_x^*} \xi(v)$ is also strictly convex, and thus gives a Finsler metric on $M$ with $F_* = p$. %(Chapter 14 of \cite{BaoChern}). 
\end{proof}

 % and give rise to an inner product on $T_x M$ that best approximates the Finsler metric to second order at $(x,v)$.%, i.e. so that if $F(x,v) = F(x,w) = 1$, then
%
%\[ \left| F(x,w) - ( \sum g_{jk}(x,v) w^j w^k )^{1/2} \right| \lesssim |v - w|^3 \]

 %For each $v \in T_x M$, the coefficients $g_{jk}(x,v)$ define a Riemannian metric on $T_x M$ which approximates the Finsler metric to second order in a neighbourhood of $v$. % i.e. the functions $w \mapsto F(x,w)$ and $w \mapsto ( \sum g_{jk}(x,v) w^j w^k )^{1/2}$ agree up to second order in a neighborhood of $v$.

%The closest analogue to the Levi-Civita connection on Finsler manifolds is the \emph{Chern connection}. To discuss the connection, let $\bar{T}$ be the \emph{distinguished section} of $\mathscr{E}$, the section given by $\bar{T}(x,v) = F(v)^{-1} v$. For two sections $X$ and $Y$ of $\mathscr{E}$, the Chern connection gives a section $\nabla_X Y$ of $\mathscr{E}$ which is (a) $C^\infty(TM)$-linear in $X$ and $\RR$-linear in $Y$ (b) \emph{torsion free}, in the sense that $\nabla_X Y - \nabla_Y X = [X,Y]$, and (c) \emph{almost metric compatable} which is a somewhat technical to state fully, but for our purposes implies that for any sections $X$ and $Y$ and $Z$ of $\mathscr{E}$, when evaluating $X \{ g_{\bar{T}}(Y,Z) \}$ at a point in $TM$ where $X$, $Y$, or $Z$ is a multiple of $\bar{T}$, one has
%
% X { g_{Tbar}(Y,Z) } = g_{Tbar}( Nabla_X Y, Z ) + g_{Tbar} ( Y, Nabla_X Z ) + 2 C_{Tbar}(Nabla_X Tbar, Y, Z )
%
%\begin{equation} \label{almostmetriccompatibility}
%    X \{ g_{\bar{T}}(Y,Z) \} = g_{\bar{T}}(\nabla_X Y, Z) + g_{\bar{T}}(Y, \nabla_X Z).
%\end{equation}
%
%%Note that the Chern connection is \emph{not metric compatible}, i.e. \eqref{almostmetriccompatibility} does not hold for arbitrary $X$, $Y$, and $Z$. There does not exist a torsion free, metric compatible affine connection on the bundle $\mathscr{E}$ unless $M$ is a Riemannian manifold. As in Riemannian geometry, the value of $\nabla_X Y$ at some point $(x,v) \in TM$ depends only on the behaviour of $Y$ along some curve $c: I \to TM$ with $c(0) = (x,v)$ and $(\pi \circ c)'(0) = X(x,v)$, and we will sometime abuse notation by writing $\nabla_X Y$ if $Y$ is only defined along such a curve.

%The analogue of the Riemann curvature tensor is the \emph{first Chern curvature tensor} $R$, a $(1,3)$ tensor defined over $\mathscr{E}$ such for three sections $X$, $Y$, and $Z$ of $\mathscr{E}$, $R(X,Y) Z$ is the section defined by
%
%\begin{equation}
%    R(X,Y) Z = \nabla_X \nabla_Y Z - \nabla_Y \nabla_X Z - \nabla_{[X,Y]} Z.
%\end{equation}
%
%Given a section $X$ of $\mathscr{E}$, the \emph{flag curvature} $K(\bar{T},X)$ is defined by the formula
%
%\begin{equation}
%    K({\bar{T},X) = \frac{g_{\bar{T}}(R(X,\bar{T}) \bar{T}, X)}{g_{\bar{T}}(X,X)} - g_{\bar{T}}(X,\bar{T})^2},
%\end{equation}
%
%which generalizes the \emph{sectional curvature} from Riemannian geometry. By homogeneity, compactness, and continuity, on any compact Finsler manifold $M$, there exists constants $\delta$ and $\Delta$ such that for all sections $X$ of $\mathscr{E}$, $\delta \leq K(\bar{T},X) \leq \Delta$.

A Finsler metric gives a length to each tangent vector on the manifold, and can thus be used to define the lengths of curves $c: I \to U_0$ by the formula
%
\begin{equation}
  L(c) = \int_I F(c,\dot{c}),
\end{equation}
%
which is invariant under reparameterization. An analysis of length minimizing curves naturally leads to a theory of geodesics on a Finsler manifold, i.e. to a theory of critical points in the space of paths between two points. The theory of geodesics on Finsler manifolds is similar to the Riemannian case, except for the interesting quirk that a geodesic from a point $p$ to a point $q$ need not necessarily be a geodesic when considered as a curve from $q$ to $p$, and so we must consider \emph{forward} and \emph{backward} geodesics\footnote{If walking up a hill is more strenuous than walking down a hill, then the optimal path to climb from the bottom of a mountain to it's summit need not be the optimal path to descend from the peak to the bottom. Indeed, Makoto Matsumoto formulated such a problem by constructing a Finsler metric upon the surface of the mountain such that geodesics correspond to optimal paths of ascent and descent \cite{Matsumoto}. Similar Finsler manifolds can be constructed to model optimal paths taken sailing under the influence of varying wind conditions, the theory being known as Zermelo's navigation problem.}. We define the \emph{forward distance} $d_+: X \times X \to [0,\infty)$ by taking the infima of paths between points. This function is a \emph{quasi-metric}, as it satisfies the triangle inequality, but is not necessarily symmetric. We define the \emph{backward distance} $d_-: X \times X \to [0,\infty)$ by setting $d_-(p,q) = d_+(q,p)$. On a general Finsler manifold one has $d_+ \neq d_-$, and these functions are distinct quasi-metrics on $M$ (though a compactness argument shows both are proportional to one another on compact manifolds). Geodesics from a point $p_0$ to a point $p_1$ need not be geodesics from $p_1$ to $p_0$ when reversed. A metric can be obtained by setting $d_X = d_X^+ + d_X^-$, and we will use this definition as the canonical metric on a Finsler manifold $X$ in the rest of this thesis.

A standard approach to an analysis of geodesics in Riemannian geometry is to rely on the \emph{fundamental theorem of Riemannian geometry}, which posits the existence of a torsion-free linear connection on the tangent bundle of a Riemannian manifold $X$ compatible with the metric, and using the connection to derive a first variation formula. A similar approach works for Finsler geometry, aside from the fact that (a) the connection is not defined on the tangent bundle and (b) the analogue of the fundamental theorem in general \emph{fails}. Because of this, there is no canonical connection that is used in the Finsler geometry literature, and depending on the application and personal preference one of several can be used, the most popular being the Cartan, Berwald, and Chern connections, the first being metric-compatible, but which can have torsion, and the latter two being torsion free, but which are almost metric-compatible. We use the Chern connection in what follows.

To define the Chern connection, recall that for a vector bundle $B$ over a manifold $X$, a connection on $B$ is an association, with each $s \in \Gamma(B)$ and $v \in T_x X$, of a vector $\nabla_v(s) \in B_x$, which is linear in $v$ and $s$, and such that for each $X \in \Gamma(X)$ and a section $s \in \Gamma(B)$, $\nabla_X(s) \in \Gamma(B)$ is smooth. We will be studying connections on the pullback bundle $\pi^*(TX)$, viewed as a vector bundle over $TX$, where $\pi: TX \to X$ is the projection map. The bundle $\pi^*(TX)$ can be viewed as a subbundle of $T(TX)$. This bundle has a distinguished section $l \in \Gamma(\pi^*(TX))$, given in coordinates by $l(x,v)_i = v_i$. Define the Cartan tensor $A$, a symmetric 3-tensor given in coordinates by
%
\begin{equation}
  A_{ijk} = \frac{F}{2} \frac{\partial^3 F^2}{\partial v^i \partial v^j \partial v^k}. 
\end{equation}
%
Note that if $X$ is a Riemannian manifold, and $F(x,v) = \langle v,v \rangle$ is the Finsler metric induced by the Riemannian metric, then the Cartan tensor $A$ vanishes identically, but on a general Finsler manifold this need not be the case.

\begin{prop}
  If $X$ is a Finsler manifold, and consider the projection map $\pi: TX \to X$. Then there exists a unique linear connection $\nabla$ on the pullback bundle $\pi^*(TM)$, viewed as a bundle over $TM$, which is torsion-free, in the sense that for any $X,Y \in \Gamma(\pi^*(TM))$, identifying $X$ and $Y$ with vector fields on $TM$,
  %
  \[ \nabla_X Y - \nabla_Y X = [X,Y], \]
  %
  and almost metric compatible, in the sense that for $X,Y,Z \in \Gamma(\pi^*(TM))$,
  %
  \[ X(g(Y,Z)) = g(\nabla_X Y, Z) + g(Y, \nabla_X Z) + A(\nabla_X l, Y, Z). \]
\end{prop}
\begin{proof}
  We do not prove this proposition, but merely define the connection. Define the formal Christoffel symbols
  %
  \begin{equation}
    \gamma^l_{jk} = \sum\nolimits_i \frac{g^{li}}{2} \left( \frac{\partial g_{ij}}{\partial x^k} + \frac{\partial g_{ki}}{\partial x^j} - \frac{\partial g_{jk}}{\partial x^i} \right),
  \end{equation}
  %
  the geodesic spray coefficients and nonlinear connection coefficients
  %
  \begin{equation}
    G^i = \sum\nolimits_{j,k} \gamma^i_{jk} v^j v^k \quad\text{and}\quad N^i_j = \frac{1}{2} \frac{\partial G^i}{\partial v^j},
  \end{equation}
  %
  the horizontal vectors
  %
  \begin{equation}
    \frac{\delta}{\delta x^a} = \frac{\partial}{\partial x^a} - \sum\nolimits_b N^b_a \frac{\partial}{\partial v^b},
  \end{equation}
  %
  and the Chern connection coefficients
  %
  \begin{equation}
    \Gamma^i_{jk} = \sum\nolimits_s \frac{g^{is}}{2} \left( \frac{\delta g_{sj}}{\delta x^k} + \frac{\delta g_{ks}}{\delta x^j} - \frac{\delta g_{jk}}{\delta x^s} \right).
  \end{equation}
  %
  Finally, the connection is defined by setting
  %
  \begin{equation}
    \left(\nabla_{\partial_{x^j}} X \right)^i = \frac{\partial X^i}{\partial x^j} + \Gamma^i_{jk} X^k \quad\text{and}\quad \left( \nabla_{\partial_{v^j}} X \right)^i = \frac{\partial X^I}{\partial v^j}.
  \end{equation}
  %
  See Theorem 2.4.1 of \cite{BaoChern} for details of the proof that this connection is torsion free and almost metric compatible, as well as the uniqueness of the connection.
\end{proof}

Now we derive a first variation formula for the energy of a curve.

\begin{prop}
  Consider a variation $c: (-\varepsilon,\varepsilon) \times I \to X$ on a Finsler manifold, not necessarily fixed at the endpoints. Define $T = \partial_t c$ be the tangent field, and $U = \partial_s c$ the variation field. Define
  %
  \begin{align*}
    E(s) &= \frac{1}{2} \int_I F(c,\dot{c})^2 = \frac{1}{2} \int_I g_T( T, T ).
  \end{align*}
  %
  Then
  %
  \[ E'(s) = g_T(U,T)|_{\partial I} - \int_I g_T(U,\nabla_T T). \]
\end{prop}
\begin{proof}
Almost metric compatibility gives that
%
\begin{equation}
  \partial_s \{ g_T(T,T) \} = U \{ g_T(T,T) \} = 2 g_T(\nabla_U T, T),
\end{equation}
%
and since $[U,T] = 0$, being torsion free and almost metric compatibility implies that
%
\begin{equation}
  g_T( \nabla_U T, T ) = g_T( \nabla_T U, T ) = \partial_t \{ g_T(U,T) \} - g_T(U, \nabla_T T).
\end{equation}
%
The Cartan tensor does not appear here, since $A_T(X,Y,Z) = 0$ if $T \in \{X,Y,Z\}$, by the Euler homogeneous function theorem. Differentiating under the integral sign thus gives that
%
\begin{equation}
  E'(s) = \frac{1}{2} \int_I \partial_s \left\{ g_T(T,T) \right\} = g_T(U,T)|_{\partial I} - \int_I g_T(U, \nabla_T T),
\end{equation}
%
which finishes the proof.
\end{proof}

A differential formula for geodesics immediately follows.

\begin{corollary}
  A constant speed curve $c: I \to X$ is a geodesic (i.e. a critical point of any length variation fixing endpoints) if and only if it satisfies the differential equation
  %
  \[ \ddot{c}^a = - \sum\nolimits_{j,k} \gamma^a_{jk}(c,\dot{c}) \dot{c}^j \dot{c}^k \quad\text{for all $1 \leq a \leq d$}. \]
  %
  Alternatively, if we set $(x,\xi) = \mathcal{L}(c,\dot{c})$, then $c$ is a geodesic if and only if it satisfies Hamilton's equations for motion
  %
  \[ \dot{x} = - (\partial_\xi F_*)(x,\xi) \quad\text{and}\quad \dot{\xi} = (\partial_x F_*)(x,\xi). \]

\end{corollary}
\begin{proof}
  The proof of the first equation follows by expanding the equation $\nabla_T T = 0$, which is necessary by the first variation formula for a curve to be a geodesic. To obtain the second, we apply the theory of Euler-Lagrange equations and Hamiltonian mechanics, as described in Sections 14 and 15 of \cite{Arnold}. The Euler-Lagrange equation for the energy functional
  %
  \begin{equation}
    E(c) = \frac{1}{2} \int_I F(c,\dot{c})^2
  \end{equation}
  %
  is equal to
  %
  \begin{equation}
    \frac{1}{2} \frac{\partial F^2}{\partial x^i}(c,\dot{c}) = \frac{1}{2} \frac{\partial^2 F^2}{\partial v^i \partial x^j}(c,\dot{c}) \dot{c}^j + g_{ij}(c,\dot{c}) \ddot{c}^j
  \end{equation}
  %
  and applying the Legendre transform to get Hamilton's equations of motions, we obtain the second pair of equations.
\end{proof}

\begin{remark}
  The Hamiltonian equations derived in this corollary will reoccur in our study of wave propagation, and will tell us that high frequency wave packet solutions to the equation $\partial_t u = i P$ travel along geodesics of the Finsler metric.
\end{remark}

Just as in the Riemannian case, one can also obtain a second variation formula involving a theory of Jacobi fields along geodesics, but this takes us too far afield. Since we will not use the second variation formula in what follows, we simply assume one of it's consequences, that geodesics are length minimizing up until the development of cut points on a manifold, which is at least as big as the \emph{injectivity radius} of the manifold, i.e. the minimum time at which the geodesic flow defined by the Hamiltonian equations above fails to be injective from a point on the manifold.

%The equations \eqref{FStarHamilton} are first order ordinary differential equations induced by the Hamiltonian vector field $( \partial_\xi F_*, - \partial_x F_* )$ on $T^* U_0$. Note that in our situation, the dual norm $F_*$ is the principal symbol $p$ of the operator $P$ we are studying, and so \eqref{FStarHamilton} is precisely the Hamiltonian flow alluded to in \eqref{qoiJIOJdoiwajfoiawjfoi}. Sufficiently short geodesics on a Finsler manifold are length minimizing. In particular, there exists $r > 0$ such that if $c$ is a geodesic between two points $x_0$ and $x_1$, and $L(c) < r$, then $d_M^+(x_0,x_1) = L(c)$.

%The analogue of the Riemann curvature tensor on a Finsler manifold is the \emph{$hh$-curvature tensor} $R$, a $(1,3)$ tensor defined on $E$ such that for three sections $X,Y,Z$ of $E$, $R(X,Y) Z$ is a section of $E$, which we define in the appendix. The analogue of sectional curvature in Finsler geometry is the \emph{flag curvature}. To define the curvature, let $T$ be the section of $E = \pi^*(TM)$ given by $T(x,v) = F(v)^{-1} v$, %and let $\omega$ be the section of $E^* = \pi^*(T^* M)$ given in coordinates by $\omega(x,v) = \sum (\partial F / \partial v^j)(x,v) dx^j$.
%This section is often called the \emph{distinguished section} in the literature. Given a section $X$ of $E$, we define the flag curvature $K(T,X)$ to be the function on $TM$ given by
%
%\[ K(T,X) = \frac{g(R(X,T) T, X)}{g(X,X) - g(X,T)^2} \]
%
%where $R$ is the \emph{$hh$-Curvature tensor}, a $(1,3)$ tensor defined on $E$ such that for three sections $X,Y,Z$ of $E$, $R(X,Y) Z$ is the section of $E$ given in coordinates by
%
%\[ R(X,Y) Z = \sum\nolimits_{i,j,k,l} Z^j X^k Y^l \left( \frac{\delta \Gamma^i_{jl}}{\delta x^k} - \frac{\delta \Gamma^i_{jk}}{\delta x^L} + \Gamma^i_{hk} \Gamma^h_{jl} - \Gamma^i_{hl} \Gamma^h_{jk} \right) \frac{\partial}{\partial x^i} \]
%
%

%To conclude, we see that any elliptic operator $P$ on a manifold $M$ gives rise to a Finsler metric on $M$ which reflects the behaviour of the principal symbol of the operator. We will see this metric arises in the study of the wave equation $\partial_t u = 2 \pi i P u$ on $M$. In particular, high-frequency wave packet solutions to the wave equation travel along \emph{geodesics} in $M$, and so an analysis of the Finsler geometry will be necessary to understand the interactions of different wave packets, which will give us an analogue of Lemma \ref{lemma4} on compact manifolds for \emph{small time interactions} between the wave packets when Assumption A holds. We will control the large time behaviour for the wave equation via a reduction to the local smoothing inequality, which is sufficient to obtain an analogue of the results \cite{HeoandNazarovandSeeger} on manifolds, when Assumption B is true.

\section{Fourier Integral Operator Techniques} \label{sec:FourierIntegral}

For any (possibly unbounded) self-adjoint operator $P$ on a Hilbert space $H$, we can use the spectral theorem for such operators to define a bounded operator $a(P)$ on $H$ for each bounded function $a: \sigma(P) \to \CC$. If $a: \RR \to \CC$ is in the Wiener algebra $A(\RR)$, then the Fourier inversion formula
%
\begin{equation}
  a(P) = \int_{-\infty}^\infty \widehat{a}(t) e^{2 \pi i t P}\; dt
\end{equation}
%
holds, in a certain sense made precise in the lemma below.

\begin{lemma}
  Suppose the Fourier transform of a bounded, continuous function $a: \RR \to \CC$ is integrable. Let $H$ be a Hilbert space, and let $P$ be a self-adjoint operator on $H$, so we may use spectral theory to define $a(P)$. Then for any $x,y \in H$,
  %
  \[ \langle a(P) x, y \rangle = \int_{-\infty}^\infty \widehat{a}(t) \langle e^{2 \pi i t P} x, y \rangle\; dt, \]
  %
  where the right hand side converges absolutely because $e^{2 \pi i t P}$ is a unitary operator.
\end{lemma}
\begin{proof}
  We recall that the operators $a(P)$ and $e^{2 \pi i t P}$ are defined in terms of a spectral resolution\footnote{See Theorem 13.33 of \cite{RudinFunc}} $E$ on $\RR$ such that for any $x,y \in H$,
  %
  \begin{equation}
    \langle Px, y \rangle = \int_{\sigma(P)} \lambda dE_{x,y}(\lambda).
  \end{equation}
  %
  We then define a bounded operator $a(P)$ such that for $x,y \in H$,
  %
  \begin{equation}
    \langle a(P) x, y \rangle = \int_{\sigma(P)} a(\lambda) dE_{x,y}(\lambda).
  \end{equation}
  %
  We may apply the Fourier inversion formula to write
  %
  \begin{equation}
    \int_{\sigma(P)} a(\lambda) dE_{x,y}(\lambda) = \int_{\sigma(P)} \left( \int_{-\infty}^\infty \widehat{a}(t) e^{2 \pi i t \lambda}\; dt \right) dE_{x,y}(\lambda).
  \end{equation}
%
Provided that $\widehat{a} \in L^1(\RR)$, the double integral is absolutely integrable, and so we may apply Fubini's theorem to interchange the order of integration, which gives the required result.
\end{proof}

Thus spectral multipliers of a self-adjoint operator $P$ can be written as averages of the semigroup $\{ e^{2 \pi i t P} \}$ under very weak assumptions on the operator. The catch is that for a general operator $P$, it is difficult to understand the behavior of the operator $e^{i t P}$ given it's abstract definition via the functional calculus. Nonetheless, in this chapter we will see that if $P \in \COP{1}(X)$, then $e^{i t P}$ is a \emph{Fourier integral operator}, which roughly means that the kernel of the operator is well approximated by oscillatory integral representation \emph{for high frequency inputs}, which is something we can exploit to obtain both geometric and frequential information about the operator. In this section, we briefly describe the relevant components we will need from the theory of Fourier integral operators for the proofs in the latter part of the thesis.

We will require methods of decomposition of functions in space and frequency simultaneously (a decomposition in \emph{phase space}). Temporarily define a \emph{micro-localization operator at a point $(x_0,\xi_0) \in T^* \RR^d$} to be an operator of the form
%
\begin{equation}
  (\Psi_{x_0,\xi_0} f)(x) = \int \phi( \xi ) e^{i \xi \cdot (x - y)} \psi(y) f(y)\; dy\; d\xi,
\end{equation}
%
where $\phi,\psi$ are smooth functions, with $\phi$ homogeneous of order zero, equal to one in a neighborhood of $\xi_0$, and $\psi$ with compact support equal to one in a neighborhood of $x_0$. Such an operator is precisely a composition of a Fourier multiplier that localized it's input in a particular frequency direction, with an operator that takes a smooth cutoff of a function. We will say such a multiplier is \emph{microlocally supported} on $\text{supp}(\psi) \times \text{supp}(\phi)$.

To motivate the definition of Fourier integral operators, consider an operator $A$ from $\RR^m$ to $\RR^n$, whose kernel is a distribution given by an oscillatory integral of the form
%
\begin{equation}
  A(x,y) = \int_{\RR^p} s(x,y,\theta) e^{i \Phi(x,y,\theta)}\; d\theta,
\end{equation}
%
where $s$ is a symbol of some fixed order, and $\Phi$ is smooth, and homogeneous of order one in the $\theta$ variable. Define the \emph{canonical relation} of $A$, the set
%
\begin{equation}
  \mathcal{C}_\Phi = \Big\{ \big(x, \nabla_x \Phi(x,y,\theta) ,y, -\nabla_y \Phi(x,y,\theta) \big) : \nabla_\theta \Phi(x,y,\theta) = 0 \Big\},
\end{equation}
%
Under the assumption that the vectors $\partial_{\theta_1} \{ \nabla_{x,y,\theta} \Phi \}, \dots \partial_{\theta_N} \{ \nabla_{x,y,\theta} \Phi \}$ are everywhere linearly independent when $\nabla_\theta \Phi = 0$, $\mathcal{C}_\Phi$ is a smooth, $n + m$ dimensional manifold\footnote{It is actually a \emph{Lagrangian submanifold} of $T^* \RR^n \times T^* \RR^m$. Thus the canonical relation of any Fourier integral operator must be a Lagrangian submanifold, though we will only use this fact implicitly}. Such a phase $\Phi$ is called \emph{nondegenerate}, and we will only deal with such phases in the sequel. If $(x_0,\xi_0,y_0,\eta_0) \not \in \mathcal{C}_\Phi$, then for micro-localization operators $\Psi_{x_0,\xi_0}$ and $\Psi_{y_0,\eta_0}$ of suitably small support, integration by parts justifies that the Fourier transform of $\Psi_{x_0,\xi_0} \circ A \circ \Psi_{y_0,\eta_0}$ is rapidly decaying, and thus the kernel of the operator $\Psi_{x_0,\xi_0} \circ A \circ \Psi_{y_0,\eta_0}$ is smooth. The canonical relation of an oscillatory integral operator thus tells us pairs of points in phase space the location where high-frequency wave packets are mapped in phase space. The main lesson of the theory of Fourier integral operators is that, to a large extent, the canonical relation \emph{determines} the behavior of the operator $A$, rather than other features of the phase $\Phi$.


%
% R^{mu + p} int chi(x,y,theta) e^{iR[ Phi(x,y,theta) + xi * x - eta * y ]}
% Nabla_theta Phi(x,y,theta) = 0
% x = y and theta = -xi = -eta
% Integration in theta gives chi^(x,y, R(x - y)), and integration in y for
% each fixed x yields a R^{-d} factor, which gives a R^mu factor.
%
% 
% 
% 
% 
%
%
% Lambda = { (H'(xi), xi) }
% int chi(x,y,theta) e^{iR[ (x - y) * theta + xi * x - y * eta ]}
%
%
%
% Consider the submanifold of T^*R^2 defined by x = y and xi = eta
% Fix (0,0,1,0)
% The Lagrangian subspace tangent to the manifold is span( (1,1,0,0), (0,0,1,1) )
% Can we find a Lagrangian subspace transverse to this space and also span((0,0,1,0),(0,0,0,1))
% The graph of a linear map from R^2 -> (R^2)^*, i.e. span( (1,0,a,b), (0,1,b,c) )
%
% The intersection of this graph with the Lagrangian subspace above is trivial
% if a != c
% So we can take a = 1, b = c = d = 0
% span( (1,0,1,1), (0,1,0,0) )
%
%
% Define x = z + z^2/2
% and w = y
%
% Then dx = (1 + z) dz
% and dw = dy
%
% So x = y becomes (z + z^2/2) = w
% and dx = dy becomes (1 + z) dz = dw
% so z = A - 1 where A = dw / dz
% and so w = (A - 1) + (A - 1)^2 / 2 = A^2/2 - 1/2
%          = ((dw / dz)^2 - 1)/2
%
% dH = [(a/b) - 1] da + [(1/2)(a/b)^2 - 1/2] db
% H(a,b) = a^2/2b - a - b/2

% This is a section of T^* R^2, and hence can be written as df(x,y) = x (dx + dy)
% i.e. f(x,y) = x^2/2 + x
% for f(x,y) = x^2/2
% 
% Define z = x + x^2/2
% x = sqrt( 1 + 2z ) - 1
% and w = y
% Then dz = (1 + x) dx and dw = dy
% So dx = dy becomes dz = sqrt(1 + 2z) dw
% and x = y becomes 2z = w^2 + 2w
% So the manifold becomes { (z,w;zeta,omega) : zeta = sqrt(1 + 2z) omega, 2z = w^2 + 2w }
%    which is equal to  { (z,w;zeta,omega) : zeta = (w + 1) omega, 2z = w^2 + 2w }
% Near ( 0,0,1,0 ),
% w = zeta / omega - 1
% z = (zeta/omega)^2 / 2 + (zeta/omega)
% dH = (a/b - 1) da + [(a/b)^2 / 2 + (a/b)] db
% a^2/2b - a + cb

% So we can take Q(x,y) = x^2/2
%
% If we now define z = x + x^2/2
% and w = y
%
% Then x = sqrt( 1 + 2z ) - 1
% so x = y becomes z = w^2/2 + w 
% dz = (1 + x) dx and dw = dy
% dx = dy becomes dz = (1 + w) dw
% { z = w + w^2/2, dz = (1 + w) dw }
% 





% Given (0,0;xi,xi) -> (0,0;e_1,0)
% If x = w + (aw^2 + bwz + cz^2),
% then xi = omega + 2aw omega + bz omega + bw zeta + 2cz zeta
% C = { z = 0,  }

%To define Fourier integral operators, we must start by introducing some basic symplectic geometry. A \emph{symplectic manifold} is a manifold $Z$ whose tangent space $T Z$ is equipped with a smoothly varying non-degenerate alternating bilinear form $\omega$. For any manifold $X$, $T^*X$ is naturally a symplectic manifold; this is because for each $(x,\xi) \in T^* X$, $T_{(x,\xi)} T^* X$ can be naturally identified with $T_x X \times T^*_x X$, and we define $\omega = \sum d\xi_i \wedge dx^i$. A \emph{Lagrangian submanifold} of a symplectic manifold $Z$ is a submanifold $W$ of $Z$ such that, with respect to $\omega$, for each $z \in Z$, $(T_z W)^\perp = T_z W$.

%Lagrangian manifolds naturally occur when using stationary phase to analyze oscillatory integrals. Let $s$ be a symbol, and consider a kernel $K$ on $\RR^m \times \RR^n$ defined by an oscillatory integral of the form
%
%\[ I(x,y) = \int_{\RR^N} s(x,y,\theta) e^{i \Phi( x,y,\theta )}\; d\theta, \]
%
%where $\Phi$ is homogeneous in $\theta$ of order one. The principle of stationary phase tells us that the high-frequency behaviour of $I$ is determined by the stationary points of the integral $(x,\theta)$ where $\nabla_\theta \Phi(x,\theta) = 0$, and each such stationary point contributes oscillatory behaviour to $I$ near $x$ at a frequency $\nabla_x \Phi(x,\theta)$. Thus we should expect that if $\psi \in C_c^\infty(\RR^d)$ has suitably small enough support around a point $x_0$, then rapid decay estimates of the form $|\widehat{\psi I}(R \xi_0)| \lesssim_M R^{-M}$ should hold for all $M > 0$ unless $(x_0,\xi_0)$ lies in the set
%
%\[ \Lambda_\Phi = \{ (x,\xi): \nabla_\theta \Phi(x,\theta) = 0\ \text{and}\ \nabla_x \Phi(x,\theta) = \xi\ \text{for some $\theta$} \}, \]
%
%which is a Lagrangian submanifold of $T^* \RR^d$. This manifold is `where the high frequency parts of $I$ live'.

A Fourier integral operator is precisely an operator whose kernel is microlocally expressible in oscillatory integrals of the form above. A \emph{Fourier integral operator} of order $\mu$ from $\RR^m$ to $\RR^n$ associated with a $n + m$ dimensional `Lagrangian' submanifold $\mathcal{C}$ of $T^* \RR^n \times T^* \RR^m$ is an operator $A$ such that for each $(x_0,\xi_0) \in T^* \RR^n$ and $(y_0,\eta_0) \in T^* \RR^m$, there exists micro-localization operators $\Psi_{x_0,\xi_0}$ and $\Psi_{y_0,\eta_0}$ at $(x_0,\xi_0)$ and at $(y_0,\eta_0)$ such that the operator $\Psi_{x_0,\xi_0} \circ A \circ \Psi_{y_0,\eta_0}$ is expressed as an oscillatory integral operator of the kind studied in the previous paragraph, with an amplitude given by a symbol of order $\nu + p/2 - (n+m)/4$.

A Fourier integral operator from an $m$-dimensional manifold $Y$ to an $n$-dimensional manifold $X$ associated with an $n + m$ dimensional Lagrangian submanifold $\mathcal{C}$ of $T^* X \times T^* Y$ is precisely an operator which, when localized in any pair of coordinate systems for $X$ and $Y$, is a Fourier integral operator of the form above. The manifold $\mathcal{C}$ is truly a geometric invariant of the operator $A$, because it is in direct correspondence to the \emph{wavefront set} of the operator $A$, which gives the location and direction of singularities of the kernel of the operator. The \emph{equivalence of phase theorem} for Fourier integral operators tells us that the particular phase used to define Fourier integral operators is largely irrelevant to the analysis of the operator, as long as the canonical relation is shared by the phase.

\begin{theorem} \label{thm:equivalenceofphase}
  Let $A$ be a Fourier integral operator of order $\nu$ between two manifolds $Y^m$ and $X^n$, associated with the canonical relation $\mathcal{C}$. Localize $A$ around $x_0 \in X$ and $y_0 \in Y$, so we may consider the operator as a Fourier integral operator from $\RR^m$ to $\RR^n$. Fix $(x_0,\xi_0,y_0,\eta_0) \in \mathcal{C}$, and consider two microlocalization operators $\Psi_{x_0,\xi_0}$ and $\Psi_{y_0,\eta_0}$ microlocally supported on two sets $\Gamma_{x_0,\xi_0} \subset T^* \RR^n$ and $\Gamma_{y_0,\eta_0} \subset T^* \RR^m$. Consider $p \geq 0$, and a non-degenerate phase $\Phi: \Gamma_{x_0,\xi_0} \times \Gamma_{y_0,\eta_0} \times \RR^p$ with $\mathcal{C}_\Phi = \mathcal{C} \cap (\Gamma_{x_0,\xi_0} \times \Gamma_{y_0,\eta_0})$. Then there exists a symbol $s$ of order $\nu + p/2 - (n+m)/4$, such that
  %
  \[ \Psi_{x_0,\xi_0} \circ A \circ \Psi_{y_0,\eta_0} = \int s(\cdot,\cdot,\theta) e^{i \Phi(x,y,\theta)}\; d\theta + C^\infty(\RR^n \times \RR^m), \]
  %
  i.e. the kernel of $\Psi_{x_0,\xi_0} \circ A \circ \Psi_{y_0,\eta_0}$ differs from the oscillatory integral operator on the right hand side by a smoothing operator.
\end{theorem}
\begin{proof}
  See Theorem 25.1.5 of \cite{Hormander4}.
\end{proof}

The simplest class of Fourier integral operators are the \emph{pseudo-differential operators}. They are precisely the Fourier integral operators from a manifold $X$ to itself, whose canonical relation is the diagonal $\Delta_X = \{ (x,\xi,x,\xi): (x,\xi) \in T^* X \}$. In any coordinate system, and for any diffeomorphism $\phi: \RR^d \to \RR^d$, the phase function $\Phi(x,y,\xi) = \xi \cdot (\phi(x) - \phi(y))$ parameterizes $\Delta_X$, and so since $p = d$, $n = d$, and $m = d$, any pseudo-differential operator of order $\nu$ on $\RR^d$ can be written, modulo a smooth operator, as an oscillatory integral operator of the form
%
\begin{equation}
  \int s(x,y,\xi) e^{i \xi \cdot (\phi(x) - \phi(y))}\; d\xi,
\end{equation}
%
where $s$ is a symbol of order $\nu$.

For our purposes, the most important Fourier integral operators are associated with solutions to wave equations on manifolds.

\begin{theorem} \label{waveisanFIOTheorem}
  Let $P$ be an elliptic operator on a manifold $X$ satisfying Assumption $A$, with principal symbol $p: T^* X \to \RR$. Let $\{ \alpha_t \}$ be the (co) geodesic flow on $T^* X$ given by the Finsler metric on $X$ induced by $p$.  Define the canonical relation $\mathcal{C} \subset T^* (X \times \RR) \times T^* X$ by setting
  %
  \[ \mathcal{C} = \{ (x,t,\xi,\tau,x',\xi') : (x,\xi) = \alpha_t(x',\xi') \quad\text{and}\quad \tau = p(x',\xi') \}. \]
  %
  Then the solution operator $Af(x,t) = e^{i t P}f(x)$ from $X$ to $X \times \RR$, which takes an initial condition $f$ on $X$ to a solution $u$ to the wave equation $\partial_t u = i P u$ is a Fourier integral operator of order $1/4$ associated with the canonical relation $\mathcal{C}$. For each fixed $t$, the operator $e^{i t P}$ is a Fourier integral operator of order $0$ associated with the canonical relation
  %
  \[ \mathcal{C}_t = \{ (x,\xi;x',\xi') : (x,\xi) = \alpha_t(x',\xi') \}. \]
\end{theorem}
\begin{proof}
  It suffices to construct a parametrix for the half-wave equation given by oscillatory integrals that fit into the canonical relations above. One can find such parametrices in many sources, such as Theorem 29.1.1 of \cite{Hormander4} or Theorem 4.1.2 of \cite{Sogge}.
\end{proof}

Given this theorem, the equivalence of phase theorem implies multiple useful oscillatory integral representations for the wave propagators $e^{i t P}$.
\begin{itemize}
  \item Consider a smooth hypersurface $\Sigma$ of $T^* X \times X$, such that for each $(x,\xi) \in T^* X$, the slice $\Sigma_{x,\xi} = \{ x' : (x,\xi,x') \in \Sigma \}$ is a smooth hypersurface in $X$ passing through $(x,\xi)$ co-normal to $\xi$. The theory of Hamilton-Jacobi equations implies the local existence of a homogeneous function $\phi$ satisfying the \emph{Eikonal equation}
  %
  \begin{equation} \label{awiodjawoidhjioq23412341234234}
    p(x, \nabla_x \phi(x,x',\xi)) = p(x',\xi)
  \end{equation}
  %
  and such that for $x' \in \Sigma_{x,\xi}$, $\phi(x,x',\xi) = 0$ and $\nabla \phi(x,x',\xi)$ is conormal to $\Sigma_{x,\xi}$. If we define $\Phi(x,t,x',\xi) = \phi(x,x',\xi) + t p(x,\xi)$, then $\mathcal{C}_\Phi$ agrees with the canonical relation $\mathcal{C}$ for the solution operator $A$, and so there exists a symbol $s$ of order zero such that, in coordinates, for $x'$ in a neighborhood of $x$, and suitably small $t$, and modulo a smooth kernel,
  %
  \begin{equation} A(x,t,x') = \int s(x,t,x',\xi) e^{i [ \phi(x,x',\xi) + t p(x,\xi) ]}\; d \xi. \end{equation}

  \item Suppose that $t > 0$ is smaller than the injectivity radius on the manifold $X$ with respect to it's Finsler metric. Then the phase 
  %
  \begin{equation} \label{onedwavephase}
    \Phi(x,x',t,\tau) = \tau ( |t| - d_+(x',x) )
  \end{equation}
  %
  for $\tau > 0$, has canonical relation agreeing with $\mathcal{C}_t$, and so there exists a symbol $s$ of order $\tfrac{d-1}{2}$ such that, modulo a smooth kernel,
  %
  \begin{equation}
    A(x,t,x') = \int_0^\infty s(x,t,x',\tau) e^{i \tau ( |t| - d_+(x',x) )}\; d\tau.
  \end{equation}
  %
  The problem with this oscillatory integral representation is that it begins to break down as $t \to 0$, and as $t$ approaches times at which cut points develop on $X$.

  \item Because of the semigroup property $\mathcal{C}_t \circ \mathcal{C}_s = \mathcal{C}_{t+s}$, we can obtain oscillatory integral representations of $e^{i t P}$ by composing the oscillatory integral representations here. We do not carry out the details, because as $t \to \infty$ the number of compositions required grows linearly, which means the oscillatory integrals are harder and harder to control in a uniform way as $t \to \infty$, and we will not use such compositions in our arguments.
  %For instance, we can write
  %
  %\begin{align*}
  %  e^{2 \pi i t P}(x,y) &= (e^{2 \pi i (t/2) P} \circ e^{2 \pi i (t/2) P})(x,y)\\
  %  &= \int s(x,t/2,w,\tau_1) s(w,t/2,y,\tau_2) e^{i \tau_1 ( d(x,w) - t/2 ) + \tau_2 ( d(w,y) - t/2 )}\; dw\; d\tau_1\; d\tau_2\\
  %  &= \int \tilde{s}(x,t,y,\tau_1,\tau_2) e^{i \tau_1(  )}
  %\end{align*}
  % tau_1 d(x,w(tau_1,tau_2)) + tau_2 d(w(tau_1,tau_2),y)
  %TODO Fix this.
  %\[ \int e^{i \tau_1 d(x,w) + \tau_2 d(w,y)}\; dw \]
\end{itemize}
%
We conclude this section with some technical calculations, that result in some useful estimates for certain Fourier integral operators that will arise later in the thesis. Our first calculation concerns `Littlewood-Paley projections' on a manifold.

\begin{theorem} \label{PseudoOsicllatoryLemma}
  Let $X$ be a compact manifold, and consider any $P \in \COP{1}(X)$. Consider a coordinate system $V$, with a set $U$ compactly contained in $V$. Fix $q \in C_c^\infty(0,\infty)$ and $R > 0$, and define $Q = q(P/R)$. Then we can find an operator $Q_\alpha$ such that for any function $u$ supported on $U$, $\| Qu - Q_\alpha u \|_{C^M(X)} \lesssim_{N,M} R^{-N} \| u \|_{L^1(X)}$ for arbitrarily large $M$ and $N$, such that $Q_\alpha u$ is supported on $V$ for any input $u$, and in the coordinate system $U$, $Q_\alpha$ is a pseudo-differential operator whose symbol $\sigma_\alpha$ is supported on $\{ (x,\xi): R/4 \leq \xi \leq 4R \}$.
\end{theorem}
\begin{proof}
  We proceed with a similar approach to Theorem 4.3.1 of \cite{Sogge}. It will be helpful to introduce a basis $\{ e_k \}$ for $L^2(X)$, with $e_k \in C^\infty(X)$ for each $k$ and $Pe_k = \lambda_k e_k$. Using \ref{waveisanFIOTheorem}, there exists $\varepsilon > 0$ such that for any function $u$ supported on $U$, for $|t| < \varepsilon$ and $u$ supported on $U$, we can write
  %
  \[ e^{2 \pi i t P} u = W_\alpha(t) u + R_\alpha(t) u, \]
  %
  where the kernel of $R_\alpha$ is smooth, and in coordinates, we can write
  %
  \[ W_\alpha(t,x,y) = \int s_0(t,x,y,\xi) e^{2 \pi i [ \phi(x,y,\xi) + t p(y,\xi) ]} \]
  %
  for an order zero symbol $s_0$ with $(x,y)$ supported on $V_\alpha \times U_\alpha$. We fix $\rho \in C_c^\infty(\RR)$ equal to one in a neighborhood of the origin and with $\rho(t) = 0$ for $|t| \geq \varepsilon / 2$. Using the Fourier inversion formula, we write
    %
    \begin{equation}
    \begin{split}
        Qu &= \int R\;\! \widehat{q}(R t) e^{2 \pi i t P}u\; dt\\
        &= \int R\;\! \widehat{q}(Rt) \Big\{ \rho(t) \tilde{W}_\alpha(t) + \rho(t) \tilde{R}(t) + (1 - \rho(t)) e^{2 \pi i t P} \Big\} u\; dt\\
        &= Q_I u + Q_{II} u + Q_{III} u.
    \end{split}
    \end{equation}
    %
    The rapid decay of $\widehat{q}$ implies that the function $\psi(t) = R \widehat{q}(Rt) (1 - \rho(t))$ satisfies bounds of the form $\| \partial_t^N \psi \|_{L^1(\RR)} \lesssim_X R^{-M}$, and so
    %
    \begin{equation} \label{psidecaybound}
        |\widehat{\psi}(\lambda)| \lesssim_{N,M} R^{-N} \lambda^{-M}.
    \end{equation}
    %
    Since $Q_{III} = \widehat{\psi}(-P)$, we can write the kernel of $Q_{III}$ as
    %
    \begin{equation}
        Q_{III}(x,y) = \sum\nolimits_\lambda \widehat{\psi}(-\lambda_k) e_k(x) \overline{e_k(y)},
    \end{equation}
    %
    Now
    %
    \begin{equation}
    \begin{split}
      P^L Q_{III}u &= \sum\nolimits_\lambda \widehat{\psi}(-\lambda_k) (P^L e_k)(x) \langle e_k, u \rangle\\
      &= \sum\nolimits_\lambda \lambda_k^L \widehat{\psi}(-\lambda_k) e_k(x) \langle e_k, u \rangle.
      \end{split}
      \end{equation}
    %
    The bounds for $\widehat{\psi}$, and the bound
    %
    \begin{equation}
      |\langle e_k, u \rangle| \leq \| e_k \|_{L^\infty(X)} \| u \|_{L^1(X)} \leq \lambda_k^{d/2 + 1} \| u \|_{L^1(X)}
    \end{equation}
    %
    imply that
    %
    \begin{equation} \label{QThreeBound}
        \| Q_{III} u \|_{C^M(X)} \lesssim_{N,M} R^{-N} \| u \|_{L^1(X)}.
    \end{equation}
    %
    Since $q$ vanishes near the origin, integration by parts yields that
    % xi^{-N} q(. / R)
    \begin{equation}
        \left| \int R \widehat{q}(Rt) \rho(t) \partial_x^\alpha \partial_y^\beta \tilde{R}(t,x,y) \right| \lesssim_N R^{-N},
    \end{equation}
    %
    and thus
    %
    \begin{equation} \label{QTwoBound}
        \| Q_{II} u \|_{C^M(X)} \lesssim_{N,M} R^{-N} \| u \|_{L^1(X)}.
    \end{equation}
    %
    Now we expand
    %
    \begin{equation}
        Q_I = \iint R \widehat{q}(Rt) \rho(t) s_0(t,x,y,\xi) e^{2 \pi i [ \phi(x,y,\xi) + t p(y,\xi) ]}\; d\xi\; dt.
    \end{equation}
    %
    We perform a Fourier series expansion, writing
    %
    \begin{equation} c_n(x,y,\xi) = \int_{-\pi}^\pi \rho(t) s_0(t,x,y,\xi) e^{-2 \pi i n t}\; dt. \end{equation}
    %
    Then the symbol estimates for $s_0$, and the compact support of $\rho$ imply that
    %
    \begin{equation} |\partial_{x,y}^\alpha \partial_\xi^\beta c_n(x,y,\xi)| \lesssim_{\alpha,\beta,N} |n|^{-N} \langle \xi \rangle^{-\beta}. \end{equation}
    %
    Using Fourier inversion we can write
    %
    \begin{equation}
    \begin{split}
        Q_I(x,y) &= \iint \sum\nolimits_n R \widehat{q}(Rt) c_n(x,y,\xi) e^{2 \pi i [ \phi(x,y,\xi) + t [ n + p(y,\xi) ] ]}\; d\xi\; dt\\
        &= \int \sum\nolimits_n q \Big( \big( n + p(y,\xi) \big) / R \Big) c_n(x,y,\xi) e^{2 \pi i \phi(x,y,\xi)}\; d\xi\\
        &= \int \tilde{\sigma}_\alpha(x,y,\xi) e^{2 \pi i \phi(x,y,\xi)}\; d\xi, 
    \end{split}
    \end{equation}
    %
    where
    %
    \begin{equation} \tilde{\sigma}_\alpha(x,y,\xi) = \sum_{n \in \ZZ} q \left( \frac{n + p(y,\xi)}{R} \right) c_n(x,y,\xi). \end{equation}
    %
    The $n$th term of this sum is supported on $R/4 - n \leq p(y,\xi) \leq 4R - n$, so in particular, if $n > 4R$ then the term vanishes. For $-\infty < n \leq 4R$, we have estimates of the form
    %
    % Good range is R/4 - n >= R/8 and 4R - n <= 8R
    % so n <= R/8 and n >= -4R
    %
    \begin{equation} \left| \partial_{x,y}^\alpha \partial_\xi^\beta \left\{ q \left( \frac{n + p(y,\xi)}{R} \right) c_n(x,y,\xi) \right\} \right| \lesssim_{\alpha,\beta,N} |n|^{-N} \langle \xi \rangle^{-\beta}. \end{equation}
    %
    For $n \geq R/8$ and $n \leq -4R$, $|\xi| \sim n$ on the support of $q( (n + p(y,\xi)) / R )$, and so the rapid decay of $n$ implies that 
    %
    \begin{equation} \left| \partial_{x,y}^\alpha \partial_\xi^\beta \left\{ q \left( \frac{n + p(y,\xi)}{R} \right) c_n(x,y,\xi) \right\} \right| \lesssim_{\alpha,\beta,N,M} |n|^{-N} \langle \xi \rangle^{-M}. \end{equation}
    %
    This means that if we define
    %
    \begin{equation} \sigma_\alpha(x,y,\xi) = \sum\nolimits_{-4R \leq n \leq R/8} q \left( \frac{n + p(y,\xi)}{R} \right) c_n(x,y,\xi). \end{equation}
    %
    and define
    %
    \begin{equation} Q_\alpha(x,y) = \int \sigma_\alpha(x,y,\xi) e^{2 \pi i \phi(x,y,\xi)}\; d\xi \end{equation}
    %
    then
    %
    \begin{equation} \Big|\partial_{x,y}^\alpha \partial_\xi^\beta \big\{ \tilde{\sigma}_\alpha - \sigma \big\}(x,y,\xi) \Big| \lesssim_{\alpha,\beta,N,M} R^{-N} \langle \xi \rangle^{-M}, \end{equation}
    %
    and so
    %
    \begin{equation} \label{QalphaApproximation}
        \| ( Q_I - Q_\alpha ) u \|_{C^M(X)} \lesssim_{N,M} R^{-N} \| u \|_{L^1(X)}.
    \end{equation}
    %
    Combining \eqref{QThreeBound}, \eqref{QTwoBound}, and \eqref{QalphaApproximation}, we conclude that
    %
    \begin{equation} \label{QApproximationTheorem}
        \| (Q - Q_\alpha) u \|_{C^M(X)} \lesssim_{N,M} R^{-N} \| u \|_{L^1(X)}.
    \end{equation}
    %
    We have thus verified the required properties of $Q_\alpha$.
\end{proof}

We can use the Littlewood-Paley projections to localize the phase support of oscillatory integrals for general Fourier integral operators.

\begin{lemma} \label{lemma:WaveOscillatoryLemmaddw}
  For each $R \geq 1$, consider a family of pseudo-differential operators 
  %
  \[ Q_R(x,y) = R^n \int_{\RR^n} q_R(x, \xi) e^{i R \xi \cdot (x - y)}\; d\xi, \]
  %
  on $\RR^n$, where the functions $\{ q_R \}$ are all supported on $\RR^d \times \{ \xi: 1/10 \leq |\xi| \leq 10 \}$, and have uniformly bounded derivatives of all orders. Consider an oscillatory integral operator
  %
  \[ A(x,y) = \int s(x,y,\theta) e^{i \Phi(x,y,\theta)}\; d\theta \]
  %
  from $\RR^n$ to $\RR^m$, where the $(x,y)$ support of $s$ is compact, and suppose that whenever $\nabla_\theta \Phi = 0$ on this support, $\nabla_y \Phi \neq 0$. If $L > 0$ is sufficiently large, and we fix $\chi \in C_c^\infty(\RR^p)$ equal to one for $1/L \leq |\xi| \leq L$, and define
  %
  \[ \tilde{A}(x,y) = \int \chi( \theta / R ) s(x,y,\theta) e^{i \Phi(x,y,\theta)}\; d\theta, \]
  %
  then
  %
  \[ \| A \circ Q_R - \tilde{A} \circ Q_R \|_{L^1(\RR^n) \to C^M(\RR^m)} \lesssim_{N,M} R^{-N} \]
\end{lemma}
\begin{proof}
  If $\delta > 0$ is chosen appropriately small, and we consider a smooth function $\alpha \in C_c^\infty(\RR)$ supported on $[-\delta,\delta]$ and with $\alpha(t) = 1$ for $|t| \leq \delta/2$. Then the two operators with kernels
  %
  \begin{equation}
    K_1(x,y) = \int (1 - \chi(\theta / R)) (1 - \alpha( |\nabla_\theta \Phi(x,y,\theta)| )) s(x,y,\theta) e^{i \Phi(x,y,\theta)}
  \end{equation}
  %
  and
  %
  \begin{equation}
    K_2(x,y) = \int \chi(\theta / R) (1 - \alpha( |\nabla_\theta \Phi(x,y,\theta) )) s(x,y,\theta) e^{i \Phi(x,y,\theta)}
  \end{equation}
  %
  are smooth and compactly supported in $x$ and $y$. Thus their Fourier transforms in the $y$-variable are rapidly decaying, which implies bounds of the form
  %
  \begin{equation}
    |\partial_x^\alpha \partial_y^\beta ( K_j \circ Q_R )(x,y)| \lesssim_{N,\alpha,\beta} R^{-N},
  \end{equation}
  %
  and thus that
  %
  \begin{equation}
    \| K_j \circ Q_R \|_{L^1(\RR^n) \to C^M(\RR^m)} \lesssim_{N,M} R^{-N}.
  \end{equation}
  %
  Our proof would be complete if we could show that the operator $B$ with kernel
  %
  \begin{equation}
  \begin{split}
    &R^n \int (1 - \chi(\theta / R)) \alpha( |\nabla_\theta \Phi(x,y,\theta)| ) \\
    &\quad\quad\quad s(x,z,\theta) q_R(z,\xi) e^{i [\Phi(x,z,\theta) + R \xi \cdot (z - y)]} q_R(z,\xi)\; d\theta\; dz\; d\xi.
  \end{split}
  \end{equation}
  %
  maps $L^1(\RR^n)$ to $C^M(\RR^m)$ with operator norm $R^{-N}$ for arbitrarily large $N$ and $M$. We can write $B = B_0 + B_1 + \cdots$ where $B_0$ is defined by restricting the integral of the kernel to $|\theta| \leq R/L$, and for $j \geq 1$, $B_j$ is defined by restricting the integral to $2^jLR \leq |\theta| \leq 2^{j+1} LR$. If $\delta$ is chosen appropriately small, then compactness implies that on the support of the integral, $C^{-1} |\theta| \leq |\nabla_z \Phi(x,z,\theta)| \leq C|\theta|$ for some $C > 0$. If $L \geq 20 C$, and $|\theta| \leq R/L$, then the $z$ gradient of the integral is greater than $R |\xi| - |\nabla_z \Phi(x,z,\theta)| \geq R/10 - CR/L \geq R/20$, and so integration by parts in the $z$-variable gives $|\partial_x^\alpha \partial_y^\beta B_0(x,y)| \lesssim_{\alpha,\beta,N} R^{-N}$, and thus that $\| B_0 \|_{L^1(\RR^n) \to C^M(\RR^m)} \lesssim_{N,M} R^{-N}$. On the other hand, if $|\theta| \sim 2^j LR$, then the $z$ gradient of the integral is greater than $|\nabla_z \Phi(x,z,\theta)| - R |\xi| \geq C^{-1} |\theta| - 10 R \gtrsim 2^j R$, and so integration by parts gives $\| B_j \|_{L^1(\RR^n) \to C^M(\RR^m)} \lesssim_{N,M} 2^{-jN} R^{-N}$, and thus summing in $j$ gives that $\| B \|_{L^1(\RR^n) \to C^M(\RR^m)} \lesssim_{N,M} R^{-N}$.
\end{proof}








\section{Periodic Geodesics and Assumption B} \label{sec:PeriodicGeodesics}

Using the Fourier integral operator methods we have described, we can now discuss the relation of Assumption B to the geometry of the manifold $X$ induced by the operator $P$.

\begin{theorem}
  Suppose that all the eigenvalues of $P$ occur in an arithmetic progression of the form $\{ a + n b \}$. Then all geodesics are closed, and have length $1/bn$ for some $n \geq 1$.
\end{theorem}
\begin{proof}
  By the functional calculus, we know that $e^{2 \pi i P/b} = e^{2 \pi i a/b} I$. In particular, it follows that the canonical relation of the operator $e^{2 \pi i P/b}$ must be equal to the canonical relation of the identity, which is the diagonal $\Delta_X \subset T^* X \times T^* X$. But Theorem \ref{waveisanFIOTheorem} says that the canonical relation is equal to $\{ (x,\xi;x',\xi'): (x,\xi) = \alpha_{1/b}(x',\xi') \}$, where $\alpha$ is the geodesic flow, and so it follows that $\alpha_{1/b}$ is the identity map, which implies the required result.
\end{proof}

The converse is \emph{almost} true.

\begin{theorem}
  Let $X$ be a compact manifold, let $P \in \COP{1}(X)$, and suppose that all geodesics are closed and have length $1/bn$. Let $a \in \ZZ$ denote the \emph{Maslov index} of the geodesics  on $X$ (a cohomological invariant of these curves). Then there exists $\tilde{P} \in \COP{1}(X)$ commuting with $P$, such that $P - \tilde{P}$ is order zero and $e^{2 \pi i \tilde{P} / b} = I$. If the \emph{subprincipal symbol} of $P$ is equal to $\pi a / 2 b$, then $P - \tilde{P}$ is order $-1$.
\end{theorem}
\begin{proof}
  See Section 29.2 of \cite{Hormander4}.
\end{proof}

The eigenvalues of $\tilde{P}$ must be integer multiples of $b$. In the second case, since $P - \tilde{P}$ is an operator of order $-1$, it follows that $e_n$ is $L^2$ normalized, $P e_n = \lambda e_n$, and $\tilde{P} e_n = n / b e_\lambda$, then
%
\[ |\lambda - n / b| = \| (P - \tilde{P}) e_\lambda \|_{L^2(X)} \lesssim \| e_n \|_{H^{-1}(X)} \lesssim n^{-1}. \]
%
It follows from this inequality that the eigenvalues of $P$ must lie in eigenbands of the form $[n/b - C/n, n/b + C/n]$, and thus `almost' lie in arithmetic progression $\ZZ$. In the case where the subprincipal symbol is a constant not equal to $\pi a / 2b$, then the eigenvalues of $P$ will lie in a shifted arithmetic progression; this is the case, for instance, when $P = \sqrt{-\Delta}$ on a Riemannian manifold; for instance, on the sphere, the eigenvalues of $P$ are equal to
%
\begin{equation}
  \sqrt{k(k+d)} = k + \frac{d-1}{2} + O(k^{-1}),
\end{equation}
%
where the $\tfrac{d-1}{2}$ term arises from the Maslov index of the great circles on the sphere.

In what follows, we need the fact that the wave equation is \emph{completely periodic} in order to control the wave equation for large times, rather than just contained in eigenbands like above.

%\part{Review of Literature}
%%!TEX root = ../main.tex

\begin{comment}

\chapter[BHS: Decoupling for FIOs]{Beltran, Hickman, and Sogge: Decoupling for Fourier Integral Operators}

The paper we now discuss extends the theory of decoupling, which was originally used to establish local smoothing for the wave equation on Euclidean space, to the setting of more general FIOs. Here we attempt to study $L^p$ to $L^p$ estimates  for Fourier integral operators given by
%
\[ Tf(x,t) = \int_{\RR^d} e^{2 \pi i \phi(x,t;\xi)} b(x,t;\xi) (1 + |\xi|^2)^{\mu/2} \widehat{f}(\xi)\; d\xi \]
%
where $b$ is a compactly supported symbol of order zero, compactly supported in $x$ and $t$, and $\phi$ is a phase function, homogeneous of degree one in the $\xi$ variable. We let
%
\[ K(x,t;y) = \int e^{2 \pi i [\phi(x,t;\xi) - y \cdot \xi]} b(x,t;\xi) (1 + |\xi|^2)^{\mu/2}\; d\xi \]
%
denote the kernel. Since $\nabla_\xi \phi(x,t;\xi)$ is homogeneous of degree zero in $\phi$, the sets
%
\[ \Sigma_{(x,t)} = \{ \nabla_\xi \phi(x,t,\xi) : \xi \in \RR^n \} \subset \RR^n_y \]
%
are usually manifolds of dimension $n-1$. They are related to the singular support of $K$.

%Let us for simplicity assume the phase function is nondegenerate, and also, that the resulting Lagrangian manifold $\Sigma \subset T^*(\RR^n_x \times \RR_t \times \RR^n_y)$, the natural projection maps $\Sigma \to T^* \RR^n_y$ and $\Sigma \to \RR^n_{x,t}$ are submersions. It follows that for an open set of $x$ and $t$ we can find a hypersurface $\Sigma_{x,t}$ in the cotangent space of $(x,t)$ upon which the operator behaves

To study the $L^p$ behaviour of $T$, we break up the behaviour of the operator dyadically in the $\xi$ variable, thus setting
%
\[ T = T_{\leq 1} + \sum_{n = 1}^\infty T_n, \]
%
where, for a given $\lambda > 0$, we let $T^\lambda$ be an operator with kernel $K^\lambda$ given by
%
\[ K^\lambda(x,t;\xi) = \int e^{2 \pi i [\phi(x,t;\xi) - y \cdot \xi]} b(x,t;\xi) (1 + |\xi|^2)^{\mu/2} \beta(\xi / \lambda)\; d\xi. \]
%
It can be verified that $T_{\leq 1}$ is a pseudo-differential operator of order 0, and is therefore bound on $L^p$ for all $1 < p < \infty$. It therefore suffices to show that as $\lambda \to \infty$,
%
\[ \| T^\lambda f \|_{L^p(\RR^d)} \lesssim \lambda^{- \varepsilon} \| f \|_{L^p(\RR^d)} \]
%
so that we may sum in $n$ in the expansion of $T$ via the triangle inequality to obtain an $L^p$ bound for the original operator.

For large $\lambda$, the principle of stationary phase tells us we should expect $K^\lambda$ to be concentrated in the set
%
\[ \{ (x,t;y) : |\nabla_\xi \phi(x,t;\xi) - y| \leq 1/\lambda\ \text{for some $\xi$} \}, \]
%
since the phase oscillates to a significant degree for $|\nabla \phi(x,t;\xi) - y| \gtrsim 1/\lambda$, roughly a $1/\lambda$ neighborhood of the singular support of $K$. Also we have $\| K^\lambda \|_{L^\infty_x L^\infty_y} \lesssim \lambda^{\mu + d}$ trivially by taking in absolute values. This gives the crude estimate that $\| K_n \|_{L^\infty_x L^1_y} \lesssim \lambda^{\mu + d - 1}$. Thus we obtain by Schur's Lemma that
%
\[ \| T^\lambda f \|_{L^1(\RR^{d+1})} \lesssim \lambda^{\mu + d - 1} \| f \|_{L^1(\RR^d)}. \]
%
We will get a much better bound by a more sophisticated decomposition of the kernels $\{ K^\lambda \}$.

For a given $\lambda$, let $\{ \xi_\nu^\lambda \}$ be a maximal, $\lambda^{-1/2}$ separated subset of the unit sphere in $\RR^n$, where $\nu$ ranges over some set $\Theta^\lambda$ with $\#(\Theta^\lambda) \sim \lambda^{(d-1)/2}$. Let
%
\[ \Gamma^\lambda_\nu = \{ \xi \in \RR^d_\xi : |\xi \cdot \xi^\lambda_\nu| \geq (1 - c \lambda^{-1/2}) \cdot |\xi| \} \]
%
for some suitably small constant $c > 0$. Let $\{ \chi^\lambda_\nu \}$ be a smooth partition of unity, homogeneous of degree zero, adapted to the $\Gamma^\lambda_\nu$. We thus have
%
\[ |D^\alpha \chi^\lambda_\nu(\xi)| \lesssim_\alpha \lambda^{|\alpha|/2} |\xi|^{1 - \alpha}. \]
%
We thus consider operators $T^\lambda_\nu$ with kernels $K^\lambda_\nu$ given by
%
\[ K^\lambda_\nu(x,t;y) = \int e^{2 \pi i (\phi(x,t;\xi) - y \cdot \xi)} b^\lambda_\nu(x,t;\xi) ( 1 + |\xi|^2 )^{\mu/2} \]
%
where
%
\[ b^\lambda_\nu(x,t;\xi) = b(x,t;\xi) \beta(\xi / \lambda) \chi^\lambda_\nu(\xi). \]
%
Stationary phase again tell us that $K^\lambda_\nu(x,t;y)$ satisfies the bounds
%
\[ |K^\lambda_\nu(x,t;y)| \lesssim_N \frac{\lambda^{\mu + (d+1)/2}}{\langle \lambda | \pi_{\xi^\lambda_\nu} (y - \nabla_\xi \phi(x,t,\xi^\lambda_\nu)| + \lambda^{1/2} | \pi_{\xi^\lambda_\nu}^\perp (y - \nabla_\xi \phi(x,t,\xi^\lambda_\nu)) | \rangle^N}. \]
%
This bound immediately yields via Schur's Lemma that for all $1 \leq p \leq \infty$,
%
\[ \| K^\lambda_\nu \|_{L^\infty_{x,t} L^1_y} \lesssim \lambda^\mu, \]
%
and thus that
%
\[ \| T^\lambda_\nu f \|_{L^\infty(\RR^{d+1})} \lesssim \lambda^\mu \| f \|_{L^\infty(\RR^d)}, \]
%
a much better bound than was obtained trivially than from the global sum.

We might hope to then combine this still fairly trivial bound with a square function estimate of the form
%
\[ \| T^\lambda_\nu f \|_{L^p(\RR^{d+1})} \lesssim_\varepsilon \lambda^\varepsilon \| S^\lambda f \|_{L^p(\RR^{d+1})} \]
%
where
%
\[ S^\lambda f = \left( \sum_\nu | T^\lambda_\nu f |^2 \right)^{1/2}, \]
%
which in some sense, captures the orthogonality of the operators $\{ T^\lambda_\nu \}$. This then yields that for $p \geq 2$, that
%
\begin{align*}
    \| T^\lambda_\nu f \|_{L^p_{x,t}} &\lesssim_\varepsilon \lambda^\varepsilon \| T^\lambda_\nu f \|_{L^p_{x,t} l^2_\nu}\\
    &\leq \lambda^\varepsilon \| T^\lambda_\nu f \|_{L^p_{x,t} l^p_\nu}\\
    &= \lambda^\varepsilon \| T^\lambda_\nu f \|_{l^p_\nu L^p_{x,t})}\\
    &\lesssim \lambda^\varepsilon \lambda^{\mu + (d-1)/2} \#(\Theta^\lambda)^{1/p}\\
    &= \lambda^{\varepsilon + \mu + (d-1) / p},
\end{align*}
%
thus giving bounds for $\mu > (d-1)/2$, i.e., the non-endpoint local smoothing.

Wolff noticed that the non-endpoint local smoothing results could be obtained with a weaker bound than a square function estimate, namely, an \emph{$l^p$ decoupling inequality} of the form
%
\[ \| T^\lambda f \|_{L^p(\RR^{d+1})} \lesssim \lambda^{\alpha(p) + \varepsilon} \| T^\lambda_\nu f \|_{l^p_\nu L^p_{x,t}}, \]
%
where if $2 \leq p \leq 2(d+1)/(d-1)$, then
%
\[ \alpha(p) = (d-1)|1/p - 1/2|, \] 
%
and for $2(d+1)/(d-1) \leq p < \infty$,
%
\[ \alpha(p) = (d-1)|1/p - 1/2| - 1/p. \]
%
The $L^p$ norm of the localized pieces is much easier to estimate. For instance, we have
%
\[ \| T^\lambda_\nu f \|_{L^\infty_{x,t}} \lesssim \lambda^\mu \| f \|_{L^\infty}, \]
%
and thus
%
\[ \| T^\lambda_\nu f \|_{l^\infty_\nu L^\infty_{x,t}} \lesssim \lambda^\mu \| f \|_{L^\infty}. \]
%
On the other hand, we have an $L^2$ energy conservation estimate of the form
%
\[ \| T^\lambda_\nu f \|_{L^2_{x,t}} \lesssim \| T^\lambda_\nu f \|_{L^\infty_t L^2_x} \lesssim \lambda^\mu \| f^\lambda_\nu \|_{L^2} \]
%
where $f^\lambda_\nu$ is the localization of $f^\lambda_\nu$ on the Fourier side to the support of $\chi^\lambda_\nu$. This immediately yields via Parseval's inequality and orthogonality that
%
\[ \| T^\lambda_\nu f \|_{l^2_\nu L^2_{x,t}} \lesssim \lambda^\mu \| f^\lambda_\nu \|_{l^2_\nu L^2_x} \lesssim \lambda^\mu \| f \|_{L^2_x}. \]
%
Interpolation thus yields that for $2 \leq p \leq \infty$,
%
\[ \| T^\lambda_\nu f \|_{l^p_\nu L^p_{x,t}} \lesssim \lambda^\mu \| f \|_{L^p}, \]
%
and thus that, together with Wolff's decoupling inequality,
%
\[ \| T^\lambda f \|_{L^p(\RR^{d+1})} \lesssim_\varepsilon \lambda^{\alpha(p) + \mu + \varepsilon} \| f \|_{L^p(\RR^d)}, \]
%
and thus we get boundedness of $T$ for $\mu < \alpha(p)$, which gives $1/p$ degrees of local smoothing.

\end{comment}

\part{New Results}
\chapter{The Boundedness of Spectral Multipliers with Compact Support} \label{chap:boundedsinglefrequencyscale}
%!TEX root = ../main.tex

In this chapter, we prove Theorem \ref{maintheorem}, which we restate below:

\thmmaintheorem*

In Section \ref{sec:AnAnalogueOf} we saw that for operators $P$ satisfying Assumption A, $\| a \|_{M^p_{\text{Dil}}(X)} \gtrsim \| a \|_{R^{s,p}[0,\infty)}$, and so it suffices to prove that $\| a \|_{M^p_{\text{Dil}}(X)} \lesssim \| a \|_{R^{s,p}[0,\infty)}$. Define $a_R(\lambda) = a(\lambda / R)$. In Section \ref{PrelimSetup}, we will see from elementary functional analysis that
%
\begin{equation} \label{TrivialLowFrequencyBound}
    \sup\nolimits_{R \leq 1} \left\| a_R \right\|_{M^p(X)} \lesssim \| a \|_{l^\infty[0,\infty)} \lesssim \| a \|_{R^{s,p}[0,\infty)}.
\end{equation}
%
On the other hand, the upper bound
%
\begin{equation} \label{dyadicMainReulst}
    \sup\nolimits_{R \geq 1} \| a_R \|_{M^p(X)} \lesssim \| a \|_{R^{s,p}[0,\infty)}.
\end{equation}
%
requires a more in depth analysis than \eqref{TrivialLowFrequencyBound}. It is here we apply the wave equation transform, writing
%
\begin{equation}
    a_R(P) = \int_{-\infty}^\infty R\;\! \widehat{a}(Rt) e^{2 \pi i t P}\; dt,
\end{equation}
%
which reduces $a_R(P)$ to studying averages associated with solutions to the half-wave equation $\partial_t = 2 \pi i P$ on $X$. We obtain \eqref{dyadicMainReulst} by studying averages to the wave equation using several methods, including:
%
\begin{itemize}
    \item[(A)] Quasi-orthogonality estimates for averages of solutions to the half-wave equation on $X$, discussed in Section \ref{estimatesforwavepackets}, which arise from a study of the geometry of the Finsler metric on $X$.

    \item[(B)] Variants of the density-decomposition arguments, which we introduced in Section \ref{sec:densitydecompositions}, to control the `small time behavior' of solutions to the half-wave equation. We carry out these arguments in section \ref{regime1firstsection}

    \item[(C)] A new strategy to reduce the `large time behavior' of the half-wave equation to an endpoint local smoothing inequality for the half-wave equation on $X$, described in Section \ref{regime2finalsection}.
\end{itemize}
%
Equations \eqref{TrivialLowFrequencyBound} and \eqref{dyadicMainReulst} thus imply Theorem \ref{maintheorem}.

\section{Preliminary Setup} \label{PrelimSetup}

Without loss of generality, by translating and dilating our multiplier, we can assume that the eigenvalues of the operator $P$ are all integers. Fix a constant $\varepsilon_X > 0$, strictly smaller than the injectivity radius of the geodesic flow on the manifold $X$ induced by the geometry of $P$. Our goal is to prove inequalities \eqref{TrivialLowFrequencyBound} and \eqref{dyadicMainReulst}. Proving \eqref{TrivialLowFrequencyBound} is simple because the operators $a_R(P)$ are smoothing operators, uniformly for $0 < R < 1$, because the spectrum of $X$ is discrete, low eigenvalue eigenfunctions are uniformly smooth, and $X$ is compact.

\begin{lemma} \label{lowjLemma}
    Let $X$ be a compact $d$-dimensional manifold, and suppose $P$ is an elliptic operator on $X$. If $1/2d < 1/p - 1/2 \leq 1/2$, and if $a$ is a regulated function, then
    %
    \[ \sup\nolimits_{R \leq 1} \| a_R \|_{M^p_{\text{Dil}}(X)} \lesssim \| a \|_{l^\infty(\Lambda)} \lesssim \| a \|_{R^{p,s}[0,\infty)}. \]
\end{lemma}
\begin{proof}
    Let $T_R = m_R(P)$. Recall that $\Lambda$ is the set of eigenvalues of $P$. The set $\Lambda \cap [0,2]$ is finite. For each $\lambda \in \Lambda$, choose a finite orthonormal basis $\mathcal{E}_\lambda$. Then we can write
    %
    \begin{equation}
        T_R = \sum\nolimits_{\lambda \in \Lambda} \sum\nolimits_{e \in \mathcal{E}_\lambda} \langle f, e \rangle e.
    \end{equation}
    %
    Since $\mathcal{V}_\lambda \subset C^\infty(X)$, H\"{o}lder's inequality implies
    %
    \begin{equation}
    \begin{split}
        \| \langle f, e \rangle e \|_{L^p(X)} &\leq \| f \|_{L^p(X)} \| e \|_{L^{p'}(X)} \| e \|_{L^p(X)} \lesssim_\lambda \| f \|_{L^p(X)}.
    \end{split}
    \end{equation}
    %
    But this means that
    %
    \begin{equation}
        \left\| T_R f \right\|_{L^p(X)} \leq \sum\nolimits_{\lambda \in \Lambda_P \cap [0,2]} \sum\nolimits_{e \in \mathcal{E}_\lambda} |m(\lambda/R)| \| \langle f, e \rangle e \|_{L^p(X)} \lesssim \| m \|_{l^\infty(\Lambda)}.
    \end{equation}
    %
    For $1/p - 1/2 > 1/2d$, the Sobolev embedding theorem and \eqref{equation12980u21u89eqiwjqwiou} imply that
    %
    \begin{equation}
        \| a \|_{l^\infty(\Lambda)} \lesssim \| a \|_{W^{s,p'}[0,\infty)} \lesssim \| a \|_{R^{s,p}[0,\infty)},
    \end{equation}
    %
    which completes the proof.
\end{proof}

We now begin to reduce the analysis of $a_R(P)$ for $R \geq 1$ to the study of the wave equation. Fix a bump function $q \in C_c^\infty(\RR)$ with $\supp(q) \subset [1/4,4]$ and $q(\lambda) = 1$ for $\lambda \in [1/2,2]$, and define $Q_R = q(P/R)$. % has range contained in the finite dimensional subspace $V_R$ of $C^\infty(M)$ spanned by eigenfunctions of $P$ with eigenvalues in $[2^{j-2},2^{j+2}]$. Since $P$ is elliptic, it is often a useful heuristic that elements of $V_R$ are `frequency localized' at a scale $R$.
We write
%
\begin{equation}
    T_R = T_R \circ Q_R = \int_{\RR} R\;\! \widehat{m}(R t) (e^{2 \pi i t P} \circ Q_R)\; dt,
\end{equation}
%
and view the operators $(e^{2 \pi i t P} \circ Q_R)$ as `frequency localized' wave propagators.

Because all eigenvalues of $P$ are integers, it follows that $e^{2 \pi i (t + n) P} = e^{2 \pi i t P}$ for any $t \in \RR$ and $n \in \ZZ$. Let $I_0$ denote the interval $[-1/2,1/2]$. We may then write
%
\begin{equation}
    T_R = \int_{I_0} b_R(t) (e^{2 \pi i tP} \circ Q_R)\; dt,
\end{equation}
%
where $b_R: I_0 \to \CC$ is the periodic function
%
\begin{equation}
    b_R(t) = \sum\nolimits_{n \in \ZZ} R \widehat{m}(R (t + n)).
\end{equation}
% Sphere 1, but we scale is by 1/2pi
% Sphere of radius 1/2 pi, has sectional curvatures kappa = 2 pi
% so 1/4
We split our analysis of $T_R$ into two regimes: regime $\text{I}$ and regime $\text{II}$. In regime $\text{I}$, we analyze the behaviour of the wave equation over times $0 \leq |t| \leq \varepsilon_X$ by decomposing this time interval into length $1/R$ pieces, and analyzing the interactions of the wave equations between the different intervals. In regime $\text{II}$, we analyze the behaviour of the wave equation over times $\varepsilon_X \leq |t| \leq 1$. Here we need not perform such a decomposition, since the $R^{s,p}$ norm gives better control on the function $b_R$ over these times.
%Heuristically, the result is just a discretization of the condition that $C_p(m) < \infty$, but using the additional fact that $b_j$ was obtained from a periodization of the Fourier transform of a function with `frequency support' on an annulus of radius $2^j$, and thus locally constant at a scale $1/2^j$. This allows us to replace $L^p$ norms with $L^1$ norms on intervals of length $1/2^j$ without incurring any loss.

\begin{lemma} \label{decompositionLemma}
    Fix $\varepsilon > 0$. Let $\mathcal{T}_R = \ZZ/R \cap [-\varepsilon, \varepsilon]$ and define $I_t = [t - 1/R, t + 1/R]$. For any function $a: [0,\infty) \to \CC$, define a periodic function $b: I_0 \to \CC$ by setting
    %
    \[ b(t) = \sum\nolimits_{n \in \ZZ} R \widehat{m}(R(t + n)). \]
    %
    Then we can write $b = \left( \sum\nolimits_{t_0 \in \mathcal{T}_R} b_{t_0}^I \right) + b^{II}$, where
    %
    \[ \supp(b_{t_0}^I) \subset I_{t_0} \quad\text{and}\quad \supp(b_R^{II}) \subset I_0 \smallsetminus [-\varepsilon,\varepsilon]. \]
    %
    Moreover, we have
    %
    \[ \left( \sum\nolimits_{t_0 \in \mathcal{T}_R} \Big[ \| b^I_{t_0} \|_{L^p(I_0)} \langle R t_0 \rangle^{s} \Big]^p \right)^{1/p} \lesssim R^{1/p'} \| a \|_{R^{s,p}[0,\infty)} \]
        %
        and
        %
    \[ \| b^{II} \|_{L^p(I_0)} \lesssim R^{1/p' - (d-1)(1/p - 1/2)} \| a \|_{R^{s,p}[0,\infty)}. \]
\end{lemma}
\begin{proof} [Proof of Lemma \ref{decompositionLemma}]
    The intervals $\{ I_{t_0} : t_0 \in \mathcal{T}_R \}$ cover $[-\varepsilon,\varepsilon]$, and so we may consider an associated partition $\mathbb{I}_{[-\varepsilon,\varepsilon]} = \sum_{t_0} \chi_{t_0}$ where $\text{supp}(\chi_{t_0}) \subset I_{t_0}$ and $|\chi_{t_0}| \leq 1$. Define $b_{t_0}^I = \chi_{t_0} b_j$ and $b^{II} = (1 - \mathbb{I}_{[-\varepsilon,\varepsilon]} ) b$. Then $b = \sum_{t_0} b_{t_0}^I + b^{II}$, and the support assumptions are satisfied. It remains to prove the required norm bounds for these choices. For each $n \in \ZZ$, define a function $b_{n}: I_0 \to \CC$ by setting $b_{n}(t) = R \widehat{m}(R (t + n))$. Then $b = \sum_n b_{n}$. Moreover,
    %
    \begin{align} \label{translationlpcalculation}
    \begin{split}
        &\left( \sum_{n \neq 0} \left[ \langle R n \rangle^{s} \| b_{n} \|_{L^p(I_0)} \right]^p \right)^{1/p}\\
        %&\quad\quad\quad \sim \left( \int_{-1/2}^{1/2} \sum_{n \neq 0} \left[ \langle R(t + n) \rangle^{s} |R \widehat{m}(R ( t + n ))| \right]^p\; dt \right)^{1/p}\\
        &\quad\quad\quad \sim \left( \int_{|t| \geq 1/2} \left[ \langle R t \rangle^{s} |R \widehat{m}(R t)| \right]^p \right)^{1/p}\\
        &\quad\quad\quad = R^{1/p'} \left( \int_{|t| \geq R/2} \left[ |t|^{s} \widehat{m}(t) \right]^p \right)^{1/p} \leq R^{1/p'} C_p(m).
    \end{split}
    \end{align}
    %
    Write $b_{t_0}^I = \sum_n b_{t_0,n}^I$ and $b^{II} = \sum_n b_{n}^{II}$, where $b_{t_0,n}^I = \chi_{t_0} b_n$ and $b_n^{II} = \mathbb{I}_{I_0 \smallsetminus [-\varepsilon, \varepsilon] } b_{n}$. Then
    %
    \begin{align} \label{zerobj0iicalculation}
    \begin{split}
        \| b_{0}^{II} \|_{L^p(I_0)} &= \left( \int_{\varepsilon \leq |t| \leq 1/2} |R \widehat{m}(R t)|^p \right)^{1/p}\\
        &= R^{1/p'} \left( \int_{R \varepsilon \leq |t| \leq R/2} |\widehat{m}(t)|^p \right)^{1/p} \lesssim R^{1/p' - s} C_p(m).
    \end{split}
    \end{align}
    %
    Using \eqref{translationlpcalculation}, \eqref{zerobj0iicalculation}, and H\"{o}lder's inequality, we conclude that
    %
    \begin{align}
    \begin{split}
        \|  b^{II} \|_{L^p(I_0)} &\leq \sum\nolimits_n \| b_{n}^{II} \|_{L^p(I_0)}\\
        &\leq \| b_{0}^{II} \|_{L^p(I_0)} + \sum\nolimits_{n \neq 0} \left[ |R n|^{s} \| b_{n}^{II} \|_{L^p(I_0)} \right] \frac{1}{|R n|^{s}}\\
        &\leq \| b_{0}^{II} \|_{L^p(I_0)} + R^{-s} \left( \sum\nolimits_{n \neq 0} \left[ |R n|^{s} \| b_{n} \|_{L^p(I_0)} \right]^p \right)^{1/p}\\ % \left( \sum_{n \neq 0} \frac{1}{|R n |^{s p'}} \right)^{1/p'}\\
        &\lesssim R^{1/p' - s} C_p(m).
    \end{split}
    \end{align}
    %
    A similar calculation shows that
    %
    \begin{align} \label{eacht0bjcalculation}
    \begin{split}
        \| b_{t_0}^I \|_{L^p(I_0)} &\leq \sum\nolimits_n \| b_{t_0,n}^I \|_{L^p(I_0)}\\
        &= \| b_{t_0,0}^I \|_{L^p(I_0)} + \sum\nolimits_{n \neq 0} \| b_{t_0,n}^I \|_{L^p(I_0)}\\
        &\lesssim \| b_{t_0,0}^I \|_{L^p(I_0)} + R^{-s} \Big( \sum\nolimits_{n \neq 0} |R n|^{s} \| b_{t_0,n}^I \|_{L^p(I_0)}^p \Big)^{1/p}\\
        &\lesssim \| b_{t_0,0}^I \|_{L^p(I_0)} + \Big( \sum\nolimits_{n \neq 0} |R n|^{s} \| b_{t_0,n}^I \|_{L^p(I_0)}^p \Big)^{1/p}.
    \end{split}
    \end{align}
    %
    Using \eqref{eacht0bjcalculation}, we calculate that
    %
    \begin{align} \label{bjt0Icalculation}
    \begin{split}
        &\left( \sum_{t_0 \in \mathcal{T}_R} \left[ \| b_{t_0}^I \|_{L^p(I_0)} \langle R t_0 \rangle^{s} \right]^p \right)^{1/p}\\
        &\quad\quad \lesssim \left( \sum_{t_0 \in \mathcal{T}_R} \left[ \| b_{t_0,0}^I \|_{L^p(I_0)} \langle R t_0 \rangle^{s} \right]^p + \sum_{n \neq 0} \left[ |R n|^{s} \| b_{t_0,n}^I \|_{L^p(I_0)} \right]^p \right)^{1/p}\\
        &\quad\quad \lesssim \left( \int_{\RR} \left[ \langle R t \rangle^{s} R \widehat{m}(R t) \right]^p dt \right)^{1/p}\\
        &\quad\quad \lesssim R^{1/p'} C_p(m).
    \end{split}
    \end{align}
    %
    Since each function $b_{t_0}^I$ is supported on a length $1/R$ interval, we have
    %
    \begin{equation}
        \| b_{t_0}^I \|_{L^1(I_0)} \lesssim R^{-1/p'} \| b_{t_0}^I \|_{L^p(I_0)},
    \end{equation}
    %
    and substituting this inequality into \eqref{bjt0Icalculation} completes the proof.
\end{proof}

The following proposition, a kind of $L^p$ square root cancellation bound, implies \eqref{dyadicMainReulst} once we take Lemma \ref{decompositionLemma} into account. % and comparing \eqref{DKAPDKAWIODJAWOI} with \eqref{ejqwoifjeoifjwqoifjwqoi} and \eqref{DWAIOJDAOIWDJWAIODJIOJD} with \eqref{DPOIJAOIWDJQWIOFJQOIVJIEOVNFNJNVNV},
Since we already proved \eqref{TrivialLowFrequencyBound}, this proposition completes the proof of Theorem \ref{maintheorem}, and will take the remainder of the chapter.

\begin{prop} \label{TjbLemma}
    Let $P$ be an elliptic operator on a $d$-dimensional compact manifold $X$ satisfying Assumptions A and B. Fix $R > 0$ and a constant $\varepsilon_X$ smaller than the injectivity radius of $X$ with respect to the geodesic flow induced by $P$, and suppose $1/(d-1) < 1/p - 1/2 < 1/2$. Consider any function $b: I_0 \to \CC$, and suppose we can write $b = \sum\nolimits_{t_0 \in \mathcal{T}_R} b_{t_0}^I + b^{II}$, where $\supp(b_{t_0}^I) \subset I_{t_0}$ and $\supp(b_R^{II}) \subset I_0 \smallsetminus [-\varepsilon_X,\varepsilon_X]$. Define operators $T^I = \sum\nolimits_{t_0 \in \mathcal{T}_R} T^I_{t_0}$ and $T^{II}$, where
    %
    \[ T_{t_0}^I = \int b_{t_0}^I(t) ( e^{2 \pi i tP} \circ Q_R )\; dt\ \ \text{and}\ \ T^{II} = \int b^{II}(t) ( e^{2 \pi i tP} \circ Q_R)\; dt. \]
    %
    Then if $s = (d-1)(1/p - 1/2)$, then
    %
    \begin{equation} \label{ejqwoifjeoifjwqoifjwqoi}
        \| T^I \|_{L^p \to L^p} \lesssim R^{-1/p'} \left( \sum\nolimits_{t_0 \in \mathcal{T}_R} \Big[ \| b^I_{t_0} \|_{L^p(I_0)} \langle R t_0 \rangle^{s} \Big]^{p} \right)^{1/p}
    \end{equation}
    %
    and
    %
    \begin{equation} \label{DPOIJAOIWDJQWIOFJQOIVJIEOVNFNJNVNV}
        \| T^{II} \|_{L^p \to L^p} \lesssim R^{s - 1/p'} \| b^{II} \|_{L^p(I_0)}.
    \end{equation}
\end{prop}

\begin{remark}
    One should view this proposition as a discretization of Theorem \ref{maintheorem}, and thus compared to Lemma \ref{lemma2} in the argument of Heo, Nazarov, and Seeger.
\end{remark}

Proposition \ref{TjbLemma} splits the main bound of the paper into two regimes: regime $\text{I}$ and regime $\text{II}$. Noting that we require weaker bounds in \eqref{DPOIJAOIWDJQWIOFJQOIVJIEOVNFNJNVNV} than in \eqref{ejqwoifjeoifjwqoifjwqoi}, the operator $T^{II}$ will not require as refined an analysis as for the operator $T^I$, and we obtain bounds on $T^{II}$ by a reduction to an endpoint local smoothing inequality in Section \ref{regime2finalsection}. On the other hand, to obtain more refined estimates for the $L^p$ norms of quantities of the form $f = T^I u$, we consider a decompositions of the form $u = \sum_{x_0} u_{x_0}$, where $u_{x_0}: X \to \CC$ is supported on a ball $B(x,1/R)$ of radius $1/R$ centered at $x_0$. We then have $\smash{f = \sum\nolimits_{(x_0,t_0)} f_{x_0,t_0}}$, where $f_{x_0,t_0} = T^I_{t_0} u_{x_0}$. To control $f$ we must establish an $L^p$ square root cancellation bound for the functions $\{ f_{x_0,t_0} \}$. In the next section, we study the $L^2$ quasi-orthogonality of these functions, which we use as a starting point to obtain the required square root cancellation.

% Finsler geometry arises in our problem because the principal symbol $p: T^* M \to [0,\infty)$ acts as a Minkowski norm on the cotangent spaces $T^* M$%, and thus induces a Finsler metric on $M$ by taking the dual Minkowski norm. % $F(x,v) = \sup\nolimits_{\xi \in S_x^*} \xi(v)$. % A very similar theory of geodesics and curvature can be developed for Finsler manifolds as for Riemannian manifolds. The main difference is that a geodesic travelling from one point $p$ to another point $q$ need not be a geodesic when the orientation of the geodesic is reversed. Thus the induced distance function $d_+$ on a Finsler manifold \emph{need not be symmetric}.

% given a homogeneous function $F: TM \to [0,\infty)$ which induces a smoothly varying \emph{Minkowski norm} on the tangent spaces of $M$. 

%A similar theory of geodesics and curvature can be developed for Finsler manifolds, and the geodesic flow on $T^* M$ induced by the Finsler metric $M$ is precisely the Hamiltonian flow induced by the function $p$. The most important difference arising in our calculations between Riemannian and Finsler geometry is that the shortest geodesic travelling from a point $x_0 \in M$ to a point $x_1 \in M$ need not agree with the shortest geodesic travelling from $x_1$ to $x_0$; we call the former a \emph{forward geodesic} from $x_0$ to $x_1$, and the latter a \emph{backward geodesic} for $x_0$ to $x_1$, and denote the lengths of these geodesics by $d_+(x_0,x_1)$ and $d_-(x_0,x_1)$ respectively; the functions $d_+$ and $d_-$ obey the triangle inequality, but are \emph{not symmetric}. We make no assumption that the reader of this paper is familiar with Finsler geometry in this paper, describing the necessary results we need, and giving explicit citations for those who wish to know more details. 

\begin{comment}

Other than this theory, the only non-standard topic in Finsler geometry we will use is a theory of \emph{approximations to normal coordinates}. In Finsler geometry, for each $x \in M$ we can use geodesics to define a diffeomorphism of a neighborhood of the origin in $T_x M$ into $M$. This map will be smooth away from the origin. However, unlike in Riemannian geometry, normal coordinates on a Finsler manifold are in general only $C^1$ at the origin, which can cause issues. Fortunately in our argument we will only need to consider an approximate normal coordinate system which is $C^\infty$, based on a method of Douglas and Thomas \cite{Douglas,Thomas}. One can find a more modern exposition and extension of the method in \cite{Pfeifer}. For each $(x,v) \in TM - 0$, and each $y \in T_p M$, we consider the system of ordinary differential equations
%
\begin{align*}
    \frac{d^2 c^i}{dt^2} = - \sum\nolimits_{a,b} \gamma^i_{ab}(c,s) \dot{c}^a \dot{c}^b \quad\text{and}\quad \dot{s} = - \sum\nolimits_{a,b} \gamma^i_{ab}(c,s) \dot{c}^a s^b,
\end{align*}
%
where $c(0) = x$, $\dot{c}(0) = y$, and $s(0) = v$. If $y$ is suitably close to the origin, we can guarantee that $c$ and $s$ exist on $[0,1]$ with $s(t) \neq 0$ for all $t \in [0,1]$. Thus the coefficients of the ordinary differential equation are \emph{smooth} on a neighbourhood of the trajectories of $c$ and $s$, which gives us \emph{smooth dependence on the initial data}. By compactness, in some pre-compact coordinate system $(x,U)$, we can find an open ball $B \subset \RR^d$ such that for all $x \in U$, all $w \in B$, and all $v \in T_x M$ with $F(x,v) = 1$, the property above is true. But if $SM = \{ (x,v) \in TM : F(x,v) = 1 \}$ denotes the sphere bundle, this means we can find an open precompact set $U \subset SM \otimes TM$ containing $SM \times 0$ and a smooth map $F: U \to M$ such that for each $(x,v) \in SM$, $F(x,v,\cdot)$ is a $C^\infty$ diffeomorphism. The inverse is thus a coordinate system $y_{x,v}$. It has the property that the geodesic starting at $x$ with tangent vector will be mapped to a straight line in the coordinate system, but not necessarily the other geodesics starting at $x$. Thus in this coordinate system $G(0,v) = 0$.
% N^a_b(0,v)
% Thus
% 0 = - \sum\nolimits_{a,b} \gamma^i_{ab}(tv,v) v^a v^b \quad\text{for $1 \leq i \leq d$},

we can find $B_x \subset T_x M$ such that this is true for all $y \in B_x$ and $v$ with $F(x,v) = 1$, and in coordinates, the. But this implies that the coefficients of the ordinary differential equations have \emph{smooth coefficients}, and so we get smooth initial dependence on the data. Choosing an open precompact set $U \subset \bigcup_x \{ x \} \times B_x \times S_x$, we get a smooth map $F: U \to M$ which restricts to a diffeomorphism $F_{x,v}: B_x$ for each $x \in M$ and $v \in T_x M$ with $F(x,v) = 1$.

A proof of these facts using more modern notation is given in Theorem 7 of \cite{Pfeifer}.

\end{comment}

\section{Quasi-Orthogonality Estimates For Wave-Packets} \label{estimatesforwavepackets}

The discussion at the end of Section \ref{PrelimSetup} motivates us to consider estimates for functions obtained by taking averages of the wave equation over a small time interval, with initial conditions localized to a particular part of space. In this section, we study the $L^2$ orthogonality of such quantities. One should compare the results of this section to the results of Lemma \ref{lemma4}. We do not exploit periodicity of the Hamiltonian flow in this section since we are only dealing with estimates for the half wave-equation for \emph{small times}. Our results here thus hold for any manifold $X$, and any operator $P$ whose principal symbol has cospheres with non-vanishing Gaussian curvature. In order to prove the required quasi-orthogonality estimates of the functions $\{ f_{x_0,t_0} \}$, we find a new connection between the Finsler geometry we introduced in section \ref{sec:geometriesinduced} and the behaviour of the operator $P$. In particular, recall the quasimetrics $d_+$ and $d_-$ introduced in that section.

\begin{prop} \label{theMainEstimatesForWave}
    Let $X$ be a compact manifold of dimension $d$, and let $P$ be a classical elliptic self-adjoint pseudodifferential operator of order one whose principal symbol satisfies the curvature assumptions of Theorem \ref{maintheorem}. Suppose $\varepsilon_X > 0$ is chosen smaller thant he injectivity radius of $X$. Then for all $R \geq 0$, the following estimates hold:
    %
    \begin{itemize}%[leftmargin=8mm]
        \item (Pointwise Estimates) Fix  $|t_0| \leq \varepsilon_X$ and $x_0 \in X$. Consider any two $L^1$ normalized functions $c: \RR \to \CC$ and $u: X \to \CC$ with $\supp(c) \subset I_{t_0}$ and $\supp(u) \subset B(x_0,1/R)$. Define $S: X \to \CC$ by setting
        %
        \[ S = \int c(t) (e^{2 \pi i t P} \circ Q_R) \{ u \}\; dt. \]
        %
        Then for any $K \geq 0$, and any $x \in X$,
        %
        \[ |S(x)| \lesssim_K \frac{R^d}{\langle R d_X(x_0,x) \rangle^{\frac{d-1}{2}}} \max\nolimits_{\pm} \Big\langle R \big| t_0 \pm d_X^\pm(x,x_0) \big| \Big\rangle^{-K}. \]

        \item (Quasi-Orthogonality Estimates) Fix $|t_0 - t_1| \leq \varepsilon_X$, and $x_0, x_1 \in X$. Consider $L^1$ normalized functions $c_0,c_1: \RR \to \CC$ and $u_0,u_1: X \to \CC$ such that, for each $\nu \in \{ 0, 1 \}$, $\supp(c_\nu) \subset I_{t_\nu}$ and $\supp(u_\nu) \subset B(x_\nu,1/R)$. Define $S_\nu: X \to \CC$ by setting
        %
        \[ S_\nu = \int c_\nu(t) (e^{2 \pi i t P} \circ Q_R) \{ u_\nu \}\; dt. \]
        %
        Then for any $K \geq 0$,
        %
        \[ \left| \langle S_0, S_1 \rangle \right| \lesssim_K \frac{R^d}{\langle R d_X(x_0,x_1) \rangle^{\frac{d-1}{2}}} \max\nolimits_{\pm} \Big\langle R \big| (t_0 - t_1) \pm d^\pm(x_0,x_1) \big| \Big\rangle^{-K}. \]
    \end{itemize}
\end{prop}

%\begin{remark} The choice of balls $B(x_0,1/R)$ in the statement above is somewhat arbitrary. We assume we have fixed some choice of open sets $B(x,\delta)$ in $M$ for each $x \in M$ and $0 < \delta < 1$, such that for each coordinate chart $F: U \to \RR^d$ and each compact set $K \subset U$, there exists $C > 1$ such that for all $x \in K$ and $0 < \delta < 1$,
%%
%\begin{equation}
%    B(F(x), C^{-1} \delta) \subset F(B(x,\delta)) \subset B(F(x), C \delta),
%\end{equation}
%
%and where the balls in $\RR^d$ are the usual Euclidean balls. The particular choice given will only effect the magnitude of $\varepsilon_X$ and the implicit constants in the statement of the proposition. Fixing an arbitrary Riemannian metric on $M$ and then letting $B(x,\delta)$ be the resulting metric balls will suffice for our purposes.
%\end{remark}

%    \begin{figure}[h]
%        \centering
%        \includegraphics[width=0.3\textwidth]{TangenciesOnSphere.png}
%    \end{figure}

\begin{remark}
    Just as in Lemma \ref{lemma4}, the pointwise estimate of Lemma \ref{theMainEstimatesForWave} tells us that the function $S$ is concentrated on a geodesic annulus of radius $|t_0|$ centered at $x_0$ and thickness $O(1/R)$. %The annulus has thickness $O(1/R)$, and on this annulus the function $S$ has magnitude at most $O(R^{\frac{d+1}{2}} |t_0|^{- \frac{d-1}{2}})$.
The quasi-orthogonality estimate tells us that the two functions $S_0$ and $S_1$ are only significantly correlated with one another if the two annuli on which the majority of the support of $S_0$ and $S_1$ lie are internally or externally tangent to one another, depending on whether $t_0$ and $t_1$ have the same or opposite sign respectively. %, and then
%
%\begin{equation} \label{DIOAWJDOIWJAFOIJWAFcCw}
%    |\langle S_0, S_1 \rangle| \lesssim R^{\frac{d+1}{2}} |t_0 - t_1|^{- \frac{d-1}{2}}.
%\end{equation}
%
%, though the upper bounds here look superficially different, because here we are using the half wave equation to define our functions $\{ S_\nu \}$, whereas in \cite{HeoandNazarovandSeeger} the analogous functions are simply defined by taking smooth functions adapted to certain annuli. Normalizing and rescaling appropriately causes the bounds to match.
\end{remark}

%We denote the induced metric on $M$ by taking the lengths of forward geodesics between points by $d_X^+: M \times M \to [0,\infty)$, and the induced metric by taking the lengths of backward geodesics by $d_M^-: M \times M \to [0,\infty)$.

% Thus $S_x^*$ has everywhere positive scalar curvature by continuity.\footnote{This argument is a higher dimensional variant of the argument of Chapter 2, Theorem 4.2 of \cite{HeinzHopf}.} But then Hadamard's ovaloid theorem (see Chapter 4, Theorem 2.1 of \cite{HeinzHopf}) implies that for each $x \in M$, the interior of $S_x^*$ is strictly convex. Thus $p$ is a strictly convex function in the $\xi$ variable.

%For each $x \in M$, define $\| \cdot \|_x^2$ to be the Legendre transform of $p(x,\cdot)^2$. Then $\| \cdot \|$ is a smoothly varying family of strictly convex norms on $TM$, and thus gives a Finsler metric on $M$. Moreover, the geodesic flow with respect to this metric is precisely the Hamiltonian flow induced by the principal symbol $p$, since $p$ is precisely the dual Finsler metric on $T^* M$ to the Finsler metric $\| \cdot \|$ on $TM$. %(the usual process of taking the Legendre transform of the Euler-Lagrange equations defining geodesics is precisely the inverse of the procedure we have used to define the metric on $M$ from the function $p$).
%Slightly abusing notation, we will sometimes denote the principal symbol $p: T^* M \to [0,\infty)$ by $\| \cdot \|$, the distinction between the norm $\| \cdot \|$ being clear depending on whether it is applied to a tangent vector or covector. We denote the induced metric on $M$ by taking the lengths of forward geodesics between points by $d_M^+: M \times M \to [0,\infty)$, and the induced metric by taking the lengths of backward geodesics by $d_M^-: M \times M \to [0,\infty)$.

\begin{proof}[Proof of Proposition \ref{theMainEstimatesForWave}]

To simplify notation, in the following proof we will suppress the use of $R$ as an index, for instance, writing $Q$ for $Q_R$. For both the pointwise and quasi-orthogonality estimates, we want to consider the operators in coordinates, so we can use Theorem \ref{thm:equivalenceofphase} to understand the wave propagators in terms of various oscillatory integrals. Consider a small quantity $\varepsilon_X' < \varepsilon_X$, to be fixed later.


We begin with a proof of the pointwise bounds. Start by covering $X$ by a finite family of suitably small open sets $\{ V_\alpha \}$, such that for each $\alpha$, there is a coordinate chart $U_\alpha$ compactly containing $V_\alpha$ and with $N(V_\alpha, 1.1 \varepsilon_X) \subset U_\alpha$. Let $\{ \eta_\alpha \}$ be a partition of unity subordinate to $\{ V_\alpha \}$. Given $u: M \to \CC$, write $u = \sum_\alpha u_\alpha$, where $u_\alpha = \eta_\alpha u$.

Let us first address the case where $\varepsilon_X' \leq |t_0| \leq \varepsilon_X$. Lemmas \ref{PseudoOsicllatoryLemma} and \ref{lemma:WaveOscillatoryLemmaddw} imply that if we define
%
\begin{equation}
    S_\alpha = \int c(t) (Q_\alpha \circ W_\alpha(t) \circ Q_\alpha) \{ u_\alpha \}\; dt,
\end{equation}
%
where $Q_\alpha$ is a pseudo-differential operator whose symbol in the coordinate system $U_\alpha$ is supported on $\{ R/4 \leq |\xi| \leq 4R \}$, and
%
\[ W_\alpha(x,y) = R^{\frac{d+1}{2}} \int_{-\infty}^\infty s(t,x,y,R\tau) e^{i R \tau \Phi(x,t,x',\tau)}\; d\xi, \]
% t = d(x,y)
% 
% 
where $s$ is smooth, and compactly supported on $L^{-1} \leq |\tau| \leq L$, and $\Phi$ is defined in \eqref{onedwavephase}, then for all $N \geq 0$,
%
\begin{equation} \label{parametrixerrroestimate}
    \left\| S - \textstyle\sum\nolimits_\alpha S_\alpha \right\|_{L^\infty(X)} \lesssim_N R^{-N}.
\end{equation}
%
This error is negligible to our analysis if $N$ is chosen appropriately large. Integrating by parts if $|t| - d^+(y,x)$ and $|t| - d^-(y,x)$ is suitably large, and then taking in absolute values gives the required bounds for each of the terms $S_\alpha$, completing the proof of the required bounds.

% If d(x_0,x) << 1, then the bound says << R^{-N} for all N
% If d(x_0,x) >> 1, then the bound says << R^{(d+1)/2} < ... >^{-N}

Now we address the case $|t_0| \leq \varepsilon_X$. Lemmas \ref{PseudoOsicllatoryLemma} and \ref{lemma:WaveOscillatoryLemmaddw} imply that if we define
%
\begin{equation}
    S_\alpha = \int c(t) (Q_\alpha \circ W_\alpha(t) \circ Q_\alpha) \{ u_\alpha \}\; dt,
\end{equation}
%
where $Q_\alpha$ is as above, and
%
\[ W_\alpha(x,y) = R^d \int s(t,x,y,\theta) e^{i [ \phi(x,y,\xi) + t p(y,\xi) ]}\; d\xi, \]
%
where $s$ is smooth and supported on $L^{-1} \leq |\theta| \leq L$, and $\phi$ is a solution to the Eikonal equation \eqref{awiodjawoidhjioq23412341234234}, then for all $N \geq 0$,
%
\begin{equation} \label{parametrixerrroestimate}
    \left\| S - \textstyle\sum\nolimits_\alpha S_\alpha \right\|_{L^\infty(X)} \lesssim_N R^{-N}.
\end{equation}
%
%This error is negligible to the pointwise bounds we want to obtain in Proposition \ref{theMainEstimatesForWave} if we choose $N \geq K - \frac{d+1}{2}$, since the compactness of $X$ implies that $d_X(x,x_0) \lesssim 1$ for all $x \in X$, and so
%
%\begin{equation}
%    R^{-N} \lesssim R^{\left( \frac{d+1}{2} - K \right)} \lesssim \frac{R^{d}}{\langle R d_X(x,x_0) \rangle^{\frac{d-1}{2}}} \Big\langle R \big| |t_0| \pm d_X^\pm(x,x_0) \big| \Big\rangle^{-K}.
%\end{equation}
%
The error in \eqref{parametrixerrroestimate} is again negligible. We bound each of the functions $\{ S_\alpha \}$ separately, combining the estimates using the triangle inequality. We continue by expanding out the implicit integrals in the definition of $S_\alpha$. In the coordinate system $U_\alpha$, we can write
%
\begin{equation} \label{SalphaDefinition}
\begin{split}
    S_\alpha(x) &= \int c(t) \sigma(x,\eta) e^{2 \pi i \eta \cdot (x - y)}\\[-6 pt]
    &\quad\quad\quad s(t,y,z,\xi) e^{2 \pi i [ \phi(y,z,\xi) + t p(z,\xi) ]}\\
    &\quad\quad\quad\quad \sigma(z,\theta) e^{2 \pi i \theta \cdot (z - w)} (\eta_\alpha u)(w)\\
    &\quad\quad\quad\quad\quad dt\; dy\; dz\; dw\; d\theta\; d\xi\; d\eta.
\end{split}
\end{equation}
%
The integral in \eqref{SalphaDefinition} looks highly complicated, but can be simplified considerably by noticing that most variables are quite highly localized. In particular, oscillation in the $\eta$ variable implies that the amplitude is negligible unless $|x - y| \lesssim 1/R$, oscillation in the $\theta$ variable implies that the amplitude is negligible unless $|z - w| \lesssim 1/R$, and the support of $u$ implies that $|w - x_0| \lesssim 1/R$. Define
%
\begin{equation} \label{DIOAJVIOEJAV8318923}
    k_1(t,x,z,\xi) = \int \sigma(x,\eta) s(t,y,z,\xi) e^{2 \pi i [ \eta \cdot (x - y) + \phi(y,z,\xi) - \phi(x,z,\xi) ]}\; dy\; d\eta,
\end{equation}
%
and
%
\begin{equation} \label{DWAIOCWOAIJFAIO213123}
\begin{split}
    k_2(t,\xi) &= \int k_1(t,x,z,\xi) \sigma(z,\theta) (\eta_\alpha u)(w)\\
    &\quad\quad\quad e^{2 \pi i [ \theta \cdot (z - w) + \phi(x,z,\xi) - \phi(x,x_0,\xi) + t p(z,\xi) - t p(x_0,\xi) ]}\; d\theta\; dw,
\end{split}
\end{equation}
%
and then set
%
\begin{equation} \label{oaisdjoai9091390}
\begin{split}
    a(x,\xi) &= \int c(t) k_2(t,R \xi) e^{2 \pi i [ (t - t_0) p(x_0, R \xi) ]} \; dt\; dz,
\end{split}
\end{equation}
%
so that $\text{supp}_\xi(a) \subset \{ \xi : 1/8 \leq |\xi| \leq 8 \}$, and
%
\begin{equation}
    S_\alpha(x) = R^d \int a(x, \xi) e^{2 \pi i R [ \phi(x,x_0,\xi) + t_0 p(x_0,\xi) ]}\; d\xi.
\end{equation}
%
Integrating by parts in $\eta$ and $\theta$ in \eqref{DIOAJVIOEJAV8318923} and \eqref{DWAIOCWOAIJFAIO213123} gives that for all multi-indices $\alpha$,
%
\begin{equation} \label{ioqejfoieqjf13412}
    |\partial_\xi^\alpha k_1(t,x,z,\xi)| \lesssim_\alpha R^{-|\alpha|} \quad\text{and}\quad |\partial_\xi^\alpha k_2(z,\xi)| \lesssim_{\alpha} R^{-|\alpha|}.
\end{equation}
%
Using the bounds in \eqref{ioqejfoieqjf13412} with the fact that $\text{supp}(c)$ is contained in a $O(1/R)$ neighborhood of $t_0$ in \eqref{oaisdjoai9091390} then implies $|\partial_\xi^\alpha a(x,\xi)| \lesssim_\alpha 1$ for all $\alpha$.
%
%We now find $\lambda: V_\alpha^* \times S^{d-1} \to (0,\infty)$ such that
%
%\[ p(x_0, \lambda(x_0,\xi) \xi ) = 1, \]
%
%and switch coordinate systems. If $\tilde{a}(x,\rho, \xi) = a(x, \rho \lambda(x_0) \xi ) (J_\xi \lambda)(x_0,\xi)$,
%a smooth family of homogeneous diffeomorphisms $F_{x_0}: \RR^d \to \RR^d$ for $x_0 \in V_\alpha^*$ such that $p(x_0,F_{x_0}(\xi)) = |\xi|$ for all $x \in \RR^d$. Then if $\tilde{a}(x,\rho, \eta) = a(x, \rho F_{x_0}(\eta) ) JF_{x_0}(\eta)$,
%then a change of variables $\xi = \lambda(x_0,\xi) \rho \eta$ for $|\eta| = 1$ gives that
%
%\begin{align*} 
%    &R^{d} \int a(x,\xi) e^{2 \pi i R [ \phi(x,x_0,\xi) + t_0 p(x_0,\xi) ]}\\
%    &\quad\quad\quad = R^d \int_0^\infty \rho^{d-1} \int_{|\eta| = 1} \tilde{a}(x,\rho,\eta) e^{2 \pi i R \rho [ \lambda(x_0,\eta) \phi(x, x_0, \eta) + t_0 ]}\; d\eta\; d\rho.
%\end{align*}
%

We now account for angular oscillation of the integral by working in a kind of 'polar coordinate' system. First we find $\lambda: V_\alpha \times S^{d-1} \to (0,\infty)$ such that for all $|\xi| = 1$,
%
\begin{equation}
    p(x_0, \lambda(x_0,\xi) \xi) = 1.
\end{equation}
%
If $\tilde{a}(x,\rho, \eta) = a(x, \rho \lambda(x_0) \xi ) \det [ \lambda(x_0,\xi) I + \xi (\nabla_\xi \lambda)(x_0,\xi)^T ]$, then
%
% F_{x_0}(xi) = L(x_0,xi) xi
% DF = L(x_0,xi) I + D_xi lambda
%
\begin{equation}
    S_\alpha(x) = R^d \int_0^\infty \rho^{d-1} \int_{|\xi| = 1} \tilde{a}(x,\rho, \xi) e^{2 \pi i R \rho [ t_0 + \phi(x, x_0, \lambda(x_0,\xi) \xi) ]}\; d\xi\; d\rho.
\end{equation}
%
Define $\Phi: S^{d-1} \to \RR$ by setting $\Phi(\xi) = \phi(x,x_0, \lambda(x_0,\xi) \xi)$.    
%
%\[ \text{Hess}_\xi(\Phi) = \text{Hess}_\xi(\phi) + t_0 \text{Hess}_\xi(p). \]
%
%For any multi-index $\alpha$,
%
%\[ |\partial_\xi^\alpha \{ \phi(x,x_0,\xi) - (x - x_0) \cdot \xi \}| \lesssim |x - x_0|^2 |\xi|^{1 - |\alpha|}, \]
%
%and so $\text{Hess}_\xi(\Phi)$ differs from $t_0 \text{Hess}_\xi(p)$ by a matrix all of whose entries are $O(|x - x_0|^2)$. The cosphere curvature assumption on $p$ implies that
%
%\[ |\text{Det}(\text{Hess}_{\xi} p)(x,\xi)| \sim 1. \]
%
%On the support of $s$, $|x - x_0| \lesssim |t|$. Thus if $\varepsilon_X$ is chosen appropriately small,
%
%\[ |\text{Det}(\text{Hess}_{\xi} \Phi)|  \gtrsim t^{d-1}. \]
%
%We claim that $(\nabla_\xi \Phi)(x,x_0,\xi) = 0$ if and only if $x$ lies on the characteristic curve through $x_0$ with cotangent vector $\xi$, and then
%
%To prove this, we rely on the property that, if we define the Hamiltonian flow $\alpha: (-\varepsilon_X, \varepsilon_X) \times T^* V_\alpha^* \to U_\alpha$ generated by $p$, then $\phi(x,x_0,\xi) = t p(x_0,\xi)$ if $x = \alpha(t,x',\xi')$, where $(x' - x) \cdot \xi = 0$ and $p(x',\xi') = p(x_0,\xi)$.
%
% For t_0 = 0, the geodesic from x_0 to x with covextor xi is the
% unique critical point.
%
% nabla_xi phi(x,x_0,xi) = Proj_{xi^Perp}(x - x_0) + O( |x - x_0|^2 )
% nabla_xi Phi = Proj_{xi^Perp}(x - x_0) + t_0 p_xi(x_0,xi) + O( |x - x_0|^2 )
%
% For each t_0, x_0, x, exists a unique xi
% t_0 = 0, then we pick xi going from x_0 to x.
%
% If we rewrite in radial coordinates
%       (t_0,r,x^,x_0)
% Provided that D_{x^} nabla_xi Phi is invertible, this is possible
%
%       phi(x,x_0,xi)
%
%   (D_{x^} nabla_xi) phi(x,x_0,xi) = D_xi [nabla_{x^} phi(x,x_0,xi)]
%                
%
%
%                   p(x, Nab_r phi, Nab_{x^} phi ) = p(x_0,xi)
%
%
% f(x,y) = 0 -> y = g(x) possible if D_yf is invertible
%
% (t(xi) + t_0) p(x_0,xi)
%
% Certainly the geodesic from x_0 to x with covector xi_0 minimizes t(xi) p(x_0,xi)
%   This would also work if p(x_0,xi) is maximized for |xi| = 1 at xi_0
%   Otherwise 
%
% Derivative is (t(xi) + t_0) ∇L(xi) + ∇t(xi) L(xi) = 0
%
% Action is (d_v L)(x,v) v
% Principle of Least Action: A(c) = int (d_v L)(c_x,c_v) c_v
%       Motion is critical point of action.
%
% H(x,xi) = xi * v - L(x,v)
%
% Action then becomes int c_xi c_v = int c_xi p_xi(c_x,c_xi)
%                                  = int p(c_x,c_xi)
%
% Given P as our Hamiltonian
%       P: T^*M -> R
%       P = Legendre Transform of L : TM -> R
%       L = Legendre Transform of P
%
%       If D_xi p(x,xi) = v
%       L(x,v) = v * xi - p(x,xi)
%
% p: R^n x R^n -> R
% psi: R^{n-1} -> R
%       where p(0,eta) = 0 and (Nabla_eta p)(0,eta) * e_n != 0
%       where (Nabla_x psi)(0) = eta'
%
% Then we can solve p(x, Nabla_x phi) = 0
%   such that phi(x',0) = psi(x')
%   and Nabla_x phi(0) = eta
%
% By the implicit function theorem, the assumptions
% uniquely determine d_x phi(x',0) = omega(x')
% in a neighborhood of 0.
%
% We start by finding a Lagrangian section of T^*M,
% which is subset of Z(p), and which contains
% (x',0,omega(x')) for all x' in a neighborhood of 0.
%
% Since Z(p) is a coisotropic submanifold, it is
% foliated by integral curves of the Hamiltonian
% vector field X_p. Find a Lagrangian manifold of R^n
% contained in Z(p), passing through (0,eta)
%
% By assumption, in a neighborhood of
% (0,eta) each of these integral curves of X_p
% projects onto a curve on R^n which is transverse
% to R^{n-1} x {0}. If we let S be the union of all
% the characteristics passing through (x',0,omega(x'))
% then S will therefore be a section of R^n.
%
% The collection (x',omega(x')) is a Lagrangian
% section of R^{n-1} x R^{n-1}, and thus is an
% isotropic submanifold M of R^n x R^n
%
% Thus TM is contained in (TM)^perp
% And the integral curves are perpendicular to
% TM since M is contained in Z(p), which
% implies the union of the curves is Lagrangian
%
% But now we've defined S, we can find phi such
% that d_x phi is S. 
%
%
% So in particular, if phi(x',0) = 0
% So vector field is perpendicular to R^{n-1}
% 
% c(0) = (q,p)
%
% F(c_1(t),y,xi) = int dF(c_1) c_1'
%           = int dF(c_1){ p_xi(c) }
%           = int c_2 { p_xi(c) }
%           = - int p(c)
%           = - t p(y,xi)
%
%The fact that $\Sigma_{x_0}$ has non-vanishing curvature means that 
%
%\[ \text{Det} ( \text{Hess}_{\hat{\eta}} \Phi ) \gtrsim |x - x_0|^{d-1}. \]
%In particular, if $\varepsilon_X$ is chosen suitably small, then the error terms are negligible. 
%
%Now the oscillatory integral has two critical points, at values $\eta$ such that $F_{x_0}(\eta)$ points directly towards or away the flow from $x_0$ to $x$. Thus the integral above can be written as
%
We claim that, if $\varepsilon_X'$ is chosen appropriately small, then in the $\xi$ variable, $\Phi$ has exactly two critical points $|\xi^+|^{-1} \xi^+$ and $|\xi^-|^{-1} \xi^-$, where $\xi^+ \in S_{x_0}^*$ is the covector corresponding to the forward geodesic from $x_0$ to $x$, and $\xi^- \in S_{x_0}^*$ is the covector corresponding to the backward geodesic from $x_0$ to $x$. Moreover,
%
\begin{equation}
    \Phi(|\xi^+|^{-1} \xi^+) = d_X^+(x_0,x) \quad\text{and}\quad \Phi(|\xi^-|^{-1} \xi^-) = - d_X^-(x_0,x),
\end{equation}
%
and the Hessian at each of these points is non-degenerate, with each eigenvalue of the Hessian having magnitude exceeding a constant multiple of $d_X^{\pm}(x_0,x)$. We prove that these properties hold for $\Phi$ in Proposition \ref{triangleLemma} of the following section, via a series of geometric arguments. It then follows from the principle of stationary phase that
%
\begin{equation}
    S_\alpha(x) = \sum_{\pm} \frac{R^{d}}{\left\langle R d_X^{\pm}(x_0,x) \right\rangle^{\frac{d-1}{2}}} \int_0^\infty \rho^{\frac{d-1}{2}} a_{\pm}(x,\rho) e^{2 \pi i R \rho [ t_0 \pm d_X^{\pm}(x_0,x)]}\; d\rho,
\end{equation}
%
where $a_{\pm}$ is supported on $|\rho| \sim 1$, and for all $\alpha$, $|\partial_\rho^\alpha a_{\pm}| \lesssim_\alpha 1$. Integrating by parts in the $\rho$ variable if $t_0 \pm d_X^{\pm}(x_0,x)$ is large, we conclude that
%
% t_0 - d^
%
% 
% 
%
\begin{equation} \label{finaloscillatoryintegralbound}
\begin{split}
    |S_\alpha(x)| \lesssim \frac{R^{d}}{\left\langle R d_X(x_0,x) \right\rangle^{\frac{d-1}{2}}} \sum_{\pm} \big\langle R |t_0 \pm d_X(x_0,x)| \big\rangle^{-K}.
\end{split}
\end{equation}
%
Combining \eqref{parametrixerrroestimate} and \eqref{finaloscillatoryintegralbound} completes the proof of the pointwise bounds.

The quasi-orthogonality arguments are obtained by a largely analogous method, and so we only sketch the proof. One major difference is that we can use the self-adjointness of the operators $Q$, and the unitary group structure of $\{ e^{2 \pi i t P} \}$, to write
%
\begin{equation}
\begin{split}
    \langle S_0, S_1 \rangle &= \int c_0(t) c_1(s) \big\langle (e^{2 \pi i t P} \circ Q) \{ u_0 \}, (e^{2 \pi i s P} \circ Q) \{ u_1 \} \big\rangle\\
    &= \int c_0(t) c_1(s) \big\langle (e^{2 \pi i (t - s) P} \circ Q^2) \{ u_0 \}, u_1 \big\rangle\\
    &= \int c(t) \big\langle (e^{2 \pi i t P} \circ Q^2) \{ u_0 \}, u_1 \big\rangle,
\end{split}
\end{equation}
%
where $c(t) = \int c_0(u) c_1(u - t)\; du$, by Young's inequality, satisfies
%
\begin{equation}
    \| c \|_{L^1(\RR)} \lesssim \| c_0 \|_{L^1(\RR)} \| c_1 \|_{L^1(\RR)} \leq 1
\end{equation}
%
and $\supp(c) \subset [ (t_0 - t_1) - 4/R, (t_0 - t_1) + 4/R]$. After this, one proceeds exactly as in the proof of the pointwise estimate. We write the inner product as
%
\begin{equation}
    \sum\nolimits_\alpha \int c(t) \big\langle (e^{2 \pi i t P} \circ Q^2) \{ \eta_\alpha u_0 \}, u_1 \big\rangle.
\end{equation}
%
Then we use Lemmas \ref{PseudoOsicllatoryLemma} and \ref{lemma:WaveOscillatoryLemmaddw} to replace $e^{2 \pi i tP} \circ Q^2$ with $Q_{\alpha}^2 \circ W_{\alpha}(t) \circ Q_{\alpha}^2$, modulo a negligible error. The integral
%
\begin{equation}
    \sum\nolimits_\alpha \int c(t) \big\langle (Q_{\alpha}^2 \circ W_{\alpha}(t) \circ Q_{\alpha}^2) \{ \eta_\alpha u_0 \}, u_1 \big\rangle
\end{equation}
%
is then only non-zero if both the supports of $u_0$ and $u_1$ are compactly contained in $U_\alpha$. Thus we can switch to the coordinate system of $U_\alpha$, in which we can express the inner product by oscillatory integrals of the exact same kind as those occurring in the pointwise estimate. Integrating away the highly localized variables as in the pointwise case, and then applying stationary phase in polar coordinates proves the required estimates.
\end{proof}

In remains to classify the critical points that arose in Proposition \ref{theMainEstimatesForWave}.

\begin{prop} \label{triangleLemma}
    Fix a bounded open set $U_0 \subset \RR^d$. Consider a Finsler metric $F: U_0 \times \RR^d \to [0,\infty)$ on $U_0$, and it's dual metric $F_*: U_0 \times \RR^d \to [0,\infty)$, which extends to a Finsler metric on an open set containing the closure of $U_0$. Fix a suitably small constant $r > 0$. Let $U$ be an open subset of $U_0$ with diameter at most $r$ which is geodesically convex (any two points are joined by a minimizing geodesic). Let $\phi: U \times U \times \RR^d \to \RR$ solve the eikonal equation
    %
    \[ F_* ( x, \nabla_x \phi(x,y,\xi) ) = F_*(y,\xi), \]
    %
    such that $\phi(x,y,\xi) = 0$ for $x \in H(y,\xi)$, where $H(y, \xi) = \{ x \in U : \xi \cdot (x - y) = 0 \}$. For each $x_0,x_1 \in U$, let $S_{x_0}^* = \{ \xi \in \RR^d: F_*(x_0,\xi) = 1 \}$ be the cosphere at $x_0$, and define $\Psi: S_{x_0}^* \to \RR$ by setting
    %
    \[ \Psi(\xi) = \phi(x_1,x_0,\xi). \]
    %
    Then the function $\Psi$ has exactly two critical points, at $\xi^+$ and $\xi^-$, where the Legendre transform of $\xi^+$ is the tangent vector of the forward geodesic from $x_0$ to $x_1$, and the Legendre transform of $\xi^-$ is the tangent vector of the backward geodesic from $x_1$ to $x_0$. Moreover,
    %
    \[ \Psi(\xi^+) = d_X^+(x_0,x_1) \quad\text{and}\quad \Psi(\xi^-) = - d_X^-(x_0,x_1), \]
    %
    and the Hessians $H_+$ and $H_-$ of $\Psi$ at these critical points, viewed as quadratic maps from $T_{\xi_{\pm}} S_{x_0}^* \to \RR$ satisfy
    %
    \[ H_+(\zeta) \geq C d_X^+(x_0,x_1) |\zeta| \quad\text{and}\quad H_-(\zeta) \leq - C d_X^-(x_0,x_1) |\zeta|, \]
    %
    where the implicit constant is uniform in $x_0$ and $x_1$.
\end{prop}

If $\Psi$ is as above, then the function $\Phi: S^{d-1} \to \RR$ obtained by setting $\Phi(\xi) = \phi( x, x_0, F_*(x,\xi)^{-1} \xi  )$ is precisely the kind of function that arose as a phase in Proposition \ref{theMainEstimatesForWave}, where $F_*$ was the principal symbol $p$ of the pseudodifferential operator we were considering. Since critical points and the Hessians of maps at critical points are stable under diffeomorphisms, %(see Chapter 5, Problem 17 of \cite{Spivak1}),
and the map $\xi \mapsto F_*(x,\xi)^{-1} \xi$ is a diffeomorphism from $S_{x_0}^*$ to $S^{d-1}$, classifying the critical points of the map $\Psi$ implies the required properties of the map $\Phi$ used in Proposition \ref{theMainEstimatesForWave}.

% then, because critical points and the Hessians of maps at critical points are stable under diffeor

%Define a function $\lambda: U \times S^{d-1} \to \RR$ by setting $\lambda(x,v) = F_*( x,v )^{-1}$ and then define $\Phi: S^{d-1} \to \RR$ by setting $\Phi(\xi) = \phi(x,x_0,\lambda(x_0,\xi) \xi)$. Let $\xi^+$ and $\xi^-$ are the two points in $S_{x_0}^*$ such that $\mathcal{L}^{-1}(x_0,\xi^+)$ is the unit tangent vector of the forward geodesic from $x_0$ to $x$, and $-\mathcal{L}^{-1}(x_0,\xi^-)$ is the unit tangent vector to the backward geodesic from $x_0$ to $x$. In this subsection, we will verify that the only critical points of $\Phi$ occur at $|\xi^+|^{-1} \xi^+$ and $|\xi^-|^{-1} \xi^-$, that $\Phi( |\xi^+|^{-1} \xi^+) = d_X^+(x_0,x)$ and $\Phi( |\xi^-|^{-1} \xi^-) = - d_X^-(x_0,x)$, and that if we let $V_+$ and $V_-$ be the subspaces of $\RR^d$ which give the tangent spaces of $S^{d-1}$ at each critical point, then the Hessians $H_{\pm} : V_{\pm} \to \RR$ of $\Phi$ at each critical point are non-degenerate, with $|H_{\pm}(\omega)| \gtrsim d_X^{\pm}(x_0,x) |\omega|$, where the implicit constant is locally uniform in $x_0$ and $x$. These properties precisely prove the properties we needed to analyze the function $\Phi$ occuring in Proposition \ref{theMainEstimatesForWave}, where $F_*$ is the principal symbol $p$ of the pseudodifferential operator we are considering.

%We will find it more geometrically natural to work with the function $\Psi: S_{x_0}^* \to \RR$ given by $\Psi(\xi) = \phi(x,x_0,\xi)$. Then $\Phi( \xi ) = \Psi( \lambda(x_0,\xi) \xi )$. The map $\xi \mapsto \lambda(x_0,\xi) \xi$ is a diffeomorphism from $S^{d-1}$ to $S_{x_0}^*$. Both critical points . Thus it suffices to prove that the only critical points of $\Psi$ occur at $\xi^+$ and $\xi^-$, that $\Psi(\xi_{\pm}) = \pm d_X^{\pm}(x_0,x)$, and that the Hessian of $\Psi$ is appropriately non-degenerate.

%It is simpler to define $\Sigma(x_0,\xi)$ when there is a Riemannian metric $g$ on $U_0$ such that $F(x,v)^2 = \sum g_{ij}(x) v^i v^j$. In this case we take a system of normal coordinates around $x_0$, and define $\Sigma(x_0,\xi)$ to be the hyperplane through the origin in these coordinates perpendicular to $\xi$.  More precisely, we consider the exponential map $\exp: W \to U$, where $W$ is an open subset of $U_0 \times \RR^d$ containing $U_0 \times \{ 0 \}$, defined so that for each $x_0 \in U_0$ and each $u \in \RR^d$, the curve $c(t) = \exp(x_0,tv)$ is a geodesic where it is well defined, i.e. solving the system
%
%\[ \ddot{c}^a = - \sum\nolimits_{j,k} \Gamma^{a}_{jk}(c) \dot{c}^j \dot{c}^k, \]
%
%with the initial conditions $c(0) = x_0$ and $c'(0) = v$. For each $\xi \in \RR^d - \{ 0 \}$, we consider the hyperplane $H_\xi \subset \RR^d$ with normal vector $\xi$, and then define $\Sigma(x_0,\xi) = \exp(x_0,H_\xi)$, which is a hypersurface perpendicular to $\xi$ since $D_v \exp(x_0,0)$ is the identity matrix. % In other words, $\Sigma(x_0,\xi)$ is the hypersurface which, in normal coordinates at $x_0$, forms a hyperplane normal to $\xi$. Since normal coordinates are smooth, and smoothly depend on $x_0$, the family of hypersurfaces $\Sigma(x_0,\xi)$ is then a smooth function of $x_0$ and $\xi$.

%We would like to use the same definition in the Finsler setting. However, the exponential map on a Finsler manifold is in general only everywhere $C^1$. %\footnote[1]{In general the exponential map is only $C^1$ unless $M$ is a \emph{Berwald Finsler manifold}, as in \cite{AkbarZadeh}.}
% Thus the surfaces $\Sigma(x_0,\xi)$ will only be $C^1$, and thus the phase $\phi$ will only be $C^1$, and thus not be smooth enough to yield a parametrix for the half-wave equation.
%
%This argument is not significantly more complex in the Finsler manifold setting than if the operator $P$ gave $M$ a Riemannian metric rather than just a Finsler metric (e.g. for the operators associated with zonal multipliers on the sphere). But we must adapt certain results in the Riemannian manifold literature to the Finsler manifold setting, which does require some extra proof. We relegate these proofs to the appendix when necessary, mentioning citations for the required result in the Riemannian manifold literature that we are adapting.
%A fix is provided by using a kind of normal coordinates at a point adapted to a particular direction, defined in terms of the \emph{autoparallel exponential map} introduced in \cite{Pfeifer}. %, though related to older methods of Douglas and Thomas \cite{Douglas,Thomas}.
%This exponential map is given by $\text{Exp}: W \to U \times \RR^d$, where $W$ is an open subset of $U_0 \times \RR^d \times \RR^d$ containing $U_0 \times \{ 0 \} \times \{ 0 \}$, and smooth away from $U_0 \times \RR^d \times \{ 0 \}$, such that for each $x_0 \in V$, and each $(v,\xi) \in \RR^d \times \RR^d$, the pair of curves $(c(t), s(t)) = \text{Exp}(x_0,tv,\xi)$ solves the system
%%
%\[ \ddot{c}^a = - \sum\nolimits_{j,k} \Gamma^a_{jk}(c,s) \dot{c}^j \dot{c}^k \quad\text{and}\quad \dot{s}^a = - \sum\nolimits_{j,k} \Gamma^a_{jk}(c,s) \dot{c}^j s^k, \]
%
%such that $c(0) = x_0$, $\dot{c}(0) = v$, and where $s(0) = \mathcal{L}^{-1}(x_0,\xi)$. Because of the choice of initial conditions here, if we let $\pi: U \times \RR^d \to U$ be the standard projection map, then $D_v(\pi \circ \text{Exp})(x_0, 0, \xi)$ is the identity, and so the map $G_{x_0,\xi} = (\pi \circ \text{Exp})(x_0,\cdot,\xi)$ is a diffeomorphism onto a neighborhood of $x_0$. % and so $E_{x_0,\xi} = (\pi \circ \exp)(x_0,\cdot,\xi)$ is a diffeomorphism in a neighborhood of the origin. If we assume $V$ to have small enough diameter, we may assume that $E_{x_0,\xi}$ is a diffeomorphism onto an open set containing $V$ for all $x_0$ and $\xi$. % We use this diffeomorphism to define a covector field $\omega$ such that $E_{x_0,\xi}^* \{ \omega \}$ is the constant covector field that everywhere takes the value $\xi$. Now we can obtain a Riemannian metric $h(x_0,\xi, \cdot)$ on $V$ via the coefficients $h^{ij}(x_0, \xi, x) = g^{ij}( x_0, \omega(x_0) )$, where $g^{ij}$ are the coefficients corresponding to Riemannian approximations of the dual metric $F_*$. We finally take a normal coordinate system $F_{x_0,\xi}$ for the metric $h$ and define 
%We define
%
%\[ \Sigma(x_0,\xi) = G_{x_0,\xi}(H_\xi), \]
%
%where $H_\xi$ is the hyperplane in $\RR^d$ perpendicular to $\xi$.

%We note that $G_{x_0,\xi}$ is in general only defined locally, which is why we must work with sufficiently small open sets $U \subset U_0$ such that for all $x_0 \in U$ and $\xi \in \RR^d$, the map $G_{x_0,\xi}$ is a diffeomorphism onto $U$. We will also assume that $U$ is small enough that each pair of points in $U$ is connected by a unique forward geodesic. This is an unproblematic assumption for the application of our arguments to Proposition \ref{theMainEstimatesForWave}, since there we were able to take an arbitrarily fine partition of unity before switching into coordinates.% Since $DE_{x_0,\xi}(0) = I$, the hypersurface $\Sigma(x_0,\xi)$ is perpendicular to $\xi$ at $x_0$, and this definition gives a smooth family of smooth hypersurfaces for any Finsler manifold.

%(TODO: does this formula look better in cotangent coordinates? ). We have need of using this exponential map is because the standard exponential map, defined in terms of geodesics, and which we can write in terms of the autoparallel exponential map as
%
%\[ (x_0,u) \mapsto \exp(x_0, u, u), \]
%
%fails to be smooth when $u = 0$ on a general Finsler manifold. On the other hand, the auto-parallel exponential map $\exp(x_0,u,v)$ is smooth away from $v = 0$. For each $x_0$ and $v \neq 0$, the inverse of the map map $u \mapsto \exp(x_0,u,v)$ gives a smooth coordinate system in a neicghborhood of $x_0$, and provides a substitute for normal coordinates, at least when in comes to geodesics pointing in the direction $v$. If 

%Now we can define $\Sigma(x_0,\xi)$ for $\xi \in T^*_{x_0} M - \{ 0 \}$. Let $v \in \RR^d$ be the Legendre transform of $\xi$ at $x_0$, i.e. the vector $v$ such that $\xi = \sum g_{ij}(x_0,v) v^j$, or equivalently, if we define
%
%\[ g^{ij} = \frac{1}{2} \frac{\partial^2 F_*}{\partial \xi^i \partial \xi^j}, \]
%
%then $v^i = \sum g^{ij}(x_0,\xi) \xi_j$. If we consider the coordinate system given by the inverse of $u \mapsto \exp(x_0,u,v)$, then the hypersurface $\Sigma(x_0,\xi)$ is then precisely the surface which in this coordinate system is the unique \emph{hyperplane} conormal to $\xi$.

%If we fix $\alpha$, then for each $x_0 \in U_\alpha$, and each $\xi \in T^*_{x_0} M$, we will also need to define an additional coordinate system $z(\cdot,x_0,\xi)$ defined in a neighborhood of $x$, which is an analogue of normal coordinates for Finsler manifolds called \emph{Douglas-Thompson normal coordinates}. Like normal coordinates, defined in terms of a system of ordinary differential equations. Namely, we have $z(x,x_0,\xi) = z$, if and only if the unique solution to the system
%
%\[ (c^i)''(t) = - \sum\nolimits_{a,b} \gamma^i_{ab}(c,s) (c^a)'(t) (c^b)'(t) \quad\text{and}\quad s'(t) = - \sum \gamma^i_{ab}(c,s) (c^a)'(t) s^b(t) \]
%
%with the initial conditions $c(0) = x_0$, $c'(0) = z$, and $s(0) = \xi$ has $c(1) = x$. If 

%\begin{remark}
%    Using the notation in Proposition \ref{theMainEstimatesForWave}, if $\alpha: S^{d-1} \to S_x^*$ is given by $\alpha(\xi) = \lambda(x_0,\xi) \xi$, then $(\Psi \circ \alpha)(\xi) = \Phi(\xi)$. Then $\alpha$ is a diffeomorphism, and so it follows from Proposition \ref{triangleLemma} that the critical points of $\Phi$ are equal to $\alpha^{-1}(\pm \xi_0)$, where $\xi_0$ is the unique covector in $S_x^*$ corresponding to the geodesic from $x_0$ to $x_1$. Since the Hessian of a function is a diffeomorphism invariant at critical points of a manifold, it also follows from Proposition \ref{theMainEstimatesForWave} that the Hessian of $\Phi$ at $\alpha^{-1}(\pm \xi_0)$ is non-degenerate, with each eigenvalue having magnitude exceeding a constant multiple of 

%    In the statement of Proposition \ref{triangleLemma}, we have applied a change of coordinates as compared to the application
%\end{remark}

In order to prove \ref{triangleLemma}, we rely on a geometric interpretation of $\Psi$ following from Hamilton-Jacobi theory.

\begin{lemma} \label{HamiltonLemma}
    Consider the setup to Proposition \ref{triangleLemma}. For any $\xi \in S_{x_0}^*$,
    %
    \[ |\Psi(\xi)| = \begin{cases} \text{the length of the shortest curve from $H(x_0,\xi)$ to $x_1$} & : \text{if}\ \Psi(\xi) > 0, \\ \text{the length of the shortest curve from $x_1$ to $H(x_0,\xi)$} & : \text{if}\ \Psi(\xi) < 0. \end{cases} \]
\end{lemma}
\begin{proof}
    We rely on a construction of $\phi$ from Proposition 3.7 of \cite{Treves2}, which we briefly describe. Fix $x_0$ and $\xi$. Then there is a unique covector field $\omega: H(x_0,\xi) \to \RR^d$ which is everywhere perpendicular to $H(x_0,\xi)$, with $\omega(x_0) = \xi$ and with $F_*(x,\omega(x)) = F_*(x_0,\xi)$ for all $x \in H(x_0,\xi)$. There exists a unique point $x(\xi) \in H(x_0,\xi)$ and a unique $t(\xi) \in \RR$, such that the unit speed geodesic $\gamma$ on $X$ with $\gamma(0) = x(\xi)$ and $\gamma'(0) = \mathcal{L}^{-1}( x(\xi), \omega(x(\xi)) )$ satisfies $\gamma( t(\xi) ) = x_1$. We then have $\Psi(\xi) = t(\xi)$. If $t(\xi)$ is negative, then $\gamma|_{[t(\xi),0]}$ is a geodesic from $x_1$ to $x_0$, and if $t(\xi)$ is positive, $\gamma|_{[0,t(\xi)]}$ is a geodesic from $x_0$ to $x_1$. Because $\gamma$ is a geodesic, the geometric interpretation then follows if $U$ is a suitably small neighborhood such that geodesics are length minimizing.
\end{proof}

% and relies on the fact discussed in the introduction that the Hamiltonian flow of the principal symbol $p$ of $P$ on $M$ is the geodesic flow of the Finsler metric induced by $p$ on $M$. Namely, for $\xi \in S_{x_0}$, the value $\psi(\xi)$ is the length of the unique geodesic starting at $x_1$ and ending at a point on $\Sigma(x_0,\xi)$, passing through $\Sigma(x_0,\xi)$ orthogonally. More precisely, there exists a unique point $x(\xi) \in \Sigma(x_0,\xi)$ and a unique value $t(\xi) \in \RR$, such that if $\eta = \eta(\xi)$ is the unique unit covector orthogonal to $\Sigma(x_0,\xi)$ at $x(\xi)$ and oriented outward from $\Sigma(x_0,\xi)$ in the same direction as $\xi$, then $x_1 = \exp_{x(\xi)}(t(\xi) \eta(\xi) )$, and we then have $\psi(\xi) = t(\xi)$.
%



%\begin{figure}[h]
%        \centering
%        \includegraphics[width=0.4\textwidth]{Geodesics.eps}
%\end{figure}

%\noindent That a unique point $x(\xi)$ exists with these properties follows if we choose the diameter of $V$ to be suitably small, depending on the Finsler metric $M$.

% to be smaller than the normal injectivity radius of $\Sigma(x_0,\xi)$ for each $\xi \in S_{x_0}$. That the point $x(\xi)$ lies in $U$ follows because $d(\Sigma(x_0,\xi), x_1) \leq d(x_0,x_1)$, and $N(x_1,d(x_0,x_1)) \subset U$. In particular, if all sectional curvatures of $M$ are upper bounded by $\kappa^2$, then $\psi$ exists if we choose the diameter of $U$ to be smaller then $(\pi/2) \kappa^{-1}$, since, by Lemma 2.3 of \cite{Grimaldi}, $(\pi/2) \kappa^{-1}$ is smaller than the normal injectivity radius of $\Sigma(x_0,\xi)$. In particular, this result is satisfied since the diameter of $U$ is bounded by $2.5 \varepsilon_M$, and $\varepsilon_M < 0.4 (\pi/2) \kappa^{-1}$.

It follows immediately from Lemma \ref{HamiltonLemma} that
%
\begin{equation}
    \Psi(\xi_+) = d_X^+(x_0,x_1) \quad\text{and}\quad \Psi(\xi_-) = - d_X^-(x_0,x_1).
\end{equation}
%
A simple geometric argument also shows $\Psi(\xi^+)$ is the maximum value of $\Psi$ on $S_{x_0}^*$, and $\Psi(\xi^-)$ is the minimum value on $S_{x_0}^*$, so that these two points are both critical. Indeed, the point $x_0$ lies in $H(x_0,\xi)$ for all $\xi \in \RR^d - \{ 0 \}$. Thus the shortest curve from $x_0$ to $x_1$ is always longer than the shortest curve from $H(x_0,\xi)$ to $x_1$. Similarily, the shortest curve from $x_1$ to $x_0$ is always longer than the shortest curve from $x_1$ to $H(x_0,\xi)$. Thus
%
\begin{equation}
    - d_X^-(x_0,x_1) \leq \Psi(\xi) \leq d_X^+(x_0,x_1),
\end{equation}
%
and so $\Psi(\xi^-) \leq \Psi(\xi) \leq \Psi(\xi^+)$ for all $\xi \in $. All that remains is to prove that $\xi^+$ and $\xi^-$ are the \emph{only} critical points of $\Psi$, and that these critical points are appropriately non-degenerate.

%In this first part of the argument, the constant $\varepsilon_X$ need only be smaller than the injectivity radius of the set $U$, which is positive since $U$ is precompact, but in the proof that $\pm \xi_0$ are nondegenerate critical points we may need to pick $\varepsilon_X$ even smaller.

In order to simplify proofs, we employ a structural symmetry to reduce the number of cases we need to analyze. Namely, if one defines the reverse Finsler metric $F_\rho(x,v) = F(x,-v)$, then $F_\rho^*(x,\xi) = F_*(x,-\xi)$, and so the associated function $\Psi^\rho$ which is the analogue of $\Psi$ for $F^\rho$ satisfies $\Psi^\rho(\xi) = -\Psi(-\xi)$. The critical points of $\Psi$ and $\Psi^\rho$ are thus directly related to one another, which allows us without loss of generality to study only points with $\Psi \leq 0$ (and thus only study geodesics beginning at $x_1$).

\begin{proof}[Proof that $\xi^+$ and $\xi^-$ are the only critical points]
    Fix $\xi^* \in T_{x_0} X - \{ \xi^{\pm} \}$. Using the notation defined above, let $x_* = x(\xi^*)$ and $t_* = t(\xi^*)$. Using the symmetry above, we may assume without loss of generality that $\Psi(\xi^*) \leq 0$. Since $\xi^-$ is not perpendicular to $H(x_0,\xi^*)$ at $x_0$, we have $x_* \neq x_0$. If $\Psi(\xi^*) < 0$, let $\gamma$ be the unique unit speed forward geodesic with $\gamma(0) = x_1$ and $\gamma(t_*) = x_*$. If $\Psi(\xi^*) = 0$, let $\gamma$ be the unique unit speed forward geodesic with $\gamma'(0)$ equal to the Legendre transform of $\xi^*$. Pick $\eta$ such that $\eta \cdot (x_* - x_0) \neq 0$, and then, for $t$ suitably close to $t_*$, define a smooth map
    %
    \begin{equation} \xi(t) = \frac{\xi^* + a(t) \eta}{F_*(x_0, \xi^* + a(t) \eta)}, \end{equation}
    %
    into $S_{x_0}^*$, where
    %
    \begin{equation} a(t) = - \frac{\xi^* \cdot ( \gamma(t) - x_0 )}{\eta \cdot ( \gamma(t) - x_0 )} \end{equation}
    %
    is defined so that $\gamma(t) \in H(x_0, \xi(t))$. Then $\Psi( \xi(t) ) = -t$, and differentiation at $t = t_*$ gives $D\Psi( \xi^* ) ( \xi'(t_*) ) = -1$. In particular, $D \Psi( \xi^* ) \neq 0$, so $\xi^*$ is not a critical point. \qedhere
    %
    % gamma(t_*) = x_*
    %   xi(t_*) = xi^*
    %
    %  xi(t) ( gamma(t) - x_0 ) = (t - t_*) [ xi'(t_*) [ gamma(t) - x_0 ] + xi^* gamma'(t) ]
    %   xi(t) = xi^* + a(t) eta
    %   xi(t) ( gamma(t) - x_0 ) = xi^* ( gamma(t) - x_0 ) + a(t) eta ( gamma(t) - x_0 )
    %                               = 0 if a(t) = - xi^* ( gamma(t) - x_0 ) / eta ( gamma(t) - x_0 )
    %   Well defined near xi^* if eta ( x_* - x_0 ) != 0
    %       Derivative in t is
    %           a'(0) = - [ [eta ( x_* - x_0 )] [xi^* gamma'(t_*)] / [ eta (x_* - x_0) ]^2 = - 1 / eta (x_* - x_0)
    %   
    %   Which is good provided that xi^* gamma'(t_*) != 0 and eta (x_* - x_0) != 0
    %
    %   And this is easy because xi^* gamma'(t_*) = 1
    %   So as long as eta is not a constant multiple of xi^* we're good.

    % Define xi(t) = xi^* + a xi^+
%    Let $v(t) = \mathcal{L}(\gamma(t), \gamma'(t))$.


    %
    %

%    For $t$ suitably close to $x_0$, we can find a smooth function $\xi(t) \in S_{x_0}^*$ with $\xi(0) = \xi^*$, and such that $\gamma(t) \in H(x_0, \xi(t))$ for each $t$. Using the fact that $\gamma|_{[t,t_*]}$ is a curve from $H(x_0,\xi(t))$ to $x_1$ of length $t_* - t$, we find that $\Psi(\xi(t)) \leq t_* - t$. Since $\Psi(\xi(0)) = t_*$, taking $t \to 0^+$ gives $D\Psi(\xi^*)(\xi'(0)) \leq -1$. In particular, $D\Psi(\xi^*) \neq 0$, and so $\xi^*$ is not a critical point. \qedhere
    \end{proof}

We now analyze the non-degeneracy of the critical points $\xi^+$ and $\xi^-$.








%It is here that the precise choice of surfaces $\Sigma(x_0,\xi)$ will aid us because to prove non-degeneracy one must look at the second-order behavior of the function $\Psi$, rather than in the last proof where it was only necessary to study first-order behavior of $\Psi$.

    % For v in T_p M, we get a linear isomorphism
    %
    % d exp_{p,v}: T_p M -> T_{exp_p(v)} M
    %
    % Given by differentiating, identifying
    % T_v T_p M with T_p M in the canonical way.
    %
    % d exp_p is *not* an isometry, but it does satisfy
    %
    % < d exp_{p,v}(v), d exp_{p,v}(w) > = < v, w >.
    %
    % Now define the Curvature Tensor
    %
    % R(X,Y)Z = Nabla_X Nabla_Y Z - Nabla_Y Nabla_X Z - Nabla_{[X,Y]} Z
    %
    % Consider alpha(s,t) = exp_p( (cos(s) v + sin(s) w) t )
    %
    % Let T = alpha_*(d_t), J(t) = alpha_*(d_s|_{(0,t)}).
    %
    % Then J is a Jacobi field, with J(t) being a tangent vector at exp_p( vt )
    %
    % J(0) = 0
    % J'(0) = w
    % J''(0) = R(T,J) T|_0 = 0
    % < J,J >''' = R(T,w) T
    %
    % Let f(t) = < J(t), J(t) >_g
    %
    % Then f(0) = < J(0), J(0) > = < 0, 0 > = 0
    %
    % And f'(0) = 2 < J'(0), J(0) > = 2 < w, 0 > = 0
    %
    % And f''(0) = 2 [ < J''(0), J(0) > + < J'(0), J'(0) > ]
    %            = 2 |w|_g^2
    % 
    % And f'''(0) = 2 [ < J'''(0), J(0) > + 3 < J''(0), J'(0) > ]
    %             = 6 < J''(0), w > = 0
    %
    % And f^(4)(0) = 2 < J^(4)(0), J(0) > + 5 < J'''(0), J'(0) > + 3 < J''(0), J''(0) >
    %              = 5 < J'''(0), J'(0) > 
    %              = 5 < R(T,w) T, w >
    %
    % So we conclude that | J(t) |^2 = t^2 + (5/24) < R(T,w) T, w > t^4
    %
    % For small s, s^{-1} [ alpha(t,s) - alpha(t,0) ] should be
    % close to J(t), so we are saying that
    % | alpha(t,s) - alpha(t,0) |^2 = s^2 [ t^2 + (5/24) < R(T,w) T, w > t^4 + O(t^5) ]
    % In particular, we can guarantee that | alpha(t,s) - alpha(t,0)| ~ st
    %
    % Now alpha(d,0) = x_0, and alpha(t,s) = x
    % and for |t - d| <= C s^2
    %
    % Now | alpha(d,s) - x_0 | ~ sd
    %
    % and | alpha(d,s) - x | <= C s^2
    %
    % so |x - x_0| ~ sd - Cs^2 ~ sd
    %
    % Argument why it must be critical: easy argument it's a maximum
    % because x_0 lies on all of the surfaces Sigma(x_0,xi), so
    % the geodesics from x_1 must hit Sigma(x_0,xi) - { x_0 } before
    % hitting x_0.


    % In particular alpha(t,s) - alpha(t,0) = s [ t^2 + (5/24) < R(T,w) T, w > t^4 + O(t^5) ]

    % 
    % exp_p( ( cos(s) v + sin(s) w ) t )
    % exp_p( ( v - O(s^2)v + sw + O(s^3)w ) t )

    % Consider |dexp_{p,v}(tw)|^2, where v and w are orthonormal. Then
    %
    %   |dexp_{p,v}(tw)|^2 = t^2 - [< R(w, v ) v, w > / 3] t^4 + O(t^5)
    %
    % By polarization identity
    %
    %   < x, y > = (1/2) [  |x + y|^2 - |x|^2 - |y|^2   ]
    %
    % < A(tw),  

\begin{proof}[Proof that $\xi^+$ and $\xi^-$ are non-degenerate]
    Using symmetry, it suffices without loss of generality to analyze the critical point $\xi^-$ rather than $\xi^+$. Let $H_-: T_{\xi^-} S_{x_0}^* \to \RR$ be the Hessian of $\Psi$ at $\xi^-$. Let $v_0 = \mathcal{L}_{x_0}^{-1} \xi^-$, and using the notation of the last argument, let $l = t(\xi^-)$. Consider a curve $\xi(a)$ valued in $S_{x_0}^*$ with $\xi(0) = \xi^-$ and $\zeta = \xi'(0)$ for some $\zeta \in T_{\xi^-}^* S_{x_0}$. Then the second derivative of $\Psi(\xi(a))$ at $a = 0$ is $H_-( \zeta )$, and so our proof would be complete if we could show that the function $L(a) = - \Psi( \xi(a) )$, which is the length of the shortest geodesic from $x_1$ to $H(x_0,\xi(a))$, satisfies $L''(0) \geq C l |\zeta|$, for a constant $C > 0$ uniform in $x_0$, $x_1$, and $\zeta$.

    Consider the partial function $\text{Exp}_{x_1}: \RR^d \to U_0$ obtained from the exponential map at $x_1$. If $r$ is chosen suitably small, there exists a neighborhood $V$ of the origin in $\RR^d$ such that $E = \text{Exp}_{x_1}|_V$ is a diffeomorphism between $V$ and $U$. Unlike in Riemannian manifolds, in general $E$ is only $C^1$ at the origin, though smooth on $V - \{ 0 \}$. However, for $|v| = 1$ and $t > 0$ we can write $E(tv) = (\pi \circ \varphi)( x_1, \mathcal{L}_{x_1} v, t)$, where the partial function $\varphi: (T^* U_0 - 0) \times \RR \to (T^* U_0 - 0)$ is the flow induced by the the Hamiltonian vector field $( \partial_\xi F_*, - \partial_x F_*)$ on $T^* U_0 - 0$, and $\pi$ is the projection map from $T^* U_0 - 0$ to $U_0$. Where defined, $\varphi$ is a smooth function since the Hamiltonian vector field is smooth, and so it follows by homogeneity and the precompactness of $U$ that the partial derivatives $(\partial^\alpha E_x)(v)$ are uniformly bounded for $v \neq 0$ and $x \in U$. It thus follows from the inverse function theorem that there exists a constant $A > 0$ such that for all $x_1$ and $x$ in $U$ with $x \neq x_1$,
    %
    \begin{equation} \label{GBounds}
        |\partial_j G_{x_1}(x)| \leq A \quad\text{and}\quad |(\partial_j \partial_k G_{x_1})(x)| \leq A.
    \end{equation}
    %
    We can also pick $A$ to be large enough that for all $x \in U$, and all $v,w \in \RR^d - \{ 0 \}$,
    %
    \begin{equation}
        A^{-1} |w|^2 \leq \sum\nolimits_{ij} g_{ij}(x,v) w^i w^j \leq A |w|^2.
    \end{equation}
    %
    and such that for all $x \in U$ and $v \in \RR^d - \{ 0 \}$,
    %
    \begin{equation} \label{gijbounds}
        |\partial_{x_k} g_{ij}(x,v)| \leq A.
    \end{equation}
    %
    %With notation now setup, we now prove $H_-$ is appropriately nondegenerate.

    Since $\zeta \in T_{\xi^-} S_{x_0}^*$, and $S_{x_0}^* = \{ \xi : F_*(x_0,\xi)^2 = 1 \}$, by Euler's homogeneous function theorem we have
    %
    \begin{equation} \label{DIOAWJDIOWAJDOIWAJ14}
        \sum\nolimits_{ij} g^{ij}(x_0,\xi^-) \xi'(0) \xi^-_j = \frac{1}{2} \sum\nolimits_{i,j} \frac{\partial^2 F_*^2}{\partial \xi_i \partial \xi_j}(x_0,\xi^-) \zeta_i \xi^-_j = \frac{1}{2} \sum \frac{\partial F_*^2}{\partial \xi_j}(x_0,\xi^-) \zeta_j = 0.
    \end{equation}
    %
    Differentiating $F_*(x_0,\xi(a))^2 = 1$ twice with respect to $a$ at $a = 0$ yields that
    %
    \begin{equation} \label{AIWODJWAO314213}
         \sum\nolimits_{i,j} g_{ij}(x_0,\xi^-) \xi''_i(0) \xi^-_j = - \sum\nolimits_{i,j} g_{ij}(x_0,\xi^-) \zeta^i \zeta^j
    \end{equation}
    %
    Define vectors $n(a)$ by setting $n^i(a) = \sum g^{ij}(x_0,\xi^-) \xi_j(a)$. Then $n(a)$ is the normal vector to $H(x_0, \xi(a))$ with respect to the inner product with coefficients $g_{ij}(x_0,v_0)$.
    %
%    \begin{equation}
%        H(x_0,\xi(a)) = \left\{ x : \sum\nolimits_{ij} g_{ij}(x_0,v_0) (x^i - x_0^i) n^j(a) = 0 \right\}.
%    \end{equation}
    %
    Let $u(a)$ be the orthogonal projection of $v_0$ onto the hyperplane $\{ v : \xi(a) \cdot v = 0 \}$ with respect to the inner product $g_{ij}(x_0,v_0)$. If we define
    %
    \begin{equation}
        c(a) = \sum g_{ij}(x_0,v_0) v_0^i n^j(a) = \sum g^{ij}(x_0,\xi^-) \xi^-_i \xi_j(a),
    \end{equation}
    %
    then $u(a) = v_0 - c(a) n(a)$. Note that \eqref{DIOAWJDIOWAJDOIWAJ14} and \eqref{AIWODJWAO314213} imply $c(0) = 1$, $c'(0) = 0$, and $c''(0) \leq - |\zeta|^2 / A$. Also $n(0) = v_0$ and $|n'(0)| \leq A |\zeta|$.
    %
    % 1 - c(a)^2 = 1 - ( 1 - a^2 |zeta|^2 / A )^2
    %            = 2 a^2 |zeta|^2 / A

    Let $x(s) = x_0 + s u(a)$ and let $R(s)$ be the length of the geodesic from $x_1$ to $x(s)$. To control $R(s)$, define $y(s) = G(x(s))$, and consider the variation $A(s,t) = E( t y(s) )$, defined so that $t \mapsto A(s,t)$ is the geodesic from $x_1$ to $x(s)$. The Gauss Lemma for Finsler manifolds (see Lemma 6.1.1 of \cite{BaoChern}) implies that
    %
    \begin{equation}
        A^{-1} R(s) \leq |y(s)| \leq A R(s).
    \end{equation}
    %
    Define $T(s) = (\partial_t A)(s,1)$ and $V(s) = (\partial_s A)(s,1)$.
    %
%    \begin{equation}
%        T(s) = (\partial_t A)(s,1) = \sum\nolimits_{i,j} \frac{\partial E^i}{\partial y^j}( y(s) ) y^j(s) e_i \quad\text{and}\quad V(s) = (\partial_s A)(s,1) = u(a).
%    \end{equation}
    %
    Then the first variation formula for geodesics implies
    %
    \begin{equation} \label{RSFirstDerivativeEquation}
        R(s) R'(s) = \sum\nolimits_{i,j} g_{ij}( x(s), T(s) ) V^i(s) T^j(s).
    \end{equation}
    %
    Again, the Gauss Lemma implies
    %
    \begin{equation} \label{Idiawjdiwaj213123}
        A^{-1} R(s) \leq |T(s)| \leq A R(s).
    \end{equation}
    %
    We can write
    %
    \begin{equation}
        T^i(s) = \sum\nolimits_{i,j} (\partial_j E^i)( y(s)) y^j(s)
    \end{equation}
    %
    and
    %
    \begin{equation}
        V^i(s) = \sum\nolimits_{i,j,k} (\partial_j E^i) (y(s)) (\partial_k G^j)( x(s)) u^k(a) = u^i(a).
    \end{equation}
    %
    In particular, $T(0) = v_0$, so $R'(0) = \sum g_{ij}(x_0, v_0) u^i(a) v_0^j = 1 - c(a)^2$. Cauchy-Schwartz applied to \eqref{RSFirstDerivativeEquation} also tells us that
    %
    \begin{equation} \label{wdkawoidj141}
        |R'(s)| \lesssim_A |u(a)|.
    \end{equation}
    %
    Note that $u(0) = 0$, so if $a$ is small enough, then we have $|u(a)| \leq l/2$. Taylor's theorem applied to \eqref{wdkawoidj141}, noting $R(0) = l$ then gives that for $|s| \leq 1$,
    %
    \begin{equation} \label{RBound}
        l/2 \leq R(s) \leq 2l.
    \end{equation}
    %
    Differentiating \eqref{RSFirstDerivativeEquation} tells us that
    %
    \begin{equation} \label{FFFF213123}
    \begin{split}
        R'(s)^2 + R''(s) R(s) &= \sum\nolimits_{i,j,k} \Bigg[ (\partial_{x_k} g_{ij})( x(s), T(s) ) u^i(a) T^j(s) u^k(a) \Bigg]\\
        &\quad\quad\quad\quad + \Bigg[ (\partial_{v_k} g_{ij} )(x(s), T(s)) u^i(a) T^j(s) (\partial_s T^k)(s)   \Bigg]\\
        &\quad\quad\quad\quad\quad\quad + \Bigg[ g_{ij}( x(s), T(s) ) u^i(a) (\partial_s T^j)(s) \Bigg].
    \end{split}
    \end{equation}
    %
    Write the right hand side as $\text{I} + \text{II} + \text{III}$. Using \eqref{gijbounds}, \eqref{Idiawjdiwaj213123}, \eqref{RBound}, and the triangle inequality gives
    %
    \begin{equation} \label{IBound}
        |\text{I}| \lesssim R(s) |u(a)|^2 \lesssim l |u(a)|^2.
    \end{equation}
    %
    Since $\partial_{v_k} g_{ij} = (1/2) (\partial^3F^2 / \partial v_i \partial v_j \partial v_k)$, applying Euler's homogeneous function theorem when summing over $j$ implies that
    %
    \begin{equation} \label{IIBound}
        \text{II} = 0.
    \end{equation}
    %
    Applying Cauchy-Schwarz and \eqref{GBounds}, we find
    %
    \begin{equation} \label{IIIBound}
        |\text{III}| \lesssim |u(a)| |T'(s)| \lesssim |u(a)|^2 [|y(s)| + 1] \lesssim (l + 1) |u(a)|^2.
    \end{equation}
    %
    But now combining \eqref{IBound}, \eqref{IIBound}, \eqref{IIIBound}, and rearranging \eqref{FFFF213123} shows that
    %
    \begin{equation}
        |R''(s)| \lesssim l^{-1} |u(a)|^2 + (1 + l^{-1}) |u(a)|^2 \lesssim l^{-1} |u(a)|^2.
    \end{equation}
    %
    Taylor's theorem implies there exists $B > 0$ depending only on $A$ and $d$ such that
    %
    \begin{equation} \label{WADAWD21312}
        |R(s) - (R(0) + s R'(0))| \leq B l^{-1} |u(a)|^2 s^2.
    \end{equation}
    %
    Since $R(0) = $l, $R'(0) = 1 - c(a)^2$, $c(0) = 1$, $u(0) = 0$, $c'(0) = 0$, $|u'(0)| \leq A |\zeta|$, and $c''(0) \leq - |\zeta|^2 / A$, we conclude from \eqref{WADAWD21312} that as $a \to 0$, if $s > 0$ then
    %
    \begin{equation}
    \begin{split}
        R(-s) &\leq l - s R'(0) + B l^{-1} |u(a)|^2 s^2\\
        &\leq l - s ( A^{-1} |\zeta|^2 a^2 ) + A^2 B l^{-1} |\zeta|^2 s^2 a^2 + O(a^3 ( s + l^{-1} s^2)).
    \end{split}
    \end{equation}
    %
    % |u(a)|^2 = sum u^i(a)^2
    % Derivative is 2 u . u'(a) = 0
    % Derivative is 2 u' u' <= A^2 |zeta|^2
    For all $s$, $L(a) \leq R(s)$. Optimizing by picking $s = l / 2 A^3 B$ gives
    %
    \begin{equation}
        L(a) \leq R(-s) \leq l - l |\zeta|^2 a^2 / 4 A^4 B + O( a^3 ).
    \end{equation}
    %
    Taking $a \to 0$ and using that $L(0) = l$ gives that $L''(0) \leq - l |\zeta|^2 / 4 A^4 B$, so setting $C = 1/4 A^4 B$, we find we have proved what was required.
\end{proof}

\section{Analysis of Regime I via Density Methods} \label{regime1firstsection}

We now begin obtaining bounds for the operator $T^I$ specified in Proposition \ref{TjbLemma} by using the quasi-orthogonality estimates of Proposition \ref{theMainEstimatesForWave}. Define a metric $d_X = d_X^+ + d_X^-$ on $X$. Given an input $u: X \to \CC$, we consider a maximal $1/R$ separated subset $\mathcal{X}_R$ of $X$, and then consider a decomposition $u = \sum_{x_0 \in \mathcal{X}_R} u_{x_0}$ with respect to some partition of unity, where $\text{supp}(u_{x_0}) \subset B(x_0,1/R)$. The balls $\{ B(x_0,1/R) : x_0 \in \mathcal{X}_R \}$ have finite overlap, and so
%
\begin{equation}
    \| u \|_{L^p(X)} \sim \left( \sum\nolimits_{x_0 \in \mathcal{X}_R} \| u_{x_0} \|_{L^p(X)}^p \right)^{1/p}.
\end{equation}
%
If we set $f_{x_0,t_0} = T_{t_0}^I \{ u_{x_0} \}$, then
%
\begin{equation} \label{DAPOCJAPWOCJAWOIFJOI}
    \left\| T^I u \right\|_{L^p(X)} = \left\| \sum\nolimits_{(x_0,t_0) \in \mathcal{X}_R \times \mathcal{T}_R} f_{x_0,t_0} \right\|_{L^p(X)}.
\end{equation}
%
In this subsection, we use the quasi-orthogonality estimates of the last section to obtain $L^2$ estimates on partial sums of a family of functions of the form $\{ {S\!}_{x_0,t_0} \}$, which are essentially $L^1$ normalized versions of the functions $\{ f_{x_0, t_0} \}$, under a density assumption on the set of indices we are summing over. Namely, we say a set $\mathcal{E} \subset \mathcal{X}_R \times \mathcal{T}_R$ has \emph{density type} $(A_0,A_1)$ if for any set $B \subset \mathcal{X}_R \times \mathcal{T}_R$ with $1/R \leq \text{diam}(B) \leq A_1/R$,
%
\begin{equation}
    \#(\mathcal{E} \cap B) \leq R A_0\; \text{diam}(B). \footnote[1]{This definition of density is chosen because it is `scale-invariant' as we change the parameter $R$, matching the definition given in Section \ref{sec:densitydecompositions} when $R = 1$. Indeed, if $X = \RR^d$, $\mathcal{X}_R = (\ZZ / R)^d$, and $\mathcal{T}_R = \ZZ/R$, then a set $\mathcal{E} \subset (\ZZ / R)^d \times (\ZZ/R)$ has density type $(A_0,A_1)$ if and only if $R\; \mathcal{E} \subset \ZZ^d \times \ZZ$ has density type $(A_0,A_1)$.}
\end{equation}
%
To obtain $L^p$ bounds from these $L^2$ bounds, in the next section we will perform a \emph{density decomposition} to break up $\mathcal{X}_R \times \mathcal{T}_R$ into families of indices with controlled density, and then apply Proposition \ref{L2DensityProposition} on each subfamily to control \eqref{DAPOCJAPWOCJAWOIFJOI} via an interpolation.

\begin{prop} \label{L2DensityProposition}
    Fix $A \geq 1$. Consider a set $\mathcal{E} \subset \mathcal{X}_R \times \mathcal{T}_R$. Suppose that for each $(x_0,t_0) \in \mathcal{E}$, we pick two measurable functions $b_{t_0}: I_0 \to \RR$ and $u_{x_0}: X \to \RR$, supported on $I_{t_0}$ and $B(x_0,1/R)$ respectively, such that $\| b_{t_0} \|_{L^1(I_0)} \leq 1$ and $\| u_{x_0} \|_{L^1(X)} \leq 1$. Define 
    %
    \[ {S\!}_{x_0,t_0} = \int b_{t_0}(t) (e^{2 \pi i t P} \circ Q_R) u_{x_0}\; dt. \]
    %
    Write $\mathcal{E} = \bigcup_{k = 0}^\infty \mathcal{E}_k$, where
    %
    \[ \mathcal{E}_0 = \{ (x,t) \in \mathcal{E}: 0 \leq |t| \leq 1/R \} \]
    %
    and for $k > 0$, define
    %
    \[ \mathcal{E}_k = \{ (x,t) \in \mathcal{E}: 2^{k-1} / R < |t| \leq 2^k / R \}. \]
    %
    Suppose that for each $k$, the set $\mathcal{E}_k$ has density type $(A,2^{k})$.
    %, i.e. so that for any set $B \subset \mathcal{X}_R \times \mathcal{T}_R$ with $\text{diam}(B) \leq 2^{k}/R$,
    %
%    \begin{equation}
%        \#( \mathcal{E}_k \cap B ) \leq R A\; \text{diam}(B).
%    \end{equation}
    %
    Then
    %
    \[ \Big\| \sum\nolimits_k \sum\nolimits_{(x_0,t_0) \in \mathcal{E}_k} 2^{k \frac{d-1}{2}} {S\!}_{x_0,t_0} \Big\|_{L^2(X)}^2 \lesssim R^d \log(A) A^{\frac{2}{d-1}} \sum\nolimits_k 2^{k(d-1)} \# \mathcal{E}_k. \]
\end{prop}

\begin{remark}
    If $\| b_{t_0} \|_{L^1(I_0)} \sim 1$ and $\| u_{x_0} \|_{L^1(X)} \sim 1$, then locally constancy from the uncertainty principle and energy conservation of the wave equation tell us that morally,
    %
    \begin{equation}
        \| S_{x_0,t_0} \|_{L^2(X)}^2 \sim R^d t_0^{d-1}.
    \end{equation}
    %
    If $\| b_{t_0} \|_{L^1(I_0)} \sim 1$ and $\| u_{x_0} \|_{L^1(X)} \sim 1$ for all $(x_0,t_0) \in \mathcal{E}$, this means that Proposition \ref{L2DensityProposition} is morally equivalent to
    %
    \begin{equation}
        \Big\| \sum\nolimits_{(x_0,t_0) \in \mathcal{E}} {S\!}_{x_0,t_0} \Big\|_{L^2(X)} \lesssim \sqrt{\log(A)} A^{\frac{1}{d-1}} \left( \sum\nolimits_{(x_0,t_0) \in \mathcal{E}} \| {S\!}_{x_0,t_0} \|_{L^2(X)}^2 \right)^{1/2}.
    \end{equation}
    %
    Thus Proposition \ref{L2DensityProposition} is a kind of square root cancellation bound, albeit with an implicit constant which grows as the set $\mathcal{E}$ increases in density, a necessity given that the functions $\{ {S\!}_{x_0,t_0} \}$ are not almost-orthogonal to one another.
\end{remark}

\begin{proof}
Write $F = \sum_k F_k$, where
%
\begin{equation}
    F_k = 2^{k \frac{d-1}{2}} \sum\nolimits_{(x_0,t_0) \in \mathcal{E}_k} {S\!}_{x_0,t_0}.
\end{equation}
%
Our goal is to bound $\| F \|_{L^2(M)}$. Applying Cauchy-Schwarz, we have
%
\begin{equation} \label{loglossbound}
    \| F \|_{L^2(X)}^2 \leq \log(A) \left( \sum\nolimits_{k \leq \log(A)} \| F_k \|_{L^2(X)}^2 + \left\| \sum\nolimits_{k \geq \log(A)} F_k \right\|_{L^2(X)}^2 \right).
\end{equation}
%
Without loss of generality, increasing the implicit constant in the final result by applying the triangle inequality, we can assume that $\{ k : \mathcal{E}_k \neq \emptyset \}$ is $10$-separated, and that all values of $t$ with $(x,t) \in \mathcal{E}$ are positive. Thus if $F_k$ and $F_{k'}$ are both nonzero functions, then $k = k'$ or $|k - k'| \geq 10$.

%We consider an expansion
%
%\begin{equation} \label{AWIOJDIOWAJF190124214}
%    \left\| \sum\nolimits_{k \geq \log(A)} F_k \right\|_{L^2(M)}^2 = \sum\nolimits_{k,k' \geq \log(A)} \langle F_k, F_{k'} \rangle.
%\end{equation}
%
Let us estimate $\langle F_k, F_{k'} \rangle$ for $k \geq k' + 10$.
%the non-diagonal terms on the right hand side of \eqref{AWIOJDIOWAJF190124214}.
We write
%
\begin{equation}
    \langle F_k, F_{k'} \rangle = \sum\nolimits_{(x_0,t_0) \in \mathcal{E}_k} \sum\nolimits_{(x_1,t_1) \in \mathcal{E}_{k'}} 2^{k \frac{d-1}{2}} 2^{k' \frac{d-1}{2}} \langle {S\!}_{x_0,t_0}, {S\!}_{x_1,t_1} \rangle.
\end{equation}
%
For each $(x_0,t_0) \in \mathcal{E}_k$, and each $k' \leq k - 10$, consider the set
%
\begin{equation}
    \mathcal{G}_0(x_0,t_0,k') = \{ (x_1,t_1) \in \mathcal{E}_{k'} : |(t_0 - t_1) - d_X^-(x_0,x_1)| \leq 2^{k' + 5} / R \},
\end{equation}
%
Also consider the sets of indices
%
\begin{equation}
    \mathcal{G}_l^+(x_0,t_0,k') = \{ (x_1,t_1) \in \mathcal{E}_{k'} : 2^{l} / R < |(t_0 - t_1) + d_X^+(x_0,x_1)| \leq 2^{l + 1} / R \}.
\end{equation}
%
and
%
\begin{equation}
    \mathcal{G}_l^-(x_0,t_0,k') = \{ (x_1,t_1) \in \mathcal{E}_{k'} : 2^{l} / R < |(t_0 - t_1) - d_X^-(x_0,x_1)| \leq 2^{l + 1} / R \}.
\end{equation}
%
If we set
%
\begin{equation}
    \mathcal{G}_0(x_0,t_0,k') = (\mathcal{G}_l^+(x_0,t_0,k') \cup \mathcal{G}_l^-(x_0,t_0,k'))
\end{equation}
%
and
%
\begin{equation}
\begin{split}
    \mathcal{G}_l(x_0,t_0,k') &= \Big( \mathcal{G}_l^+(x_0,t_0,k') \cup \mathcal{G}_l^-(x_0,t_0,k') \Big)\\
    &\quad\quad\quad\quad\quad\quad - \bigcup\nolimits_{r < l} \Big(\mathcal{G}_r^+(x_0,t_0,k') \cup \mathcal{G}_r^-(x_0,t_0,k') \Big).
\end{split}
\end{equation}
%
Then $\mathcal{E}_{k'}$ is covered by $\mathcal{G}_0(x_0,t_0,k')$ and $\mathcal{G}_l(x_0,t_0,k')$ for $k' + 5 \leq l \leq 10 \log R$. Define
%
\begin{equation}
    B_0(x_0,t_0,k') = \sum\nolimits_{(x_1,t_1) \in \mathcal{G}_0(x_0,t_0,k')} 2^{k \frac{d-1}{2}} 2^{k' \frac{d-1}{2}} |\langle {S\!}_{x_0,t_0}, {S\!}_{x_1,t_1} \rangle|,
\end{equation}
%
and
%
\begin{equation}
    B_l(x_0,t_0,k') = \sum\nolimits_{(x_1,t_1) \in \mathcal{G}_l(x_0,t_0,k')} 2^{k \frac{d-1}{2}} 2^{k' \frac{d-1}{2}} |\langle {S\!}_{x_0,t_0}, {S\!}_{x_1,t_1} \rangle|.
\end{equation}
%
We thus have
%
\begin{equation}
    \langle F_k, F_{k'} \rangle \leq \sum\nolimits_{(x_0,t_0) \in \mathcal{E}_k} B_0(x_0,t_0,k') + \sum\nolimits_{(x_0,t_0) \in \mathcal{E}_k} \sum\nolimits_{k' + 5 \leq l \leq 10 \log R} B_l(x_0,t_0,k').
\end{equation}
%
Using the density properties of $\mathcal{E}$, we can control the size of the index sets $\mathcal{G}_{\bullet}(x_0,t_0,k')$, and thus control the quantities $B_\bullet(x_0,t_0,k')$. The rapid decay of Proposition \ref{theMainEstimatesForWave} means that only $\mathcal{G}_0(x_0,t_0,k')$ needs to be estimated rather efficiently:
%
\begin{itemize}%[leftmargin=8mm]
    \item Start by bounding the quantities $B_0(x_0,t_0,k')$. If $(x_1,t_1) \in \mathcal{G}_0(x_0,t_0,k')$, then
    %
    \begin{equation}
        |d_X^-(x_0,x_1) - (t_0 - t_1)| \leq 2^{k'+5}/R,
    \end{equation}
    %
    Thus if we consider $\text{Ann}(x_0,t_0,k') = \{ x_1: |d_X^-(x_0,x_1) - t_0| \leq 2^{k'+8}/R \}$, which is a geodesic annulus of radius $\sim 2^k / R$ and thickness $O(2^{k'}/R)$, then
    %
    \begin{equation}
        \mathcal{G}_0(x_0,t_0,k') \subset \text{Ann}(x_0,t_0,k') \times [ 2^{k'}/R, 2^{k'+1}/R ].
    \end{equation}
    %
    The latter set is covered by $O(2^{(k-k')(d-1)})$ balls of radius $2^{k'}/R$, and so the density properties of $\mathcal{E}_{k'}$ implies that
    %
    \begin{equation} \label{G0Size}
        \#( \mathcal{G}_0(x_0,t_0,k') ) \lesssim A 2^{(k-k')(d-1)} 2^{k'}. 
    \end{equation}
    %
    Since $k \geq k' + 10$, for $(x_1,t_1) \in \mathcal{G}_0(x_0,t_0,k')$ we have $d_X(x_0,x_1) \gtrsim 2^k / R$ and so
    %
    \begin{equation} \label{InnerProductG0Size}
        \langle S_{x_0,t_0}, S_{x_1,t_1} \rangle \lesssim R^d 2^{-k \left( \frac{d-1}{2} \right)}.
    \end{equation}
    %
    But putting together \eqref{G0Size} and \eqref{InnerProductG0Size} gives that
    %
    \begin{align*}
        B_0(x_0,t_0,k') &\leq (2^{k \frac{d-1}{2}} 2^{k' \frac{d-1}{2}}) ( A 2^{(k-k')(d-1)} 2^{k'} )  ( R^d 2^{-k \left( \frac{d-1}{2} \right)} )\\
        &= A R^d 2^{k(d-1)} 2^{-k' \frac{d-3}{2}}.
    \end{align*}
    %
    Thus for each $k$, since $d \geq 4$,
    % d = 2: get an extra O( log k ) factor
    \begin{equation} \label{AAAlowbounds}
        \sum\nolimits_{(x_0,t_0) \in \mathcal{E}_k} \sum\nolimits_{k' \in [\log(A), k - 10]} B_0(x_0,t_0,k') \lesssim R^{d} 2^{k (d-1)} \# \mathcal{E}_k.
    \end{equation}

    \item Next we bound $B_l(x_0,t_0,k')$ for $k' + 5 \leq l \leq k - 5$. The set $\mathcal{G}_l^+(x_0,t_0,k')$ is empty in this case. Thus
    %
    \begin{equation}
        \mathcal{G}_l(x_0,t_0,k') \subset ( \text{Ann} \cup \text{Ann}' ) \times [t_0 - 2^{k'} / R, t_0 + 2^{k'} / R],
    \end{equation}
    %
    where
    %
    \begin{equation}
        \text{Ann} = \{ x \in X: |d_X^-(x_0,x) - (t_0 - 2^{k'}) / R| \leq 100 \cdot 2^l / R \}
    \end{equation}
    %
    and
    %
    \begin{equation}
        \text{Ann}' = \{ x \in X: |d_X^-(x_0,x) - (t_0 + 2^{k'}) / R| \leq 100 \cdot 2^l / R \},
    \end{equation}
    %
    These are geodesic annuli of thickness $O(2^l / R)$ and radius $\sim 2^k$. Thus $\mathcal{G}_l(x_0,t_0,k')$ is covered by $O( 2^{(l-k')} 2^{(k-k')(d-1)} )$ balls of radius $2^{k'} / R$, and the density of $\mathcal{E}_{k'}$ implies that
    %
    \begin{equation}
        \#(\mathcal{G}_l(x_0,t_0,k')) \lesssim R A\; 2^{(l-k')} 2^{(k-k')(d-1)} 2^{k'} / R = A 2^{l} 2^{(k-k')(d-1)}.
    \end{equation}
    %
    For $(x_1,t_1) \in \mathcal{G}_l(x_0,t_0,k')$, $d_X(x_0,x_1) \sim 2^k / R$, and thus Proposition \ref{theMainEstimatesForWave} implies
    %
    \begin{equation}
        |\langle {S\!}_{x_0,t_0}, {S\!}_{x_1,t_1} \rangle| \lesssim R^d 2^{-k \frac{d-1}{2}} 2^{-lK}.
    \end{equation}
    %
    Thus for any $K \geq 0$,
    %
    \begin{equation}
    \begin{split}
        B_l(x_0,t_0,k') &\lesssim_K \Big( A 2^{l} 2^{(k-k')(d-1)} \Big)  R^{d} 2^{k \frac{d-1}{2}} 2^{k' \frac{d-1}{2}} \Big( 2^{-k \frac{d-1}{2}} 2^{-lK} \Big)\\
        &\lesssim A R^d 2^l 2^{k(d-1)} 2^{-k' \frac{d-1}{2}} 2^{-lK}.
    \end{split}
    \end{equation}
    %
    Picking $K > 1$, we conclude that
    % -d/2 + 1/2
    \begin{equation} \label{AAAlBoundSmall}
        \sum_{(x_0,t_0) \in \mathcal{E}_k} \sum_{k' \in [\log(A), k - 10]} \sum_{l \in [k' + 10, k - 5]} B_l(x_0,t_0,k') \lesssim R^{d} 2^{k (d-1)} \# \mathcal{E}_k.
    \end{equation}

    \item Finally, let's bound $B_l(x_0,t_0,k')$ for $k - 5 \leq l \leq 10 \log R$. If either $(x_1,t_1) \in \mathcal{G}_l^-(x_0,t_0,k')$ or $(x_1,t_1) \in \mathcal{G}_l^+(x_0,t_0,k')$, then $d_X(x_0,x_1) \lesssim 2^l / R$. So $\mathcal{G}_l(x_0,t_0,k')$ is covered by $O( 2^{(l-k')d} )$ balls of radius $2^{k'} / R$, and thus
    %
    \begin{equation}
        \#(\mathcal{G}_l(x_0,t_0,k')) \lesssim R A\; 2^{(l-k')d} (2^{k'} / R) = A 2^{(l-k')d} 2^{k'}.
    \end{equation}
    %
    For $(x_1,t_1) \in \mathcal{G}_l(x_0,t_0,k')$, we have no good control over $d_X(x_0,t_1)$ aside from the trivial estimate $d_X(x_0,x_1) \lesssim 1$. Thus Proposition \ref{theMainEstimatesForWave} yields a bound of the form
    %
    \begin{equation}
        |\langle {S\!}_{x_0,t_0}, {S\!}_{x_1,t_1} \rangle| \lesssim R^d 2^{-lK}.
    \end{equation}
    %
    Thus we conclude that
    %
    \begin{equation}
    \begin{split}
        B_l(x_0,t_0,k') &\lesssim_N R^{d} 2^{k \frac{d-1}{2}} 2^{k' \frac{d-1}{2}} \Big( A 2^{(l-k')d} 2^{k'} \Big) \Big( 2^{-lN} \Big)\\
        &= A R^d 2^{k \frac{d-1}{2}} 2^{-k' \frac{d-1}{2}} 2^{-lN}
    \end{split}
    \end{equation}
    %
    Picking $K > d$, we conclude that
    %
    \begin{equation} \label{AAAlBoundBig}
        \sum\nolimits_{(x_0,t_0) \in \mathcal{E}_k} \sum\nolimits_{k' \in [\log(A), k - 10]} \sum\nolimits_{l \in [k+10,\log R]} B_l(x_0,t_0,k')  \lesssim R^d.
    \end{equation}
\end{itemize}
%
The three bounds \eqref{AAAlowbounds}, \eqref{AAAlBoundSmall} and \eqref{AAAlBoundBig} imply that
%
\begin{equation} \label{DADAOWIDJAWOIDJAWDaweq13412}
    \sum\nolimits_k \sum\nolimits_{k' \in [\log(A), k]} |\langle F_k, F_{k'} \rangle| \lesssim R^d \sum\nolimits_k 2^{k (d-1)} \# \mathcal{E}_k.
\end{equation}
%
In particular, combining \eqref{DADAOWIDJAWOIDJAWDaweq13412} with \eqref{loglossbound}, we have
%
\begin{equation} \label{DPOWADPAWKDPOWAKDOPWAK}
    \| F \|_{L^2(X)}^2 \lesssim \log(A) \left( \sum\nolimits_k \| F_k \|_{L^2(X)}^2 + R^d \sum\nolimits_k 2^{k (d-1)} \# \mathcal{E}_k \right).
\end{equation}
%
Next, consider some parameter $a$ to be determined later, and decompose the interval $[2^{k} / R, 2^{k+1} / R]$ into the disjoint union of length $A^a / R$ intervals of the form
%
\begin{equation}
    I_{k,\mu} = [ 2^{k} / R + (\mu - 1) A^a / R, 2^{k} / R + \mu A^a / R] \quad\text{for $1 \leq \mu \leq 2^k/A^a$}.
\end{equation}
%
We thus consider a further decomposition $\mathcal{E}_k = \bigcup \mathcal{E}_{k,\mu}$, where $F_k = \sum F_{k,\mu}$. As before, increasing the implicit constant in the Proposition, we may assume without loss of generality that the set $\{ \mu: \mathcal{E}_{k,\mu} \neq \emptyset \}$ is $10$-separated. We now estimate
%
\begin{equation}
    \sum\nolimits_{\mu \geq \mu' + 10} |\langle F_{k,\mu}, F_{k,\mu'} \rangle|.
\end{equation}
%
For $(x_0,t_0) \in \mathcal{E}_{k,\mu}$ and $l \geq 1$, define
%
\begin{equation}
    \mathcal{H}_l(x_0,t_0,\mu') = \Big\{ (x_1,t_1) \in \mathcal{E}_{k,\mu'} : \frac{2^l A^a}{2R} \leq \max(d_X(x_0,x_1), t_0 - t_1) \leq \frac{2^l A^a}{R} \Big\}.
\end{equation}
%
Then $\bigcup_{l \geq 1} \mathcal{H}_l(x_0,t_0,\mu')$ covers $\bigcup_{\mu \geq \mu' + 10} \mathcal{E}_{k,\mu'}$. Set
%
\begin{equation}
    B'_l(x_0,t_0,\mu') = \sum\nolimits_{(x_1,t_1) \in \mathcal{H}_l(x_0,t_0,\mu')} 2^{k(d-1)} |\langle {S\!}_{x_0,t_0}, {S\!}_{x_1,t_1} \rangle|.
\end{equation}
%
Then
%
\begin{equation}
    \langle F_{k,\mu}, F_{k,\mu'} \rangle \leq \sum\nolimits_{(x_0,t_0) \in \mathcal{E}_{k,\mu}} \sum\nolimits_l B'_l(x_0,t_0,\mu').
\end{equation}
%
We now bound the constants $B'_l$. Pick a constant $r$ such that $d_X \leq 2^r d_X^+$ and $d_X \leq 2^r d_X^-$. As in the estimates of the quantities $B_l$, the quantities where $l$ is large have negligible magnitude:
%
\begin{itemize}
    \item For $l \leq k - a \log_2 A + 10 r$, we have $2^l A^a / R \lesssim 2^k / R$. The set $\mathcal{H}_l(x_0,t_0,\mu')$ is covered by $O(1)$ balls of radius $2^l A^a / R$, and density properties imply
    %
    \begin{equation}
        \# \mathcal{H}_l(x_0,t_0,\mu') \lesssim (R A) (2^l A^a / R) = A^{a+1} 2^l
    \end{equation}
    %
    For $(x_1,t_1) \in \mathcal{H}_l(x_0,t_0,\mu')$, we claim that
    %
    \begin{equation}
        2^{k(d-1)} |\langle {S\!}_{x_0,t_0}, {S\!}_{x_1,t_1} \rangle| \lesssim R^{d} 2^{k(d-1)} (2^l A^a)^{- \frac{d-1}{2}}.
    \end{equation}
    %
    Indeed, for such tuples we have
    %
    \begin{equation}
        d_X(x_0,x_1) \gtrsim 2^l A^a / R \quad\text{or}\quad \min\nolimits_{\pm} |d_X^{\pm}(x_0,x_1) - (t_0 - t_1)| \gtrsim 2^l A^a / R,
    \end{equation}
    %
    and the estimate follows from Proposition \ref{theMainEstimatesForWave} in either case. Since $d \geq 4$, we conclude that
    %
    \begin{align} \label{BBBEquation}
    \begin{split}
        &\sum\nolimits_{l \in [1, k - a \log_2 A + 10]} B'_l(x_0,t_0,,\mu')\\
        &\quad\quad\quad\quad \lesssim \sum\nolimits_{l \in [1, k - a \log_2 A + 10]} R^{d} (2^{k(d-1)}) (2^l A^a)^{- \frac{d-1}{2}} (A^{a+1} 2^l)\\
        &\quad\quad\quad\quad \lesssim \sum\nolimits_{l \in [1, k - a \log_2 A + 10]} R^{d}  2^{k(d-1)} 2^{-l \frac{d-3}{2}} A^{1 - a \left( \frac{d-3}{2} \right)}\\
        &\quad\quad\quad\quad \lesssim R^{d} 2^{k(d-1)} A^{1 - a \left( \frac{d-3}{2} \right)}.
    \end{split}
    \end{align}

    \item For $l > k - a \log_2 A + 10 r$, a tuple $(x_1,t_1) \in \mathcal{E}_k$ lies in $\mathcal{H}_l(x_0,t_0,\mu')$ if and only if $2^l A^a / 2 R \leq d_X(x_0,x_1) \leq 2^l A^a / R$, since we always have
    %
    \begin{equation}
         t_0 - t_1 \leq 2^{k+1}/R < 2^{l+r} A^a / 8R.
    \end{equation}
    %
    and so $d_X(x_0,x_1) \geq 2^l A^a / 2R$. And so 
    %
    \[ |(t_0 - t_1) - d_X^-(x_0,x_1)| \geq 2^{-r} 2^l A^a / 2R - 2^{k+1} / R \geq 2^{-r} 2^l A^a / 4R. \]
    %
    Also $|(t_0 - t_1) + d_X^+(x_0,x_1)| \geq |d_X^+(x_0,x_1) \geq 2^{-r} 2^l A^a / 4R$. Thus we conclude from Proposition \ref{theMainEstimatesForWave} that
    %
    \begin{equation}
        2^{k(d-1)} |\langle {S\!}_{x_0,t_0}, {S\!}_{x_1,t_1} \rangle| \lesssim_K R^{d} 2^{k(d-1)} (2^l A^a)^{- K}.
    \end{equation}
    %
    Now $\mathcal{H}_l(x_0,t_0,\mu')$ is covered by $O( (2^{l-k} A^a)^d )$ balls of radius $2^k / R$, and the density properties of $\mathcal{E}_k$ thus imply that
    %
    \begin{equation}
        \#(\mathcal{H}_l(x_0,t_0,\mu')) \lesssim (RA) (2^{l-k} A^a)^d ( 2^k / R ) \lesssim A^{1 + ad} 2^{ld} 2^{-k(d-1)}.
    \end{equation}
    %
    Thus, picking $K > \max(d,1+ad)$, we conclude that
    %
    \begin{align} \label{BBB2}
    \begin{split}
        &\sum\nolimits_{l \geq k - a \log_2 A + 10} B'_l(x_0,t_0,\mu')\\
        &\quad \lesssim R^{d} \sum\nolimits_{l \geq k - a \log_2 A + 10} (2^{k(d-1)}) (2^l A^a)^{-M} A^{1 + ad} 2^{ld} 2^{-k(d-1)} \lesssim R^{d}.
    \end{split}
    \end{align}
    \end{itemize}
    %
    Combining \eqref{BBBEquation} and \eqref{BBB2}, and then summing over the tuples $(x_0,t_0) \in \mathcal{E}_{k,\mu}$, we conclude that
    %
    \begin{equation} \label{DOUIAWJDOIAWJVIO}
        \sum\nolimits_{\mu \geq \mu' + 10} |\langle F_{k,\mu}, F_{k,\mu'} \rangle| \lesssim R^{d} \left( 1 + 2^{k(d-1)} A^{1 - a \left( \frac{d-3}{2} \right)} \right) \# \mathcal{E}_{k,\mu}.
    \end{equation}
    %
    Now summing in $\mu$, \eqref{DOUIAWJDOIAWJVIO} implies that
    %
    \begin{equation} \label{DAOWDHAODWWID}
        \| F_k \|_{L^2(X)}^2 \lesssim \sum\nolimits_\mu \| F_{k,\mu} \|_{L^2(X)}^2 + R^{d} \left( 1 + 2^{k(d-1)} A^{1 - a \left( \frac{d-3}{2} \right)} \right) \# \mathcal{E}_k.
    \end{equation}
%
The functions in the sum defining $F_{k,\mu}$ are highly coupled, and it is difficult to use anything except Cauchy-Schwarz to break them apart. Since $\# ( \mathcal{T}_R \cap I_{k,\mu}) \sim A^a$, if we set $F_{k,\mu} = \sum_{t \in \mathcal{T}_R \cap I_{k,\mu}} F_{k,\mu,t}$, where
%
\begin{equation}
    F_{k,\mu,t} = \sum\nolimits_{(x_0,t) \in \mathcal{E}_{k,\mu}} 2^{k \frac{d-1}{2}} {S\!}_{x_0,t}.
\end{equation}
%
Then Cauchy-Schwarz implies that
%
\begin{equation} \label{IOJDAOIWDJAWOIJF}
    \| F_{k,\mu} \|_{L^2(X)}^2 \lesssim A^a \sum\nolimits_{t \in \mathcal{T}_R \cap I_{k,\mu}} \| F_{k,\mu,t} \|_{L^2(X)}^2.
\end{equation}
%
Since the elements of $\mathcal{X}_R$ are $1/R$ separated, the functions in the sum defining $F_{k,\mu,t}$ are quite orthogonal to one another; Proposition \ref{theMainEstimatesForWave} implies that for $x_0 \neq x_1$,
%
\begin{equation}
    |\langle {S\!}_{x_0,t}, {S\!}_{x_1,t} \rangle| \lesssim R^d ( R d_X(x_0,x_1) )^{-K} \quad\text{for all $K \geq 0$}.
\end{equation}
%
Thus
%
\begin{equation}
    \| F_{k,\mu,t} \|_{L^2(X)}^2 \lesssim R^{d} 2^{k(d-1)} \# (\mathcal{E}_k \cap (X \times \{ t \})).
\end{equation}
%
But this means that
%
\begin{equation} \label{eoqiejoiwjdoiaevjoa}
    A^a \sum\nolimits_{t \in \mathcal{T}_R \cap I_{k,\mu}} \| F_{k,\mu,t} \|_{L^2(X)}^2 \lesssim R^{d} 2^{k(d-1)} A^a \# \mathcal{E}_{k,\mu}.
\end{equation}
%
Thus \eqref{DAOWDHAODWWID}, \eqref{IOJDAOIWDJAWOIJF}, and \eqref{eoqiejoiwjdoiaevjoa} imply that
%
\begin{equation}
\begin{split}
    \| F_k \|_{L^2(X)}^2 &\lesssim \sum\nolimits_\mu \| F_{k,\mu} \|_{L^2(X)}^2 + R^{d} \left( 1 + 2^{k(d-1)} A^{1 - a \left( \frac{d-3}{2} \right)} \right) \# \mathcal{E}_k\\
    &\lesssim R^{d} \left( 2^{k(d-1)} A^a + (1 + 2^{k(d-1)} A^{1 - a \left( \frac{d-3}{2} \right)} \right) \# \mathcal{E}_k.
\end{split}
\end{equation}
% u^a = u^{1 - a(d-3)/2}
% a ( (d-1)/2 ) = 1
%
Optimizing by picking $a = 2 / (d-1)$ gives that
%
\begin{equation} \label{OICJOAIEVJAIOJFAOIJRIO}
    \| F_k \|_{L^2(X)}^2 \lesssim R^{d} 2^{k(d-1)} A^{\frac{2}{d-1}} \# \mathcal{E}_k.
\end{equation}
%
The proof is completed by combining \eqref{DPOWADPAWKDPOWAKDOPWAK} with \eqref{OICJOAIEVJAIOJFAOIJRIO}.
\end{proof}

Combining the $L^2$ analysis of Section \ref{regime1firstsection} with a density decomposition argument, we can now prove the following Lemma, which completes the analysis of the operator $T^I$ in Proposition \ref{TjbLemma}.

\begin{lemma} \label{regime1Lemma}
    Using the notation of Proposition \ref{TjbLemma}, let $T^I = \sum\nolimits_{t_0 \in \mathcal{T}_R} T^I_{t_0}$, where
    %
    \[ T^I_{t_0} = b_{t_0}^I(t) (e^{2 \pi i t P} \circ Q_R)\; dt. \]
    %
    Then for $1 \leq p < 2 (d-1) / (d+1)$,
    %
    \[ \| T^I u \|_{L^p(X)} \lesssim R^{-1/p'} \left( \sum\nolimits_{t_0 \in \mathcal{T}_R} \Big[ \| b^I_{t_0} \|_{L^p(I_0)} \langle R t_0 \rangle^{s} \Big]^p \right)^{1/p} \| u \|_{L^p(X)}. \]
\end{lemma}

We prove Lemma \ref{regime1Lemma} via a \emph{density decomposition} argument, adapted from the methods of \cite{HeoandNazarovandSeeger}. Given a function $u: X \to \CC$, we use a partition of unity to write
%
\begin{equation}
    u = \sum\nolimits_{x_0 \in \mathcal{X}_R} u_{x_0},
\end{equation}
%
where $u_{x_0}$ is supported on $B(x_0,1/R)$, and
%
\begin{equation}
\begin{split}
    \left( \sum\nolimits_{x_0 \in \mathcal{X}_R} \| u_{x_0} \|_{L^1(X)}^p \right)^{1/p} &\lesssim R^{-d/p'} \left( \sum\nolimits_{x_0 \in \mathcal{X}_R} \| u_{x_0} \|_{L^p(X)}^p \right)^{1/p} \lesssim R^{-d/p'} \| u \|_{L^p(X)}.
\end{split}
\end{equation}
% L^p is H^p R^{-1}
% L^1 is H^p R^{-p}
Define
%
\begin{equation}
    \mathcal{X}_{a} = \{ x_0 \in \mathcal{X}_R: 2^{a-1} < \| u_{x_0} \|_{L^1(X)} \leq 2^a \}
\end{equation}
%
and let
%
\begin{equation}
    \mathcal{T}_{b} = \{ t_0 \in \mathcal{T}_R: 2^{b-1} < \| b_{t_0}^I \|_{L^1(X)} \leq 2^b \}.
\end{equation}
%
Define functions $f_{x_0,t_0} = T_{t_0}^I u_{x_0}$. Lemma \ref{regime1Lemma} follows from the following result.

\begin{lemma} \label{LpBoundLemma}
    Fix $u \in L^p(X)$, and consider $\mathcal{X}_{a}$, $\mathcal{T}_{b}$, and $\{ f_{x_0,t_0} \}$ as above. For any function $c: \mathcal{X}_R \times \mathcal{T}_R \to \CC$, and $1 < p < 2 (d-1) / (d+1)$,
    %
    \begin{align*}
    &\Bigg\| \sum\nolimits_{a,b} \sum\nolimits_{(x_0,t_0) \in \mathcal{X}_{a} \times \mathcal{T}_{b}} 2^{-(a+b)} \langle R t_0 \rangle^{\frac{d-1}{2}} c(x_0,t_0) f_{x_0,t_0} \Big\|_{L^p(X)}\\
    &\quad\quad\quad\quad \lesssim R^{ d / p'} \left( \sum\nolimits_{a,b} \sum\nolimits_{(x_0,t_0) \in \mathcal{X}_{a} \times \mathcal{T}_{b}} |c(x_0,t_0)|^p \langle R t_0 \rangle^{d-1} \right)^{1/p}.
    \end{align*}
\end{lemma}

To see how Lemma \ref{LpBoundLemma} implies Lemma \ref{regime1Lemma}, set $c(x_0,t_0) = 2^{a+b} \langle R t_0 \rangle^{- \frac{d-1}{2}}$ for $x_0 \in \mathcal{X}_{a}$ and $t_0 \in \mathcal{T}_{b}$. Then Lemma \ref{LpBoundLemma} implies that
%
\begin{equation}
\begin{split}
    &\| T^I u \|_{L^p(X)}\\
    &\quad = \left\| \sum f_{x_0,t_0} \right\|_{L^p(X)}\\
    &\quad \lesssim R^{d/p'} \left( \sum\nolimits_{(x_0,t_0)} \left[ \| b_{t_0}^I \|_{L^1(\RR)} \| u_{x_0} \|_{L^1(X)} \langle R t_0 \rangle^{s} \right]^p \right)^{1/p}\\
    &\quad \lesssim R^{-1/p'} \left( \sum\nolimits_{t_0} \Big[ \| b_{t_0}^I \|_{L^p(I_0)} \langle R t_0 \rangle^{s} \Big]^p \right)^{1/p}\left( \sum\nolimits_{x_0} \left[ \| u_{x_0} \|_{L^1(X)}  R^{d/p'} \right]^p \right)^{1/p}\\
    &\quad \lesssim R^{-1/p'} \left( \sum\nolimits_{t_0} \Big[ \| b_{t_0}^I \|_{L^p(I_0)} \langle R t_0 \rangle^{s} \Big]^p \right)^{1/p} \| u \|_{L^p(X)}.
\end{split}
\end{equation}
% C = alpha(p) + d/p'
%   = (d+1)/2 - 1/p
%
Thus we have proved Lemma \ref{regime1Lemma}. We take the remainder of this section to prove Lemma \ref{LpBoundLemma} using a density decomposition argument.

\begin{proof}[Proof of Lemma \ref{LpBoundLemma}]

For $p = 1$, this inequality follows simply by applying the triangle inequality, and applying the pointwise estimates of Proposition \ref{theMainEstimatesForWave}. By methods of interpolation, to prove the result for $p > 1$, we thus only need only prove a restricted strong type version of this inequality. In other words, we can restrict $c$ to be the indicator function of a set $\mathcal{E} \subset \mathcal{X}_R \times \mathcal{T}_R$. Write $\mathcal{E} = \bigcup_{k \geq 0} \mathcal{E}_{k,a,b}$, where
%
\begin{equation}
    \mathcal{E}_{0,a,b} = \{ (x,t) \in \mathcal{E} \cap (\mathcal{X}_{a} \times \mathcal{T}_{b}) : |t| \leq 1/R \}
\end{equation}
%
and for $k > 0$, let
%
\begin{equation}
    \mathcal{E}_{k,a,b} = \{ (x,t) \in \mathcal{E} \cap (\mathcal{X}_{a} \times \mathcal{T}_{b}) : 2^{k-1} / R < |t| \leq 2^{k} / R \}.
\end{equation}
%
Write
%
\begin{equation}
    F_k = \sum\nolimits_{a,b} \sum\nolimits_{(x_0,t_0) \in \mathcal{E}_{k,a,b}} 2^{k \left( \frac{d-1}{2} \right)} 2^{-(a+b)} f_{x_0,t_0}.
\end{equation}
%
Our proof will be completed if we can show that
%
\begin{equation} \label{oDOIAWJCVOIEJOIJER1312s}
    \Big\| \sum\nolimits_k F_k \Big\|_{L^p(X)} \lesssim R^{d ( 1 - 1/p )} \Big( \sum\nolimits_k 2^{k(d-1)} \# \mathcal{E}_k \Big)^{1/p}.
\end{equation}
%
To prove \eqref{oDOIAWJCVOIEJOIJER1312s}, we perform a density decomposition on the sets $\{ \mathcal{E}_k \}$. For $u \geq 0$, let $\widehat{\mathcal{E}}_k(u)$ be the set of all points $(x_0,t_0) \in \mathcal{E}_k$ that are contained in a ball $B$ with $\text{rad}(B) \leq 2^{k} / 100 R$, such that $\#( \mathcal{E}_k \cap B ) \geq R 2^{u} \text{rad}(B)$. Then define
%
\begin{equation}
    \mathcal{E}_k(u) = \widehat{\mathcal{E}}_k(u) - \bigcup\nolimits_{u' > u} \widehat{\mathcal{E}}_k(u').
\end{equation}
%
Because the set $\mathcal{E}_k$ is $1/R$ discretized, we have
%
\begin{equation}
    \mathcal{E}_k = \bigcup\nolimits_{u \geq 0} \mathcal{E}_k(u).
\end{equation}
%
Moreover $\mathcal{E}_k(u)$ has density type $(R 2^{u}, 2^{k} / 100 R)$, and thus by a covering argument, also has density type $(C_d R 2^{u}, 2^{k} / R)$ for $C_d = 1000^d$. Furthermore, there are disjoint balls $B_{k,u,1},\dots,B_{k,u,N_{k,u}}$ of radius at most $2^{k} / 100 R$ such that
%
\begin{equation}
    \sum\nolimits_n \text{rad}(B_{k,u,n}) \leq 2^{-u} / R \# \mathcal{E}_k.
\end{equation}
%
and such that $\mathcal{E}_k(u)$ is covered by the balls $\{ B_{k,u,n}^* \}$, where, for a ball $B$, $B^*$ denotes the ball with the same center as $B$, but 5 times the radius. Now write
%
\begin{equation}
    F_{k,u} = \sum\nolimits_{a,b} \sum\nolimits_{(x_0,t_0) \in \mathcal{E}_{k,a,b}(u)} 2^{k \left( \frac{d-1}{2} \right)} 2^{-(a+b)} f_{x_0,t_0}.
\end{equation}
%
Using the density assumption on $\mathcal{E}_k(u)$, we can apply Lemma \ref{L2DensityProposition} of the last section, which implies that, with ${S\!}_{x_0,t_0} = 2^{-(a+b)} f_{x_0,t_0}$ for $(x_0,t_0) \in \mathcal{E}_{k,a,b}(u)$,
%
\begin{equation} \label{DOIWAJOIAJVOIWAJFOIWF}
\begin{split}
    \Big\| \sum\nolimits_k F_{k,u} \Big\|_{L^2(X)} \lesssim R^{d/2} \left( u^{1/2} 2^{u \left( \frac{1}{d-1} \right)} \right) \left( \sum\nolimits_k 2^{k(d-1)} \# \mathcal{E}_k \right)^{1/2}.
\end{split}
\end{equation}
%
Let $(y_{k,u,n}, t_{k,u,n})$ denote the center of $B_{k,u,n}$. Then
%
\begin{equation}
    \sum\nolimits_{(x_0,t_0) \in [B_{k,u,n} \cap \mathcal{E}_k(u)]} 2^{-(a+b)} f_{x_0,t_0}
\end{equation}
%
has mass concentrated on the geodesic annulus $\text{Ann}_{k,u,n} \subset X$ with center $y_{k,u,n}$, with radius $t_{k,u,n} \sim 2^{k} / R$, and with thickness $5\; \text{rad}(B_{k,u,n})$. Thus
%
\begin{equation}
    \sum\nolimits_n |\text{Ann}_{k,u,n}| \lesssim \sum\nolimits_{n} (2^{k} / R)^{d-1} \text{rad}(B_{k,u,n}) \leq (2^{k}/R)^{d-1} R^{-1} 2^{-u} \# \mathcal{E}_k.
\end{equation}
%
If we set $\Lambda_u = \bigcup_k \bigcup_n \text{Ann}_{k,u,n}$, then
%
\begin{equation}
    |\Lambda_u| \lesssim R^{-d} 2^{-u} \sum\nolimits_k 2^{k(d-1)} \# \mathcal{E}_k
\end{equation}
%
Since $1/p - 1/2 > 1/(d-1)$, so that $s > 1$, H\"{o}lder's inequality implies that
%
\begin{equation}
\begin{split}
    \Big\| \sum\nolimits_k F_{k,u} \Big\|_{L^p(\Lambda_u)} &\lesssim |\Lambda_u|^{1/p - 1/2} \Big\| \sum\nolimits_k F_{k,u} \Big\|_{L^2(\Lambda_{k,u})}\\
    &\lesssim \left( R^{-d} 2^{-u} \sum\nolimits_k 2^{k(d-1)} \# \mathcal{E}_k \right)^{1/p - 1/2}\\
    &\quad\quad\quad\quad \left( R^{d} \left( u 2^{u \left( \frac{2}{d-1} \right)} \right) \sum\nolimits_k 2^{k(d-1)} \# \mathcal{E}_k \right)^{1/2}\\
    &= R^{d(1-1/p)} \left( u^{1/2} 2^{-u \left( \frac{s - 1}{d - 1} \right)}  \right) \left( \sum\nolimits_k 2^{k(d-1)} \# \mathcal{E}_k \right)^{1/p}\\
    &\lesssim R^{d(1 - 1/p)} 2^{-u \varepsilon} \left( \sum\nolimits_k 2^{k(d-1)} \# \mathcal{E}_k \right)^{1/p}
\end{split}
\end{equation}
%
for some suitable small $\varepsilon > 0$. For each $(x_0,t_0) \in \mathcal{E}_{k,a,b}(u) \cap B_{k,u,n}$, we calculate using the pointwise bounds for the functions $\{ f_{x_0,t_0} \}$ that
%
%\begin{align*}
%        \| 2^{-(l+r)} f_{x_0,t_0} \|_{L^q(\text{Ann}_n^c)}^q &= 2^{jdq} \int_{\text{Ann}_j^c} \langle 2^j d_X(x,x_0) \rangle^{- q \left( \frac{d-1}{2} \right)} \langle 2^j |t_0 - d_X(x,x_0)| \rangle^{-qM}\; dx \\
%        &\lesssim 2^{jq \left( \frac{d+1}{2} - M \right)} \int_{5\; \text{rad}(B_n)}^{O(1)} (2^{k-j} + s)^{(d-1) - q \left( \frac{d-1}{2} \right)} s^{-qM}\; ds\\
%    &\lesssim 2^{k(d-1)(1-q/2)} 2^{j[ q \left( d - M \right) - (d-1) ]} \text{rad}(B_n)^{1 - qM}.
%\end{align*}
\begin{equation}
\begin{split}
        \| 2^{-(a+b)} f_{x_0,t_0} \|_{L^1(\Lambda(u)^c)} &= R^{d} \int_{\text{Ann}_R^c} \langle R d_X(x,x_0) \rangle^{- \left( \frac{d-1}{2} \right)} \langle R |t_0 - d_X(x,x_0)| \rangle^{-M}\; dx\\
        &\lesssim R^{ \left( \frac{d+1}{2} - M \right)} \int_{5\; \text{rad}(B_{k,u,n})}^{O(1)} ( t_{k,u,n} + s)^{\left( \frac{d-1}{2} \right)} s^{-M}\; ds\\
        &\lesssim R^{ \left( \frac{d+1}{2} - M \right)} \text{rad}(B_{k,u,n})^{1-M} t_{k,u,n}^{\left( \frac{d-1}{2} \right)}\\
        &\lesssim 2^{k \left( \frac{d-1}{2} \right)} (R \text{rad}(B_{k,u,n}))^{1 - M}.
\end{split}
\end{equation}
%
Thus
%
%\[ \| 2^{k \left( \frac{d-1}{2} \right)} 2^{-(l+r)} f_{x_0,t_0} \|_{L^q(\text{Ann}_n^c)} \lesssim 2^{k \left( \frac{d-1}{q} \right) } 2^{jd/q'} (2^j \text{rad}(B_n))^{1/q - M} \]\
\begin{equation}
    \| 2^{k \left( \frac{d-1}{2} \right)} 2^{-(a+b)} f_{x_0,t_0} \|_{L^1(\text{Ann}_n^c)} \lesssim 2^{k(d-1)} (R \text{rad}(B_{k,u,n}))^{1 - M}
\end{equation}
%
% 2^{(k-j)[-p(d-1)/2]} rad(B_n)^{d-pM}
%      >> 2^{(k-j)( -p(d-1)/2 + d - pM )}
%
% 2^{(k-j)( d - p(d-1)/2 - pM )}
% 
%
Because the set of points in $\mathcal{E}_k$ is $1/R$ separated, there are at most $O( (R \text{rad}(B_{k,u,n}))^{d+1} )$ points in $\mathcal{E}_k(u) \cap B_{k,u,n}$, and so the triangle inequality implies that
% \mathcal{E}_{k,l,r} - \widehat{\mathcal{E}}_{k,l,r} = \bigcup_{l,r} 
%\begin{align*}
%    &\left\| \sum\nolimits_{l,r} \sum\nolimits_{(x_0,t_0) \in (\mathcal{E}_{k,l,r} - \widehat{\mathcal{E}}_{k,l,r}) \cap B_n} 2^{k \left( \frac{d-1}{2} \right)} 2^{-(l+r)} f_{x_0,t_0} \right\|_{L^q(\text{Ann}_n^c)}\\
%    &\quad\quad \lesssim 2^{k \left( \frac{d-1}{q} \right)} 2^{jd/q'} (2^j \text{rad}(B_n))^{d + 1 + 1/q - M}
%\end{align*}
\begin{equation}
\begin{split}
    &\Big\| \sum\nolimits_{a,b} \sum\nolimits_{(x_0,t_0) \in \mathcal{E}_{k,a,b}(u) \cap B_{k,u,n}} 2^{k \left( \frac{d-1}{2} \right)} 2^{-(a+b)} f_{x_0,t_0} \Big\|_{L^1(\Lambda(u)^c)}\\
    &\quad\quad\quad \lesssim 2^{k (d - 1)} (R \text{rad}(B_{k,u,n}))^{d + 2 - M}.
\end{split}
\end{equation}
%
Since $\# \mathcal{E}_k \cap B_{k,u,n} \geq R 2^{u}\ \text{rad}(B_{k,u,n})$, and $\mathcal{E}_k$ is $1/R$ discretized, we must have
%
\begin{equation}
    \text{rad}(B_{k,u,n}) \geq (2^u / 2^d)^{\frac{1}{d-1}}\; 1/R.
\end{equation}
%
Thus
%
% 2^{jd(p-1)}
%
\begin{equation}
\begin{split}
    \Big\| \sum\nolimits_k F_{k,u} \Big\|_{L^1(\Lambda(u)^c)} &\lesssim_X \sum\nolimits_k \sum\nolimits_n 2^{k (d-1)} (R \text{rad}(B_{k,u,n}))^{d + 2 - M}\\
    &\lesssim \sum\nolimits_k 2^{k(d-1)} \Big( R \min\nolimits_n \text{rad}(B_{k,u,n}) \Big)^{d + 1 - M}\\
    &\quad\quad\quad\quad \left( \sum\nolimits_n R \text{rad}(B_{k,u,n}) \right) \\
    &\lesssim \sum\nolimits_k 2^{k (d-1)} 2^{u \left( \frac{d+1-M}{d-1} \right)} \left( 2^{-u} \# \mathcal{E}_k \right)\\
    &\lesssim 2^{u \left( \frac{2-M}{d-1} \right)} \sum\nolimits_k 2^{k (d-1)} \# \mathcal{E}_k
\end{split}
\end{equation}
%
Picking $M > 2 + (1 - 1/p)(1/p - 1/2)^{-1}$, and interpolating with the bounds on $\| \sum_k F_{k,u} \|_{L^2(X)}$ yields that
%
\begin{equation}
\begin{split}
    \Big\| \sum\nolimits_k F_{k,u} \Big\|_{L^p(\Lambda(u)^c)} &\lesssim \left( 2^{u \left( \frac{2-M}{d-1} \right)} \right)^{2/p - 1} \left( R^{d/2} \left( u^{1/2} 2^{u \left( \frac{1}{d-1} \right)} \right) \right)^{2(1 - 1/p)}\\
    &\quad\quad\quad\quad \left( \sum 2^{k(d-1)} \# \mathcal{E}_k \right)^{1/p} \\
    &\lesssim 2^{-u \varepsilon} R^{d(1 - 1/p)} \sum\nolimits_k \left( \sum 2^{k(d-1)} \# \mathcal{E}_k \right)^{1/p}.
\end{split}
\end{equation}
%
So now we know
%
\begin{equation}
    \Big\| \sum\nolimits_k F_{k,u} \Big\|_{L^p(X)} \lesssim R^{d(1-1/p)} 2^{- u \varepsilon} \left( \sum\nolimits_k 2^{k(d-1)} \# \mathcal{E}_k \right)^{1/p}.
\end{equation}
%
The exponential decay in $u$ allows us to sum in $u$ to obtain that
%
\begin{equation}
    \Big\| \sum\nolimits_u \sum\nolimits_k F_{k,u} \Big\|_{L^p(X)} \lesssim R^{d(1 - 1/p)} \left( \sum\nolimits_k 2^{k(d-1)} \# \mathcal{E}_k \right)^{1/p}.
\end{equation}
%
This is precisely the bound we were required to prove.
\end{proof}



\begin{comment}



which, within the range $p < 2(d-1)/(d+1)$ we are considering, and since we are assuming $\lambda \gtrsim_d 2^{jd}$, satisfies
%
\begin{align*}
        \sum_n |\text{Ann}_n| &\lesssim \log(\lambda/2^{jd})^{(2-p) \left( \frac{d-1}{2} \right)} (2^{k-j})^{d-1} 2^{-j} \left( 2^{jd} / \lambda \right)^{(2-p) \left( \frac{d-1}{2} \right)} \# \mathcal{E}_k\\
        &= \log(\lambda/2^{jd})^{(2-p) \left( \frac{d-1}{2} \right)} (\lambda/2^{jd})^{p \left( \frac{d+1}{2} \right) - (d-1)} 2^{jd(p-1)} \Big( 2^{k(d-1)} \# \mathcal{E}_k \Big) \lambda^{-p}\\
        &\lesssim 2^{jd(p-1)} \Big( 2^{k(d-1)} \# \mathcal{E}_k \Big) \lambda^{-p}.
\end{align*}
%
The last inequality uses the fact that for any $\varepsilon, \delta > 0$, and $x \geq 10^{d+1}$,
%
\[ \log(x)^\delta x^{-\varepsilon} \lesssim_{\delta,\varepsilon} 1. \]


% 2^{k-j} in diameter
% So contained in a ball with radius 2^{k-j}
%   Place balls of radius 1/20 in this ball. Then the union of these balls is
%   contained in a ball of radius 21/20. Therefore there are at most 21^d
%   such balls
% On each of these balls, << 2^{j+u} rad(B) / 20


% p / p' = p (1 - 1/p) = p - 1
% When p = 1 the inequality says
% Sum_k 2^{k (d-1)/2} |S_{x_0,t_0}|_{L^1} << R^{-1} sum_k 2^{k(d-1)} #(E_k)
%
% d_g(x,x_0) ~ 2^k / R
%
% Has height R^{d-1} 2^{-k (d-1)/2}
%
% on an annulus of thickness 1/R, and radius 2^k/R
%
% So has L^1 norm R^{-1} 2^{k(d-1)/2}
%
% SO WE DO GET THE p = 1 INEQUALITY!
%
% So for weak type estimate, by interpolation,
% we only have to consider large lambda superlevel sets
%
% LOW DENSITY PART: DEALT WITH USING L2 ESTIMATE
% HIGH DENSITY PART: Since lambda is large,
%           can use essential support of function.
%



This is equivalent to showing that for any $\lambda > 0$,
%
\begin{align} \label{FSumLPInfBound}
\begin{split}
    &\bigg| \Big\{ x: \Big| \sum\nolimits_{a,b}  \sum\nolimits_k \sum\nolimits_{(x_0,t_0) \in \mathcal{E}_{k,a,b}} 2^{k \left( \frac{d-1}{2} \right)} 2^{-(a+b)} f_{x_0,t_0}(x) \Big| \geq \lambda \Big\} \bigg|\\
    &\quad\quad\quad \lesssim  2^{j d (p - 1)} \left( \sum_k 2^{k(d-1)} \# \mathcal{E}_k \right) \lambda^{-p}.
\end{split}
\end{align}
%
Applying Markov's inequality, as well as the $p = 1$ inequality, we find that
%
\[ \bigg| \Big\{ x: \Big|\sum\nolimits_{a,b} \sum\nolimits_{k} \sum\nolimits_{(x_0,t_0) \in \mathcal{E}_{k,a,b}} 2^{k \left( \frac{d-1}{2} \right)} 2^{-(a+b)} f_{x_0,t_0}(x) \Big| \geq \lambda \Big\} \bigg| \lesssim \left( \sum_k 2^{k(d-1)} \# \mathcal{E}_k \right) \lambda^{-1} \]
%
We have
%
\[ \left( \sum_k 2^{k(d-1)} \# \mathcal{E}_k \right) \lambda^{-1} \lesssim 2^{jd(p-1)} \left( \sum_k 2^{k(d-1)} \# \mathcal{E}_k \right) \lambda^{-p} \]
% lambda^{p-1} << 2^{jd(p-1)}
for $\lambda \leq 10^{d+2} 2^{jd}$, so we may assume $\lambda \geq 10^{d+2} 2^{jd}$ in what follows. We then employ the method of density decompositions introduced in \cite{HeoandNazarovandSeeger}. Fix a quantity $A \geq 10^{d+1}$, to be chosen later. For each $k$, we consider the collection $\mathcal{B}_k$ of all balls $B$ with radius at most $2^{k-j} / 10$ such that $\# \mathcal{E}_k \cap B \geq (2^j A) \text{rad}(B)$. Applying the Vitali covering lemma, we can find a disjoint family of balls $\{ B_1, \dots, B_N \}$ in $\mathcal{B}_k$ such that the balls $\{ B_1^*, \dots, B_N^* \}$ obtained by dilating the balls by 5 cover $\bigcup \mathcal{B}_k(\lambda)$. Then
%
\[ \sum_n \text{rad}(B_n) \leq 2^{-j} A^{-1} \# \mathcal{E}_k. \]
%
Then the set $\widehat{\mathcal{E}}_k = \mathcal{E}_k - \bigcup \mathcal{B}_k(\lambda)$ has density type $(10^d 2^j A, 2^{k-j})$. Thus we can apply Lemma \ref{L2DensityProposition} of the last section, which implies that, with ${S\!}_{x_0,t_0} = 2^{-(a+b)} f_{x_0,t_0}$ for $(x_0,t_0) \in \mathcal{E}_{k,a,b}$,
%
\[ \Big\| \sum\nolimits_{a,b} \sum\nolimits_k \sum\nolimits_{(x_0,t_0) \in \widehat{\mathcal{E}}_{k,a,b}} 2^{k \left( \frac{d-1}{2} \right)} 2^{-(a+b)} f_{x_0,t_0} \Big\|_{L^2(M)}^2 \lesssim 2^{jd} \log(A) A^{\frac{2}{d-1}} \sum_k 2^{k(d-1)} \# \mathcal{E}_k. \]
%2^{j(d-2)} \log(A) A^{\frac{2}{d-1}} \sum_k 2^{k(d-1)} \# \mathcal{E}_k. \]
%
Applying Chebyshev's inequality, we conclude that
%
\begin{align*}
    &\bigg| \Big\{ x: \Big|\sum\nolimits_{a,b} \sum\nolimits_k \sum\nolimits_{(x_0,t_0) \in \widehat{\mathcal{E}}_k} 2^{k \frac{d-1}{2}} 2^{-(a+b)} f_{x_0,t_0}(x)\Big| \geq \lambda / 2 \Big\} \bigg|\\
    &\quad\quad\quad \lesssim 2^{jd} \log(A) A^{\frac{2}{d-1}} \Big( \sum_k 2^{k(d-1)} \# \mathcal{E}_k \Big) \lambda^{-2}.
\end{align*}
% 
% TODO: Bound A^s log(A) by some power of A for large A.
%
% 2^{jd} A^{2/(d-1)} / lambda^2 << 2^{jd(p-1)} / lambda^p
% A << (lambda / 2^{jd})^{(2-p)(d-1)/2}
%
% A^{2/(d-1) + log log A} << (L/2^{jd})^{2-p}
% If A <= (L/2^{jd})^{(2-p)(d-1)/2}
% then log log A <= C_{p,d} + log log (L / 2^{jd})
% A^{2/(d-1) + C_{p,d} + log log (L / 2^{jd})} <= (L / 2^{jd})^{2-p}
% (2/(d-1) + C_{p,d} + log log (L / 2^{jd})) log A <= (2-p) log (L / 2^{jd})
% log A <= (2-p) log (L / 2^{jd}) / (C + log log(L / 2^{jd}))
% A = (L / 2^{jd})^{(2-p) / (C + log log (L / 2^{jd})) }
% So if L = 2^{2^N} 2^{jd},
% A = 2^{(2^N - jd)(2-p)/(C + N)}
%
Choose
%
\[ A = \log(\lambda / 2^{jd})^{\left( \frac{d-1}{2} \right)} \left( \lambda / 2^{jd} \right)^{(2-p) \left( \frac{d-1}{2} \right)}, \]
%
so that
%
\[ (\log A)^{\left( \frac{d-1}{2} \right)} A \lesssim (\lambda / 2^{jd})^{(2-p) \left( \frac{d-1}{2} \right)}. \]
%
Then
%
\begin{align} \label{ChebyshevFirstBound}
\begin{split}
    &\bigg| \bigg\{ x: \bigg|\sum_{a,b} \sum_k \sum_{(x_0,t_0) \in \widehat{\mathcal{E}}_k} 2^{k \frac{d-1}{2}} 2^{-(a+b)} f_{x_0,t_0}(x)\bigg| \geq \lambda / 2 \bigg\} \bigg|\\
    &\quad\quad\quad \lesssim 2^{jd(p-1)} \left( \sum_k 2^{k(d-1)} \# \mathcal{E}_k \right) \lambda^{-p}.
\end{split}
\end{align}
%
\eqref{ChebyshevFirstBound} gives a good enough bound for $\widehat{\mathcal{E}}_k$. Conversely, we exploit the clustering of the sets $\mathcal{E}_k - \widehat{\mathcal{E}}_k$ to bound
%
\[ \bigg| \Big\{ x: \Big| \sum\nolimits_{a,b} \sum\nolimits_k \sum\nolimits_{(x_0,t_0) \in \mathcal{E}_k - \widehat{\mathcal{E}}_k} 2^{k \frac{d-1}{2}} 2^{-(a+b)} f_{x_0,t_0}(x) \Big| \geq \lambda / 2 \Big\} \bigg| \]
%
We have found balls $B_1^*, \dots, B_N^*$, each with radius at most $2^{k-j} / 10$, such that
%
\[ \sum \text{rad}(B_n) \leq 2^{-j} A^{-1} \# \mathcal{E}_k. \]
%
Let $(y_n,r_n)$ denote the center of the ball $B_n$. Then the function
%
\[ \sum\nolimits_{(x_0,t_0) \in B_n \cap (\mathcal{E}_k - \widehat{\mathcal{E}}_k)} 2^{-(a+b)} f_{x_0,t_0} \]
%
has mass concentrated on the geodesic annulus $\text{Ann}_n \subset M$ centred at $y_n$, with radius $r_n \sim 2^{k-j}$ and thickness $5\; \text{rad}(B_n)$. But we then calculate that
%
\[ \sum_n |\text{Ann}_n| \lesssim \sum_n (2^{k-j})^{d-1} \text{rad}(B_n) \leq (2^{k-j})^{d-1} 2^{-j} A^{-1} \# \mathcal{E}_k \]
%
which, within the range $p < 2(d-1)/(d+1)$ we are considering, and since we are assuming $\lambda \gtrsim_d 2^{jd}$, satisfies
%
\begin{align*}
        \sum_n |\text{Ann}_n| &\lesssim \log(\lambda/2^{jd})^{(2-p) \left( \frac{d-1}{2} \right)} (2^{k-j})^{d-1} 2^{-j} \left( 2^{jd} / \lambda \right)^{(2-p) \left( \frac{d-1}{2} \right)} \# \mathcal{E}_k\\
        &= \log(\lambda/2^{jd})^{(2-p) \left( \frac{d-1}{2} \right)} (\lambda/2^{jd})^{p \left( \frac{d+1}{2} \right) - (d-1)} 2^{jd(p-1)} \Big( 2^{k(d-1)} \# \mathcal{E}_k \Big) \lambda^{-p}\\
        &\lesssim 2^{jd(p-1)} \Big( 2^{k(d-1)} \# \mathcal{E}_k \Big) \lambda^{-p}.
\end{align*}
%
The last inequality uses the fact that for any $\varepsilon, \delta > 0$, and $x \geq 10^{d+1}$,
%
\[ \log(x)^\delta x^{-\varepsilon} \lesssim_{\delta,\varepsilon} 1. \]
% p < 2(d-1)/(d+1)
% p - (2 - p)(d-1)/2 = 2(d-1)/(d+1) - (2(d-1)/(d+1))
%
%\begin{align*}
%     2^{-jd} & \left( \frac{\lambda^{2-p}}{2^{j(d+1)(1 - p/2)}} \right)^{- \frac{d-1}{2}} ( 2^{k(d-1)} \# \mathcal{E}_k )\\
%     &\quad\quad = 2^{j \left[ \left( \frac{d-1}{2} \right) (d+1)(1 - p/2) - d \right]} \lambda^{p - (2-p) \left( \frac{d-1}{2} \right)} ( 2^{k(d-1)} \# \mathcal{E}_k ) \lambda^{-p}\\
%     &\quad\quad = 2^{j \left[ \left( \frac{d-1}{2} \right) (d+1)(1 - p/2) - d \right]} \lambda^{p \left( \frac{d+1}{2} \right) - (d-1)} ( 2^{k(d-1)} \# \mathcal{E}_k ) \lambda^{-p}\\
%     &\quad\quad \leq 2^{j \left[ \left( \frac{d-1}{2} \right) (d+1)(1 - p/2) - d + \left\{ p \left( \frac{d+1}{2} \right) - (d-1) \right\} \left( \frac{d+1}{2} \right) \right]}\\
%     &\quad\quad = 2^{j \left[ - d + (p/4) [ d^2 + d + 2 ] \right]} ( 2^{k(d-1)} \# \mathcal{E}_k ) \lambda^{-p}\\
%     &\quad\quad = 2^{j \left[ p \left( \frac{d+1}{2} \right) - 1 \right]} 2^{j(d-1)[ pd/4 - 1]}
%\end{align*}
% - d
%
%\begin{align*}
%    (2^{k-j})^{d-1} 2^{-j} &  \left( \frac{\lambda^{2-p}}{2^{j \left[ (d-1) - p \left( \frac{d+1}{2} \right) \right]}} \right)^{- \frac{d-1}{2}} \# \mathcal{E}_k\\
%    &= 2^{j \left[ \frac{d^2 - 4d + 1}{2} - p  \left( \frac{d^2 - 1}{4} \right) \right]}\; \lambda^{p \left( \frac{d+1}{2} \right) - (d-1)}  \; 2^{k(d-1)} \# \mathcal{E}_k\; \lambda^{-p}\\
%    &\lesssim 2^{j \left[ \frac{d^2 - 4d + 1}{2} - p  \left( \frac{d^2 - 1}{4} \right) \right]} 2^{j \left[ p \left( \frac{d^2+2d + 1}{4} \right) - \left( \frac{d^2 - 1}{2} \right) \right]}  \; \left( 2^{k(d-1)} \# \mathcal{E}_k \right)\; \lambda^{-p}\\
%    &= 2^{j \left[ p \left( \frac{d + 1}{2} \right) - (2d - 1) \right]}   \; \left( 2^{k(d-1)} \# \mathcal{E}_k \right)\; \lambda^{-p}\\
%    &\lesssim 2^{j \left[ p \left( \frac{d+1}{2} \right) - 1 \right]}   \; \left( 2^{k(d-1)} \# \mathcal{E}_k \right)\; \lambda^{-p}
%\end{align*}
%
%\textcolor{red}{TODO: LAST INEQUALITY SEEMS SUSPICIOUSLY LOOSE.}
%
Defining $\text{Bad}_k = \bigcup_n \text{Ann}_n$, and $\text{Bad} = \bigcup_k \text{Bad}_k$, we have
%
\begin{align} \label{BadSet}
    |\text{Bad}| \lesssim 2^{jd (p-1)} \sum\nolimits_k 2^{k(d-1)} \# \mathcal{E}_k\; \lambda^{-p}.
\end{align}
%
This is an exceptional set with size appropriate for the proof of the $L^{p,\infty}$ bound. All that remains is to account for the negligible error terms associated with the functions $f_{x_0,t_0}$, i.e. their behaviour on $\text{Bad}^c$. For each $(x_0,t_0) \in (\mathcal{E}_{k,a,b} - \widehat{\mathcal{E}}_{k,a,b}) \cap B_n$, we calculate using the pointwise bounds for the functions $\{ f_{x_0,t_0} \}$ that
%
%\begin{align*}
%        \| 2^{-(l+r)} f_{x_0,t_0} \|_{L^q(\text{Ann}_n^c)}^q &= 2^{jdq} \int_{\text{Ann}_j^c} \langle 2^j d_g(x,x_0) \rangle^{- q \left( \frac{d-1}{2} \right)} \langle 2^j |t_0 - d_g(x,x_0)| \rangle^{-qM}\; dx \\
%        &\lesssim 2^{jq \left( \frac{d+1}{2} - M \right)} \int_{5\; \text{rad}(B_n)}^{O(1)} (2^{k-j} + s)^{(d-1) - q \left( \frac{d-1}{2} \right)} s^{-qM}\; ds\\
%    &\lesssim 2^{k(d-1)(1-q/2)} 2^{j[ q \left( d - M \right) - (d-1) ]} \text{rad}(B_n)^{1 - qM}.
%\end{align*}
\begin{align*}
        \| 2^{-(a+b)} f_{x_0,t_0} \|_{L^1(\text{Ann}_n^c)} &= 2^{jd} \int_{\text{Ann}_j^c} \langle 2^j d_g(x,x_0) \rangle^{- \left( \frac{d-1}{2} \right)} \langle 2^j |t_0 - d_g(x,x_0)| \rangle^{-M}\; dx \\
        &\lesssim 2^{j \left( \frac{d+1}{2} - M \right)} \int_{5\; \text{rad}(B_n)}^{O(1)} (2^{k-j} + s)^{\left( \frac{d-1}{2} \right)} s^{-M}\; ds\\
    &\lesssim 2^{k \left( \frac{d-1}{2} \right)} 2^{j(1 - M)} \text{rad}(B_n)^{1 - M}.
\end{align*}
%
Thus
%
%\[ \| 2^{k \left( \frac{d-1}{2} \right)} 2^{-(l+r)} f_{x_0,t_0} \|_{L^q(\text{Ann}_n^c)} \lesssim 2^{k \left( \frac{d-1}{q} \right) } 2^{jd/q'} (2^j \text{rad}(B_n))^{1/q - M} \]
\[ \| 2^{k \left( \frac{d-1}{2} \right)} 2^{-(a+b)} f_{x_0,t_0} \|_{L^1(\text{Ann}_n^c)} \lesssim 2^{k \left( d - 1 \right) } (2^j \text{rad}(B_n))^{1 - M} \]
%
% 2^{(k-j)[-p(d-1)/2]} rad(B_n)^{d-pM}
%      >> 2^{(k-j)( -p(d-1)/2 + d - pM )}
%
% 2^{(k-j)( d - p(d-1)/2 - pM )}
% 
%
Because the set of points in $\mathcal{E}_k$ is $2^{-j}$ separated, there are at most $O( (2^j \text{rad}(B_n))^{d+1} )$ points in $(\mathcal{E}_{k,a,b} - \widehat{\mathcal{E}}_{k,a,b}) \cap B_n$, and so the triangle inequality implies that
% \mathcal{E}_{k,l,r} - \widehat{\mathcal{E}}_{k,l,r} = \bigcup_{l,r} 
%\begin{align*}
%    &\left\| \sum\nolimits_{l,r} \sum\nolimits_{(x_0,t_0) \in (\mathcal{E}_{k,l,r} - \widehat{\mathcal{E}}_{k,l,r}) \cap B_n} 2^{k \left( \frac{d-1}{2} \right)} 2^{-(l+r)} f_{x_0,t_0} \right\|_{L^q(\text{Ann}_n^c)}\\
%    &\quad\quad \lesssim 2^{k \left( \frac{d-1}{q} \right)} 2^{jd/q'} (2^j \text{rad}(B_n))^{d + 1 + 1/q - M}
%\end{align*}
\begin{align*}
    &\Big\| \sum\nolimits_{a,b} \sum\nolimits_{(x_0,t_0) \in \mathcal{E}_{k,a,b} - \widehat{\mathcal{E}}_{k,a,b}} 2^{k \left( \frac{d-1}{2} \right)} 2^{-(a+b)} f_{x_0,t_0} \Big\|_{L^1(\text{Ann}_n^c)} \lesssim 2^{k \left( d - 1 \right)} (2^j \text{rad}(B_n))^{d + 2 - M}.
\end{align*}
%
Since $\# \mathcal{E}_k \cap B_n \geq 2^j A\ \text{rad}(B_n)$, and because $\mathcal{E}_k$ is $2^{-j}$ discretized, we must have
%
\[ \text{rad}(B_n) \geq (A / 2^d)^{\frac{1}{d-1}}\; 2^{-j}. \]
%
%
%
% CALCULATIONS q = 1:
%
%   L^1 norm for fixed ball is 2^{k(d-1)} ( 2^j \text{rad}(B_n) )^{d + 2 - M}
%       Summing over, using the fact that sum 2^j rad(B_n) <= A^{-1} #(E_k)
%       and that 2^j rad(B_n) >> A^{1/(d-1)}
%   
%       2^{k(d-1)} A^{-1} (2^j rad(B_n))^{d + 1 - M} #(E_k)
%       2^{k(d-1)} A^{-(M - 2)/(d-1)} #(E_k)
%       A^{-(M-2)/(d-1)} 2^{k(d-1)} #(E_k)
%       (L/2^{jd)}^{-(2-p)(M-2)/2}) 2^{k(d-1)} #(E_k)
%           Choosing M Appropriately
%       L^{1-p} 2^{jd(p-1)} 2^{k(d-1)} #(E_k)
%
%       But this means the points above L have measure
%
%       2^{jd(p-1)} 2^{k(d-1)} #(E_k) L^{-p}
%
%       Consider x_n
%
%       With 2^{u eps} <= x_n <= 2^k
%
%       And sum x_n <= 2^{-u} A

Suppose we pick $M$ to be any integer larger than $3 + 2(p-1)/(2-p)$. Then we conclude that for $\lambda \geq 10^{d+1} 2^{jd}$, using the bound on $\sum \text{rad}(B_n)$ and the individual upper bounds on $\text{rad}(B_n)$, imply that
%
% 2^{jd(p-1)}
%
\begin{align*}
    &\Big\| \sum\nolimits_{a,b} \sum\nolimits_{(x_0,t_0) \in \mathcal{E}_{k,a,b} - \widehat{\mathcal{E}}_{k,a,b}} 2^{k \left( \frac{d-1}{2} \right)} 2^{-(a+b)} f_{x_0,t_0} \Big\|_{L^1(\text{Bad}_k^c)}\\
    &\quad\quad\quad \lesssim_X \sum_n 2^{k (d-1)} (2^j \text{rad}(B_n))^{d + 2 - M}\\
    &\quad\quad\quad \lesssim 2^{k(d-1)} A^{- \frac{M - d - 2}{d - 1}} A^{-1} \# \mathcal{E}_k \\
    &\quad\quad\quad \lesssim 2^{k (d-1)} A^{- \frac{M-3}{d-1}} \# \mathcal{E}_k\\
    &\quad\quad\quad \lesssim 2^{k(d-1)} \left( \log(\lambda / 2^{jd})^{\left( \frac{d-1}{2} \right)} (\lambda / 2^{jd})^{(2-p) \left( \frac{d - 1}{2} \right)} \right)^{- \frac{M-3}{d-1}} \# \mathcal{E}_k\\
    &\quad\quad\quad \lesssim 2^{jd(p-1)} \Big( 2^{k(d-1)} \# \mathcal{E}_k \Big) \lambda^{1-p}.
%    &\quad\quad\quad \lesssim 2^{j \left[\alpha(p) + d/p' \right]} ( \lambda / 2^{jd} )^{- (2 - p) \left( \frac{d-1}{2} \right) \left( \frac{M-2}{d-1} \right)} \left( 2^{k \left( \frac{d - 1}{2} \right)} \# \mathcal{E}_k \right)\\
%    &\quad\quad\quad \lesssim 2^{jd/p'} \left( 2^{k \left( \frac{d - 1}{2} \right)} \# \mathcal{E}_k \right)
\end{align*}
%
The last inequality again used the fact that $\log(x)^\delta x^{-\varepsilon} \lesssim 1$ for any $\delta,\varepsilon > 0$ and $x \geq 10^{d+1}$.
% j(d-1)
%
%     \left( A^{-1} \# \mathcal{E}_k \right) (A / 2^d)^{\frac{d+1-M}{d-1}}\\
%    &\quad\quad\quad \lesssim_n 2^{k \left( \frac{3d - 3}{2} \right)} A^{\frac{2 - M}{d - 1}} \# \mathcal{E}_k\\
%    &\quad\quad\quad \lesssim_n 2^{k \left( \frac{3d - 3}{2} \right)} \left( \lambda / 2^{jd} \right)^{\frac{(2-p)(2 - M)}{2}} \# \mathcal{E}_k\\
%    &\quad\quad\quad \lesssim 2^{jd(p-1)} 2^{k(d-1)} \Big( 2^{k \left( \frac{d-1}{2} \right)} \# \mathcal{E}_k \Big) \lambda^{1-p}
%
But this means that
%
\begin{align*}
    &\Big\| \sum\nolimits_k \sum\nolimits_{a,b} \sum\nolimits_{(x_0,t_0) \in \mathcal{E}_{k,a,b} - \widehat{\mathcal{E}}_{k,a,b}} 2^{k \left( \frac{d-1}{2} \right)} 2^{-(a+b)} f_{x_0,t_0} \Big\|_{L^1(\text{Bad}^c)}\\
    &\quad\quad\quad\quad \lesssim \Big( 2^{jd(p-1)} \sum_k 2^{k(d-1)} \# \mathcal{E}_k \Big) \lambda^{1-p}.
\end{align*}
%
Applying Markov's inequality, this is enough to justify that
%
\begin{align} \label{LastMarkovRemainderBound}
\begin{split}
    &\bigg| \Big\{ x \in \text{Bad}^c: \Big|\sum\nolimits_{a,b} \sum\nolimits_k \sum\nolimits_{(x_0,t_0) \in \mathcal{E}_{k,a,b} - \widehat{\mathcal{E}}_{k,a,b}} 2^{-(a+b)} f_{x_0,t_0}(x)\Big| \geq \lambda / 2 \Big\} \bigg|\\
    &\quad\quad\quad\quad\quad\quad\quad\quad\quad\quad\quad \lesssim 2^{jd(p-1)} \left( \sum_k 2^{k(d-1)} \# \mathcal{E}_k \right) \lambda^{-p}.
\end{split}
\end{align}
%
Putting \eqref{ChebyshevFirstBound}, \eqref{BadSet}, and \eqref{LastMarkovRemainderBound} together implies that for $\lambda \geq 2^{jd}$,
%
\begin{align*}
    &\bigg| \Big\{ x: \Big|\sum\nolimits_{a,b} \sum\nolimits_k \sum\nolimits_{(x_0,t_0) \in \mathcal{E}_{k,a,b}} 2^{-(a+b)} f_{x_0,t_0}(x)\Big| \geq \lambda \Big\} \bigg| \\
    &\quad\quad\quad \lesssim 2^{jd(p-1)} \left( \sum_k 2^{k(d-1)} \# \mathcal{E}_k \right) \lambda^{-p}
\end{align*}
%
This completes the proof of \eqref{FSumLPInfBound}, and thus the theorem.
\end{proof}

% The sum of S_{x_0,t_0}, where (x_0,t_0) range over some ball B_j
% has measure concentrated on a radius ~ 2^k / R annulus with thickness rad(B_j), a set with measure (2^k / R)^{d-1} rad(B_j).

%
% Picking A = (L / R^{d-1})^{ps_n}
% gives slightly too large a bound,
% by a factor log(L / R^{d-1})
%
% If we can replace ps_p by ps_p - epsilon
% we're probably good.
%
% Or even (L / R^{d-1})^{ps_p} log ( L / R^{d-1} )^{-O(1)}
%
% Then
%
% We're left with dealing with the concentrated case
% points are covered by balls B_1 ... B_N with radius at most 2^k / R
% and with sum rad(B_i) <= R^{-1} A^{-1} #E_k
%
% <= R^{-1} (L / R^{d-1})^{-(d-1)(1 - p/2)} log(L / R^{d-1})^{O(1)}
%
% The sum of S_{x_0,t_0}, where (x_0,t_0) range over some ball B_j
% has measure concentrated on a radius ~ 2^k / R annulus with thickness rad(B_j), a set with measure (2^k / R)^{d-1} rad(B_j).
%
% Summing over j, the total measure of these annulus are
%
% << (2^k / R)^{d-1} R^{-1} (L / R^{d-1} )^{-(d-1)(1 - p/2)} log(L / R^{d-1})^{O(1)}
% 2^{k(d-1)} R^{-d + (d-1)p} L^{-p} #(E_k)
%

% This is good provided that
% R^{d-2} L^{2p/(d-1) - 2} <= L^{-p} R^{(d-1)p - d}
% R^{(2-p)(d-1)} <= L^{2-p - 2p/(d-1)}
% R^{(2-p)(d-1)} <= L^{2 - p( (d+1)/(d-1) )}
% L >= R^{(d-1) [ (2-p)/(2 - p(d+1)/(d-1))]} = R^{(d-1) + eps_p}
% The right exponent exceeds d-1 in the range we are considering
% Shouldn't this be bad?

% s(lambda))^{2p/(d-1)} <= lambda^{2-p} R^{-(2-p)(d-1)}
% s(lambda) <= lambda^{ s_p } R^{ - (d-1)s_p } = ( lambda / R^{d-1} )^{s_p}
% p = 2 (d-1)/ (d+1)

% (2 - p)(d-1) / 2p > 1 in the range we are considering

% log_2(s(lambda)) s(lambda) <= lambda^{(2-p)(d-1)/2p} R^{[(d-1)^2(p-1) - d(d-1)]/2p}


% \log_2(\lambda) lambda^{(d+1)/(d-1)[p - 2(d-1)/(d+1)]} <= R^{2 + (d-1)p - 2d}
% if
% R^{(2-p)(d-1)} <= lambda^{2 - p(d+1)/(d-1)}
% which holds iff
% lambda >= R^{(2-p)(d-1)/[2 - p(d+1)/(d-1)]}
% L^{(2p)/(d-1) - 2 + p} <= R^{(d-1)p - d + 2 - d} 

% d > (p+2)/(2-p)
% 


\end{comment}








\section{Analysis of Regime II via Local Smoothing} \label{regime2finalsection}

In this section, we bound the operators $\{ T^{II} \}$, by a reduction to an endpoint local smoothing inequality, namely, the inequality that
%
\begin{equation} \label{thelocalsmoothinginequality}
    \| e^{2\pi i t P} f \|_{L^{p'}(X) L^{p'}_t(I_0)} \lesssim \| f \|_{L^{p'}_{s - 1/p'}}.
\end{equation}
%
This inequality is proved in Corollary 1.2 of \cite{LeeSeeger} for $1 < p < 2(d-1)/(d+1)$ for classical elliptic pseudodifferential operators $P$ satisfying the cosphere assumption of Theorem \ref{maintheorem}. The range of $p$ here is also precisely the range of $p$ in Theorem \ref{maintheorem}. Alternatively, Lemma \ref{LpBoundLemma} can be used to prove \eqref{thelocalsmoothinginequality} independently of \cite{LeeSeeger} in the same range by a generalization of the method of Section 10 of \cite{HeoandNazarovandSeeger}.

\begin{lemma} \label{LocalSmoothingLargeTimesTheorem}
    Using the notation of Proposition \ref{TjbLemma}, let
    %
    \[ T^{II} = \int b^{II}(t) (e^{2 \pi i t P} \circ Q_R)\; dt. \]
    %
    For $1 < p < 2 (d-1)/(d+1)$, we then have
    %
    \[ \| T^{II} u \|_{L^p(X)} \lesssim R^{s - 1/p'} \| b^{II} \|_{L^p(I_0)} \| u \|_{L^p(X)}. \]
\end{lemma}
\begin{proof}
    For each $R$, the \emph{class} of operators of the form $\{ T^{II} \}$ formed from a given function $b^{II}$ is closed under taking adjoints. Indeed, if $T^{II}$ is obtained from $b^{II}$, then $(T^{II})^*$ is obtained from the multiplier $\overline{b^{II}}$. Because of this self-adjointness, if we can prove that
    %
    \begin{equation}
        \| T^{II} u \|_{L^{p'}(X)} \lesssim R^{s - 1/p'} \| b^{II} \|_{L^p(I_0)} \| u \|_{L^{p'}(X)},
    \end{equation}
    %
    then we obtain the required result by duality. We apply this duality because it is easier to exploit local smoothing inequalities in $L^{p'}(X)$ since now $p' > 2$.

    We begin by noting that the operators $\{ Q_R \}$, being a bounded family of order zero pseudo-differential operators, are uniformly bounded on $L^{p'}(X)$. Thus
    %
    \begin{equation}
    \begin{split}
        \| T^{II} u \|_{L^{p'}(X)} &= \Big\| \Big( \int_{I_0} b^{II}(t) e^{2 \pi i tP} (Q_R u) \Big) \Big\|_{L^{p'}(X)}\\
        &\lesssim \Big\| \Big( \int_{I_0} b^{II}(t) e^{2 \pi i tP} (Q_R u) \Big) \Big\|_{L^{p'}(X)}.
    \end{split}
    \end{equation}
    %
    Applying H\"{o}lder and Minkowski's inequalities, we find that
    %
    \begin{equation}
    \begin{split}
        \| T^{II}u \|_{L^{p'}(X)} &\leq \| b^{II} \|_{L^p(\RR)} \Big\| \Big( \int_{I_0} |e^{2 \pi i t P} (Q_R u)|^{p'} \Big)^{1/p'} \Big\|_{L^{p'}(X)}.
    \end{split}
    \end{equation}
    %
    Applying the endpoint local smoothing inequality \eqref{thelocalsmoothinginequality}, we conclude that
    %
    \begin{equation}
    \begin{split}
        \| T^{II} u \|_{L^{p'}(X)} &\lesssim \| b^{II} \|_{L^p(\RR)}  \| e^{2 \pi i P} (Q_R u) \|_{L^{p'}_t L^{p'}_x}\\
        &\lesssim  \| b^{II} \|_{L^p(\RR)}  \| Q_R u \|_{L^q_{s - 1/p'}(X)},
    \end{split}
    \end{equation}
    %
    Bernstein's inequality for compact manifolds (see \cite{Sogge}, Section 3.3) gives
    %
    \begin{equation}
        \| Q_R u \|_{L^{q}_{s - 1/p'}(X)} \lesssim R^{s - 1/p'} \| u \|_{L^p(X)}.
    \end{equation}
    %
    Thus we conclude that
    %
    \begin{equation}
        \| T^{II}u \|_{L^{p'}(X)} \lesssim R^{s - 1/p'} \| b^{II} \|_{L^p(I_0)} \| u \|_{L^{p'}(X)},
    \end{equation}
    %
    which completes the proof.
\end{proof}

Combining Lemma \ref{regime1Lemma} and Lemma \ref{LocalSmoothingLargeTimesTheorem} completes the proof of Proposition \ref{TjbLemma}, and thus of inequality \eqref{dyadicMainReulst}. Since \eqref{TrivialLowFrequencyBound} was already proven as a consequence of Lemma \ref{lowjLemma}, this completes the proof of Theorem \ref{maintheorem}.

%Indeed, if we fix an arbitrary point $v_0 \in T_x^*M$, and consider the smallest closed ball $B \subset T_x^* M$ centered at $v_0$ and containing $S_x^*$, then the sphere $\partial B$ must share the same tangent plane as $S_x^*$ at some point. All principal curvatures of $\partial B$ are positive, and at this point all principal curvatures of $S_x^*$ must be greater than the principal curvatures of $\partial B$, since $S_x^*$ curves away faster than $\partial B$ in all directions. By continuity, we conclude that the principal curvatures are everywhere positive. 





\begin{comment}

\section{Necessary Conditions}

Let $m: [0,\infty) \to \CC$ be a function such that the spectral multiplier operators
%
\[ \{ m(\sqrt{-\Delta} / R) \} \]
%
are uniformly bounded on $L^p(S^d)$. Define $\psi: [0,\infty) \to [0,\infty)$ such that $\sqrt{-\Delta} = \psi(P)$ for an elliptic pseudodifferential operator $P$ commuting with $\sqrt{-\Delta}$, and with $\sigma(P) \subset \NN$. Define $m_R(\lambda) = m(\psi(\lambda)/R)$, so that 
%
\[ m(\sqrt{-\Delta} / R) = m_R(P). \]
%
If $\text{supp}(m) \subset [1/2,2]$, then $\text{supp}(m_R) \subset [R/4,4R]$ for suitably large $R$. If we define
%
\[ b_R(t) = \sum_{n = 0}^\infty m_R(n) \cos(2 \pi n t), \]
%
then we can write
%
\[ m_R(P) = 2 \int_0^{1/2} b_R(t) \cos(2 \pi P t)\; dt \]
%
Consider a normal coordinate system $\mathfrak{x}$ centered at a point $x_0 \in S^d$, and use it to define a family of bump functions $f_R: S^d \to \CC$ for $R \geq 1$ by setting $f_R = R^{d/p} f(R \mathfrak{x})$ for some fixed $f \in C_c^\infty(\RR^d)$ supported in a small neighbourhood of the origin. If we consider a partition of unity $\chi_0, \chi_1, \chi_2$ subordinate to the cover $[0,10/R)$, $(5/R, 1/2 - 5/R)$, $(1/2 - 10/R, 1/2]$, then we can write
%
\[ m_R(P) \{ f_R \} = 2(g_{R,0} + g_{R,1} + g_{R,2} + g_{R,\infty}) \]
%
where
%
\[ g_{R,j} = \int_0^{1/2} \chi_j(t) b_R(t) C_j(t) \{ f_R \}\; dt \]
%
and
%
\[ g_{R,\infty} = \int_0^{1/2} b_R(t) C_\infty(t) \{ f_R \}\; dt, \]
%
where $C_j(t)$ is a parametrix for $\cos(2 \pi P t)$, and $C_\infty(t)$ is an operator with a smooth kernel in $C^\infty([0,1/2] \times S^d \times S^d)$. Now applying Fourier series, we can write
%
\[ g_{R,\infty} = \sum_{n = 0}^\infty m_R(n) \widehat{C}_\infty(n) \{ f_R \}\; dt, \]
%
where $\widehat{C}_\infty(n)$ is rapidly decaying in $n$. But since $m_R(n)$ is supported on $[R/4,4R]$, we conclude that
%
\[ |\widehat{C}_\infty(n) \{ f_R \}| \lesssim_N R^{-N} |n|^{-10d}, \]
%
and thus
%
\[ \| g_{R,\infty} \|_{L^\infty(S^d)} \lesssim_N R^{-N} \| m \|_{L^\infty}\quad\text{for all $N \geq 0$}. \]
%
One can show (via a Fourier inversion, interpolation, and pseudodifferential operator argument) that for $x \in S^d$ with $d(x,x_0) \geq C_0 / R$,
%
\[ |g_{R,0}(x)| \lesssim_N \| m \|_{L^\infty} R^{d/p} |R d(x,x_0) |^{-d-N} \quad\text{for all $N \geq 0$}. \]
%
Similarily, if $x_0^* \in S^d$ is the antipode of $x_0$, then for $d(x,x_0^*) \geq C_0 / R$,
%
\[ |g_{R,2}(x)| \lesssim_N \| m \|_{L^\infty} R^{d/p} |R d(x,x_0^*) |^{-d-N} \quad\text{for all $N \geq 0$}. \]
%
If we let $\Omega = \{ x \in S^d: d(x,x_0) \geq C_0 / R\ \text{and}\ d(x,x_0^*) \geq C_0 / R \}$, then
%
\[ \| g_{R,0} \|_{L^p(\Omega)} + \| g_{R,2} \|_{L^p(\Omega)} \lesssim \| m \|_{L^\infty} R^{-d/p^*} \]
%
Now we introduce the Hadamard parametrix, writing
%
% \[ C_1(t,x,y) = A_d\; t \sum_{j = 0}^\infty \frac{( -1 )^j}{4^j \Gamma(j - \frac{d-1}{2})} W_j(x,y) ( d(x,y)^2 - t^2 )^{j - \frac{d+1}{2}} \]
\[ C_1(t,x,y) = t \int_0^\infty a(x,y,\tau) e^{i \tau ( d(x,y)^2 - t^2 )}\; d\tau, \]
%
where $a$ is a symbol of order $(d-1)/2$ in the $\tau$ variable, whose principal symbol is a power of the volume density on the manifold $M$. In particular, $a(x,y,\tau) > 0$ for sufficiently large $\tau$. We consider a partition of unity $\tilde{\chi}_0, \tilde{\chi}_1, \tilde{\chi}_2$ subordinate to $[0,1/5), (1/10, 10), (5,\infty)$ and use this parition to partition the regions of integration in the $\tau$ variable, writing $g_{R,1} = g_{R,1,0} + g_{R,1,1} + g_{R,1,2}$, where
%
\[ g_{R,1,j}(x) = \int \int a(x,y,\tau) \tilde{\chi}_j(\tau / R) b_R(t) t e^{i \tau ( d(x,y)^2 - t^2 ) } f_R(y)\; d\tau\; dt\; dy. \]
%
Integration by parts in the $t$ variable shows that for $j \in \{ 0, 2 \}$ $\| g_{R,1,j} \|_{L^\infty} \lesssim_N R^{-N} \| m \|_{L^\infty}$. Rescaling, writing $a_R(x,y,\tau) = R^{- \frac{d-1}{2}} a(x,y,R \tau) \tilde{\chi}_j(\tau)$ we can write
%
\begin{align*}
    g_{R,1,1}(x) &= R^{\frac{d+1}{2} + \frac{d}{p}} \iiint a_R(x,y,\tau) b_R(t) t e^{i R \tau ( d(x,y)^2 - t^2 )} f(Ry)\; d\tau\; dt\; dy.
%    &= R^{\frac{d+1}{2} + \frac{d}{p}} \iint (\mathcal{F}_\tau a_R)(x,y,R [ d(x,y)^2 - t^2 ]) b_R(t) t f(Ry)\; dt\; dy
\end{align*}
%
For our purposes (TODO: Justify Later) we may assume $d(x,y) \approx d(x,0) + y \cdot x / |x|$, and that $d(x,y)^2 \approx d(x,0)^2 + 2 d(x,0)  (y \cdot x / |x|)$. Thus
%
\[ g_{R,1,1}(x) \approx R^{\frac{d+1}{2} - \frac{d}{p'}} \iint a_R(x,\tau) b_R(t) t e^{i R \tau [d(x,0)^2 - t^2]}. \]
%
Thus
%
\[ g_{R,1,1}(x) \approx R^{\frac{d+1}{2} - \frac{d}{p'}} \int \widehat{a}_R \big(x, R[d(x,0)^2 - t^2] \big) b_R(t) t\; dt \]

\end{comment}

\chapter{Combining Scales Via Atomic Decompositions} \label{chap:spectralatomicdchapter}
%!TEX root = ../main.tex

In this chapter, we discuss a prospective method of combining frequency scales for spectral multipliers, proving Theorem \ref{atomicscalestheorem}. We take an operator $P$ satisfying Assumptions A, B, and C, and show we can combine estimates between frequency scales under an assumption which is \emph{slightly stronger} than controlling each part of the operator separately. We do this by adapting the techniques of Heo, Nazarov, and Seeger from \cite{HeoandNazarovandSeeger2} to the case of compact manifolds, which we previous discussed in Section \ref{sec:combiningscaleswithatomicdecompositions}. We now restate the setup to Theorem \ref{atomicscalestheorem}.
    
Fix $k \geq 0$, and consider a pair of maximal $2^{-k}$ separated subsets $\mathcal{X}_k$ and $\mathcal{T}_k$ of $X$ and of $[0,\Pi]$, where $1/\Pi$ is the spacing between points in the arithmetic progression that the eigenvalues of the operator $P$ are contained in. Fix a family of $L^1$-normalized functions $\mathfrak{b} = \{ b_{t_0} \}$ and $\mathfrak{u} = \{ u_{x_0} \}$, with $u_{x_0}$ supported on a $2^{1-k}$ neighborhood of $x_0$, and $b_{t_0}$ on a $2^{1-k}$ interval centered at $t_0$. Also fix a bump function $q \in C_c^\infty(\RR)$ with $\supp(q) \subset [1/4,4]$ and $q(\lambda) = 1$ for $\lambda \in [1/2,2]$, and define $Q_k = q(P/2^k)$, which we view as `frequency localizations' at a scale $2^k$. For each $(x_0,t_0) \in \mathcal{X}_k \times \mathcal{T}_k$, define $S\!_{x_0,t_0} = \int b_{t_0}(t) (\cos(2 \pi i t P) \circ Q_k) \{ u_{x_0} \}$, a time average of a frequency localized solution to the wave equation $\partial_t^2 u = - P^2 u$. Finally, define an operator $A_k$ from functions on $\mathcal{X}_k \times \mathcal{T}_k$ to functions on $X$ by $A_k \{ c \} = \sum\nolimits_{(x_0,t_0) \in \mathcal{X}_k \times \mathcal{T}_k} c(x_0,t_0) S\!_{x_0,t_0}$. We assume uniform bounds on such operators.

\vspace{0.5em}

\noindent \fbox{\parbox{\textwidth}{\textbf{Assumption} $\text{Wave-Bound}(p)$:
There exists a constant $C_0 > 0$ such that
%
\begin{align*}
    \left\| A_k \{ c \} \right\|_{L^p(X)} \leq C_0\; 2^{k d/p'} \left( \sum\nolimits_{(x_0,t_0) \in \mathcal{X}_k \times \mathcal{T}_k} \left[ |c(x_0,t_0)| \langle 2^k t_0 \rangle^{s} \right]^p \right)^{1/p}.
\end{align*}
% but with 2^{kd/p'}          < 2^k t_0 >^{ (d-1)/2 }
%
uniformly in $k$, $\mathfrak{b}$, and $\mathfrak{u}$, with $s = (d-1)(1/p - 1/2)$.
%
%\begin{align*}
%    \left\| A_k \{ c \} \right\|_{L^{p_*}(X)} \leq C\; 2^{kd} \left( 2^{-k(2d+1)} \sum\nolimits_{(x_0,t_0) \in \mathcal{X}_k \times \mathcal{T}_k} |c(x_0,t_0)|^{p_*} \langle 2^k t_0 \rangle^{d-1} \right)^{1/p_*}.
%\end{align*}
% 2^{kd/p'} BUT PICK UP 2^{-k(d+1)}
% SO 2^{- k(1 + d/p)}
}}
\vspace{0.4em}

As we have seen in the previous chapter, such a bound naturally arises in the study of averages of the wave equation. In particular, $\text{Wave-Bound}(p,d)$ implies that for any function $a$, if $a_k = \chi(t) a(2^k t)$ for some $\chi \in C_c^\infty(\RR)$ with $1 = \sum \chi(\lambda/2^k)$ for $\lambda \neq 0$, then
%
\begin{equation}
    \| a_k(P/2^k) \|_{L^p(X) \to L^p(X)} \lesssim \| a \|_{R^{s,p}[0,\infty)} \quad\text{for $s = (d-1)(1/p - 1/2)$},
\end{equation}
%
uniformly in $k$. The main result of this section is that one can also `sum these bounds' to bound $a(P) = \sum a_k(P/2^k)$.

\thmatomicscalestheorem*

%Such a theorem implies a characterization of $L^p$ boundedness for rescaled spectral multipliers. Since $\text{Wave-Bound}(p,d)$ has been proved for $1/p - 1/2 > 1/2d$, in this range, this paper completes the proof of the characterization of $L^p$ boundedness for general multipliers.

%In order to study the behaviour of the multipliers $T_k = m_k(P/2^k)$, it is natural to represent the operator in terms of the wave equation on $M$, so that we can exploit geometric information about the behaviour of waves on $M$. We thus write
%
%\[ T_k = \int_0^\infty a_k(t) \cos(2 \pi t P), \]
%
%where $2^{-k} a_k(\cdot / 2^k)$ is the cosine transform of $m_k$. Given that $T_k$ is supported on the union of eigenspaces in the eigenband $[2^{k-1}, 2^{k+1}]$, so that the operator, and all inputs are `frequency localized' at a scale $2^k$, uncertainty principle heuristics suggest that it might be profitable to consider a further decomposition in spacetime at a scale $2^{-k}$; consider maximal $2^{-k}$ separated discrete subsets $\mathcal{X}_k$ and $\mathcal{T}_k$ of $M$ and $[0,\infty)$ respectively,  and consider an associated pair of partitions of unity $\{ \chi_{x_0} \}$ and $\{ \eta_{t_0} \}$ adapted to the balls of radius $2^{1-k}$ centered at the points in $\mathcal{X}_k$ and $\mathcal{T}_k$. Then given $u \in L^p(M)$, we can write
%
%\[ T_k u = \sum\nolimits_{(x_0,t_0) \in \mathcal{X}_k \times \mathcal{T}_k} c(x_0,t_0) f_{x_0,t_0}, \]
%
%where
%
%\[ f_{x_0,t_0} = \int_0^\infty a_{t_0} \cos(2 \pi t P) \{ u_{x_0} \}\; dt, \]
%
%where $a_{t_0}$ and $u_{x_0}$ are $L^1$-normalized multiples of $\eta_{t_0} a_k$ and $\chi_{x_0} u$ respectively, and $c(x_0,t_0) = \| \chi_{x_0} u \|_{L^1(M)} \| \eta_{t_0} a_k \|_{L^1(\RR)}$. One can verify that uniform bounds of the form
%
%\[ \| T_k \|_{L^p(M) \to L^p(M)} \lesssim C_p(m) \]
%
%follow if we could show an `$L^p$-cancellation' inequality of the form
%
%\[ \left\| \sum\nolimits_{x_0,t_0} \langle 2^k t_0 \rangle^{\frac{d-1}{2}} c(x_0,t_0) f_{x_0,t_0} \right\|_{L^p(M)} \lesssim 2^{k \beta(p,d)} \left( \sum\nolimits_{x_0,t_0} |c(x_0,t_0)|^p \langle 2^k t_0 \rangle^{d-1} \right)^{1/p}, \]
% ( Sum |chi_{x_0} u|_{L^1}^p )^{1/p} << 2^{-kd/p^*} |u|_{L^p}
% 
% ( Sum H^p W )^{1/p}
% ( (H W)^p W^{1 - p} )^{1/p}
%
% We have bounds on
%   int |m_k^(t)|^p <t>^{(d-1)(1 - p/2)}
%     = int |a_k(t)|^p <2^k t>^{(d-1)(1 - p/2)}
%     = Sum int |a_{t_0}(t)|^p <2^k t>^{(d-1)(1 - p/2)}
%     = Sum < 2^k t_0 >^{(d-1)(1 - p/2)} |a_{t_0}|_{L^p}^p
%     = Sum < 2^k t_0 >^{(d-1)(1 - p/2)} |a_{t_0}|_{L^1}^p 2^{k(p - 1)}
%   Thus (Sum |a_{t_0}|_{L^1}^p < 2^k t_0 >^{(d-1)(1 - p/2)} )^{1/p} << 2^{-k/p^*} C_p(m)
%
%
% ( Sum |a_{t_0}|^p < 2^k t_0 >^{d-1} )^{1/p}
%
%where $\beta(p,d) = (d + 1)/p^*$. In this paper, we will establish general bounds for multipliers, under an assumption that such a bound holds uniformly in $k$.

%\[ Tu = \sum\nolimits_k m_k( P / 2^k ) \{ u \}, \]
%
%where $\sup_k C_p(m_k) < \infty$, and $\text{supp}(m_k) \subset (1/2,2)$. It is conjectured that under these assumptions, for $|1/p - 1/2| > 1/2d$, the operator $T$ is bounded on $L^p(S^d)$. The goal of this paper is to obtain such bounds, under the assumption that certain bounds associated with the wave equation on $S^d$ hold uniformly at each frequency scale. The novelty in this paper is thus in obtaining a general method to efficiently combine bounds on operators at each frequency scale together.

% The finiteness of $C_p(m)$ is also necessary to control `the high frequency behaviour' of the function, in a certain sense. Namely, if the functions $\{ \chi_k \}$ are adapted to $(1/2,2)$, uniformly in $k$, then one can show that for $1/p - 1/2 > 1/2d$,
%
%\[ C_p(\text{Dil}_\rho m) \lesssim C_p(m). \]
%
%It is a result of Mitjagin that
%
%\[ \limsup\nolimits_{\rho \to \infty} \| (\text{Dil}_\rho m)(P) \|_{L^p(M) \to L^p(M)} \gtrsim C_p(m). \]
%
%Thus we conclude that
%
%\[ \limsup\nolimits_{\rho \to \infty} \| (\text{Dil}_\rho m)(P) \|_{L^p(M) \to L^p(M)} \sim \sup C_p(m). \]
%
%Thus we obtain necessary and sufficient conditions for $L^p$ boundedness `in the high frequency regime' (as $\rho \to \infty$).

Since the right hand side of the inequality in Theorem \ref{atomicscalestheorem} is invariant under dilations of $a$, it suffices to prove a bound of the form $\| a(P) \|_{L^p(X) \to L^p(X)} \lesssim \| a \|_{R^{s,q}[0,\infty)}$. To prove Theorem \ref{atomicscalestheorem}, we use an analogue of the technique of atomic decompositions introduced in Section \ref{sec:combiningscaleswithatomicdecompositions}. The following lemma describes the properties of this atomic decomposition useful to us, and is proved in an appendix.

\begin{lemma} \label{atomicdecompositionlemma}
    Consider coordinate charts $\{ U_\alpha \}$ covering $X$. Then, for any measurable function $u: X \to \CC$, if we consider the dyadic decomposition $u = \sum u_k$, where $u_k = Q_k u$ is a quasimode with eigenvalue $2^k$, then we have a further decomposition
    %
    \[ u_k = \sum\nolimits_\alpha \sum\nolimits_H \sum\nolimits_{W \in \mathcal{W}_{\alpha,H}} A_{\alpha,k,H,W}, \]
    %
    where $H$ ranges over powers of $2$. For each $\alpha$ and $H$, $\mathcal{W}_{\alpha,H}$ is a family of almost disjoint dyadic cubes in the coordinate system $U_\alpha$ whose union is a set $\Omega_{\alpha,H}$, such that the following properties hold:
    %
    \begin{itemize}
        \item The 10-fold dilates $\{ W^* : W \in \mathcal{W}_{\alpha,H} \}$ have the bounded overlap property.

        \item If $l(W) = 2^l$, then $A_{\alpha,k,H,W} = 0$ for $k < -l$.

        \item For each $W$, $\text{supp}(A_{\alpha,k,H,W}) \subset W$, but as $H$ varies, the functions $\{ A_{\alpha,k,H,W} \}$ have disjoint support.

        \item For each $H$,
        %
        \[ \left( \sum\nolimits_k \sum\nolimits_\alpha \sum\nolimits_{W \in \mathcal{W}_{\alpha,H}} \| A_{\alpha,k,H,W} \|_{L^2(X)}^2 \right)^{1/2} \lesssim H |\Omega_{\alpha,H}|, \]

        \item For any choice of indices $k(\alpha,W)$ for each $\alpha$ and $W$, we have
        %
        \[ \left( \sum\nolimits_\alpha \sum\nolimits_{W \in \mathcal{W}_{\alpha,H}} |W| \| A_{\alpha,k(\alpha,W),H,W} \|_{L^\infty(X)}^p \right)^{1/p} \lesssim H |\Omega_{\alpha,H}|. \]

        \item For each $\alpha$,
        %
        \[ \left( \sum\nolimits_H H^p |\Omega_{\alpha,H}| \right)^{1/p} \lesssim \| u \|_{L^p(X)}. \]
    \end{itemize}
\end{lemma}
\begin{proof}
    Without loss of generality, by performing a partition of unity and then working in one of the coordinate systems $U_\alpha$, we may assume $u$ is a function on $\RR^d$. But then the decomposition is precisely that of Section 7 of \cite{HeoandNazarovandSeeger}, i.e. using the $L^p$ boundedness of a square function of Peetre \cite{Peetre}, and the level sets of that square function applied to $u$, in order to form the atoms.
\end{proof}

To exploit this atomic decomposition, we write
%
\begin{equation}
    a(P) u = \sum\nolimits_k m_k(P/2^k) \{ u_k \} = \sum\nolimits_\alpha \sum\nolimits_k \sum\nolimits_H \sum\nolimits_{W \in \mathcal{W}_{\alpha,H}} m_k(P/2^k) \{ A_{\alpha,k,H,W} \}.
\end{equation}
%
We regroup this sum as
%
\begin{equation}
    \sum\nolimits_\alpha \sum\nolimits_k \sum\nolimits_H \sum\nolimits_{l \geq 0} \sum\nolimits_{\substack{W \in \mathcal{W}_{\alpha,H}\\l(W) = l - k}} a_k(P/2^k) \{ A_{\alpha,k,H,W} \}.
\end{equation}
%
For each $k$ and $l$, we write $a_k(P/2^k) = T_{k,l,\text{Short}} + T_{k,l,\text{Long}}$, where
%
\begin{equation}
    T_{k,l,\text{Short}} = \int_0^\infty \chi( 2^{k-l} t ) 2^k\;\! \widehat{a}_k(2^k t) \cos(2 \pi i t P)
\end{equation}
%
and
%
\begin{equation}
    T_{k,l,\text{Long}} = \int_0^\infty (1 - \chi(2^{k-l} t )) 2^k\;\! \widehat{a}_k(2^k t) \cos(2 \pi i t P). 
\end{equation}
%
For $f = Tu$, we thus write $f = f_{\text{Short}} + f_{\text{Long}}$, where
%
\begin{equation}
    f_{\text{Short}} = \sum\nolimits_\alpha \sum\nolimits_k \sum\nolimits_H \sum\nolimits_{l \geq 0} \sum\nolimits_{\substack{W \in \mathcal{W}_{\alpha,H}\\l(W) = l - k}} T_{k,l,\text{Short}} \{ A_{\alpha,k,H,W} \} = \sum f_{\alpha,k,H,W,\text{Short}}. 
\end{equation}
%
and
%
\begin{equation}
    f_{\text{Long}} = \sum\nolimits_\alpha \sum\nolimits_k \sum\nolimits_H \sum\nolimits_{l \geq 0} \sum\nolimits_{\substack{W \in \mathcal{W}_{\alpha,H}\\l(W) = l - k}} T_{k,l,\text{Long}} \{ A_{\alpha,k,H,W} \} = \sum f_{\alpha,k,H,W,\text{Long}},
\end{equation}
%
and analyze each part using separate techniques.

\section{Short Range Bounds}

To obtain short range bounds, we exploit the propogation speed of the operators $\cos(2 \pi i t P)$, and the bounded overlap of the sets $W_{\alpha,H}$ for a fixed $H$. To obtain an $L^q$ bound for $f_{\text{Short}}$, we interpolate between an $L^2$ bound and an $L^1$ bound, at a fixed quantity $H$. So write $f_{\text{Short}} = \sum_\alpha \sum_k \sum_H f_{\alpha,k,H}$.
%To obtain $L^2$ estimates, it becomes necessary to estimate the inner products $\langle f_{\alpha,k,H,W}, f_{\alpha,k,H,W'} \rangle$.
%
%\[ \sum\nolimits_{W' \cap W^* = \emptyset} \langle f_{\alpha,k,H,W}, f_{\alpha,k,H,W'} \rangle \]
In $L^2$, different frequencies are orthogonal, so that
%
\begin{equation}
\begin{split}
    \| f_{H,\text{Short}} \|_{L^2(X)} \lesssim \left( \sum\nolimits_k \left\| \sum\nolimits_{W \in \mathcal{W}_{\alpha,H}} f_{\alpha,k,H,W,\text{Short}} \right\|_{L^2(X)}^2 \right)^{1/2}.
\end{split}
\end{equation}
%
We note that by the finite propogation speed of $\cos(2 \pi t P)$, $\langle f_{\alpha,k,H,W,\text{Short}}, f_{\alpha,k,H,W',\text{Short}} \rangle = 0$ if $W^* \cap (W')^* = \emptyset$. By the bounded overlap property of the sets $\{ W^* \}$, these functions are almost orthogonal, and so we conclude that
%
\begin{equation}
    \| f_{H,\text{Short}} \|_{L^2(X)} \lesssim \left( \sum\nolimits_k \sum\nolimits_{W \in \mathcal{W}_{\alpha,H}} \left\| f_{\alpha,k,H,W,\text{Short}} \right\|_{L^2(X)}^2 \right)^{1/2}.
\end{equation}
%
We can write $T_{k,l,\text{Short}} = a_{k,l}(P / 2^k)$, where
%
\begin{equation}
    a_{k,l}(t) = [2^l \widehat{\chi}(2^l \cdot) * a_k].
\end{equation}
%
Now
%
\begin{equation}
    \| a_{k,l}(P/2^k) \|_{L^2(X) \to L^2(X)} = \| a_{k,l} \|_{L^\infty(\RR)} \lesssim \| a_k \|_{L^\infty(\RR)} \lesssim \| a \|_{R^{s,p}[0,\infty)}, 
\end{equation}
%
and so
%
\begin{equation}
    \| f_{\alpha,k,H,W} \|_{L^2(X)} = \| T_{k,l,\text{Short}} \{ A_{\alpha,k,H,W} \} \|_{L^2(X)} \lesssim \| a \|_{R^{s,p}[0,\infty)} \| A_{\alpha,k,H,W} \|_{L^2(X)}.
\end{equation}
%
So
%
\begin{equation}
    \| f_{H,\text{Short}} \|_{L^2(X)} \lesssim \| a \|_{R^{s,p}[0,\infty)} H |\Omega_H|^{1/2}.
\end{equation}
%
By the finite propogation speed of the wave equation, $f_{H,\text{Short}}$ is supported on the set $\{ x: (M \chi_{\Omega_H})(x) \geq 1/10^d \}$, where $M$ is the Hardy-Littlewood maximal function; by the weak $L^1$ boundedness of $M$, this set has measure $O(|\Omega_H|)$. Thus we can use H\"{o}lder's inequality to conclude that for any $r \in [1,2]$,
%
\begin{equation}
    \| f_{H,\text{Short}} \|_{L^1(X)} \lesssim \| a \|_{R^{s,p}[0,\infty)} |\Omega_H|^{1/2} H |\Omega_H|^{1/2} = \| a \|_{R^{s,p}[0,\infty)} H |\Omega_H|.
\end{equation}
%
Performing a real interpolation between $r = 1$ and $r = 2$, which allows us to sum over the dyadic height scales $H$, we conclude that
%
\begin{equation}
    \| f_{\text{Short}} \|_{L^p(X)} \lesssim  \| a \|_{R^{s,p}[0,\infty)} \left( \sum H^p |\Omega_H| \right)^{1/p} \lesssim \| a \|_{R^{s,p}[0,\infty)} \| u \|_{L^p(X)}.
\end{equation}
%
This completes the analysis of the short range interactions.

\section{Long Range Bounds}

The long range interactions require slightly more work. Define an operator
%
\begin{equation}
\begin{split}
    S_{l,k} C = \sum\nolimits_{z_0} \sum\nolimits_{t_0 \geq 2^{l-k}} C(z_0,t_0) \sum\nolimits_{x_0 \in Q(z_0)} F_{x_0,t_0},
\end{split}
\end{equation}
%
where $U_{z_0} = \sum_{x_0 \in Q(z_0)} u_{x_0}$ and $F_{x_0,t_0} = \int b_{t_0}(t) \cos(2 \pi t P) \{ U_{z_0} \}\; dt$ for $L^1$ normalized functions $\{ b_{t_0} \}$. The bound $\text{Wave-Bound}(q)$ implies an exponential decay in $l$ on the $L^p$ operator norm of $S_{l,k}$.

\begin{lemma} \label{lemma:scaleupbound}
    Suppose $1 \leq p \leq q$, and that $\text{Wave-Bound}(q)$ is true. Then
    % |b_{t_0}|_{L^1} << 2^{-k/q}
    \begin{align*}
        &\| S_{l,k} \{ C \} \|_{L^p(X)}  \lesssim 2^{- l \varepsilon} \left( \sum\nolimits_{z_0} \sum\nolimits_{t_0 \geq 2^{l-k}} \left[ \langle 2^k t_0 \rangle^s |C(z_0,t_0)| 2^{(l-k)d/p} \| U_{z_0,t_0} \|_{L^\infty(X)} \right]^p \right)^{1/p},
    \end{align*}
    %
    where $s = (d-1)(1/p - 1/2)$, and
    %
    \[ \varepsilon = - \left( \frac{d-1}{2} \right) \left( \frac{1/p - 1/q}{1 - 1/q} \right), \]
    %
    which, in particular, is positive for $p < q$.
\end{lemma}
\begin{proof}
    By applying the triangle inequality, we may assume that the support of $C$ is $10$-separated in the $z_0$ variable. Thus any point $x_0$ is contained in a unique sidelength cube $Q(z(x_0))$ with $z(x_0)$ in $\text{supp}_z(C)$, or is not contained in any such cube. If we define $c(x_0,t_0) = C(z(x_0),t_0) \| u_{x_0} \|_{L^1(X)}$, then we can apply $\text{Wave-Bound}(q)$ and H\"{o}lder's inequality to conclude that
    %
    \begin{equation}
    \begin{split}
      \| S_{l,k} C \|_{L^q(X)} \lesssim \left( \sum\nolimits_{x_0} \sum\nolimits_{t_0 \geq 2^{l-k}} \left[ |C(z(x_0),t_0)| \| u_{x_0} \|_{L^q(X)} \langle 2^k t_0 \rangle^{s(q)} \right]^q \right)^{1/q},
    \end{split}
    \end{equation}
    %
    % H^qW^q
    % vs H^q W
    % << 1/R^{1-1/q}
    where $s(q) = (d-1)(1/q - 1/2)$. Since the supports of the functions $\{ u_{x_0} \}$ are almost disjoint, and since $U_{z_0}$ is supported on a set of measure $2^{(l-k)d}$,
    %
    \begin{equation}
        \sum\nolimits_{z(x_0) = z_0} \| u_{x_0} \|_{L^q(X)}^q \lesssim \| U_{z_0} \|_{L^q(X)}^q \lesssim 2^{(l-k)d} \| U_{z_0} \|_{L^\infty(X)}^q.
    \end{equation}
    %
    %
    Substituting this bound into the prior bound yields that
    %
    \begin{equation}
    \begin{split}
        \| S_{l,k} C \|_{L^q(X)} &\lesssim \left( \sum\nolimits_{z_0} \sum\nolimits_{t_0 \geq 2^{l-k}} \left[ |C(z_0,t_0)| \big[ 2^{(l-k)d/q} \| U_{z_0} \|_{L^\infty(X)} \big] \langle 2^k t_0 \rangle^{s(q)} \right]^q \right)^{1/q}.
    \end{split}
    \end{equation}
    % H^p W^{p/2}
    % vs. H^p W
    % 2^{k(d - (2d+1)/p)}
    % Provided that epsilon(q) <= -1/q' we're fine
    %       epsilon(1) = 0 WHICH IS GOOD
    %       epsilon(p_*) = d - (2d+1)/p_*
    % d - (2d+1)/p_* <= -1/p_*'
    % Holds iff
    %   d - (2d+1)/p_* <= -1 + 1/p_*
    % p_* <= 2
    We will interpolate this bound with an $L^1$ bound with exponential decay, which will yield the result. We should expect $F_{t_0,z_0}$ to be concentrated on a set of measure $O( 2^{l-k} t_0^{d-1} )$, namely, the annulus $\text{Ann}_{t_0,z_0}$ of width $O(2^{l-k})$ upon a sphere of radius $t_0$ centered at $z_0$. By H\"{o}lder's inequality, we find that
    % Xaybe use F = T^I_{x_0} U_{z_0} = \sum f_{x_0,t_0}
    \begin{equation}
    \begin{split}
        \| F_{t_0,z_0} \|_{L^1(\text{Ann}_{t_0,z_0})} &\lesssim \left( 2^{l-k} t_0^{d-1} \right)^{1/2} \| F_{t_0,z_0} \|_{L^2(X)}\\
        &\lesssim 2^{\frac{l-k}{2}} t_0^{\frac{d-1}{2}} \| U_{z_0} \|_{L^2(X)}\\
        &\lesssim 2^{(l-k) \left( \frac{d+1}{2}\right)} t_0^{\frac{d-1}{2}} \| U_{z_0} \|_{L^\infty(X)}.
    \end{split}
    \end{equation}
    %
    Here we used the fact that $F_{t_0,z_0} = \int b_{t_0}(t) ( \cos(2 \pi t P) \circ Q_k ) \{ U_{z_0} \}$, that $\| b_{t_0} \|_{L^1(\RR)} \leq 1$,  and that $\| \cos(2 \pi t P) \circ Q_k \|_{L^2(X) \to L^2(X)} \lesssim 1$. so that we can apply the triangle inequality to conclude that $\| T_{t_0}^I \|_{L^2(X) \to L^2(X)} \lesssim \| b_{t_0} \|_{L^1(X)} \leq 1$. On the other hand, we can use Lemmas \ref{PseudoOsicllatoryLemma} and \ref{lemma:WaveOscillatoryLemmaddw} to prove that $(\cos(2 \pi t P) \circ Q_k)(x,y) \lesssim_N [2^k |d(x,y) - t|]^{-N}$, uniformly for $|t| \lesssim 1$, and thus that
    %
    \begin{equation}
        \| F_{t_0,z_0} \|_{L^1(\text{Ann}_{t_0,z_0}^c)} \lesssim_N 2^{-lN} 2^{-k} t_0^{d-1} \| U_{z_0} \|_{L^1(X)} \lesssim 2^{-lN} 2^{-k(d+1)} t_0^{d-1} \| U_{z_0} \|_{L^\infty(X)}.
    \end{equation}
    %
    Since $t_0 \geq 2^{l-k}$, we have
    %
    \begin{equation}
        \| F_{t_0,z_0} \|_{L^1(\text{Ann}_{t_0,z_0}^c)} \lesssim_N 2^{-lN} 2^{-k \left( \frac{d+3}{2} \right)} t_0^{\frac{d-1}{2}} \| U_{z_0} \|_{L^\infty(X)},
    \end{equation}
    %
    which is smaller than the $L^1$ norm bound on $\text{Ann}_{t_0,z_0}$.
    %
    Summing in $t_0$ and $z_0$ using the triangle inequality gives that
    %
    \begin{equation}
    \begin{split}
        &\| S_{l,k} C \|_{L^1(X)}\\
        &\quad \lesssim 2^{(l-k) \left( \frac{d+1}{2}\right)} \sum\nolimits_{z_0} \sum\nolimits_{t_0 \geq 2^{l-k}} t_0^{\frac{d-1}{2}} |C(z_0,t_0)| \| U_{z_0} \|_{L^\infty(X)}\\
        &\quad \lesssim 2^{-l \left( \frac{d-1}{2} \right)} \sum\nolimits_{z_0} \sum\nolimits_{t_0 \geq 2^{l-k}} |C(z_0,t_0)| \left[  2^{(l-k)d} \| U_{z_0} \|_{L^\infty(X)}\right] \langle 2^k t_0 \rangle^{\frac{d-1}{2}}.
    \end{split}
    \end{equation}
    %
    % 2^{k (d-1)(2)}
    %
    The argument is concluded by interpolation.
    % -l(d-1)/2
    %
\end{proof}

We now use this lemma to control the function $f_{\text{Long}}$. Because of the exponential decay in $l$ given by the lemma above, we may sum in $l$ trivially using the triangle inequality for $1 \leq q < p$. Using $L^2$ orthogonality, if $f_{\text{Long}} = \sum_k f_{\text{Long},k}$, then we find that
%
\begin{equation}
    \left\| \sum\nolimits_k f_{\text{Long},k} \right\|_{L^p(X)} = \left( \sum \| f_{\text{Long},k} \|_{L^p(X)}^p \right)^{1/p}.
\end{equation}
%
Now write $f_{\text{Long},k} = \sum f_{\text{Long},\alpha,l,k}$ with
%
\begin{equation}
    f_{\text{Long},\alpha,l,k} = \sum\nolimits_H \sum\nolimits_{\substack{W \in \mathcal{W}_{\alpha,H}\\l(W) = l - k}} T_{k,l,\text{Long}} \{ A_{\alpha,k,H,W} \}.
\end{equation}
%
Applying Lemma \ref{lemma:scaleupbound} with $l$ and $k$ as above, $C(z_0,t_0) =  \| b_{t_0} \|_{L^1(X)}$, and with $U_{z_0} = \sum\nolimits_H A_{\alpha,k,H,Q(z_0)}$, we conclude that %where $W = [z_0, z_0 + 2^{l-k}]$ we conclude that
% W^{1/p - 1}
\begin{equation}
\begin{split}
    \| f_{\text{Long},\alpha,l,k} \|_{L^p(X)} &\lesssim 2^{-l \varepsilon} \left( \sum\nolimits_{z_0} \sum\nolimits_{t_0 \geq 2^{l-k}} [ \langle 2^k t_0 \rangle^{s} \| b_{t_0} \|_{L^1(X)} ]^p \big[ 2^{(l-k)d} \| U_{z_0} \|_{L^\infty(X)}^p \big] \right)^{1/p}\\
    &\lesssim \| a \|_{R^{s,p}[0,\infty)} 2^{-l \varepsilon} \left(  \sum\nolimits_H \sum\nolimits_{\substack{W \in \mathcal{W}_{\alpha,H}\\l(W) = l-k}} |W| \| A_{\alpha,k,H,W} \|_{L^\infty(X)}^p \right)^{1/p}.
\end{split}
\end{equation}
% L^1 = H W
% L^p = 
Summing in $k$ using $L^p$ orthogonality, and summing over $\alpha$ trivially, we find that
%
\begin{equation}
    \| f_{\text{Long},l} \|_{L^p(X)} \lesssim 2^{-l \varepsilon}  \| a \|_{R^{s,p}[0,\infty)} \left( \sum\nolimits_H H^p |\Omega_H| \right) \lesssim 2^{-l \varepsilon}  \| a \|_{R^{s,p}[0,\infty)} \| u \|_{L^p(X)}. 
\end{equation}
%
Summing in $l$ trivially gives $\| f_{\text{Long}} \|_{L^p(X)} \lesssim \| a \|_{R^{s,p}[0,\infty)} \| u \|_{L^p(X)}$, completing the proof of the long range estimates, and thus the proof.

\section{Wave Bounds on the Sphere}

We end this chapter with a proof that $\text{Wave-Bound}(p)$ holds on $S^d$ for $1/p - 1/2 > 1/(d-1)$, which completes the proof of Theorem \ref{maintheoremsphere}.

\begin{lemma}
    The operator $P_{\text{SH}}$ satisfies $\text{Wave-Bound}(p)$ for $1/p - 1/2 > (d-1)^{-1}$.
\end{lemma}
\begin{proof}
    Here $\Pi = 1$. If $\varepsilon > 0$, and a given function $c$ satisfies $c(x_0,t_0) = 0$ for $t_0 \in [1/2 - \varepsilon, 1/2 + \varepsilon]$, then the bound has already been proven in this paper; indeed, it follows from Lemma \ref{LpBoundLemma}, where $R = 2^k$, since for $t_0 \leq 1/2$ we can write $2 \cos(2 \pi t P) = e^{2 \pi i t P} + e^{-2 \pi i t P}$, and for $t_0 \geq 1/2$ we can write $2 \cos(2 \pi t P) = e^{2 \pi i (t - 1) P} + e^{-2 \pi i (t - 1) P}$. Applying Lemma \ref{LpBoundLemma} to each piece of the exponential, we have that, if $f_{x_0,t_0}$ is as in the definition of the assumption $\text{Wave-Bound}(p)$,
    %
    \begin{equation}
    \begin{split}
        &\left\| \sum\nolimits_{\substack{(x_0,t_0) \in \mathcal{X}_k \times \mathcal{T}_k\\|t_0| \leq 1/2 - \varepsilon}} c(x_0,t_0) f_{x_0,t_0} \right\|_{L^p(S^d)} \lesssim 2^{kd/p'} \left( \sum\nolimits_{(x_0,t_0) \in \mathcal{X}_k \times \mathcal{T}_k} \Big[ |c(x_0,t_0)| \langle R t_0 \rangle^{s} \Big]^p \right)^{1/p}
    \end{split}
    \end{equation}
    %
    and
    %
    \begin{equation}
    \begin{split}
        &\left\| \sum\nolimits_{\substack{(x_0,t_0) \in \mathcal{X}_k \times \mathcal{T}_k\\|t_0| \geq 1/2 + \varepsilon}} c(x_0,t_0) f_{x_0,t_0} \right\|_{L^p(S^d)} \\
        &\quad\quad \lesssim 2^{kd/p'} \left( \sum\nolimits_{\substack{(x_0,t_0) \in \mathcal{X}_k \times \mathcal{T}_k\\|t_0| \geq 1/2 + \varepsilon}} \Big[ |c(x_0,t_0)| \langle R (t_0 - 1) \rangle^{s} \Big]^p \right)^{1/p}\\
        &\quad\quad \lesssim 2^{kd/p'} \left( \sum\nolimits_{\substack{(x_0,t_0) \in \mathcal{X}_k \times \mathcal{T}_k\\|t_0| \geq 1/2 + \varepsilon}} \Big[ |c(x_0,t_0)| \langle R t_0 \rangle^{s} \Big]^p \right)^{1/p}.
    \end{split}
    \end{equation}
    %
    It remains to prove the result for $c$ supported on $[1/2 - \varepsilon, 1/2 + \varepsilon]$. To obtain this result, if $Ux = -x$ is the reflection operator on $S^d$, then for $t \in [1/2 - \varepsilon, 1/2 + \varepsilon]$, the operators $U \circ e^{2 \pi i t P}$ have the same conical relation as $e^{2 \pi i t P}$, and so writing out these operators using oscillatory integrals following the proof of Proposition \ref{theMainEstimatesForWave}, one can prove that the operators $S\!_{x_0,t_0} = U \circ \int b_{1/2 + t_0} e^{2 \pi i t P} \{ u_{x_0} \}$ satisfy the pointwise and orthogonality estimates of Proposition \ref{theMainEstimatesForWave}. Thus we can apply Lemma \ref{LpBoundLemma} to obtain bounds of the form
    %
    \begin{equation}
    \begin{split}
        &\left\| \sum\nolimits_{\substack{(x_0,t_0) \in \mathcal{X}_k \times \mathcal{T}_k\\t_0 \in [1/2 - \varepsilon, 1/2 + \varepsilon]}} \langle 2^k t_0 \rangle^{\frac{d-1}{2}} c(x_0,t_0) f_{x_0,t_0} \right\|_{L^p(S^d)} \\
        &\quad\quad \lesssim 2^{kd/p'} \left( \sum\nolimits_{\substack{(x_0,t_0) \in \mathcal{X}_k \times \mathcal{T}_k\\|t_0| \geq 1/2 + \varepsilon}} \Big[ |c(x_0,t_0)| \langle R (t_0 - 1/2) \rangle^{s} \Big]^p \right)^{1/p}\\
        &\quad\quad \lesssim 2^{kd/p'} \left( \sum\nolimits_{\substack{(x_0,t_0) \in \mathcal{X}_k \times \mathcal{T}_k\\|t_0| \geq 1/2 + \varepsilon}} \Big[ |c(x_0,t_0)| \langle R t_0 \rangle^{s} \Big]^p \right)^{1/p}.
    \end{split}
    \end{equation}
    %
    Combining these three bounds proves $\text{Wave-Bound}(p)$ holds.
\end{proof}

\begin{remark}
    We note that this analysis requires an understanding of the global geometric of the geodesic flow on $S^d$, in particular, that points flow into a conjugate point exactly as they exit a starting point. A similar analysis may yield $\text{Wave-Bound}(p)$ for operators on the rank one symmetric spaces, but is unlikely to be easy to obtain on an arbitrary Zoll manifold with periodic geodesic flow, due to the absense of information about the geometric properties of conjugate points.
\end{remark}

\part{Appendices}
%!TEX root = ../main.tex

%\renewcommand{\thechapter}{A}
\chapter{Elementary Theorems} \label{cha:elementary_theorems}

In this appendix, we list several results that occur frequently in the harmonic analysis literature. The first result, Euler's homogeneous function theorem, provides a way to relate a homogeneous function and it's derivatives.

\begin{theorem}[Euler's Homogeneous Function Theorem] \label{thm:Euler}
    Suppose $f: {\dot{\RR}{\vphantom{\RR}}^p} \to \CC$ is a smooth, homogeneous function of order $s$. Then for any $x \in {\dot{\RR}{\vphantom{\RR}}^p}$, $x \cdot \nabla f(x) = s f(x)$.
\end{theorem}
\begin{proof}
    Differentiate the relation $f(ax) = a^s f(x)$ with respect to $a$, and then set $a = 1$.
\end{proof}

The Sobolev embedding theorem allows us to trade smoothness for integrability.

\begin{theorem}[The Sobolev Embedding Theorem] \label{Chp:Sobolev}
    Suppose $X$ is a $d$-dimensional manifold, or $X = \RR^d$. Fix $1 \leq p \leq q < \infty$ and let $r = d(1/p - 1/q)$. Then for any $s \in \RR$, $W^{s,q}(X) \subset W^{s-r,p}(X)$, and if $s > d/p$, $W^{s,p}(X) \subset C_b(X)$, the space of continuous, bounded functions, where these inclusions are continuous.
\end{theorem}
\begin{proof}
    For the case where $X = \RR^d$, see Theorem 6.2.4 of \cite{GrafakosModern}. To obtain the result on a general compact manifold, apply the Sobolev embedding theorem in each coordinate system.
\end{proof}

\begin{comment}
\begin{theorem}[Bernstein's Inequality]
    Fix $1 \leq p \leq \infty$, $s \in \RR$, and $R > 0$. If the Fourier transform of a function $f: \RR^d \to \CC$ is supported on $\{ \xi : 0 \leq |\xi| \leq R \}$, then
        %
    \[ \| f \|_{W^{s,p}(\RR^d)} \sim \langle R \rangle^s \| f \|_{L^p(\RR^d)}. \]% \quad\text{and}\quad \| f \|_{\dot{B}^{s,p}_r(\RR^d)} \sim R^s \| f \|_{L^p(\RR^d)}. \]
    Similarily, suppose $X$ is a compact manifold, and $P$ is a classical, elliptic, self-adjoint operator of order one on $X$. If $f: \RR^d \to \CC$ can be written as a linear combination of eigenfunctions of $P$, whose eigenvalues are contained in the interval $[0,R]$, then
        %
        \[ \| f \|_{W^{s,p}(X)} \sim \langle R \rangle^s \| f \|_{L^p(\RR^d)} \]
\end{theorem}
\begin{proof}
    For the result on $\RR^d$, see Proposition 5.3 of \cite{Wolff}.
\end{proof}
\end{comment}

Interpolation is essential for reducing bounds to Lebesgue norms in which the boundedness of the operators is simpler to understand. If $X$ and $Y$ are measure spaces, recall that an operator $T$ mapping elements of $L^p(X)$ to measurable functions on $Y$ is said to be of \emph{restricted weak type} $(p,q)$ if it has the property that for any measurable set $E \subset X$,
%
\begin{equation}
    \| T \mathbb{I}_E \|_{L^{q,\infty}(Y)} \lesssim \| \mathbb{I}_E \|_{L^p(X)} = |E|^{1/p}.
\end{equation}
%
where $\mathbb{I}_E$ is the indicator function of $E$, and the implicit constant is uniform in $E$.

\begin{theorem}[The Marcinkiewicz Interpolation Theorem] \label{thm:marci}
    Let $X$ and $Y$ be measure spaces, and fix exponents $1 \leq p_0,p_1 < \infty$ and $1 \leq q_0,q_1 < \infty$. For $\theta \in [0,1]$, define $p_\theta$ and $q_\theta$ so that
    %
    \[ 1/p_\theta = \theta/p_1 + (1-\theta)/p_0 \quad\text{and}\quad 1/q_\theta = \theta/q_1 + (1-\theta)/q_0. \]
    %
    Let $T$ be a linear operator mapping elements of $L^{p_0}(X) + L^{p_1}(X)$ to measurable functions on $Y$. Then if $T$ is of restricted weak type $(p_0,q_0)$ and $(p_1,q_1)$, then for any $\theta \in (0,1)$, the operator $T$ is bounded from $L^{p_\theta}(X)$ to $L^{q_\theta}(Y)$.
\end{theorem}
\begin{proof}
    See Theorem 1.4.19 of \cite{Grafakos}.
\end{proof}

Schur's Test for integral kernels is a simple, but often optimal method, for obtaining the boundedness of integral operators from an $L^1$ norm, or into an $L^\infty$ norm. Interpolation can then be used to obtain certain bounds for operators with respect to other norms.

\begin{theorem}[Schur's Test for Integral Kernels] \label{thm:Schurl}
    Let $X$ and $Y$ be measure spaces.
    %
    \begin{itemize}
        \item If $K_T \in L^\infty(Y) L^p(X)$ and $f \in L^1(Y)$, then for almost every $x$, the integral
        %
        \[ Tf(x) = \int K_T(x,y) f(y)\; dy \]
        %
        is absolutely integrable, and defines an operator from $L^1(Y)$ to $L^p(X)$ such that
        %
        \[ \| T \|_{L^1(Y) \to L^p(X)} \leq \| K_T \|_{L^\infty(X) L^p(Y)}. \]

        \item If $K_T \in L^\infty(X) L^{p'}(Y)$, then for each $f \in L^p(Y)$, and for almost every $x$, the integral
        %
        \[ Tf(x) = \int K_T(x,y) f(y)\; dy \]
        %
        is absolutely integrable, defining an operator $T$ from $L^p(Y)$ to $L^\infty(X)$, such that
        %
        \[ \| T \|_{L^p(Y) \to L^\infty(X)} \leq \| K_T \|_{L^\infty(Y) L^{p'}(X)}. \]
    \end{itemize}
\end{theorem}
\begin{proof}
    For any measurable $f: Y \to \CC$, define $T^\dagger f: X \to [0,\infty]$ by setting
    %
    \begin{equation}
        T^\dagger f(x) = \int |K_T(x,y)| f(y)|\; dy.
    \end{equation}
    %
    Applying Minkowski's inequality and H\"{o}lder's inequality gives the bounds
    %
    \begin{equation}
        \| T^\dagger f \|_{L^p(X)} \leq \| K_T \|_{L^\infty(X) L^p(Y)} \| f \|_{L^1(Y)}
    \end{equation}
    %
    and
    \begin{equation}
        \| T^\dagger f \|_{L^\infty(X)} \leq \| K_T \|_{L^\infty(Y) L^{p'}(X)} \| f \|_{L^p(Y)}.
    \end{equation}
    %
    These bounds immediately imply the required results for the integral operator $T$.
\begin{comment}
    If $K_T \in L^\infty(Y) L^p(X)$, then for any measurable function $f$ on $Y$, the integral
    %
    \begin{equation} \label{equation189203u190248231094583129}
        \left( \int \left( \int |K_T(x,y)| |f(y)|\; dy \right)^p\; dx \right)^{1/p}
    \end{equation}
    %
    is well defined, though possibly infinite. Applying Minkowski's inequality, we find that
    %
    \begin{equation} \label{equatyion12890u3091284587u93125tuy98324hfv9i8euwhf}
    \begin{split}
        \left( \int \left( \int |K_T(x,y)| |f(y)|\; dy \right)^p\; dx \right)^{1/p} &\leq \int \left( \int |K_T(x,y)|^p |f(y)|^p\; dx \right)^{1/p}\; dy\\
        &\leq \int \| K_T \|_{L^p(X)} |f(y)|\; dy\\
        &\leq \| K_T \|_{L^\infty(X) L^p(Y)} \| f \|_{L^1(Y)}.
    \end{split}
    \end{equation}
    %
    Thus if $f \in L^1(Y)$, \eqref{equation189203u190248231094583129} is finite, and so $\int |K_T(x,y)| |f(y)|\; dy$ is finite for almost every $x \in X$, so that $Tf$ is well defined as a measurable function on $X$. But the triangle inequality and \eqref{equatyion12890u3091284587u93125tuy98324hfv9i8euwhf} imply the bound $\| Tf \|_{L^\infty(X)} \leq \| K_T \|_{L^\infty(X) L^p(Y)} \| f \|_{L^1(Y)}$.
\end{comment}
\end{proof}

We will rely on the following `$L^p$ orthogonality' result.

\begin{lemma} \label{lem:LpOrthogonalityEasyLemma}
    Let $X$ be a measure space. Suppose $\{ T_j \}$ are operators on $X$ which satisfy
    %
    \[ \| T_j \|_{L^1(X) \to L^{1,\infty}(X)} \lesssim 1, \]
    %
    uniformly in $j$, and suppose that the bound
    %
    \[ \left\| \sum T_j f_j \right\|_{L^2(X) \to L^2(X)} \lesssim \left( \sum\nolimits_j \| f_j \|_{L^2(X)}^2 \right)^{1/2} \]
    %
    holds. Then for $1 < p \leq 2$,
    %
    \[ \left\| \sum T_j f_j \right\|_{L^p(X)} \lesssim \left( \sum \| f_j \|_{L^p(X)}^p \right)^{1/p}. \]
\end{lemma}
\begin{proof}
    Define a vector-valued operator $T$ from sequences of functions on $X$ to functions on $X$, such that for a sequence $f = \{ f_j \}$, $Tf = \sum T_j f_j$. Then by assumption,
    %
    \begin{equation}
        \| Tf \|_{L^2(X)} \lesssim \| f \|_{l^2(\NN) L^2(X)},
    \end{equation}
    %
    and by the triangle inequality,
    %
    \begin{equation}
        \| Tf \|_{L^{1,\infty}(X)} \lesssim \| f \|_{l^1(\NN) L^1(X)}.
    \end{equation}
    %
    Real interpolation (i.e. the Marcinkiewicz interpolation theorem) thus tells us that for $1 < p < \infty$,
    %
    \begin{equation}
        \| Tf \|_{L^p(X)} \lesssim \| f \|_{l^p(\NN) L^p(X)},
    \end{equation}
    %
    which implies the required result.
\end{proof}

Finally, we will often use a helpful interpolation lemma.

\begin{lemma}[A Real Interpolation Lemma] \label{theoremrealinterpolation}
    Consider a family of functions $\{ f_H \}$, where $H$ ranges over the dyadic numbers, and fix $0 < p_0 < p_1 < \infty$. If for each $H$,
    %
    \[ \| f_H \|_{L^{p_0}(X)} \lesssim H W_H^{1/p_0} \quad\text{and}\quad \| f_H \|_{L^{p_1}(X)} \lesssim H W_H^{1/p_1}, \]
    %
    with implicit constants uniform in $H$, then for any $p_0 < p < p_1$,
    %
    \[ \left\| \sum\nolimits_H f_H \right\|_{L^p(X)} \lesssim \left( \sum\nolimits_H H^p W_H \right)^{1/p}. \]
\end{lemma}
\begin{proof}
    See Lemma 2.2 of \cite{HeoandNazarovandSeeger}.
\end{proof}

In an application of Lemma \ref{theoremrealinterpolation}, the quantities $H$ normally stand for the `height' of a given input, and the quantities $\{ W_H \}$ the `width', and so this interpolation result shows that if we are not proving results at an endpoint, it suffices to prove bounds for `a fixed height scale'.









%\renewcommand{\thechapter}{B}
\chapter{Distributions Defined by Oscillatory Distributions} \label{cha:distributions_defined_by_oscillatory_distributions}

The kernels of many of the Schwartz operators $T$ we study in this thesis have kernels which are defined by certain formal expression of the form
%
\begin{equation}
    K_T(x,y) = \int_{\RR^p} a(x,y,\theta) e^{i \phi(x,y,\theta)}\; d\theta,
\end{equation}
%
where $a$ is a symbol, and $\phi$ is smooth and homogeneous of order one.  In this appendix we show how these formal expressions can be interpreted in a way that ensures these expressions are well defined as distributions, provided that $\nabla_{x,y} \phi$ and $\nabla_\theta \phi$ have no common zeros. This ensures that, when $K_T$ is integrated against a pair of test functions on $\RR^m$ and $\RR^n$, the phase becomes rapidly oscillatory at high frequencies, which ensures convergence of appropriate integrals. We call a smooth, homogeneous function $\phi$ such that $\nabla_{x,y,\theta} \phi$ is non-vanishing a \emph{phase function}.

\begin{theorem}
    Let $\Omega$ be an open subset of $\RR^n$. Let $a: {\Omega} \times\; {\dot{\RR}{\vphantom{\RR}}^p} \to \CC$ be a symbol, and let $\phi: W\times\; \dot{\RR}{\vphantom{\RR}}^p \to \RR$ be a phase function. Fix $\chi \in C_c^\infty(\RR^p)$, equal to one in a neighborhood of the origin. The smooth functions
    %
    \[ I^{a,\theta}_R = \int_{\RR^p} a(x,\theta) \chi(\theta / R) e^{2 \pi i \phi(x,\theta)}\; d\theta \]
    %
    converge in the weak $*$ topology to a distribution $I$, independent of the choice of $\chi$. 
\end{theorem}
\begin{proof}
    If $\mu < -p$, then
    %
    \begin{equation}
        \int_{\RR^p} a(x,\theta) e^{2 \pi i \phi(x,\theta)}\; d\theta
    \end{equation}
    %
    is absolutely integrable, and the result follows from the dominated convergence theorem. For $\mu \geq -p$, fix $f \in C_c^\infty(\RR^d)$ supported on a compact set $K \subset \Omega$. Our goal is to show the quantities
    %
    \begin{equation}
        \langle I^{a,\theta}_R, f \rangle = \int a(x,\theta) \chi(\theta / R) e^{2 \pi i \phi(x,\theta)}\; d\theta
    \end{equation}
    %
    converges as $R \to \infty$ to a quantity independent of the choice of $\chi$, and that moreover, there exists $N > 0$ such that
    %
    \begin{equation}
        |\langle I^{a,\theta}_R, f \rangle| \lesssim \| f \|_{C^M(\RR^d)},
    \end{equation}
    %
    uniformly in $R$. Write
    %
    \begin{equation}
        a_R(x,\theta) = a(x, \theta) \chi(\theta / R).
    \end{equation}
    %
    We will assume without loss of generality that $a(x,\theta) = 0$ for $|\theta| \leq 1$, since the quantity
    %
    \begin{equation}
        \int a(x,\theta) \mathbb{I} \big(|\theta| \leq 1 \big) e^{2 \pi i \phi(x,\theta)}\; d\theta
    \end{equation}
    %
    defines a function in $C^\infty(\RR^d)$. We now apply the principle of non-stationary phase, i.e. integrating by parts. Consider the \emph{homogenized gradient}
    %
    \begin{equation}
        (\nabla^H\! \phi)(x,\theta) = ( \nabla_x \phi, |\theta| \nabla_\theta \phi(x,\theta) ).
    \end{equation}
    %
    Then $\nabla_{x,\theta}^H \phi$ is homogeneous of degree one in the $\theta$ variable, and $|\nabla_{x,\theta}^H \phi| > 0$. We note that
    %
    \begin{equation}
        (\nabla_{x,\theta}^H e^{2 \pi i \phi} ) = (2 \pi i) e^{2 \pi i \phi} \nabla_{x,\theta}^H \phi.
    \end{equation}
    %
    Note also that the \emph{formal transpose} of $\nabla_{x,\theta}^H$ is the differential operator $L$ given for pairs of functions $F_1: \RR^d \times \RR^p \to \RR^d$ and $F_2: \RR^d \times \RR^p \to \RR^p$ by the formula
    %
    \begin{equation}
    \begin{split}
        L(F_1, F_2) &= - ( \nabla_x \cdot F_1 + \nabla_\theta \cdot ( |\theta| F_2 ))\\
        &= - \left( \nabla_x \cdot F_1 + |\theta|^{-1} (\theta \cdot F_2) + |\theta| (\nabla_\theta \cdot F_2) \right),
    \end{split}
    \end{equation}
    i.e. so that for smooth, compactly supported functions $F_1$, $F_2$, and $G$,
    %
    \begin{equation}
        \int_{\RR^d \times \RR^p} (F_1,F_2) \cdot \nabla^H_{x,\theta} G = \int_{\RR^d \times \RR^p} L(F_1,F_2) \cdot G.
    \end{equation}
    %
    Thus we conclude that
    %
    \begin{equation}
    \begin{split}
        \langle I^{a,\theta}_R, f \rangle &= \frac{1}{2 \pi i} \int \frac{a \cdot f}{|\nabla_{x,\theta}^H \phi|^2} \left( \nabla_{x,\theta}^H e^{2 \pi i \phi} \right) \cdot (\nabla_{x,\theta}^H \phi)\\
        &= \frac{1}{2 \pi i} \int L \left\{ \frac{a \cdot f}{|\nabla_{x,\theta}^H \phi|^2} \nabla_{x,\theta}^H \phi \right\} e^{2 \pi i \phi}.
    \end{split}
    \end{equation}
    %
    Expanding out the differential operator $L$, we see we can write
    %
    \begin{equation}
        \langle I^{a,\phi}_R, f \rangle = \sum\nolimits_{|\alpha| \leq 1} \langle I^{a_\alpha, \phi}_R, \partial^\alpha \! f \rangle,
    \end{equation}
    %
    where $a_\alpha$ is a symbol of order at most $\mu - 1$. Iterating this argument, we find that for any $N > 0$, we can write
    %
    \begin{equation}
        \langle I^{a,\theta}_R, f \rangle = \sum\nolimits_{|\alpha| \leq N} \langle I^{a_\alpha, \phi}_{R}, \partial^\alpha\! f \rangle,
    \end{equation}
    %
    where $a_{\alpha}$ is a symbol of order at most $\mu - N$. If $N > \mu + p$, then we can apply the dominated convergence theorem to each term on the right hand side, from which the result follows.
\end{proof}









%\renewcommand{\thechapter}{C}
\chapter{Pseudo-Differential Operators} \label{appendixpsueiodjaweiodj}

In this appendix we briefly introduce the theory of pseudo-differential operators relevant to this thesis. Recall that a \emph{pseudo-differential operator} on $\RR^d$ of order $s \in \RR$ is a bounded Schwartz operator $T$ whose kernel is an oscillatory integral distribution of the form
%
\begin{equation}
    K_T(x,y) = \int_{\RR^d \times \RR^d} a(x,\xi) e^{2 \pi i \xi \cdot (x - y)}\; d\xi,
\end{equation}
%
where $a: {\RR^d} \times {\dot{\RR}{\vphantom{\RR}}^d} \to \CC$ is a fixed symbol of order $s$. The operator $T$ is often written $a(x,D)$, because if $a$ is a polynomial in $\xi$, i.e.
%
\begin{equation}
    a(x,\xi) = \sum\nolimits_\alpha c_\alpha(x) \xi^\alpha,
\end{equation}
%
then the pseudo-differential operator $a(x,D)$ is really the differential operator $\sum c_\alpha(x) D^\alpha$, where $D_j = (2\pi i)^{-1} \partial_j$ is a self-adjoint normalization of the partial derivative operator.

An important property of pseudo-differential operators is that they are \emph{pseudo-local}, in the sense that the Schwartz kernel $K_T$ is a smooth function away from the diagonal $\Delta_{\RR^d} = \{ (x,x): x \in \RR^d \}$, and $K_T$ and all of it's derivatives rapidly decay away from the diagonal\footnote{See e.g. Proposition 1 of Chapter 5, Section 4 of \cite{BigStein} for a proof of this rapid decay.}, i.e. if we write $k_T(x,z) = K_T(x,x+z)$, then
%
\begin{equation} \label{eq1029312094312093821094u29th}
    | \partial_x^\alpha \partial_z^\beta k_T(x,z)| \lesssim_{\alpha,\beta,N} |z|^{-d-s-|\beta| - N} \quad\text{for all $N > 0$}.
\end{equation}
%
Using duality, one can extend the definition of the operator $T$ to the space of tempered distributions, and the estimates in \eqref{eq1029312094312093821094u29th} imply that if $f$ is a distribution smooth in a neighborhood of a given point, then $Tf$ is smooth in a neighborhood of this point.
 % and satisfies, for $|x - y| \geq 1$, estimates of the form
%
%\[ |\partial_x^\alpha \partial_y^\beta K_T(x,y)| \lesssim_{\alpha,\beta,N} \langle x - y \rangle^{-N} \]
%
%for arbitrarily large $N > 0$.

A symbol $a$ of order $s$ is \emph{classical} if it has an expansion of the form
%
\begin{equation}
    a \sim \sum\nolimits_{k = 0}^\infty a_{s - k},
\end{equation}
%
where $a_{s-k}: {\RR^d} \times  {\dot{\RR}{\vphantom{\RR}}^d} \to \CC$ is a smooth function which is homogeneous of order $s - k$ in the second variable. A pseudo-differential operator $T = a(x,D)$ is then \emph{classical} if the symbol $a$ is classical. We then call $a_s$ the \emph{principal symbol} of the operator $T$. A classical pseudo-differential operator is \emph{elliptic} if it's principal symbol is non-vanishing.

All this generalizes to the setting of compact manifolds by working in coordinates. Let $X$ be a compact manifold, and consider a cover of $X$ by open sets $\{ U_\alpha \}$, where the closure of $U_\alpha$ is contained in an open set $V_\alpha$ which is diffeomorphic to a pre-compact subset $W_\alpha$ of $\RR^d$ by a diffeomorphism $F_\alpha: V_\alpha \to W_\alpha$. Consider a partition of unity $\{ \psi_\alpha \}$ of $X$, subordinate to the cover $\{ U_\alpha \}$, and also consider a smooth function $\phi_\alpha$ on $\RR^d$, equal to one on $F_\alpha(U_\alpha)$ but vanishing outside of $W_\alpha$. A Schwartz operator $T$ on a compact $d$-dimensional manifold $X$ is a \emph{pseudo-differential operator} of order $s > 0$ if it's Schwartz kernel $K_T$, defined with respect to any fixed smooth volume density on $X$, agrees with a smooth function away from the diagonal $\Delta_X = \{ (x,x): x \in X \}$, and if for each $\alpha$, the operators $T_\alpha$ on $\RR^d$ given by
%
\begin{equation}
    (T_\alpha f)(x) = \phi_\alpha(x)\; T \{ \psi_\alpha \cdot (f \circ F_\alpha) \} (F_\alpha^{-1}(x))
\end{equation}
%
are pseudo-differential operators on $\RR^d$. The operator $T$ is classical if each of the operators $T_\alpha$ is classical, and if $T_\alpha$ has principal symbol $p_\alpha(x,\xi)$ we can then define the principal symbol of $T$ to be the function $p: T^* X - \{ 0 \} \to \CC$ defined by
%
\begin{equation}
    p(x,\xi) = \sum \psi_\alpha(x) p_\alpha(x, DF_\alpha(x)^{-T} \xi).
\end{equation}
%
One verifies using the change-of-variable formulas for pseudo-differential operators\footnote{See Theorem 18.1.17 of \cite{Hormander3}.} that if $T$ is a pseudo-differential operator for one choice of cover $\{ U_\alpha \}$, then it is a pseudo-differential operator for \emph{all} choices of covers, and that for a classical pseudo-differential operator the function $p$ is invariant of this choice when it is interpreted as defined on the cotangent bundle of $X$. A classical pseudo-differential operator $T$ on a compact manifold is then \emph{elliptic} if it's principal symbol is non-vanishing.

%\chapter{Future Work}
%%!TEX root = ../main.tex

\section{Exploiting Tangency Bounds in $\RR^3$ and $\RR^4$}

The results of Heo, Nazarov, and Seeger only apply when $d \geq 4$. Cladek found a method to get an initial radial multiplier conjecture result in $\RR^3$, and an improvmeent of the bounds obtained by Heo, Nazarov, and Seeger when $d = 3$. The idea is to exploit the fact that one need only prove a version of \ref{lemma2} for a set $\mathcal{E} = \mathcal{E}_X \times \mathcal{E}_R$, where $\mathcal{E}_X$ is a one-separated family of points, and $\mathcal{E}_R$ are a family of radii. One can then exploit this Cartesian product structure when analyzing functions of the form
%
\[ F = \sum_{(x,r) \in \mathcal{E}} \chi_{x,r}, \]
%
in particular, improving upon the result of \cite{HeoandNazarovandSeeger}.

\subsection{Result in 3 Dimensions}

As in \cite{HeoandNazarovandSeeger}, Cladek first performs a density decomposition, i.e. writing
%
\[ F = \sum F_k^m \]
%
where
%
\[ F_k^m = \sum_{(x,r) \in \mathcal{E}_k(2^m)} \chi_{x,r}. \]
%
Cladek then interpolates between an $L^0$ bound and an $L^2$ bound on the resulting functions. The $L^0$ bound is exactly the same bound used in \cite{HeoandNazarovandSeeger}.

\begin{theorem}
    For the function $F$, we have
    %
    \[ |\text{supp}(F_k^m)| \lesssim 2^{-m} 4^k \# \mathcal{E}_k \]
    %
    and thus
    %
    \[ |\text{supp}(F^m)| \lesssim \sum_k 2^{-m} 4^k \# \mathcal{E}_k. \]
\end{theorem}

The $L^2$ bound is improved upon, which is what allows us to obtain a new result in three dimensions.

\begin{lemma} \label{cladeksl2}
    Suppose $\mathcal{E} = \bigcup_k \mathcal{E}_k$ is a one-separated set, where $\mathcal{E}_k \subset \RR^d \times [2^k,2^{k+1})$ is a set of density type $(2^m, 2^k)$. Then
    %
    \[ \left\| \sum_{(x,r) \in \mathcal{E}} \chi_{x,r} \right\|_{L^2(\RR^d)} \lesssim_\varepsilon 2^{[(11/13) + \varepsilon] m} \sum_k 4^k \# \mathcal{E}_k. \]
\end{lemma}

Interpolation thus yields that for a set of density type $2^m$ as in this Lemma,
%
\[ \| \sum_{(x,r) \in \mathcal{E}} \chi_{x,r} \|_{L^p(\RR^d)} \lesssim_\varepsilon 2^{-m(1/p - 12/13 - \varepsilon)} ( \sum_k 4^k \# \mathcal{E}_k )^{1/p}. \]
%
If $1 < p < 13/12$, this sum is favorable in $m$, and may be summed without harm to prove the radial multiplier conjecture for unit scale radial multipliers in this range.

\begin{proof} [Proof of Lemma \ref{cladeksl2}]
    Write
    %
    \[ F_k = \sum_{(x,r) \in \mathcal{E}_k} \chi_{x,r}. \]
    %
    As before, we can throw away terms for $k \leq 10 m$, i.e. obtaining that
    %
    \[ \| \sum F_k \|_{L^2(\RR^d)} \lesssim m^{1/2} \left( \sum_k \| F_k \|_{L^2(\RR^d)}^2 + \sum_{10m < k < k'} |\langle F_k, F_{k'} \rangle| \right)^{1/2}. \]
    %
    Our proof thus splits into two cases: where the radii are incomparable, and where the radii are comparable.

    TODO:
\end{proof}

\subsection{Results in 4 Dimensions}

TODO

\section{Manifolds With Periodic Geodesic Flow}

\section{Abstract Reductions In Manifolds With Constant Sectional Curvature}

%\appendix
%\chapter{Appendix Title}
%Appendix goes here...

\singlespacing
\printbibliography

\end{document}
